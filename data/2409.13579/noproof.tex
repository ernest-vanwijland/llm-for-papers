

\documentclass[authorcolumns,numberwithinsect]{no-lipics-v2022}

\pdfoutput=1
\usepackage{enumitem}


\def\sd#1{{\color{red!50!black}{\bf Simon says:} #1}}
\newcommand{\ijac}{\todo[green,inline]}

\makeatletter

\newcommand{\indmatch}{\textsc{IndMatch}}
\newcommand{\subsprob}{\textsc{Sub}}
\newcommand{\indsubsprob}{\textsc{IndSub}}
\newcommand{\homsprob}{\textsc{Hom}}
\newcommand{\homs}[2]{\mathsf{Hom}(#1 \to #2)}
\newcommand{\embs}[2]{\mathsf{Emb}(#1 \to #2)}
\newcommand{\auts}{\mathsf{Aut}}
\newcommand{\colHoms}{\mathsf{colHom}}
\newcommand{\surhoms}[2]{\mathsf{SurHom}(#1 \to #2)}
\newcommand{\lovasz}{Lov{\'{a}}sz}


\newcommand{\matchings}[2]{M(#1, #2)}
\newcommand{\subs}[2]{\mathsf{Sub}(#1 \to #2)}
\newcommand{\indsubs}[2]{\mathsf{IndSub}(#1 \to #2)}
\newcommand{\indset}{\textsc{IS}}
\newcommand{\calG}{\mathcal{G}}
\newcommand{\isk}{\mathsf{IS}_k}
\newcommand{\tw}{\mathsf{tw}}
\newcommand{\id}{\mathsf{id}}
\newcommand{\colisk}{\mathsf{ColIS}_k}
\newcommand{\W}{\mathrm{W}}
\newcommand{\E}{\mathbf{E}}
\newcommand{\N}{\mathbb{N}}
\newcommand{\Q}{\mathbb{Q}}
\newcommand{\No}{\N_0}
\newcommand{\R}{\mathbb{R}}
\newcommand{\scH}{\mathcal{H}}
\newcommand{\scU}{\mathcal{U}}
\newcommand{\scG}{\mathcal{G}}
\newcommand{\scF}{\mathcal{F}}
\newcommand{\scI}{\mathcal{I}}
\newcommand{\homsupp}{\chi}
\newcommand{\ppart}{\ensuremath{\mathsf{Part}}}

\newcommand{\ccStyle}[1]{\mathrm{#1}}
\newcommand{\ccP}{\ccStyle{P}}
\newcommand{\ccSharpP}{\#\ccStyle{P}}
\newcommand{\ccFPT}{\ccStyle{FPT}}
\newcommand{\ccW}[1]{\ccStyle{W}\text{[#1]}}
\newcommand{\ccSharpW}[1]{\#\ccStyle{W}\text{[#1]}}

\newcommand{\zsig}[2]{\ensuremath{#1\lvert_{0\mapsto#2}}}
\newcommand{\parthoms}[2]{\mathsf{Hom}_{\phi}(#1 \to #2)}

\newcommand{\parthomsX}[3]{\mathsf{Hom}^{\phi}_{#3}(#1 \to #2)}

\newcommand{\necphoms}[3]{\ensuremath{\mathsf{Hom}_{\mathsf{cp}}}(#1 \to_{#2} #3)}
\newcommand{\coledgesubs}[2]{\mbox{\ensuremath{\mathsf{ColEdgeSub}(#1 \to #2)}}}
\newcommand{\edgesubs}[2]{\mbox{\ensuremath{\mathsf{EdgeSub}(#1 \to #2)}}}
\newcommand{\edgesubsprob}{\text{\sc{EdgeSub}}}
\newcommand{\cpindsubsprob}{\text{\sc{cp-IndSub}}}
\newcommand{\coledgesubsprob}{\text{\sc{ColEdgeSub}}}
\newcommand{\clique}{\textsc{Clique}}
\newcommand{\is}{\textsc{IndSet}}
\newcommand{\colis}{\textsc{ColIndSet}}
\newcommand{\match}{\textsc{Match}}
\newcommand{\colmatch}{\textsc{ColMatch}}
\newcommand{\cphomsprob}{\text{\sc{cp-Hom}}}
\newcommand{\colhomsprob}{\text{\sc{col-Hom}}}
\newcommand{\listhomsprob}
{\text{\sc{list-Hom}}}
\newcommand{\bind}{\beta_{\mathsf{ind}}}
\newcommand{\m}{{\mathrm{m}}}
\newcommand{\mind}{\m_{\mathsf{ind}}}
\newcommand{\fptred}{\leq^{\mathsf{FPT}}_{\mathsf{T}}}
\newcommand{\fptinterred}{\equiv^{\mathsf{FPT}}_{\mathsf{T}}}
\newcommand{\fptlinred}{\leq^{\mathsf{FPT-lin}}_{\mathsf{T}}}
\newcommand{\fptinterlinred}{\equiv^{\mathsf{FPT-lin}}_{\mathsf{T}}}
\newcommand{\all}{{\mathcal{U}}}
\def\fracture#1#2{\ensuremath{#1\raisebox{.2ex}{\rotatebox[origin=c]{-15}{$\sharp$}}#2}}



\newcommand{\free}[1]{{#1}\text{-free}}
\newcommand{\holantprob}{\text{\sc{p-Holant}}}
\newcommand{\holantprobstar}{\text{\sc{p-Holant}}^{\mathsf{Hcol}}}

\newcommand{\ind}{\texttt{ind}}
\newcommand{\homscp}{\mathsf{Hom}_{\mathsf{cp}}}
\newcommand{\embscp}{\mathsf{Emb}_{\mathsf{cp}}}
\newcommand{\colembs}{\mathsf{colEmb}}
\newcommand{\colsubs}{\mathsf{colSub}}
\newcommand{\cphoms}{\mathsf{cp}\text{-}\mathsf{Hom}}
\newcommand{\torus}{\ensuremath{\pmb{\circledcirc}}}
\newcommand{\colms}{\mathsf{ColMatch}}
\newcommand{\holant}{\mathsf{Holant}}
\newcommand{\colholant}{\mathsf{Holant}_\mathsf{col}}
\def\fracture#1#2{\ensuremath{#1\raisebox{.2ex}{\rotatebox[origin=c]{-15}{$\sharp$}}#2}}
\newcommand{\colholantprob}{\textsc{p-ColHolant}}
\newcommand{\Hcolholantprob}{\textsc{p-HomColHolant}}

\usepackage{tikz, tikz-cd}
\usetikzlibrary{matrix}
\usetikzlibrary{shapes}
\usetikzlibrary{arrows,decorations.markings, tikzmark}
\usetikzlibrary{decorations.pathreplacing,calligraphy,backgrounds}
\usetikzlibrary{arrows.meta,arrows}
\usetikzlibrary{conference}
\usetikzlibrary{cd}
\usepackage{todonotes}
\newcommand{\agob}[1]{\todo[color=blue!40!red!50]{\footnotesize AG: #1}}

\newcommand{\panos}[1]{\todo[inline,color=green!30!]{\footnotesize Panos: #1}}

\newcommand{\Panos}[1]{\todo[color=green!30!]{\footnotesize Panos: #1}}

\newcommand{\redc}[2][red,fill=red]{\tikz[baseline=-0.5ex]\draw[#1,radius=#2] (0,0) circle ;}
\newcommand{\greenc}[2][green!80!blue,fill=green!80!blue]{\tikz[baseline=-0.5ex]\draw[#1,radius=#2] (0,0) circle ;}
\newcommand{\bluec}[2][blue,fill=blue]{\tikz[baseline=-0.5ex]\draw[#1,radius=#2] (0,0) circle ;}
\newcommand{\yellowc}[2][yellow!50!orange,fill=yellow!50!orange]{\tikz[baseline=-0.5ex]\draw[#1,radius=#2] (0,0) circle ;}
\newcommand{\blackc}[2][black,fill=black]{\tikz[baseline=-0.5ex]\draw[#1,radius=#2] (0,0) circle ;}
\newcommand{\brownc}[2][cyan,fill=cyan]{\tikz[baseline=-0.5ex]\draw[#1,radius=#2] (0,0) circle ;}

\newcommand*{\boldone}{\text{\usefont{U}{bbold}{m}{n}1}}


\newcommand{\he}{\mathsf{he}}  

\newcommand{\old}[1]{\textcolor{blue!70!black}{#1}}
\newcommand{\new}[1]{\textcolor{red!70!black}{#1}}

\newcommand{\crnt}[1]{\vec{\textup{#1}}}

\newcommand{\rk}{r}
\newcommand{\ar}{\mathsf{ar}}
\newcommand{\uncolholant}{\mathsf{UnColHolant}}
\newcommand{\uncolholantprob}{\textsc{p-UnColHolant}}


 \nolinenumbers

\title{Parameterised Holant Problems}



\author{Panagiotis Aivasiliotis}{Hasso Plattner Institute, University of Potsdam}{panos.aivasiliotis@hpi.de}{}{}

\author{Andreas Göbel}{Hasso Plattner Institute, University of Potsdam}{andreas.goebel@hpi.de}{}{}

\author{Marc Roth}{School of Electronic Engineering and Computer Science, Queen Mary University of London}{m.roth@qmul.ac.uk}{}{}

\author{Johannes Schmitt}{Department of Mathematics, ETH Zürich}{johannes.schmitt@math.ethz.ch}{}{}
\authorrunning{P. Aivasiliotis, A. Göbel, M. Roth, J. Schmitt}

\keywords{holant problems, counting problems, parameterised algorithms, fine-grained complexity theory, homomorphisms}

\acknowledgements{The authors would like to thank Miriam Backens for their insightful feedback on this work. Andreas Göbel was funded by the project PAGES (project No. 467516565) of the German Research Foundation (DFG).
Johannes Schmitt was supported by the Swiss National Science Foundation (project No. 219369) and SwissMAP.}

\begin{document}
\maketitle

\begin{abstract}
    We investigate the complexity of parameterised holant problems $\textsc{p-Holant}(\mathcal{S})$ for families of symmetric signatures~$\mathcal{S}$. The parameterised holant framework has been introduced by Curticapean in 2015 as a counter-part to the classical and well-established theory of holographic reductions and algorithms, and it constitutes an extensive family of coloured and weighted counting constraint satisfaction problems on graph-like structures, encoding as special cases various well-studied counting problems in parameterised and fine-grained complexity theory such as counting edge-colourful $k$-matchings, graph-factors, Eulerian orientations or, more generally, subgraphs with weighted degree constraints. We establish an exhaustive complexity trichotomy along the set of signatures $\mathcal{S}$: Depending on the signatures, $\textsc{p-Holant}(\mathcal{S})$ is either
    \begin{itemize}
        \item[(1)] solvable in FPT-near-linear time, i.e., in time $f(k)\cdot \tilde{\mathcal{O}}(|x|)$, or
        \item[(2)] solvable in ``FPT-matrix-multiplication time'', i.e., in time $f(k)\cdot {\mathcal{O}}(n^{\omega})$, where $n$ is the number of vertices of the underlying graph, but not solvable in FPT-near-linear time, unless the Triangle Conjecture fails, or
        \item[(3)] $\#\W[1]$-complete and no significant improvement over the naive brute force algorithm is possible unless the Exponential Time Hypothesis fails. 
    \end{itemize}
    This classification reveals a significant and surprising gap in the complexity landscape of parameterised Holants: Not only is every instance either fixed-parameter tractable or $\#\W[1]$-complete, but additionally, every FPT instance is solvable in time (at most) $f(k)\cdot {\mathcal{O}}(n^{\omega})$. 
    We show that there are infinitely many instances of each of the types; for example, all constant signatures yield holant problems of type (1), and the problem of counting edge-colourful $k$-matchings modulo $p$ is of type ($p$) for $p\in\{2,3\}$. 

    Finally, we also establish a complete classification for a natural uncoloured version of parameterised holant problem $\textsc{p-UnColHolant}(\mathcal{S})$, which encodes as special cases the non-coloured analogues of the aforementioned examples. We show that the complexity of $\textsc{p-UnColHolant}(\mathcal{S})$ is different: Depending on $\mathcal{S}$ all instances are either solvable in FPT-near-linear time, or $\#\W[1]$-complete, that is, there are no instances of type (2). 
\end{abstract}



\newpage

\section{Introduction}
Inspired by Valiant's work on holographic algorithms~\cite{Valiant08}, the so-called \emph{holant framework}, first introduced in the conference version~\cite{CaiLX09} of~\cite{CaiLX14}, constitutes one of the most powerful and ubiquitous tools for the analysis of computational counting problems. Holants, defined momentarily, strictly generalise counting constraint satisfaction problems (``$\#\textsc{CSP}$s'') and are able to model various (in)famous counting problems such as counting perfect matchings, graph factors, Eulerian orientations, and proper edge-colourings~\cite{CaiG21} (see also~\cite{CaiL11}). Moreover, the holant framework has been used for the analysis of the complexity of computing partition functions from statistical physics (see e.g.\ \cite{CaiFX18,CaiF23}), and it has shown to allow for the application of tools from quantum information theory, particular of entanglement, to the analysis of counting problems~\cite{Backens18}.

Formally, an instance of a holant problem\footnote{We present here the case of signatures with Boolean domain, but we point out that more general versions have been studied (see e.g.\ \cite{CaiG21}).} is a pair of a graph~$G$ and an assignment from vertices~$v$ of~$G$ to \emph{signatures} $s_v$, where each $s_v$ is a function with values in algebraic complex numbers and with domain $\{0,1\}^{\mathsf{deg}(v)}$. The value of the holant on input $(G,\{s_v\}_{v\in V(G)})$ is then defined as
\begin{equation}\label{eq:intro_classical_holant}
    \sum_{\alpha: E(G) \to \{0,1\}}~ \prod_{v\in V(G)} s_v(\alpha|_{E(v)}) \,,
\end{equation}
where $\alpha|_{E(v)}$ is the restriction of $\alpha$ on the edges incident to $v$. For example, let $G$ be a $d$-regular graph, and set
for each $v\in V(G)$ of degree $d$ the signature $s_v$ as the $d$-ary function $\mathsf{hw}^d_{=1}$ that outputs~$1$ if precisely one edge incident to $v$ is mapped by~$\alpha$ to~$1$, and~$0$ otherwise. Then~\eqref{eq:intro_classical_holant} is equal to the number of perfect matchings of $G$, that is, for the signature $\mathsf{hw}^d_{=1}$, the holant problem is equivalent to counting perfect matchings in $d$-regular graphs.

Since their inception in 2009, the holant framework has seen immense success in the quest of charting the complexity landscape of computational counting problems~\cite{CaiL11,CaiLX14,CaiGW16,LinW17,Backens18}. In a majority of the previous works, the central question was to determine the complexity of evaluating the Holant~\eqref{eq:intro_classical_holant} depending on the allowed signatures. For example, if each $s_v$ is the signature of having an even number of $1$s, then \eqref{eq:intro_classical_holant} can be computed by counting the number of solutions to a system of linear equations over $\mathbb{Z}/2\mathbb{Z}$, which can be done in polynomial time; this approach generalises to families of \emph{affine signatures}~\cite{ParityHolant13}. On the other hand, if we allow the signatures $\mathsf{hw}^i_{=1}$ for $i\in \mathbb{N}$, then the holant problem becomes $\#\mathrm{P}$-hard\footnote{$\#\mathrm{P}$ is the class of all (counting) problems polynomial-time reducible to $\#\textsc{SAT}$, the problem of counting satisfying assignments of a Boolean formula. By a result of Toda~\cite{Toda91} $\#\mathrm{P}$-hard problems are at least as hard as all problems in the polynomial-time hierarchy $\mathrm{PH}$.} since it is at least as hard as the $\#\mathrm{P}$-complete problem of counting perfect matchings~\cite{Valiant79,Valiant79b}.

It has turned out that, in numerous settings, the complexity of evaluating holants is either solvable in polynomial time, or $\#\mathrm{P}$-hard, and the dichotomy criterion only depends on the set of allowed signatures, that is, there are no instances of intermediate complexity.\footnote{In contrast, by (the counting version of) a result of Ladner~\cite{Ladner75}, assuming that $\mathrm{FP}\neq \#\mathrm{P}$, there are counting problems in $\#\mathrm{P}$ that are neither solvable in polynomial time, nor $\#\mathrm{P}$-hard.} Among others, key results include, but are not limited to, the complete complexity dichotomy for Boolean symmetric holants~\cite{CaiGW16}, for non-negative real-valued holants~\cite{LinW17}, and for holants with constant unary signatures~\cite{Backens18}.

\subsection{Parameterised Complexity Meets Holant Problems}
Introduced by Downey and Fellows in the early 90s~\cite{DowneyF92}, the field of parameterised complexity theory, also called multivariate algorithmics, investigates the complexity of computational problems not only along the size $|x|$ of an input $x$, but also along a parameter $k=\kappa(x)$ taking into account one or more (structural) properties of $x$. The notion of efficient algorithms is then relaxed from polynomial-time algorithms to fixed-parameter tractable (FPT) algorithms, which are required to run in time $f(k)\cdot |x|^{O(1)}$, where $f$ can be any computable function. The notion of fixed-parameter tractability formalises the existence of efficient algorithms if the parameter of the problem input is significantly smaller than the input size. For example, in the database query evaluation problem, the input is a pair $x=(\varphi,D)$ of a query $\varphi$ and a database $D$. Choosing the size of the query as a parameter, i.e., setting $k=|\varphi|$, an FPT algorithm for this problem can then be thought as an efficient algorithm for instances in which the size of the query is significantly smaller than the size of the database, which is true for realistic instances. 

Independently introduced by McCartin~\cite{McCartin06} and by Flum and Grohe~\cite{FlumG04}, the field of parameterised \emph{counting} complexity theory aims to apply the tools and methods from parameterised algorithmics to computational counting problems. In the context of counting problems, the notion of tractability is naturally
still given by FPT algorithms, and the notion of intractability is given by $\#\W[1]$-hardness. In a nutshell, the class $\#\W[1]$ can be thought as a parameterised equivalent of $\#\mathrm{P}$ and its canonical complete problem is the problem of, given a positive integer $k$ and a graph $G$, counting the number of $k$-cliques in $G$, parameterised by $k$. Under standard assumptions such as the Exponential Time Hypothesis, $\#\W[1]$-hard problems are not fixed-parameter tractable (see e.g.\ \cite{Chenetal05,Chenetal06,CyganFKLMPPS15}).

In addition to early key results such as the classification of the parameterised homomorphism counting problem by Dalmau and Jonson~\cite{DalmauJ04}, and the resolution of the complexity of the problem of counting $k$-matchings by Curticapean~\cite{Curticapean13}, the field of parameterised counting recently witnessed significant advancements with the introduction of the framework of graph motif parameters~\cite{CurticapeanDM17} which has subsequently been used to fully resolve the parameterised and fine-grained complexity of numerous network pattern counting problems~\cite{Roth17,DellRW19,RothSW20,BressanR21,BeraGLSS22,PenaS22,BLR2023stoc,GishbolinerLSY23,DoringMW24,Curticapean24,CurticapeanN24,DoringMW25}.

\paragraph*{Parameterised Holant Problems}
In the present work we investigate holant problems under the lens of parameterised (counting) complexity theory. In fact, parameterised holant problems have already been introduced and used almost a decade ago~\cite{Curticapean15}, but no attempt has been made to establish exhaustive classification results comparable to the classical holant dichotomies. 

Let us now introduce the parameterised holant framework following the approach of~\cite{Curticapean15}. To this end, let $\mathcal{S}$ be a finite set of symmetric signatures --- think for now of a signature in $\mathcal{S}$ as a function from $\{0,1\}^\ast$ to algebraic complex numbers; we will see later that we have to be very careful about which functions we can allow as signatures. A $k$-\emph{edge-coloured signature grid} over $\mathcal{S}$ then consists of a graph $G$, a (not necessarily proper) edge-colouring $\xi:E(G)\to [k]$, and an assignment from vertices $v\in V(G)$ to signatures in $\mathcal{S}$. We write $s_v\in \mathcal{S}$ for the signature assigned to vertex $v$. We say that an assignment $\alpha:E(G) \to \{0,1\}$ is \emph{edge-colourful} if $\alpha$ maps exactly $k$ edges to $1$, and each edge colour (w.r.t.\ $\xi$) is hit exactly once. The holant value of the $k$-edge-colourful signature grid $\Omega=(G,\xi,\{s_v\}_{v\in V(G)})$ is then defined as
\[ \holant(\Omega)= \sum_{\substack{\alpha: E(G)\to \{0,1\}\\\text{edge-colourful}}} ~\prod_{v\in V(G)} s_v(\alpha|_{E(v)}) \,.\]

For a set of signatures $\mathcal{S}$, the problem $\holantprob(\mathcal{S})$ gets as input a $k$-edge-coloured signature grid $\Omega$ over $\mathcal{S}$ and outputs $\holant(\Omega)$. The problem parameter is $k$. Similarly as in the case of classical Holants, our goal is to identify precisely the signature sets $\mathcal{S}$ for which $\holantprob(\mathcal{S})$ becomes tractable. However, we ask for fixed-parameter tractability, rather than for polynomial-time tractability, that is, our goal is the construction of algorithms running in time $f(k)\cdot |\Omega|^{O(1)}$. Analogously, we define the (arguably more natural)\footnote{While the uncoloured parameterised Holant problem is most likely more natural to readers outside of the field of parameterised counting, we decided to follow the notation of Curticapean~\cite{Curticapean15} and denote the coloured Holant problem by $\holantprob(\mathcal{S})$. To avoid any confusion we thus denote the uncoloured Holant problem by $\text{\sc{p-UnColHolant}}(\mathcal{S})$.} \emph{uncoloured} version $\text{\sc{p-UnColHolant}}(\mathcal{S})$, in which the signature grid does not come with a $k$-edge-colouring, and we instead sum over all $\alpha:E(G)\to\{0,1\}$ that map exactly $k$ edges to $1$. We will see in the applications of our main results that both $\holantprob(\mathcal{S})$ and $\text{\sc{p-UnColHolant}}(\mathcal{S})$ encode various well-studied parameterised counting problems, such as counting (coloured or uncoloured) $k$-matchings, counting (coloured or uncoloured) $k$-factors, and, to some extent, counting the number of weight-$k$ solutions of systems of linear equations.

As indicated previously, before stating our main results, we have to discuss some subtleties about the definition of signatures. In classical holant theory, each signature has a fixed arity $d$ and only allows for inputs in $\{0,1\}^d$. This implies that only vertices of degree $d$ can be equipped with such a signature. As a consequence, if we would also enforce a fixed arity for each signature, then for any \emph{finite} set of signatures $\mathcal{S}$, the only valid inputs to $\holantprob(\mathcal{S})$ and $\text{\sc{p-UnColHolant}}(\mathcal{S})$ would be signature grids, the underlying graphs of which have maximum degree $d$, where $d$ equals the maximum arity of the signatures in $\mathcal{S}$. Using standard tools from parameterised algorithmics, such as the bounded search-tree paradigm, the problems $\holantprob(\mathcal{S})$ and $\text{\sc{p-UnColHolant}}(\mathcal{S})$ would then always be fixed-parameter tractable. At the same time, modelling key parameterised counting problems such as counting $k$-matchings would require an infinite set of signatures. Therefore, we leverage the setting by allowing signatures to be functions with domain $\{0,1\}^\ast$ --- for example, setting $\mathsf{hw}_{\leq 1}$ to be the signature that maps an input tuple to $1$ if and only if at most one of its elements is $1$, the problems $\holantprob(\{\mathsf{hw}_{\leq 1}\})$ and $\text{\sc{p-UnColHolant}}(\{\mathsf{hw}_{\leq 1}\})$ become the problems of counting edge-colourful and uncoloured $k$-matchings, respectively. Moreover, in this work, we will restrict ourselves to \emph{symmetric} signatures, that is, signatures $s$ satisfying $s(x)=s(y)$ whenever $x$ can be obtained from $y$ by permuting its entries. Since the domain of signatures is $\{0,1\}^\ast$, the value of $s(x)$ only depends on the Hamming weight, i.e., the number of $1$s, in $x$. For notational convenience, we thus define signatures as functions from $\mathbb{N}$ to algebraic complex numbers.

\begin{definition}[Signatures]\label{def:signatures_intro}
    A \emph{signature} is a computable function $s$ from non-negative integers to algebraic complex numbers. Moreover, we require $s(0)\neq 0$.
\end{definition}
We note that the requirement $s(0)\neq 0$ makes sure that a vertex $v$ equipped with $s$ is allowed to not ``participate'' in the $k$-edge-subset chosen by $\alpha$; 
if more than $2k$ vertices were equipped with signatures that allow for $s(0)=0$, then the holant value would be trivially $0$.
As the core of our technical work applies to signatures that fulfil the $s(0)\neq0$ requirement, we focus our discussion on such signatures.
However, for completeness, we show in Section~\ref{sec:sig0} how all our results can be extended to the case in which signatures $s$ with $s(0)=0$ are allowed.

Now, with this definition of signatures, we can simplify the notation in the definition of parameterised holant problems as follows; for the remainder of the paper, we will use the subsequent notation:

\begin{definition}[The Parameterised Holant Problem]\label{def:intro_param_holant}
    Let $\mathcal{S}$ be a finite set of signatures. The problem $\holantprob(\mathcal{S})$ gets as input a
    $k$-edge-coloured signature grid $\Omega=(G,\xi,\{s_v\}_{v\in V(G)})$ with $s_v\in \mathcal{S}$ for all $v\in V(G)$. The output is
    \[ \holant(\Omega):= \sum_{\substack{A \subseteq E(G)\\|A|=k,~\xi(A)=[k]}}~\prod_{v\in V(G)} s_v(|A \cap E(v)|)\,, \]
    where $E(v)$ denotes the set of edges incident to $v$. The problem parameter is $k$.
\end{definition}
\begin{definition}[The Parameterised Uncoloured Holant Problem]\label{def:intro_param_uncol_holant}
    Let $\mathcal{S}$ be a finite set of signatures. The problem $\text{\sc{p-UnColHolant}}(\mathcal{S})$ gets as input a positive integer $k$, and a signature grid $\Omega=(G,\{s_v\}_{v\in V(G)})$ with $s_v\in \mathcal{S}$ for all $v\in V(G)$. The output is
    \[ \holant(\Omega,k):= \sum_{\substack{A \subseteq E(G)\\|A|=k}}~\prod_{v\in V(G)} s_v(|A \cap E(v)|)\,. \]
    The problem parameter is $k$.
\end{definition}

\subsection{Our Contributions}
We provide a complexity ``trichotomy'' for $\holantprob(\mathcal{S})$, and a complexity dichotomy for $\text{\sc{p-UnColHolant}}(\mathcal{S})$, both along the permitted signatures $\mathcal{S}$. 

For the statement of our results, we first introduce signature fingerprints and types of signature sets.

\begin{definition}[Signature Fingerprints]\label{def:fingerprint_intro}
Let $d$ be a positive integer and let $s$ be a signature. The \emph{fingerprint} of $d$ and $s$ is defined as
    \[\chi(d,s) := \sum_{\sigma} (-1)^{|\sigma|-1} (|\sigma|-1)! \cdot \prod_{B \in \sigma} \frac{s(|B|)}{s(0)} \,,\]
    where the sum is over all partitions of $[d]$.
\end{definition}

We point out that $\chi(d,s)$ is equal to a weighted sum over all evaluations of the M\"obius function of the lattice of partitions of the set $[d]$; the details necessary for this work are provided in Section~\ref{sec:parts_quots_mobius} and we refer the reader to e.g.\ \cite[Chapter 3]{Stanley11} for further reading.

\begin{definition}[Types of Signature Sets]\label{def:sigtype_intro}
    Let $\mathcal{S}$ be a finite set of signatures. We say that $\mathcal{S}$ is of type
    \begin{itemize}
        \item[(I)] $\mathbb{T}[\mathsf{Lin}]$ if $\chi(d,s)=0$ for all $s\in \mathcal{S}, d\geq 2$, 
        \item[(II)] $\mathbb{T}[\omega]$ if $\chi(d,s)=0$ for all $s\in \mathcal{S}, d\geq 3$, but there exists $s\in \mathcal{S}$ with $\chi(2,s)\neq 0$, and
        \item[(III)] $\mathbb{T}[\infty]$ otherwise, i.e., there exists $s\in \mathcal{S}$ and $d\geq 3$ such that $\chi(d,s)\neq 0$.
    \end{itemize}
\end{definition}

We point out that there are infinitely many signature sets of each type --- we provide an easy construction in the appendix. For now, let us discuss some natural examples which also help us gain some initial understanding of the properties of the signature fingerprints.
\begin{itemize}
    \item[(I)] Unsurprisingly, any signature set containing only a constant function $s(x)=c\neq 0$ for all $x\in \mathbb{N}$ makes the problem easy.  In that case, we have, for each $d$, that
    $\chi(d,s) := \sum_{\sigma} (-1)^{|\sigma|-1} (|\sigma|-1)!$, since $s(|B|)/s(0)$ will always be $1$. However, as indicated previously, this alternating sum is equal to $\sum_{\rho}\mu(\rho,\top)$, where $\mu$ is the M\"obius function of the partition lattice, and $\top$ is the coarsest partition of the $d$-element set. This sum is well-known to be $0$ for every $d\geq 2$ (see e.g.\ \cite[Section~3.7]{Stanley11}), implying that $\mathcal{S}=\{s_c\}$ is indeed of type $\mathbb{T}[\mathsf{Lin}]$. Analogously, any finite signature set $\mathcal{S}$ containing only constant functions is of type $\mathbb{T}[\mathsf{Lin}]$ as well.

    As a slightly more interesting example, a similar argument applies to the function $s(x):= 2^{ax+b}$ for any pair of rational numbers $a$ and $b$, since $s(|B|)/s(0)= 2^{ax}$, implying that $\prod_{B\in \sigma} s(|B|)/s(0) = 2^{a\sum_{B\in \sigma}|B|} = 2^{ad}$, and thus $\chi(d,s) := 2^{ad} \cdot \sum_{\sigma} (-1)^{|\sigma|-1} (|\sigma|-1)! =0$, for $d \geq 2$.
    \item[(II)] For type $\mathbb{T}[\mathsf{\omega}]$, we consider the evaluation of $\holantprob(\mathcal{S})$ modulo $2$ (in Section~\ref{sec:modular} we show how to lift our results on the edge-coloured holant problem to modular counting). Then we set $\mathcal{S}=\{\mathsf{hw}_{\leq 1}\}$, and we recall that $\mathsf{hw}_{\leq 1}(x)$ evaluates to $1$ if $x\in \{0,1\}$ and to $0$ otherwise. Then the holant problem is precisely the problem of counting edge-colourful $k$-matchings modulo $2$. Now note that we have $(|\sigma|-1)!=0$ modulo $2$ whenever $|\sigma| \geq 3$, and, for $s=\mathsf{hw}_{\leq 1}$, $s(|B|)=0$ whenever $|B|\geq 2$. Therefore, for any $d\geq 3$, we have that $\chi(d,s)=0$ modulo $2$.

    However, note that $\chi(2,s)=1$ modulo $2$: The set $[2]$ only has two partitions $\bot_2=\{\{1\},\{2\}\}$ and $\top_2=\{\{1,2\}\}$, and observe that $\top_2$ contains a block $B$ of size $2$, hence $s(|B|)$ and the contribution of $\top_2$ to $\chi(2,s)$ vanishes. Therefore 
    \[ \chi(2,s) = (-1)^{|\bot_2|-1}(|\bot_2|-1)! \prod_{B\in \bot_2} \frac{s(|B|)}{s(0)} = 1 \mod 2 \,. \]
    Thus $\mathcal{S}=\{\mathsf{hw}_{\leq 1}\}$ is indeed of type $\mathbb{T}[\mathsf{\omega}]$ when computation is done modulo $2$.
    \item[(III)] A natural example for type $\mathbb{T}[\mathsf{\infty}]$ is the signature $\mathsf{even}(x)$ which maps $x$ to $1$ if $x$ is even, and to $0$ otherwise. Let us fix $d=4$ and set $s=\mathsf{even}$. Observe that \[\prod_{B\in \sigma}\frac{s(|B|)}{s(0)} = \begin{cases}
        1 & \forall B\in \sigma: |B|=0 \mod 2\\
        0 & \text{otherwise}
    \end{cases}\]
    There are precisely four partitions of $[4]$ that only contain even sized blocks: $\{\{1,2\},\{3,4\}\}$, $\{\{1,3\},\{2,4\}\}$, $\{\{1,4\},\{2,3\}\}$, and $\{\{1,2,3,4\}\}$. The former three each contribute $(-1)^{2-1} (2-1)! = -1$ to $\chi(4,s)$, and the latter contributes $(-1)^{1-1}(1-1)!=1$ to $\chi(4,s)$. Thus $\chi(4,s)=-3+1=-2\neq 0$ and $\mathcal{S}=\{\mathsf{even}\}$ is, as promised, of type $\mathbb{T}[\mathsf{\infty}]$. 
\end{itemize}

In some cases, e.g.\ for signatures with co-domain $\{0,1\}$, it will be possible to simplify the definition of the types. Moreover, we will show that $\mathbb{T}[\mathsf{Lin}]$ can always be simplified if $s(0)=1$ for each signature (and we will later see that we can always make this assumption without loss of generality).
\begin{lemma}\label{lem:simplify_type_1_intro}
Let $\mathcal{S}$ be a finite set of signatures such that $s(0)=1$ for all $s\in \mathcal{S}$. Then
$\mathcal{S}$ is of type $\mathbb{T}[\mathsf{Lin}]$ if and only if, for each $s\in \mathcal{S}$ and $n\in \mathbb{N}$, we have $s(n)=s(1)^n$.\qed
\end{lemma}

We are now able to state our main results. Some of the lower bounds rely on two well-known hardness assumptions from fine-grained complexity theory: The Exponential Time Hypothesis (ETH), and the Triangle Conjecture. We introduce both in Section~\ref{sec:prelims_fgct}; in a nutshell, ETH asserts that $3$-$\textsc{SAT}$ cannot be solved in sub-exponential time in the number of variables, and the Triangle conjecture asserts that there is no linear-time algorithm for finding a triangle in a graph.  
\begin{mtheorem}[Complexity Trichotomy for $\textsc{p-Holant}(\mathcal{S})$]\label{main_thm}
    Let $\mathcal{S}$ be a finite set of signatures.
    \begin{itemize}
        \item[(I)] If $\mathcal{S}$ is of type $\mathbb{T}[\mathsf{Lin}]$, then $\textsc{p-Holant}(\mathcal{S})$ can be solved in FPT-near-linear time, that is, there is a computable function $f$ such that $\textsc{p-Holant}(\mathcal{S})$ can be solved in time $f(k)\cdot \tilde{\mathcal{O}}(|V(\Omega)|+|E(\Omega)|)$.
        \item[(II)] If $\mathcal{S}$ is of type $\mathbb{T}[\omega]$, then $\textsc{p-Holant}(\mathcal{S})$ can be solved in FPT-matrix-multiplication time, that is, there is a computable function $f$ such that $\textsc{p-Holant}(\mathcal{S})$ can be solved in time $f(k)\cdot \mathcal{O}(|V(\Omega)|^{\omega})$. Moreover, $\textsc{p-Holant}(\mathcal{S})$ cannot be solved in time $f(k)\cdot \tilde{\mathcal{O}}(|V(\Omega)|+|E(\Omega)|)$ for any function $f$, unless the Triangle Conjecture fails.
        \item[(III)] Otherwise, that is, if $\mathcal{S}$ is of type $\mathbb{T}[\infty]$, $\textsc{p-Holant}(\mathcal{S})$ is $\#\W[1]$-complete. Moreover, $\textsc{p-Holant}(\mathcal{S})$ cannot be solved in time $f(k)\cdot |V(\Omega)|^{o(k/\log k)}$ for any function $f$, unless the Exponential Time Hypothesis fails.  \qed
    \end{itemize}
\end{mtheorem}
At the time of writing this paper the matrix multiplication exponent $\omega$ is known to be bounded by $2\leq \omega\leq 2.371552$~\cite{WilliamsXXZ24}.
Note that the lower bound in (III) matches, up to a factor of $1/\log k$ in the exponent, the running time of the brute force algorithm --- which runs in time $|\Omega|^{\mathcal{O}(k)}$ ---- making this bound (almost) tight. Moreover, the factor $1/\log k$ is not an artifact of our proofs, but a consequence of the notoriously open problem of whether ``you can beat treewidth''~\cite{Marx10}.

We emphasise that our complexity trichotomy above is \emph{much} stronger than an FPT vs $\#\W[1]$ classification result: Under ETH and the Triangle Conjecture, the best possible exponent of $|\Omega|$ in the running time is either $1$, or between $1$ and $\omega$, or lower bounded by $o(k/\log k)$. In particular, there are no FPT instances requiring an exponent larger than $\omega$.



Perhaps surprisingly, we discover that the complexity changes for the uncoloured holant problem; we will see in Section~\ref{sec:uncoloured} that $\text{\sc{p-UnColHolant}}(\mathcal{S})$ appears to be \emph{strictly} harder than $\textsc{p-Holant}(\mathcal{S})$.

\begin{mtheorem}[Complexity Dichotomy for $\text{\sc{p-UnColHolant}}(\mathcal{S})$]\label{thm:main_uncol}
    Let $\mathcal{S}$ be a finite set of signatures. 
    \begin{itemize}
        \item[(I)] If $\mathcal{S}$ is of type $\mathbb{T}[\mathsf{Lin}]$, then $\text{\sc{p-UnColHolant}}(\mathcal{S})$ can be solved in FPT-near-linear time.
        \item[(II)] Otherwise $\text{\sc{p-UnColHolant}}(\mathcal{S})$ is $\#\W[1]$-complete. If, additionally, $\mathcal{S}$ is of type $\mathbb{T}[\infty]$, then $\text{\sc{p-UnColHolant}}(\mathcal{S})$ cannot be solved in time $f(k)\cdot |V(\Omega)|^{o(k/\log k)}$, unless ETH fails. \qed
    \end{itemize}
\end{mtheorem}


\paragraph*{Explicit Tractability Criteria}
For both of our main classifications, the tractability criteria appear to be implicit and hard to verify in the sense that it might be non-trivial to decide for a concrete set of signatures whether it yields a tractable instance of a parameterised Holant problem.

We address this issue by using Lemma~\ref{lem:simplify_type_1_intro}, and we obtain an explicit classification for signatures satisfying $s(0)=1$; intuitively, in those instances the vertices not incident to any of the chosen edges do not contribute to the holant value. We provide below the corresponding dichotomy for the uncoloured Holant problem; in combination with the assumption $s(0)=1$ this variant accounts for the arguably most natural instantiation of parameterised Holant problems, and in this case, the tractability criterion is directly and easily verifyable.
\begin{mtheorem}
    Let $\mathcal{S}$ be a finite set of signatures such that $s(0)=1$ for all $s\in \mathcal{S}$. 
    \item[(I)] If  $s(n)=s(1)^n$ for all $s\in \mathcal{S}$ and $n\geq 1$, then $\text{\sc{p-UnColHolant}}(\mathcal{S})$ can be solved in FPT-near-linear time.
        \item[(II)] Otherwise $\text{\sc{p-UnColHolant}}(\mathcal{S})$ is $\#\W[1]$-complete. \qed
\end{mtheorem}





\subsubsection{Applications of our Main Results}

We now discuss new complexity classifications of specific problems that transpire from our main theorems. The complexities of these problems where previously known only for special cases.

\paragraph*{Counting Factors of size $k$}
Given a graph $G$ and a function $f:V(G)\to \mathcal{P}(\mathbb{N})$, an $f$-\emph{factor} of $G$ is a subset of edges $A$ such that $|E(v)\cap A| \in f(v)$ for all $v\in V(G)$. Moreover, given a $k$-edge-colouring $\xi$ of $G$ a factor is called colourful (w.r.t.\ $\xi$) if it contains precisely one edge per colour.

Graph factors have been studied since at least the 70s (c.f.\ \cite{Plummer07} for a survey) --- see also~\cite{MarxSS21,MarxSS22} for more recent results under the lens of parameterised complexity. Let us consider the following two problems:
\begin{definition}
    Let $\mathcal{B}$ be a finite non-empty subset of $\mathcal{P}(\mathbb{N})$. 
    \begin{itemize}
    \item $\textsc{ColFactor}(\mathcal{B})$ expects as input a graph $G$, a $k$-edge-colouring $\xi$ of $G$ and a mapping $f:V(G) \to \mathcal{B}$, and outputs the number of colourful $f$-factors of $G$. The parameter is $k$.
    \item $\textsc{Factor}(\mathcal{B})$ expects as input a graph $G$, a positive integer $k$, and a mapping $f:V(G) \to \mathcal{B}$, and outputs the number of $f$-factors of size $k$ in $G$. The parameter is $k$.
\end{itemize}
\end{definition}

In other words, $\textsc{Factor}(\mathcal{B})$ and $\textsc{ColFactor}(\mathcal{B})$ can be seen as coloured and uncoloured parameterised holant problems on signatures with co-domain $\{0,1\}$. Those allow us to express counting (colourful) $k$-edge-subgraphs with pre-specified degree constraints, subsuming among others, the problems of counting (colourful) $k$-matchings, $k$-partial cycle covers~\cite{BlaserC12}, or, more generally, $d$-regular $k$-edge subgraphs. We prove the following classification in Corollary~\ref{cor:factor_classification_col} for the coloured setting, and in Corollary~\ref{cor:factor_classification_uncol} for the uncoloured setting. 

\begin{theorem}\label{thm:factor_classification}
    If $\mathcal{B}$ contains a set $\{0\} \subsetneq S \subsetneq \mathbb{N}$ then the problems $\textsc{ColFactor}(\mathcal{B})$ and $\textsc{Factor}(\mathcal{B})$ are $\#\W[1]$-complete, and cannot be solved in time $f(k)\cdot n^{o(k/\log k)}$ for any function~$f$, unless the Exponential Time Hypothesis fails. Otherwise both problems are solvable in FPT-near-linear time.\qed
\end{theorem}


As a consequence of Theorem~\ref{thm:factor_classification} we do not only immediately infer the known parameterised hardness results for counting $k$-partial cycle covers~\cite{BlaserC12} and $k$-matchings~\cite{Curticapean13}, but we also obtain the following, significantly more general, lower bounds.

\begin{corollary}
    Let $d\geq 1$ be any fixed integer. The problem of counting $d$-regular $k$-edge subgraphs in an $n$-vertex graph $G$ is $\#\W[1]$-complete when parameterised by $k$, and cannot be solved in time $f(k)\cdot n^{o(k/\log k)}$ for any function~$f$, unless the Exponential Time Hypothesis fails. The same holds true if $G$ comes with a $k$-edge-colouring and the goal is to count edge-colourful $d$-regular subgraphs.\qed
\end{corollary}


\paragraph*{Modular Counting of (Colourful) Matchings}
The parameterised counting problems $\oplus\textsc{ColMatch}$ and $\oplus\textsc{Match}$ ask, respectively, to compute the number of colourful $k$-matchings in a $k$-edge-coloured graph and the number of $k$-matchings in an (uncoloured) graph; here a $k$-matching is colourful if it contains precisely one edge per colour. Both problems are parameterised by $k$.

Recently, Curticapean, Dell, and Husfeldt analysed the parameterised complexity of counting small subgraphs, modulo a fixed prime $p$~\cite{CurticapeanDH21}. One of their results is an FPT algorithm for the problem $\oplus\textsc{Match}$. Using a standard trick based on inclusion-exclusion, it can be shown that the edge-colourful variant $\oplus\textsc{ColMatch}$ reduces to $\oplus\textsc{Match}$ via parameterised reductions. Therefore, it is not surprising that $\oplus\textsc{ColMatch}$ is fixed-parameter tractable as well. However, we can prove the stronger fact that $\oplus\textsc{ColMatch}$ can be solved in FPT-matrix-multiplication time. Moreover, we also show that neither $\oplus\textsc{ColMatch}$ nor $\oplus\textsc{Match}$ can be solved in FPT-near-linear time; the proof of the subsequent result can be found in Section~\ref{sec:modular}:
\begin{theorem}\label{thm:colmatch_mod_2_intro}
    $\oplus\textsc{ColMatch}$ can be solved in FPT-matrix-multiplication time. Moreover, neither $\oplus\textsc{ColMatch}$, nor $\oplus\textsc{Match}$ can be solved in FPT-near-linear time, unless the Triangle Conjecture fails.\qed
\end{theorem}
Interestingly, to the best of our knowledge, it is (and remains) unknown whether $\oplus\textsc{Match}$ can also be solved in FPT-matrix-multiplication time, since our main result for the uncoloured holant problem does \emph{not} extend to modular counting.


\paragraph*{Counting Weight-$k$ Solutions to Systems of Linear Equations}
We provide an intractability result for the following coding problem (see e.g.\ \cite{BerlekampMT78,DowneyFVW99}), also known as $\#\textsc{Weighted-XOR-SAT}$, or the counting version of $\textsc{Exact-Even-Set}$.\footnote{Note that the problem of \textbf{counting} solutions of Hamming weight $k$ is equivalent to counting solutions of weight \emph{at most} $k$: For one direction, the number of solutions of Hamming weight at most $k$ can be obtained by adding the solutions of Hamming weight $\ell$ for $\ell=0,\dots,k$. For the other direction, the number of solutions of Hamming weight $k$ is equal to the difference of number of solutions of Hamming weight at most $k$ and at most $k-1$.} 

\begin{corollary}\label{cor:coding_hard}
    The problem of counting the Hamming weight $k$ solutions of a system of linear equations $A\vec{x}=0$ over $\mathbb{Z}/2\mathbb{Z}$ is $\#\W[1]$-hard when parameterised by $k$, and cannot be solved in time $f(k)\cdot n^{o(k/\log k)}$ for any function~$f$, unless the Exponential Time Hypothesis fails. This holds true even if the matrix $A$ is promised to contain at most two $1$s per column.
\end{corollary}
\begin{proof}\textcolor{red}{TOPROVE 0}\end{proof}
We point out that the absence of an FPT algorithm for the problem in the previous result is not surprising, given hardness results of related parameterised decision problems~\cite{DowneyFVW99}, as well as the breakthrough (hardness) result on the $\textsc{EvenSet}$ problem due to Bhattacharyya et al.\ \cite{BhattacharyyaGS18}. However, the latter results do not come with tight lower bounds under ETH, and~\cite{BhattacharyyaGS18} even requires as lower bound assumption a stronger, approximate version of ETH, called $\mathsf{GAP}$-ETH. Moreover, neither result applies to the restriction to input matrices with at most two $1s$ per column. 


We further remark that Corollary~\ref{cor:coding_hard} does not contradict the tractability results of Creignou and Vollmer~\cite{CreignouV15}, and of Marx~\cite{Marx05} on a seemingly similar weighted satisfiability problem, since both~\cite{CreignouV15} and~\cite{Marx05} enforce a constant upper bound on the number of variables in each equation.

Finally, note that, Corollary~\ref{cor:coding_hard} also shows that the tractability criteria for parameterised Holants significantly differ from classical holant problems, as $\mathsf{even}$ is an affine signature, and affine signatures yield polynomial time tractability if given an instance of a classical Holant~\cite{ParityHolant13}.

\subsection{Technical Contributions}
It turns out that we do not need to rely on any of the well-established tools for analysing holant problems, such as matchgates~\cite{CaiC07}, combined signatures~\cite{CurticapeanX15}, or holographic reductions~\cite{Valiant08,CaiL11}, despite the fact that some of those tools have already been adapted to the realm of parameterised counting by Curticapean~\cite{Curticapean15}.
Instead, similarly to most works on exact parameterised counting throughout the last seven years~\cite{Roth17,DellRW19,RothSW20,BressanR21,BLR2023stoc,DoringMW24,CurticapeanN24,DoringMW25}, we crucially rely on the framework of complexity monotonicity of graph motif parameters as introduced by Curticapean, Dell, and Marx~\cite{CurticapeanDM17}. In a nutshell, we show that both the coloured and the uncoloured holant problems can be cast as a finite linear combination of homomorphism counts. Let us make this explicit for the uncoloured version --- in fact, the coloured version is easier to analyse, but requires a more extensive set-up on edge-coloured graphs, which we omit from the introduction for the sake of conceptual clarity.

Fix a finite set $\mathcal{S}=\{s_1,\dots,s_\ell\}$ of signatures. We write $\mathcal{G}(\mathcal{S})$ for the set of all signature grids over $\mathcal{S}$. Given two signature grids $\Omega_H=(H,\{s^H_v\}_{v\in V(H)})$ and $\Omega_G=(G,\{s^G_v\}_{v\in V(G)})$ in $\mathcal{G}(\mathcal{S})$, a \emph{homomorphism} from $\Omega_H$ to $\Omega_G$ is a graph homomorphism $\varphi$ from $H$ to $G$ such that $s^G_{\varphi(v)}=s^H_v$ for all $v\in V(H)$, that is $\varphi$ must preserve signatures. We write $\#\homs{\Omega_H}{\Omega_G}$ for the number of homomorpisms from $\Omega_H$ to $\Omega_G$.

Using M\"obius inversion, it is not hard to show that, for each $k$, there is a finitely supported function $\zeta_{\mathcal{S},k}$ from $\mathcal{G}(\mathcal{S})$ to algebraic complex numbers such that, for each signature grid $\Omega$ over $\mathcal{S}$, we have
\begin{equation}\label{eq:to_hombasis_intro}
    \mathsf{UnColHolant}(\Omega,k) = \prod_{i\in \ell}s_i(0)^{n_i} \cdot \sum_{\Omega_H\in \mathcal{G}(\mathcal{S})} \zeta_{\mathcal{S},k}(\Omega_H) \cdot \#\homs{\Omega_H}{\Omega}\,.
\end{equation}
As mentioned before, a similar transformation exists for the colourful holant problem.
Now, an adaptation of the principle of ``complexity monotonicity''~\cite{CurticapeanDM17} will imply that evaluating the linear combination~\eqref{eq:to_hombasis_intro} is \emph{precisely as hard as} evaluating its hardest term $\#\homs{\Omega_H}{\Omega}$ with a non-zero coefficient $\zeta_{\mathcal{S},k}(\Omega_H)\neq 0$. Fortunately, the complexity of evaluating the individual terms $\#\homs{\Omega_H}{\Omega}$ is very well understood: under standard assumptions from fine-grained and parameterised complexity theory, the evaluation is hard if the treewidth (see Section~\ref{sec:prelims}) of $\Omega_H$ is large, and the evaluation is easy if the treewidth of $\Omega_H$ is small. For this reason, the goal of understanding the complexity of parameterised holant problems can be reduced to the purely combinatorial problem of understanding the coefficient function $\zeta_{\mathcal{S},k}$, and its analogue in the coloured setting.

In previous results, this approach has mostly been used for establishing \emph{lower bounds} on pattern counting problems on graphs~\cite{FockeR22,DoringMW24,Curticapean24,CurticapeanN24}, in which case it suffices to find a high-treewidth term that survives with a non-zero coefficient, and it has turned out that even finding \emph{one} such a term can constitute solving highly challenging combinatorial problems~\cite{RothSW20,PeyerimhoffRSSVW23,DoringMW24,DoringMW25}. Using the framework for upper bounds is usually even more difficult since it makes it necessary to find an upper bound on the treewidth of \emph{all} graphs surviving with a non-zero coefficient (see e.g.\ \cite{PeyerimhoffRSV21,BeraGLSS22,BressanR21,BLR23}). In the current work, we solve this task as precisely as possible; intuitively, our main combinatorial result reads as follows:

\begin{center}
    \textit{The maximum treewidth of the graphs surviving in the ``homomorphism basis'' of \emph{any} instance of a parameterised Holant problem (coloured or uncoloured) is either $1$, $2$, or unbounded.}
\end{center}
Concretely, in the coloured setting, we show that 
\begin{enumerate}
    \item[(1)] for $\mathcal{S}$ of type $\mathbb{T}[\mathsf{Lin}]$ all terms of treewidth $\geq 2$ vanish,
    \item[(2)] for $\mathcal{S}$ of type $\mathbb{T}[\omega]$ all terms of treewidth $\geq 3$ vanish, but at least one term of treewidth $2$ survives, and
    \item[(3)] for $\mathcal{S}$ of type $\mathbb{T}[\infty]$, terms of arbitrary high treewidth survive.
\end{enumerate}
In the uncoloured setting (specifically in Equation~\eqref{eq:to_hombasis_intro}), we show that 
\begin{enumerate}
    \item[(1)] for $\mathcal{S}$ of type $\mathbb{T}[\mathsf{Lin}]$ all terms of treewidth $\geq 2$ vanish,
    \item[(2)] for $\mathcal{S}$ of type $\mathbb{T}[\omega]$ or $\mathbb{T}[\infty]$, terms of arbitrary high treewidth survive.
\end{enumerate}

We consider the combinatorial analysis of the coefficients $\zeta$ to be the central technical contribution of this work. Moreover, we emphasize that, by understanding the coefficients in this detail, our work is, to the best of our knowledge, the only application of the framework of complexity monotonicity achieving a classification of a general family of counting problems into FPT and $\#\mathrm{W}[1]$-complete cases in which the exponents of all FPT cases are almost precisely determined under assumptions from fine-grained complexity theory. 


\subsection{Conclusion and Future Directions}
In this work, we focused on \emph{symmetric} signatures for parameterised Holant problems. We provided complete classifications for both the coloured and the uncoloured variant of the problem, not only identifying precisely those instances that allow for FPT algorithms, but also proving almost tight bounds for the best possible run-time exponents under assumptions from fine-grained complexity theory.

We identify the following questions as starting points for future research on parameterised Holants:
\begin{itemize}
    \item Asymmetric signatures. In classical Holant theory, after the case of symmetric (boolean) signatures had been solved~\cite{CaiGW16}, a significant amount of effort has been put in understanding asymmetric signatures, which involve more intricate cases than the classifications for symmetric signatures \cite{CaiFX18,CaiF23,meng2025}. We expect that the tools we set up for the study of parameterised Holants can be generalised to apply to the asymmetric setting via encoding the orderings of activated incident edges to a vertex by imposing  constraints on the set of tuples of edge-colours.
    \item Polynomial Time vs.\ $\#\mathrm{P}$-hardness. The FPT cases of our classifications are all ``real'' FPT cases in the sense that the running time of our algorithms yields superpolynomial overhead in the parameter $k$. For a future direction, we propose to study for which cases the superpolynomial overhead is necessary, under standard assumptions from (counting) complexity theory such as $\mathrm{FP}\neq \#\mathrm{P}$.
\end{itemize}


\subsection{Organisation of the Paper}

We start with introducing the required preliminary material in Section~\ref{sec:prelims}. In particular, we revisit the framework of \emph{fractured graphs} as introduced in~\cite{PeyerimhoffRSSVW23}, which will be a central ingredient for the classification of $\holantprob(\mathcal{S})$.
Afterwards, in Section~\ref{sec:col_holants_equivalences}, we consider an intermediate problem $\holantprobstar(\mathcal{S})$ and show it to be interreducible with $\holantprob(\mathcal{S})$ under parameterised linear-time reductions. This allows us then to prove Main Theorem~\ref{main_thm} by classifying first the intermediate problem $\holantprobstar(\mathcal{S})$, which, for technical reasons, is much more convenient to work with than $\holantprob(\mathcal{S})$. In Section~\ref{sec:modular}, we present the extension of Main Theorem~\ref{main_thm} to counting modulo a fixed prime $p$.
Next, in Section~\ref{sec:sig0}, we show how our classification for $\holantprob(\mathcal{S})$ in Main Theorem~\ref{main_thm} can be extended to the case in which we allow signatures $s$ with $s(0)=0$.
Finally, in Section~\ref{sec:uncoloured}, we prove our classification for the uncoloured holant problem (Main Theorem~\ref{thm:main_uncol}), and we encapsulate the analysis of the coefficient function $\zeta$ for the uncoloured problem in Section~\ref{sec:analysis_of_zeta}. For reasons of accessibility, we present our analysis of $\zeta$ in Section~\ref{sec:analysis_of_zeta} in multiple steps, starting with the special cases that encapsulate the central ideas, and then iteratively and carefully generalising the arguments and results to obtain a thorough understanding of $\zeta$ in the full and unrestricted case.
We also include an appendix with some proofs that are omitted in the main paper, and a construction of infinitely many signature families of each of the three types in our classifications.

\newpage
\tableofcontents

\pagebreak


\section{Preliminaries}\label{sec:prelims}

For a finite set $S$, we denote the cardinality of~$S$ by~$|S|$ or $\#S$. Graphs in this work are undirected, simple and without loops, unless stated otherwise. 
Given a vertex $v$ of a graph $G$, we write $E(v)$ for the set of all edges of $G$ that are incident to $v$. Let $\mathcal{N}_{G}(v) = \{u \in V(G) \mid \{u, v\} \in E(G)\}$ denote the neighbourhood of a vertex $v\in G$ and let $d_G(v)=|\mathcal{N}_G(v)|$ denote the degree of $v$. 
For a subset $J\subseteq V(G)$, we denote by $G[J]$ the induced subgraph graph of $G$ with vertices $J$ and edges $\{u,v\}\in E(G)$ with $u,v\in J$. Throughout this paper, given a graph $G$, a set $S$, and a mapping $m: V(G) \to S$, we will always freely allow ourselves to extend $m$ to the edges of $G$ by just setting $m(\{u,v\}):=\{m(u),m(v)\}$. For a function $f:A\to B$ and a subset $C\subseteq A$, we write $f|_A: C \to B$ for the restriction of $f$ to $C$.

A \emph{tree-decomposition} of a graph $G$ is a pair of a tree $T$ and a mapping $\beta: V(T) \to 2^{V(G)}$ such that $\bigcup_{t\in V(T)} \beta(t)=V(G)$, for each edge $e$ of $G$ there is $t\in V(T)$ such that $e \subseteq \beta(t)$, and for each vertex $v$ of $G$ the graph $T[\{t\in V(T)\mid v\in \beta(t)\}]$ is connected.
The width of $(T,\beta)$ is $\max\{|\beta(t)|-1 \mid t\in V(t)\}$, and the \emph{treewidth} of $G$, denoted by $\mathsf{tw}(G)$ is the minimum possible width of a tree-decomposition of $G$. 
It will sometimes be convenient to consider tree-decompositions $(T,\beta)$ where $T$ is a \emph{rooted} tree with some fixed root $r$. Then,
for $t \in V(T)\setminus\{r\}$, let $\sigma(t)$ denote \textit{the separator at $t$}, that is $\sigma(t) = \beta(t)\,\cap\,\beta(t')$, where $t'$ is the parent of $t$ in $T$. For the root $r$, we set $\sigma(r) = \emptyset$. Let also $\gamma(t)\subseteq V(H)$ denote \textit{the cone at $t$}, that is, $\gamma(t)$ is the union of the $\beta(\hat{t})$ for all descendants $\hat{t}$ of $t$.



\subsection{Homomorphisms, Embeddings, and Automorphisms}
A \emph{homomorphism} from a graph $H$ to a graph $G$ is a mapping $\varphi: V(H) \to V(G)$ such that $\varphi(e)\in E(G)$ for each edge $e\in E(H)$. We write $\homs{H}{G}$ for the set of all homomorphisms from $H$ to $G$. An \emph{embedding} from $H$ to $G$ is a homomorphism from $H$ to $G$ that is injective (on the vertices of $H$), and we write $\embs{H}{G}$ for the set containing all embeddings from $H$ to $G$. Finally, an embedding from $H$ to itself is called an automorphism of $H$, and we write $\auts(H)$ for the set of all automorphisms of $H$. Note that, for $\pi \in \auts(H)$, we have $\{u,v\}\in E(H)\Leftrightarrow \{\pi(u),\pi(v)\}\in E(H)$ for each pair of vertices $u,v \in V(H)$.


\subsection{Partitions, Quotient Graphs, and M\"obius Functions}\label{sec:parts_quots_mobius}
A partition $\rho$ of a finite set $S$ is a set of pairwise disjoint and non-empty blocks $\rho=\{B_1,\dots,B_t\}$ such that $\dot\cup_{i=1}^t B_i = S$; we emphasise that the order of the blocks does not matter. We write $|\rho|$ for the number of blocks of $\rho$. Given two partitions $\rho$ and $\sigma$ of $S$, we say that $\sigma$ \emph{refines} $\rho$, denoted by $\sigma \leq \rho$, if for each block $B^\rho\in \rho$ there are blocks $B_1^\sigma,\dots,B_\ell^\sigma$ of $\sigma$ such that $\dot\cup_{i=1}^\ell B^\sigma_i = B^\rho$. Intuitively, this means that $\sigma$ can be decomposed into ``subpartitions'' $\sigma_B$ of $B$ for each block $B$ in $\rho$. We write $\bot_S$ for the \emph{finest partition}, that is, for the partition that contains blocks $\{s\}$ for each $s\in S$, and we write $\top_S$ for the \emph{coarsest partition}, that is, for the partition containing only one block $B=S$. We might drop the subscript and just write $\bot$ and $\top$ if~$S$ is clear from the context.
Given two partitions $\sigma \leq \rho$, the \emph{M\"obius function} of $\sigma$ and $\rho$ is defined as follows.\footnote{In fact, the M\"obius function is \emph{defined} using the incidence algebra of a partially ordered set (see e.g.\ \cite[Chapter 3.6]{Stanley11}), and Definition~\ref{def:mobius} is a lemma on the M\"obius function over the poset of partition refinement. However, since we will not rely on any further properties of the incidence algebra of that poset, we allow ourselves to introduce the M\"obius function via the explicit formula in Definition~\ref{def:mobius} for reasons of self-containment.}

\begin{definition}[The M\"obius function of partitions (cf.\ \cite{Stanley11})]\label{def:mobius}
    Let $S$ be a finite set and let $\rho=\{B_1,\dots,B_t\}$ be a partition of $S$. Let furthermore $\sigma$ be a partition of $S$ with $\sigma \leq \rho$. For each $i\in [t]$, let $\sigma^i$ be the subpartition of $\sigma$ refining $\{B_i\}$, that is, $\sigma = \dot{\cup}_{i=1}^t \sigma^i$ and $\sigma^i$ is a partition of $B_i$ for all $i\in [t]$. Then the \emph{M\"obius function} of $\sigma$ and $\rho$ is defined as follows:
    \[ \mu_S(\sigma,\rho):=  \prod_{i=1}^t (-1)^{|\sigma^i|-1}(|\sigma^i|-1)! \,,\]
    where $0!=1$ as usual. We will drop the subscript $S$ of $\mu$ if it is clear from the context.
\end{definition}





\paragraph*{Quotient Graphs}
Given a graph $H$, we write $\ppart(H)$ for the set of all partitions of $V(H)$. Given a partition $\rho$ of $V(H)$, the \emph{quotient graph} $H/\rho$ has as vertices the blocks of $\rho$ and two blocks $B_1$ and $B_2$ are made adjacent if (and only if) there are vertices $u\in B_1$ and $v\in B_2$ such that $\{u,v\} \in E(H)$. Note that $H/\rho$ \emph{might contain self-loops}. We write $h_\rho: V(H) \to V(H/\rho)$ for the mapping that assigns each vertex $v\in V(H)$ the block $B$ containing $v$, and we observe that $h_\rho$ is a homomorphism. We write $\mu_H$ for the M\"obius function $\mu_{V(H)}$ of partitions of $V(H)$, and we might again drop the subscript if it is clear from the context. 

\subsection{$H$-coloured graphs and fractured graphs}\label{sec:col_graphs_fractures}



Given a graph $H$, an $H$\emph{-coloured graph} is a pair $(G,h)$ of a graph $G$ and a homomorphism, called $H$-\emph{colouring}, $h\in \homs{G}{H}$.
Given a graph $H$ and an $H$-coloured graph $(G,h)$, a homomorphism $\varphi \in \homs{H}{G}$ is called \emph{colour-prescribed}, w.r.t.\ $h$, if $h(\varphi(v))=v$ for all $v\in V(H)$. We write $\cphoms(H \to (G,h))$ for the set of all colour-prescribed homomorphisms (w.r.t.\ $h$) from $H$ to $G$.

\begin{definition}[Fracture]\label{def:fracture}
    Let $H$ be a graph without isolated vertices. A \emph{fracture} of $H$ is a tuple $\vec\rho=(\rho_v)_{v\in V(H)}$ where $\rho_v$ is a partition of $E(v)$ for each $v\in V(H)$.
\end{definition}


Let $H$ be a graph without isolated vertices, we denote by $\mathcal{F}(H)$ the set of all fractures of $H$. Given $\vec{\rho}\in\mathcal{F}(H)$ and a vertex $v$ of $H$, we will write $\vec{\rho}(v)$ for the partition in $\vec{\rho}$ corresponding to vertex $v$. Given two fractures $\vec{\sigma}$ and $\vec{\rho}$ of a graph $H$, we slightly overload notation and denote point-wise partition refinement by $\leq$, that is, we write $\vec{\sigma}\leq \vec{\rho}$ if $\vec{\sigma}(v) \leq \vec{\rho}(v)$ for all vertices $v$ of $H$. The M\"oebius function extends from partitions to fractures as follows:\footnote{Again, the M\"obius function of fractures (of a graph $H$) is normally defined via the incidence algebra of the poset of fractures of $H$ and pointwise partition refinement. As observed in~\cite{PeyerimhoffRSSVW23}, the resulting poset is isomorphic to the product of $|V(H)|$ partition lattices and the M\"obius function factorises along the individual partitions.}

\begin{definition}[The M\"obius function of fractures]
    Let $\vec{\sigma} \leq \vec{\rho}$ be fractures of a graph $H$. The \emph{M\"obius} function of $\vec{\sigma}$ and $\vec{\rho}$ is defined as follows: 
    \[\vec{\mu}_H := \prod_{v\in V(H)} \mu_{E(v)}(\vec{\sigma}(v),\vec{\rho}(v))\,.\]
    We will drop the subscript $H$ if the graph is clear from the context.
\end{definition}

For notational convenience, we will always use $\vec{\mu}$ to denote the M\"obius function of fractures, and $\mu$ to denote the M\"obius function of partitions.

Informally, a fracture $\vec{\rho}$ of a graph $H$ is an instruction on how to split, or ``fracture'' the vertices of~$H$. Concretely, for each $v\in V(H)$, we replace $v$ by vertices $v^B$ for each $B \in \vec{\rho}(v)$. Then we add an edge between two vertices $u^B$ and $v^{B'}$ if and only if $\{u,v\}\in B \cap B'$. Formally, we state the definition as in~\cite{PeyerimhoffRSSVW23}; consult Figure~\ref{fig:fracture} for an illustration.

\begin{figure}[t!]
    \centering
    \begin{tikzpicture}[scale=1.75]

        \node[vertex,inner sep=.4ex,label={[label distance=.03]below:\(v\)}] (m) at (0, 0) {};

        \draw[very thick,red] (m) -- ++(135:1);
        \node[label={[label distance=.02]below:\(e_1\)}] at (135:0.7) {};
        \draw[very thick,green!80!blue] (m) -- ++(180:1);
        \node[label={below:\(e_2\)}] at (180:0.5) {};
        \draw[very thick,blue] (m) -- ++(-135:1);
        \node[label={[label distance=.02]below:\(e_3\)}] at (-135:0.5) {};

        \draw[very thick,yellow!50!orange] (m) -- ++(45:1);
        \node[label={[label distance=.03]below:\(e_4\)}] at (45:0.7) {};
        \draw[very thick] (m) -- ++(0:1);
        \node[label={below:\(e_5\)}] at (0:0.5) {};
        \draw[very thick,cyan] (m) -- ++(-45:1);
          \node[label={[label distance=.03]below:\(e_6\)}] at (-45:0.5) {};


        \begin{scope}[shift={(5,0)}]
            \begin{scope}[scale=2.4]
                \kowaen{0,0}{-90/90/white,90/270/white}{1};
            \end{scope}

            \node[label={[label distance=.03]below:\(v^{B_1}\)}]  at (1-2) {};

            \draw[very thick,red] (1-2) -- ++(135:1);
            
            \draw[very thick,green!80!blue] (1-2) -- ++(180:1);
            
            \draw[very thick,blue] (1-2) -- ++(-135:1);
            \begin{scope}[shift=(1-2)]
            \node[label={below:\(e_2\)}] at (180:0.5) {};
            \node[label={[label distance=.02]below:\(e_1\)}] at (135:0.7) {};
            \node[label={[label distance=.02]below:\(e_3\)}] at (-135:0.5) {};
            \end{scope}

            \node[label={[label distance=.03]below:\(v^{B_2}\)}]  at
                (1-1) {};

            \draw[very thick,yellow!50!orange] (1-1) -- ++(45:1);
            \draw[very thick] (1-1) -- ++(0:1);
            \draw[very thick,cyan] (1-1) -- ++(-45:1);
            \begin{scope}[shift=(1-1)]
            \node[label={[label distance=.03]below:\(e_6\)}] at (-45:0.5) {};
             \node[label={below:\(e_5\)}] at (0:0.5) {};
             \node[label={[label distance=.03]below:\(e_4\)}] at (45:0.7) {};
            \end{scope}

            \begin{scope}[scale=1.8]
                \kowaen{0,0}{-90/90/black,90/270/black}{1};
            \end{scope}

        \end{scope}
    \end{tikzpicture}
    \caption{\label{fig:fracture} Illustration of the construction of a
        fractured graph taken from~\cite{PeyerimhoffRSSVW23}. The left picture shows a vertex $v$ of a graph~$H$ with incident
        edges $E_H(v)=\{e_1,\dots,e_6\}$.
The right
        picture shows the splitting of $v$ in the construction of the fractured
        graph~$\fracture{H}{\vec{\rho}}$ for a fracture $\vec{\rho}$ satisfying that the partition
        $\vec{\rho}(v)$ contains two blocks $B_1 =\{e_1,e_2,e_3\}$ and $B_2=\{e_4,e_5,e_6\}$.
}
\end{figure}

\begin{definition}[Fractured graphs $\fracture{H}{\vec{\rho}}$]\label{def:fract_graph}
 Let $H$ be a graph without isolated vertices, and let $\vec{\rho}\in\mathcal{F}(H)$. Write $M_H$ for the matching containing one copy of each edge of $H$, that is, $V(M_H)=\bigcup_{e \in E(H)}\{u_e,v_e\}$, and $E(M_H)= \{\{u_e,v_e\} \mid e \in E(H)\}$.

 Let $\tau$ be the partition on $V(M_H)$ that places two vertices $u_e,v_f$ into the same block if and only if $u=v$ and there exists $B\in \vec{\rho}(u)$ with $e,f \in B$. Then the \emph{fractured graph} $\fracture{H}{\vec{\rho}}$ is defined to be the quotient $M_H/\tau$. We write $v^B$ for the vertex of $\fracture{H}{\vec{\rho}}$ corresponding, in this construction, to vertex $v\in V(H)$ and block $B\in \vec{\rho}(v)$. 
\end{definition}


Given a graph $H$ and $\vec{\rho}\in\mathcal{F}(H)$, the fractured graph $\fracture{H}{\vec{\rho}}$ admits an $H$-colouring $h_{\vec{\rho}}$ which maps $v_B$ to $v$ for each $v\in V(H)$ and $B \in \vec{\rho}(v)$. We refer to $h_{\vec{\rho}}$ as the \emph{canonical $H$-colouring} of $\fracture{H}{\vec{\rho}}$. We next introduce a version of colour-preserving homomorphisms from fractured graphs.

\begin{definition}[$\homscp,\embscp$]
    Let $H$ be a graph and let $\vec{\rho}\in\mathcal{F}(H)$. Furthermore, let $(G,h)$ be an $H$-coloured graph. We set
    \[ \homscp(\fracture{H}{\vec{\rho}} \to (G,h))= \{\varphi \in \homs{\fracture{H}{\vec{\rho}}}{G} \mid \forall x \in V(\fracture{H}{\vec{\rho}}): h(\varphi(x)) = h_{\vec{\rho}}(x) \} \,, \]
    where $h_{\vec{\rho}}$ is the canonical $H$-colouring of $\fracture{H}{\vec{\rho}}$. The set $\embscp(\fracture{H}{\vec{\rho}} \to (G,h))$ is defined similarly for embeddings.
\end{definition}
In other words, $\homscp(\fracture{H}{\vec{\rho}} \to (G,h))$ and $\embscp(\fracture{H}{\vec{\rho}} \to (G,h))$ contain, respectively, the homomorphisms and embeddings from $\fracture{H}{\vec{\rho}} $ to $G$ that maps vertices $v_B$ of $\fracture{H}{\vec{\rho}}$ (i.e., $v\in V(H)$ and $B$ is a block of $\vec{\rho}(v)$) to vertices in $G$ coloured by $h$ with $v$.





\subsection{Parameterised and Fine-Grained Complexity Theory}
We provide a concise introduction to parameterised counting complexity in what follows, and we refer the reader to the standard textbook~\cite{CyganFKLMPPS15} for a comprehensive treatment of parameterised algorithms and to~\cite[Chapter 14]{FlumG06} and~\cite{Curticapean15} for a detailed overview over parameterised counting problems.  


A \emph{parameterised counting problem} is a pair of a function $P:\{0,1\}^\ast \to \mathbb{Q}$ and a computable parameterisation\footnote{We note that some authors require the parameterisation to be computable in polynomial time, which is the case for all parameterisations encountered in this work.} $\kappa: \{0,1\}^\ast \to \mathbb{N}$. For example, $\#\textsc{Clique}$ denotes the parameterised counting problem of computing the number of $k$-cliques in a graph $G$, and it is parameterised by $k$, that is, $\kappa(G,k)=k$.
A parameterised counting problem $(P,\kappa)$ is called \emph{fixed-parameter tractable} (FPT) if there is a computable function $f$ and an algorithm $\mathbb{A}$ such that, on input $x$, computes $P(x)$ in time $f(\kappa(x))\cdot |x|^{O(1)}$. We call $\mathbb{A}$ an \emph{FPT} algorithm w.r.t.\ parameterisation $\kappa$.

For the purpose of this work, we will be interested in the exact constant in the exponent of $|x|$ in the running time of an FPT algorithm; in the definition below we use $\tilde{O}(n)$ to hide poly-logarithmic factors, i.e., $t\in \tilde{O}(n)$ if there is a constant $d$ such that $t\in O(n\log^dn)$.

\begin{definition}[FPT-near-linear time and FPT-matrix-multiplication time]
    An algorithm $\mathbb{A}$ is called an \emph{FPT-near-linear time algorithm} w.r.t.\ parameterisation $\kappa$ if there is a computable function $f$ such that $\mathbb{A}$ runs in time $f(\kappa(x))\cdot \tilde{O}(|x|)$. Moreover, we call $\mathbb{A}$ an \emph{FPT-matrix-multiplication time algorithm} if there is a computable function $f$ such that $\mathbb{A}$ runs in time $f(\kappa(x))\cdot O(|x|^{\omega})$, where $\omega$ is the matrix multiplication exponent.\footnote{At the time of writing of this work, the best known bound for $\omega$ is $2\leq \omega\leq 2.371552$~\cite{WilliamsXXZ24}.}

    We say that a parameterised counting problem is solvable in FPT-near-linear time (resp.\ FPT-matrix-multiplication time) if it can be solved by an FPT-near-linear time (resp.\ FPT-matrix-multiplication time) algorithm.
\end{definition}


We proceed by introducing two notions for reductions between parameterised counting problems. We note that FPT Turing-reductions are standard~\cite[Chapter 14]{FlumG06}, but \emph{linear} FPT Turing-reductions are less so, since we have to be careful with the number of oracle queries.

\begin{definition}[Parameterised Reductions]
    Let $(P,\kappa)$ and $(P',\kappa')$ be parameterised counting problems. An \emph{FPT Turing-reduction} from $(P,\kappa)$ to $(P',\kappa')$ is an algorithm $\mathbb{A}$ with the following properties: There exists a computable function $f$ such that
    \begin{enumerate}
        \item on input $x$, $\mathbb{A}$ computes $P(x)$ in time $f(\kappa(x))\cdot |x|^{O(1)}$, and
        \item $\mathbb{A}$ has oracle access to $P'$, and, on input $x$, each oracle query $y$ posed by $\mathbb{A}$ satisfies $\kappa'(y)\leq f(\kappa(x))$.
    \end{enumerate}
    We write $(P,\kappa)\fptred (P',\kappa')$ if an FPT Turing-reduction exists. Moreover, we write $(P,\kappa)\fptinterred (P',\kappa')$ if $(P,\kappa)\fptred (P',\kappa')$ and $(P',\kappa')\fptred (P,\kappa)$.

    A \emph{linear FPT Turing-reduction} from $(P,\kappa)$ to $(P',\kappa')$ is an algorithm $\mathbb{A}$ with the following properties: There exists a computable function $f$ such that
    \begin{enumerate}
        \item on input $x$, $\mathbb{A}$ computes $P(x)$ in time $f(\kappa(x))\cdot O(|x|)$.
        \item $\mathbb{A}$ has oracle access to $P'$, and, on input $x$, each oracle query $y$ posed by $\mathbb{A}$ satisfies $\kappa'(y)\leq f(\kappa(x))$, and the number of oracle queries must be bounded by $f(\kappa(x))$.
    \end{enumerate}
    We write $(P,\kappa)\fptlinred (P',\kappa')$ if a linear FPT Turing-reduction exists. Moreover, we write $(P,\kappa)\fptinterlinred (P',\kappa')$ if $(P,\kappa)\fptlinred (P',\kappa')$ and $(P',\kappa')\fptlinred (P,\kappa)$.
\end{definition}

It is well-known that $(P,\kappa)$ is FPT if $(P,\kappa)\fptred (P',\kappa')$ and $(P',\kappa')$ is FPT. For the purpose of this work, we establish a more fine-grained version of this fact via linear FPT Turing-reductions.

\begin{lemma}
    Let $d\geq 0$ and $c\geq 1$ be reals, and let $(P,\kappa)$ and $(P',\kappa')$ be parameterised counting problems such that $(P,\kappa)\fptlinred (P',\kappa')$. Assume that $(P',\kappa')$ can be solved in time $f(\kappa'(x))\cdot O(\log^d(|x|)\cdot |x|^c)$ for some computable function $f$. Then there is a computable function $g$ such that $(P,\kappa)$ can be solved in time $g(\kappa(x))\cdot O(\log^d(|x|)\cdot |x|^c)$. 
\end{lemma}
\begin{proof}\textcolor{red}{TOPROVE 1}\end{proof}

\subsubsection{Lower Bounds and Fine-Grained Complexity Theory}\label{sec:prelims_fgct}
For this work, we rely on the following two hardness assumptions from fine-grained complexity theory.

\begin{hypothesis}[ETH~\cite{ImpagliazzoP01,ImpagliazzoPZ01}]
    The \emph{Exponential Time Hypothesis} (ETH) asserts that $3\textsc{-SAT}$ cannot be solved in $\exp(o(n))$, where $n$ is the number of variables of the input formula. 
\end{hypothesis}

\begin{hypothesis}[Triangle Detection Conjecture~\cite{Abboud14-TriangleConj}]
    The \emph{Triangle Detection Conjecture} asserts that there is a positive real $\gamma >0$ such that, in the word RAM model of $O(\log n)$ bits, there is no (deterministic or randomised) algorithm that decides whether a graph with $m$ edges contains a triangle in (expected) time $O(m^{1+\gamma})$.
\end{hypothesis}

In addition to running time lower bounds based on the previous assumptions, we will also establish hardness results for the parameterised complexity class $\#\W[1]$, which can be thought of as the parameterised counting equivalent of $\mathrm{NP}$~\cite[Chapter 14]{FlumG06}. A parameterised counting problem $(P,\kappa)$ is $\#\W[1]$-\emph{hard} if $\#\textsc{Clique}\fptred (P,\kappa)$, and it is $\#\W[1]$-\emph{complete} if $\#\textsc{Clique}\fptinterred (P,\kappa)$.\footnote{For readers familiar with structural parameterised complexity theory, we note that, originally, containment in $\#\W[1]$ is defined via parameterised parsimonious reductions~\cite[Chapter 14]{FlumG06}. However, it has since become standard to use parameterised Turing-reductions for the definition of $\#\W[1]$-completeness instead; see \cite[Chapter 2.3.1]{Roth19} for a more comprehensive discussion.} It is well-known that $\#\W[1]$-hard problems are not FPT unless ETH fails~\cite{Chenetal05,Chenetal06,CyganFKLMPPS15}. 

For our analysis of the complexity of $\holantprob$, the following coloured homomorphism counting problem will be a key ingredient.

\begin{definition}[$\#\cphomsprob(\mathcal{H})$]
    Let $\mathcal{H}$ be a class of graphs. The problem $\#\cphomsprob(\mathcal{H})$ expects as input a graph $H\in \mathcal{H}$ and an $H$-coloured graph $(G,h)$, and outputs $\#\cphoms(H \to (G,h))$. The parameter is $|H|$
\end{definition}

We rely on the following (conditional) lower bounds on $\#\cphomsprob(\mathcal{H})$.

\begin{lemma}\label{lem:cphom_lower_bounds}
    Let $\mathcal{H}$ be a recursively enumerable\footnote{This is a standard technical condition, required to avoid discussing non-uniform FPT algorithms and parameterised reductions. All classes considered in this work will be recursively enumerable.} class of graphs.
    \begin{enumerate}
        \item If $\mathcal{H}$ contains a triangle then $\#\cphomsprob(\mathcal{H})$ cannot be solved in FPT-near-linear time, unless the Triangle Conjecture Fails.
        \item If $\mathcal{H}$ has unbounded treewidth, then $\#\cphomsprob(\mathcal{H})$ is $\#\W[1]$-hard and, assuming ETH, cannot be solved in time \[f(|H|)\cdot |V(G)|^{o(\mathsf{tw}(H)/\log(\mathsf{tw}(H)))}\]
        for any function $f$.
    \end{enumerate}
\end{lemma}
\begin{proof}\textcolor{red}{TOPROVE 2}\end{proof}


\subsubsection{Algorithms for counting coloured homomorphisms}\label{sec:algo_count_col_homs}
For what follows, given two graphs $H$ and $G$ with (not necessarily proper) vertex colourings $\nu_H$ and $\nu_G$, respectively, we write $\homs{(H,\nu_H)}{(G,\nu_G)}$ for the set of homomorphisms $\varphi$ from $H$ to $G$ that agree on the vertex colourings, i.e., $\nu_H(v)=\nu_G(\varphi(v))$ for all $v\in V(H)$. 

\begin{definition}[$\#\colhomsprob(\mathcal{H})$]
    Let $\mathcal{H}$ be a class of graphs. The problem $\#\colhomsprob(\mathcal{H})$ expects as input a graph $H\in \mathcal{H}$, a graph $G$, and (not necessarily proper) vertex colourings $\nu_H$ and $\nu_G$ of $H$ and $G$. and outputs $\#\homs{(H,\nu_H)}{(G,\nu_G)}$. The parameter is $|H|$.
\end{definition}

It is well-known, in fact, folklore, that counting homomorphisms from a graph $H$ to a graph $G$ can be counted in near-linear time (in $|V(G)|+ |E(G)|$) if $H$ is acyclic (i.e., if it has treewidth $1$) (see e.g.\ \cite[Theorem 7]{BeraGLSS22} for a formal statement and proof). Moreover, the same holds true for the more general problem of counting answers to acyclic conjunctive without quantified variables (see e.g.\ \cite[Theorem 12]{BraultBaron13}). Interpreting vertex-colours as unary predicates, we obtain as an immediate corollary:

\begin{fact}\label{fact:colhoms_lintime}
    $\#\colhomsprob(\mathcal{H})$ can be solved in FPT-near-linear time if $\mathcal{H}$ only contains acyclic graphs.
\end{fact}

Next, we adapt a result due to Curticapean, Dell and Marx~\cite[Theorem 1.7]{CurticapeanDM17} from uncoloured homomorphisms to coloured homomorphisms to obtain, for $\mathcal{H}$ containing graphs of treewidth at most~$2$, an algorithm for $\#\colhomsprob(\mathcal{H})$ via fast matrix multiplication; the proof of (a generalisation of) the following lemma can be found in the Appendix~\ref{sec:appendix_fastMM}.

\begin{lemma}\label{lem:colhom_matrix_multi}
    Let $\mathcal{H}$ be a class of graphs of treewidth at most $2$. Then $\#\colhomsprob(\mathcal{H})$ can be solved in time $f(|H|)\cdot \mathcal{O}(|V(G)|^{\omega})$ for some computable function $f$. \qed
\end{lemma}



\subsection{Parameterised Holant Problems}
For a smoother presentation we will first consider the following types of signatures.
\begin{definition}
    A \emph{signature} is a computable function $s:\mathbb{N} \to \mathbb{Q}$ with $s(0)\neq 0$.
\end{definition}
In Section~\ref{sec:sig0} we show how to deal with signatures $s$ allowing $s(0)=0$. We further point out that, in classical holant literature, the symbol $f$ is used for signature functions. However, since $f$ is co-notated with another role in the world of parameterised algorithms, we decided to use the symbol $s$ instead. 

\begin{definition}[Edge-Coloured Signature Grids]
    Let $\mathcal{S}$ be a finite set of signatures and let $k$ be a positive integer.
    A $k$-\emph{edge-coloured signature grid} over $\mathcal{S}$ is a triple $\Omega=(G,\xi,\{s_v\}_{v\in V(G)})$ of a graph $G$, a mapping $\xi:E(G) \to [k]$ called the $k$-\emph{edge-colouring}, and a collection of signatures $\{s_v\}_{v\in V(G)}$ with $s_v \in \mathcal{S}$ for all $v\in V(G)$.

    A subset of edges $A\subseteq E(G)$ of $G$ is called \emph{colourful} if $|A|=k$ and $\xi(A)=[k]$, that is, $A$ contains precisely one edge per colour.
\end{definition}

\begin{definition}[Edge-Colourful Holants]
    Let $\Omega=(G,\xi,\{s_v\}_{v\in V(G)})$ be a $k$-edge-colourful signature grid. Define
    \[ \holant(\Omega) = \sum_{\substack{A \subseteq E(G)\\ A \text{ colourful}}} \prod_{v\in V(G)} s_v(|A \cap E(v)|) \]
\end{definition}

We are now able to define the parameterised holant problem.

\begin{definition}[$\textsc{p-Holant}(\mathcal{S})$]
    Let $\mathcal{S}$ be a finite set of signatures. The problem $\textsc{p-Holant}(\mathcal{S})$ expects as input a positive integer $k$ and a $k$-edge-coloured signature grid $\Omega=(G,\xi,\{s_v\}_{v\in V(G)})$ over $\mathcal{S}$. The output is $\holant(\Omega)$, and the problem is parameterised by $k$. 
\end{definition}

For technical reasons, we will also consider the following restricted version of $\textsc{p-Holant}(\mathcal{S})$:
\begin{definition}[$\holantprobstar(\mathcal{S})$]
    Let $\mathcal{S}$ be a finite set of signatures. The problem $\holantprobstar(\mathcal{S})$ expects as input a graph $H$, an $H$-coloured graph $(G,h)$, and a collection $\{s_v\}_{v\in V(G)}$ of signatures in $\mathcal{S}$, such that for any pair of vertices $u,v$ of $G$ we have $h(u)=h(v)$ implies $s_u=s_v$. The output is $\holant(G,h,\{s_v\}_{v\in V(G)})$, and the problem is parameterised by $|H|$.
\end{definition}

Note that we slightly abuse notation in the previous definition by using the $H$-colouring $h$ of $G$ as the edge-colouring of the signature grid, that is, we assign an edge $e=\{u,v\}$ of $G$ the colour $h(e):=\{h(u),h(v)\} \in E(H)$. Formally, we can fix any bijection $b:E(H) \to [|E(H)|]$ and define the edge-colouring of the signature grid by setting $\xi(e):=b(h(e))$. For the sake of avoiding notational clutter, we will omit making $b$ explicit and just refer to $h$ as the edge-colouring of the signature grid in the remainder of the paper. 

In this way, observe that $\holantprobstar(\mathcal{S})$ is a restriction of $\holantprob(\mathcal{S})$ since we allow as edge-colourings only those that correspond to an underlying $H$-colouring. Thus, clearly:

\begin{fact}\label{fact:easy_direction_equivalence}
    Let $\mathcal{S}$ be a finite set of signatures. We have
    \[ \holantprobstar(\mathcal{S}) \fptlinred \holantprob(\mathcal{S}) \,.\]
\end{fact}

Finally, we define the \emph{uncoloured} parameterised Holant problem; in what follows an \emph{uncoloured signature grid} over $\mathcal{S}$ is just a pair $\Omega=(G,\{s_v\}_{v\in V(G)})$ with $s_v\in \mathcal{S}$ for all $v\in V(G)$.
\begin{definition}[$\text{\sc{p-UnColHolant}}(\mathcal{S})$]
    Let $\mathcal{S}$ be a finite set of signatures. The problem $\text{\sc{p-UnColHolant}}(\mathcal{S})$ gets as input a positive integer $k$, and a signature grid $\Omega=(G,\{s_v\}_{v\in V(G)})$ over $\mathcal{S}$. The output is
    \[ \holant(\Omega,k):= \sum_{\substack{A \subseteq E(G)\\|A|=k}}~\prod_{v\in V(G)} s_v(|A \cap E(v)|)\,. \]
    The problem parameter is $k$.
\end{definition}

\begin{remark}
    For finite sets of signatures $\mathcal{S}$, the problems $\textsc{p-Holant}(\mathcal{S})$ and $\textsc{p-UnColHolant}(\mathcal{S})$ can always be reduced to $\#\textsc{Clique}$ w.r.t.\ parameterised Turing-reductions, since we will see that both problems can easily be cast as a linear combination of homomorphism counts, which always reduces to $\#\textsc{Clique}$~\cite{DalmauJ04}. For this reason, we will only prove $\#\W[1]$-hardness when establishing $\#\W[1]$-completeness of our Holant problems.
\end{remark}

\section{Equivalence of $\holantprob$ and $\holantprobstar$}\label{sec:col_holants_equivalences}

In this section, we will prove that, for each finite set of signatures $\mathcal{S}$, the problems $\holantprob(\mathcal{S})$ and $\holantprobstar(\mathcal{S})$ are equivalent w.r.t.\ FPT near-linear-time reductions. 

For the proof, we will first introduce a class of coloured graphs that will be extremely useful in the proof of the aforementioned equivalence. We will only rely on this particular family of coloured graphs in the current section.

\subsection{$(\ell_1,\ell_2)$-Coloured Graphs}

An $(\ell_1,\ell_2)$\emph{-coloured graph} is a triple $(G,\nu,\xi)$ of a graph $G$, a mapping $\nu: V(G) \to S_1$ for a set $S_1$ of size $\ell_1$ (called the $\ell_1$-vertex-colouring), and a mapping $\xi: E(G) \to S_2$ for a set $S_2$ of size $\ell_2$ (called the $\ell_2$-edge-colouring). 

\begin{remark}
   An $H$-coloured graph $(G,h)$ naturally induces an $(\ell_1,\ell_2)$-coloured graph $(G,\nu_h,\xi_h)$ where $\nu_h=h$ and $\xi_h(e)=h(e)$ for all $e\in E(H)$.
\end{remark}


A homomorphism from $(G_1,\nu_1,\xi_1)$ to $(G_2,\nu_2,\xi_2)$ is a homomorphism $h\in \homs{G_1}{G_2}$ such that $\nu_2(h(v))=\nu_1(v)$ for all $v\in V(G_1)$ and $\xi_2(h(e))= \xi_1(e)$ for all $e\in E(G)$. We write $\homs{(G_1,\nu_1,\xi_1)}{(G_2,\nu_2,\xi_2)}$ for the set of all such homomorphisms. Embeddings and isomorphisms between $(\ell_1,\ell_2)$-coloured graphs are defined likewise. We write $(G_1,\nu_1,\xi_1)\cong (G_2,\nu_2,\xi_2)$ if $(G_1,\nu_1,\xi_1)$ and $(G_2,\nu_2,\xi_2)$ are isomorphic.

A subgraph of an an $(\ell_1,\ell_2)$-coloured graph $(G,\nu,\xi)$ is an $(\ell_1,\ell_2)$-coloured graph $(H,\nu|_{V(H)},\xi|_{E(H)})$ where $H$ is a subgraph of~$G$. 
We write $\subs{(H,\nu_H,\xi_H)}{(G,\nu_G,\xi_G)}$ for the set of all subgraphs of $(G,\nu_G,\xi_G)$ that are isomorphic to $(H,\nu_H,\xi_H)$.

We write $\auts(G,\nu,\xi)$ for the set of all automorphisms~$a$ of~$G$ such that $\nu(a(v))=\nu(v)$ and $\xi(a(e))=\xi(e)$ for all $v \in V(G)$ and $e\in E(G)$. The following identity is well-known; we include a proof only for reasons of self-containment:

\begin{proposition}\label{prop:col_subs_to_embs}
    $\#\embs{(H,\nu_H,\xi_H)}{(G,\nu_G,\xi_G))} = \#\auts(H,\nu_H,\xi_H)\cdot \#\subs{(H,\nu_H,\xi_H)}{(G,\nu_G,\xi_G)}$.
\end{proposition}
\begin{proof}\textcolor{red}{TOPROVE 3}\end{proof}

Next we extend the notion of quotient graphs to $(\ell_1,\ell_2)$-coloured graphs; this requires us to restrict to partitions that do not identify vertices or edges with distinct colours.

\begin{definition}[Colour-Consistent Partitions]
    Let $(H,\nu,\xi)$ be an $(\ell_1,\ell_2)$-coloured graph. A partition $\rho \in \ppart(H)$ is called \emph{colour-consistent} (w.r.t.\ $\nu$ and $\xi$) if the following two constraints are satisfied:
    \begin{itemize}
        \item[(I)] If two vertices $u$ and $v$ are in the same block of $\rho$, then $\nu(u)=\nu(v)$.
        \item[(II)] If two edges $e_1$ and $e_2$ of $H$ are mapped to the same edge of $H/\rho$ by $h_\rho$, then $\xi(e_1)=\xi(e_2)$. 
    \end{itemize}
    We write $\ppart(H,\nu,\xi)$ for the set of all colour-consistent partitions of $(H,\nu,\xi)$.
\end{definition}

\begin{definition}[Quotient Graphs of $(\ell_1,\ell_2)$-Coloured Graphs]
    Given an $(\ell_1,\ell_2)$-coloured graph $(H,\nu,\xi)$ and a colour-consistent partition $\rho \in \ppart(H,\nu,\xi)$ we define $(H,\nu,\xi)/\rho := (H/\rho,\nu/\rho,\xi/\rho)$, where $\nu/\rho$ assigns a block the colour of its members, and $\xi/\rho$ assigns an edge $e\in E(H/\rho)$ the colour of the edges of $H$ that are mapped to $e$ by $h_\rho$.
\end{definition}
Observe that the previous construction of quotients for $(\ell_1,\ell_2)$-coloured graphs is well-defined as colour-consistent partitions can only lead to identifications of vertices and edges with the same colours.

Given an $(\ell_1,\ell_2)$-coloured graph $(H,\nu,\xi)$, we consider the poset of colour-consistent partitions $\ppart(H,\nu,\xi)$ with partition refinement. We write $\mu_{(H,\nu,\xi)}$ for the M\"obius function of this poset,\footnote{See e.g.\ \cite[Chapter 3.7]{Stanley11} for the definition of the M\"obius function of a poset. Since we do not need any additional properties of $\mu_{(H,\nu,\xi)}$, we avoid stating the definition in this paper.} and we simplify notation by setting $\mu_{(H,\nu,\xi)}(\rho) := \mu_{(H,\nu,\xi)}(\bot,\rho)$. The proof of the subsequent transformation follows by M\"obius inversion over the poset of colour-consistent partitions and reads almost verbatim as the proof for the uncoloured setting~\cite[Chapter 5.2.3]{Lovasz12}.

\begin{lemma}\label{lem:col_embs_to_homs}
    Let $(H,\nu_H,\xi_H)$ and $(G,\nu_G,\xi_G)$ be  $(\ell_1,\ell_2)$-coloured graphs. We have
    \[ \#\embs{(H,\nu_H,\xi_H)}{(G,\nu_G,\xi_G)} = \!\!\!\!\sum_{\rho \in \ppart(H,\nu_H,\xi_H)}\!\!\!\! \mu(\rho) \cdot \#\homs{(H,\nu_H,\xi_H)/\rho}{(G,\nu_G,\xi_G)} \]
    where $\mu=\mu_{(H,\nu_H,\xi_H)}$. \qed
\end{lemma}

\begin{corollary}\label{cor:col_sub_to_hom}
     Let $(H,\nu_H,\xi_H)$ and $(G,\nu_G,\xi_G)$ be  $(\ell_1,\ell_2)$-coloured graphs. We have
    \[ \#\subs{(H,\nu_H,\xi_H)}{(G,\nu_G,\xi_G)} = \#\auts(H,\nu_H,\xi_H)^{-1}\!\!\!\!\!\!\sum_{\rho \in \ppart(H,\nu_H,\xi_H)}\!\!\!\!\!\! \mu(\rho) \cdot \#\homs{(H,\nu_H,\xi_H)/\rho}{(G,\nu_G,\xi_G)} \]
    where $\mu=\mu_{(H,\nu_H,\xi_H)}$.
\end{corollary}
\begin{proof}\textcolor{red}{TOPROVE 4}\end{proof}

Next, we establish some required algebraic properties of $(\ell_1,\ell_2)$-coloured graphs.
\begin{definition}[$\Gamma(\ell_1,\ell_2,S)$]
    Let $\ell_1,\ell_2$ be positive integers, and let $S$ be a set of size $\ell_1$. We define $\Gamma(\ell_1,\ell_2,S)$ as the set of all isomorphism types of $(\ell_1,\ell_2)$-coloured graphs $(H,\nu,\xi)$ with $\nu: V(H)\to S$ and $\xi: E(H) \to [\ell_2]$. Let also $\Gamma_{\text{\sf{inj}}}(\ell_1, \ell_2, S)$ denote the subset of $\Gamma(\ell_1,\ell_2,S)$ containing those graphs with \emph{injective} edge-colouring $\xi$. 
\end{definition}


\noindent Given $(H,\nu_H,\xi_H),(F,\nu_F,\xi_F) \in \Gamma(\ell_1,\ell_2,S)$, we define their \emph{Tensor product} $(H,\nu_H,\xi_H)\otimes(F,\nu_F,\xi_F)$ as follows:
\begin{itemize}
    \item[(1)] The vertex set is $\{ (u,v) \mid \nu_H(v)=\nu_F(v) \}$,
    and the vertex colouring $\nu_{H\otimes F}$ assigns a vertex $(u,v)$ the colour $\nu_H(u)(=\nu_F(v))$.
    \item[(2)] Two vertices $(u,v)$ and $(u',v')$ are made adjacent if $\{u,u'\}\in E(H)$ and $\{(v,v')\}\in E(F)$, and $\xi_H(\{u,u'\})=\xi_F(\{v,v'\})$. The edge colouring $\xi_{H\otimes F}$ assigns an edge $\{(u,v),(u',v')\}$ the colour $\xi_H(\{u,u'\})(=\xi_F(\{v,v'\}))$.
\end{itemize}

\begin{proposition}\label{prop:semigroup}
     Let $\ell_1,\ell_2$ be positive integers, and let $S$ be a set of size $\ell_1$. Then $(\Gamma(\ell_1,\ell_2,S),\otimes)$ is a semigroup.
\end{proposition}
\begin{proof}\textcolor{red}{TOPROVE 5}\end{proof}


\begin{proposition}\label{prop:linear}
     Let $(F,\nu_F,\xi_F),(G,\nu_G,\xi_G),(H,\nu_H,\xi_H) \in \Gamma(\ell_1,\ell_2,S)$. We have 
     \begin{align*}
         ~&~\#\homs{(F,\nu_F,\xi_F)}{(G,\nu_G,\xi_G)\otimes (H,\nu_H,\xi_H)} \\
         =&~\#\homs{(F,\nu_F,\xi_F)}{(G,\nu_G,\xi_G)} \cdot \#\homs{(F,\nu_F,\xi_F)}{(H,\nu_H,\xi_H)}\,. 
     \end{align*}
\end{proposition}
\begin{proof}\textcolor{red}{TOPROVE 6}\end{proof}

The final ingredient is the following proposition. Its  proof follows the same lines as the classical argument of~\lovasz (see~\cite[Chapter 5.4]{Lovasz12}), but for reasons of self-containment we include a proof.


\begin{proposition}\label{prop:distinct}
    Let $(F,\nu_F,\xi_F),(H,\nu_H,\xi_H) \in \Gamma_{\text{\sf{inj}}}(\ell_1,\ell_2,S)$. If $(F,\nu_F,\xi_F)\ncong (H,\nu_H,\xi_H)$ then there exists $(G,\nu_G,\xi_G)\in \Gamma_{\text{\sf{inj}}}(\ell_1,\ell_2,S)$ such that
    \[ \#\homs{(F,\nu_F,\xi_F)}{(G,\nu_G,\xi_G)} \neq \#\homs{(H,\nu_H,\xi_H)}{(G,\nu_G,\xi_G)} \,.\]
\end{proposition}
\begin{proof}\textcolor{red}{TOPROVE 7}\end{proof}

\subsection{Statement and Proof of the Equivalence}

\begin{lemma}\label{lem:main_equivalence_hard}
    Let $\mathcal{S}$ be a finite set of signatures. We have
    
    \[ \holantprob(\mathcal{S}) \fptinterlinred \holantprobstar(\mathcal{S}) \]
\end{lemma}
\begin{proof}\textcolor{red}{TOPROVE 8}\end{proof}

\section{Classification for $\holantprobstar$}

We start with the following transformation, which can be considered a weighted version of the (first part of the) transformation in~\cite[Lemma 4.1]{PeyerimhoffRSSVW23}, and which follows similar arguments. However, due to various technicalities regarding the vertex signatures, we provide a proof nevertheless. 

\begin{lemma}\label{lem:holant_star_to_cpembs}
Let $H$, $(G,h,\{s_v\}_{v\in V(G)})$ be an instance of $\holantprobstar(\mathcal{S})$ for some finite set of signatures $\mathcal{S}$. Assume that $V(H)=\{v_1,\dots,v_z\}$, and, for each $i\in [z]$, set $n_i$ as the number of vertices of $G$ coloured by $h$ with $v_i$, and let $s_i$ be the signature of the vertices coloured by $h$ with $v_i$. 
    Then
    \[ \holant(G,h,\{s_v\}_{v\in V(G)}) = \prod_{i=1}^z s_i(0)^{n_i} \cdot \sum_{\vec{\sigma}\in \mathcal{F}(H)} \#\embscp(\fracture{H}{\vec{\sigma}} \to (G,h)) \cdot \left(\prod_{i=1}^z \prod_{B \in \vec{\sigma}(v_i)} \frac{s_i(|B|)}{s_i(0)}  \right) \]
\end{lemma}
\begin{proof}\textcolor{red}{TOPROVE 9}\end{proof}


\begin{lemma}\label{lem:holant_star_to_cphoms}
    Let $H$, $(G,h,\{s_v\}_{v\in V(G)})$ be an instance of $\holantprobstar(\mathcal{S})$ for some finite set of signatures $\mathcal{S}$. Assume that $V(H)=\{v_1,\dots,v_z\}$, and, for each $i\in [z]$, set $n_i$ as the number of vertices of $G$ coloured by $h$ with $v_i$, and let $s_i$ be the signature of the vertices coloured by $h$ with $v_i$. 
    Then
    \[ \holant(G,h,\{s_v\}_{v\in V(G)}) = \prod_{i=1}^z s_i(0)^{n_i} \cdot \sum_{\vec{\sigma}\in \mathcal{F}(H)} \sum_{\vec{\rho} \geq \vec{\sigma}} \vec{\mu}(\vec{\sigma},\vec{\rho}) \cdot \#\homscp(\fracture{H}{\vec{\rho}} \to (G,h)) \cdot \left(\prod_{i=1}^z \prod_{B \in \vec{\sigma}(v_i)} \frac{s_i(|B|)}{s_i(0)}  \right) \,.\]
\end{lemma}
\begin{proof}\textcolor{red}{TOPROVE 10}\end{proof}

Next we collect the coefficient for individual fractures in the previous lemma. 

\begin{definition}
   Let $H$ be a graph, and let $\mathcal{S}$ be a finite set of signatures. Assume that $V(H)=\{v_1,\dots,v_z\}$, and let $\vec{s}=(s_1,\dots,s_z)$ be a $z$-tuple of (not necessarily distinct) signatures in $\mathcal{S}$. For a fracture $\vec{\rho}$ of $H$ we define
   \[ \mathsf{coeff}_{H,\vec{s}}(\vec{\rho})= \sum_{\vec{\sigma} \leq \vec{\rho}} \vec{\mu}(\vec{\sigma},\vec{\rho}) \cdot \left(\prod_{i=1}^z \prod_{B \in \vec{\sigma}(v_i)} \frac{s_i(|B|)}{s_i(0)}  \right) \,.\]
   We might drop the subscript $H,\vec{s}$ if it is clear from the context.
\end{definition}

\begin{corollary}\label{cor:collect_coeffs}
    Let $(H,(G,h,\{s_v\}_{v\in V(G)}))$ be an instance of $\holantprobstar(\mathcal{S})$ for some finite set of signatures $\mathcal{S}$. Assume that $V(H)=\{v_1,\dots,v_z\}$, and, for each $i\in [z]$, set $n_i$ as the number of vertices of $G$ coloured by $h$ with $v_i$, and let $s_i$ be the signature of the vertices coloured by $h$ with $v_i$. Moreover, set $\vec{s}=(s_1,\dots,s_z)$.
    Then
    \[ \holant(G,h,\{s_v\}_{v\in V(G)}) \cdot \prod_{i=1}^z s_i(0)^{-n_i} = \sum_{\vec{\rho}\in \mathcal{F}(H)} \mathsf{coeff}_{H,\vec{s}}(\vec{\rho})  \cdot \#\homscp(\fracture{H}{\vec{\rho}} \to (G,h))  \,.\]
\end{corollary}
\begin{proof}\textcolor{red}{TOPROVE 11}\end{proof}

We continue by analysing the coefficients $\mathsf{coeff}(\vec{\rho})$ in detail. For convenience, we first recall and expand the definition of signature fingerprints:

\begin{definition}
    Let $\rho$ be a partition of a finite set, and let $s$ be a signature. We define 
    \[\chi(\rho,s) := \sum_{\sigma \leq \rho} \mu(\sigma,\rho) \cdot \prod_{B \in \sigma} \frac{s(|B|)}{s(0)}\,.\]
    Moreover, given a positive integer $d$, we define the \emph{signature fingerprint} of $d$ and $s$ as follows:
    \[\chi(d,s) := \sum_{\sigma} (-1)^{|\sigma|-1} (|\sigma|-1)! \cdot \prod_{B \in \sigma} \frac{s(|B|)}{s(0)} \,,\]
    where the sum is over all partitions of $[d]$.
\end{definition}

\begin{lemma}\label{lem:just_distributivity}
    Let $\rho$ be a partition of a finite set, let $B_1,\dots,B_t$ be the blocks of $\rho$, and let $s$ be a signature. We have $\chi(\rho,s) =  \prod_{i=1}^t \chi(|B_i|,s)$.
\end{lemma}
\begin{proof}\textcolor{red}{TOPROVE 12}\end{proof}

\begin{lemma}\label{lem:coeffs_done}
    Let $H$ be a graph with vertices $V(H)=\{v_1,\dots,v_z\}$, let $\vec{\rho}$ be a fracture of $H$, and let $\vec{s}=(s_1,\dots,s_z)$ be a $z$-tuple of signatures. We have
    \[\mathsf{coeff}_{H,\vec{s}}(\vec{\rho}) = \prod_{i=1}^z \prod_{B\in \vec{\rho}(v_i)} \chi(|B|,s_i) \,.\]
\end{lemma}
\begin{proof}\textcolor{red}{TOPROVE 13}\end{proof}

Having understood the coefficients of the homomorphism expansion of $\holant(\mathcal{S})$ in terms of the signature fingerprints, we are now able to prove our main classification result.

First of all, we associate each (finite) set of signatures with a class of graphs defined as follows.

\begin{definition}[Homomorphism Supports $\homsupp(\mathcal{S})$ and $\homsupp_\top(\mathcal{S})$]
    Let $\mathcal{S}$ be a finite set of signatures. We say that a graph $F$ is \emph{weakly supported} by $\mathcal{S}$ if there is a graph $H$ with $V(H)=[z]$ (for some $z$), a $z$-tuple $\vec{s}=(s_1,\dots,s_z)$ of signatures in $\mathcal{S}$, and a fracture $\vec{\rho}$ of $H$ such that
    $F \cong \fracture{H}{\vec{\rho}}$ and
        \[\prod_{i=1}^z \prod_{B\in \vec{\rho}(v_i)} \chi(|B|,s_i)\neq 0\,.\]
The \emph{weak homomorphism support} of $\mathcal{S}$, denoted by $\homsupp(\mathcal{S})$, is defined to be the class of all graphs weakly supported by $\mathcal{S}$. 
        
Moreover, we say that a graph $H$ with $V(H)=[z]$  is \emph{strongly supported} by $\mathcal{S}$ if there is a $z$-tuple $\vec{s}=(s_1,\dots,s_z)$ of signatures in $\mathcal{S}$ such that 
    \[  \prod_{i=1}^z \chi(d_i,s_i) \neq 0\,,\]
    where $d_i$ is the degree of the $i$-th vertex. The \emph{strong homomorphism support} of $\mathcal{S}$, denoted by $\homsupp_\top(\mathcal{S})$, is defined to be the class of all graphs strongly supported by $\mathcal{S}$. 
\end{definition}


\begin{lemma}\label{lem:holant_sandwich}
    Let $\mathcal{S}$ be a finite set of signatures. We have
    \[\#\cphomsprob(\homsupp_\top(\mathcal{S})) \fptlinred \holantprobstar(\mathcal{S}) \fptlinred \#\colhomsprob(\homsupp(\mathcal{S})) \,.\]
\end{lemma}
\begin{proof}\textcolor{red}{TOPROVE 14}\end{proof}

The final ingredient for our trichotomy result provides upper and lower bounds on the treewidth of graphs in $\homsupp(\mathcal{S})$ and $\homsupp_\top(\mathcal{S})$.

\begin{lemma}\label{lem:hom_support_characterisation}
    Let $\mathcal{S}$ be a finite set of signatures. 
    \item[(1)] If $\mathcal{S}$ is of type $\mathbb{T}[\mathsf{Lin}]$ then $\homsupp(\mathcal{S})$ only contains acyclic graphs.
    \item[(2)] If $\mathcal{S}$ is of type $\mathbb{T}[\omega]$ then $\homsupp_\top(\mathcal{S})$ contains the triangle, but $\homsupp(\mathcal{S})$ only contains graphs of treewidth at most $2$.
    \item[(3)] If $\mathcal{S}$ is of type $\mathbb{T}[\infty]$ then there exists a constant $d\geq 3$ such that $\homsupp_\top(\mathcal{S})$ contains all $d$-regular graphs
\end{lemma}
\begin{proof}\textcolor{red}{TOPROVE 15}\end{proof}

We are now able to prove~\cref{main_thm}, which we restate for the readers convenience.
\begin{theorem}[\cref{main_thm}, restated]
    Let $\mathcal{S}$ be a finite set of signatures.
    \begin{itemize}
        \item[(I)] If $\mathcal{S}$ is of type $\mathbb{T}[\mathsf{Lin}]$, then $\textsc{p-Holant}(\mathcal{S})$ can be solved in FPT-near-linear time, that is, there is a computable function $f$ such that $\textsc{p-Holant}(\mathcal{S})$ can be solved in time $f(k)\cdot \tilde{\mathcal{O}}(|V(\Omega)|+|E(\Omega)|)$.
        \item[(II)] If $\mathcal{S}$ is of type $\mathbb{T}[\omega]$, then $\textsc{p-Holant}(\mathcal{S})$ can be solved in FPT-matrix-multiplication time, that is, there is a computable function $f$ such that $\textsc{p-Holant}(\mathcal{S})$ can be solved in time $f(k)\cdot \mathcal{O}(|V(\Omega)|^{\omega})$. Moreover, $\textsc{p-Holant}(\mathcal{S})$ cannot be solved in time $f(k)\cdot \tilde{\mathcal{O}}(|V(\Omega)|+|E(\Omega)|)$ for any function $f$, unless the Triangle Conjecture fails.
        \item[(III)] Otherwise, that is, if $\mathcal{S}$ is of type $\mathbb{T}[\infty]$, $\textsc{p-Holant}(\mathcal{S})$ is $\#\W[1]$-complete. Moreover, $\textsc{p-Holant}(\mathcal{S})$ cannot be solved in time $f(k)\cdot |V(\Omega)|^{o(k/\log k)}$ for any function $f$, unless ETH fails. 
    \end{itemize}
\end{theorem}
\begin{proof}\textcolor{red}{TOPROVE 16}\end{proof}

\subsection{Consequences for Modular Counting}\label{sec:modular}
Our analysis of $\holantprob(\mathcal{S})$ applies verbatim to the case of counting modulo a fixed prime $p$, if we restrict ourselves to signatures $s$ with $s(0)\neq 0\mod p$. For the formal statement, we define, for each prime $p$, the problem $\holantprob_p(\mathcal{S})$ to be the version of $\holantprob(\mathcal{S})$ where we output the value of the holant modulo $p$. Likewise, we define the types $\mathbb{T}_p[\mathsf{lin}]$, $\mathbb{T}_p[\omega]$, and $\mathbb{T}_p[\infty]$ by evaluating the fingerprints modulo $p$.

For our lower bounds, we require the parameterised complexity class $\mathsf{Mod}_p\text{-}\W[1]$, which consists of all parameterised counting problems reducible to the problem of counting $k$-cliques modulo $p$, parameterised by $k$ (see~\cite{CurticapeanDH21}). Moreover, we will rely on the \emph{randomised} Exponential Time Hypothesis rETH, which is identical to ETH except for additionally ruling out sub-exponential time \emph{randomised}, bounded-error algorithms for $3\textsc{-SAT}$. We note that some authors already state ETH in a way to account for randomised algorithms~\cite{CurticapeanDH21}; however, to avoid confusion, we emphasise the need for rETH in our result on modular counting. 

\begin{theorem}\label{thm:modular_classification}
     Let $p$ be a prime, and let $\mathcal{S}$ be a finite set of signatures with $s(0)\not\equiv 0 \mod p$ for each $s\in \mathcal{S}$.
    \begin{itemize}
        \item[(I)] If $\mathcal{S}$ is of type $\mathbb{T}_p[\mathsf{Lin}]$, then $\holantprob_p(\mathcal{S})$ can be solved in FPT-near-linear time.
        \item[(II)] If $\mathcal{S}$ is of type $\mathbb{T}_p[\omega]$, then $\holantprob_p(\mathcal{S})$ can be solved in FPT-matrix-multiplication time. Moreover, $\holantprob_p(\mathcal{S})$ cannot be solved in FPT-near-linear time, unless the Triangle Conjecture fails.
        \item[(III)] Otherwise, that is, if $\mathcal{S}$ is of type $\mathbb{T}_p[\infty]$, $\holantprob_p(\mathcal{S})$ is $\mathsf{Mod}_p\text{-}\W[1]$-hard. Moreover, $\holantprob(\mathcal{S})$ cannot be solved in time $f(k)\cdot |V(\Omega)|^{o(k/\log k)}$ for any function $f$, unless rETH fails.  \qed
    \end{itemize}
\end{theorem}
\begin{proof}\textcolor{red}{TOPROVE 17}\end{proof}


We provide the following example application of our trichotomy for modular counting; recall that the problem $\oplus\textsc{ColMatch}$ gets as input a positive integer $k$ and a $k$-edge-coloured graph $G$, and the output is the parity of the number of edge-colourful $k$-matchings in $G$; the parameter is $k$.
\begin{corollary}
    $\oplus\textsc{ColMatch}$ can be solved in FPT-matrix-multiplication time. Moreover, it cannot be solved in FPT-near-linear time, unless the Triangle Conjecture fails.
\end{corollary}
\begin{proof}\textcolor{red}{TOPROVE 18}\end{proof}

\begin{proof}\textcolor{red}{TOPROVE 19}\end{proof}

Finally, as mentioned as example in the abstract, it is easy to see that Theorem~\ref{thm:modular_classification} also implies hardness of counting edge-colourful $k$-matchings modulo $p$, for any prime $p>2$, since the type changes to $\mathbb{T}_p[\infty]$ in that case. We omit stating this example as a theorem since it has already been shown to be hard in previous work, using different methods, by Curticapean, Dell, and Husfeldt~\cite{CurticapeanDH21}.

\section{Extension to Signatures Allowing $s(0) = 0$}\label{sec:sig0}

So far, we have only considered signatures $s$ restricted to $s(0) \neq 0$. In this section, we lift this restriction and establish, similar to \Cref{main_thm}, a trichotomy for $\holantprob(\mathcal{S})$ for any finite set $\mathcal{S}$ of signatures $s$ without requiring $s(0) \neq 0$ (cf. \Cref{def:signatures_intro}).

\subsection{List Homomorphisms}

For our proofs, we need to extend the notion of coloured homomorphisms to list homomorphisms given below. We then show that all algorithmic results for counting coloured homomorphisms, mentioned in \Cref{sec:algo_count_col_homs}, also apply to counting list homomorphisms.

\begin{definition}\label{def:listHoms}
Let $H, G$ be two graphs and let $\mathcal{L} = (L_v)_{v\in V(H)}$ be a collection of sets $L_v \subseteq V(G)$. We set
\[
\homs{H}{G}[\mathcal{L}] = \{\phi \in \homs{H}{G} \mid \forall v \in V(H) : \phi(v) \in L_v\}.
\]
\end{definition}

\begin{definition}
Let $\mathcal{H}$ be a class of graphs. The problem $\#\listhomsprob(\mathcal{H})$ expects as input a graph $H \in \mathcal{H}$, a graph $G$, and a collection $\mathcal{L} = (L_v)_{v\in V(H)}$ of sets $L_v \subseteq V(G)$, and outputs $\#\homs{H}{G}[\mathcal{L}]$.
\end{definition}

As in the case of coloured homomorphisms (see \Cref{fact:colhoms_lintime}), we can interpret the lists $L_v$ as unary predicates and show that computing $\#\homs{H}{G}[\mathcal{L}]$ reduces to counting answers to acyclic conjunctive queries without quantified variables, which can be done in FPT-near-linear time (see e.g.\ \cite[Theorem 7]{BeraGLSS22}).

\begin{lemma}\label{lem:listHomsLinear}
Let $\mathcal{H}$ be a class of acyclic graphs. Then, $\#\listhomsprob(\mathcal{H})$ can be solved in time $f(|H|)\cdot\tilde{\mathcal{O}}(|V(G)| + |E(G)|)$, for some computable function $f$.   \qed 
\end{lemma}

Next, we adapt a result due to Curticapean, Dell and Marx from uncoloured homomorphisms to list homomorphisms to obtain, for $\mathcal{H}$ containing graphs of treewidth at most $2$, an algorithm for $\#\listhomsprob(\mathcal{H})$ via fast matrix multiplication; the proof of the following lemma can be found in \Cref{lem:AppendixListHomsMatrix}.

\begin{lemma}\label{lem:listHomsMatrix}
Let $\mathcal{H}$ be a class of graphs of treewidth at most 2.
Then $\#\listhomsprob(\mathcal{H})$ can be solved in time $f(|H|)\cdot \mathcal{O}(|V(G)|^{\omega})$ for some computable function $f$. Here, $\omega$ is the matrix multiplication exponent. \qed
\end{lemma}

\subsection{The Tractable Cases}
The criterion for the classification of $\holantprob(\mathcal{S})$ is the type of the set of signatures $s \in \mathcal{S}$, for which $s(0) \neq 0$ holds. We first deal with the tractable cases of the classification. We show that these cases reduce to a tractable restriction of the holant problem considered in \Cref{main_thm}, that considers signature grids with a restricted number of vertices with signatures of type $\mathbb{T}[\infty]$ (introducing a parameter that upper-bounds the latter number). To this end, we first prove some algorithmic results, we will need next.



\begin{definition}
Let $G_1, G_2$ be two graphs with respective vertex-colorings $\nu_{G_1} : V(G_1) \to [\ell]$ and $\nu_{G_2} : V(G_2) \to [\ell]$, for some $\ell \in \mathbb{N}$. For a subset $X \subseteq V(G_1)$ and $\phi \in \homs{(G_1[X], \nu_{G_1}|_{X})}{(G_2, \nu_{G_2})}$, we define,
\begin{equation*}
\parthomsX{(G_1, \nu_{G_1})}{(G_2, \nu_{G_2})}{X} = \{h \in \homs{(G_1, \nu_{G_1})}{(G_2, \nu_{G_2})} : h|_{X} = \phi\}\,.    
\end{equation*}
\end{definition}

\begin{lemma}\label{lem:partialHomsFPT}
Let $H, G$ be two graphs with respective vertex colorings $\nu_H: V(H) \to [\ell]$ and $\nu_G : V(G) \to [\ell]$, for some $\ell \in \mathbb{N}$. Further, let $V(H) = X_H \,\dot\cup\, Y_H$ and $V(G) = X_G \,\dot\cup\, Y_G$ such that $\nu_H(X_H) \subseteq \nu_G(X_G)$, $\nu_H(Y_H) \subseteq \nu_G(Y_G)$, and $\nu_G(X_G)\cap\nu_G(Y_G) = \emptyset$, and let $\phi \in \homs{(H[X_H], \nu_H|_{X_H})}{(G[X_G], \nu_G|_{X_G})}$.
\begin{enumerate}
    \item If every vertex $v \in Y_H$ has degree 1 then we can compute $\#\parthomsX{(H, \nu_H)}{(G, \nu_G)}{X_H}$ in $f(|H|)\cdot\tilde{\mathcal{O}}(|V(G)| + |E(G)|)$ time, for some computable function $f$.
    \item If every vertex $v \in Y_H$ has maximum degree at most 2, then we can compute $\#\parthomsX{(H, \nu_H)}{(G, \nu_G)}{X_H}$ in $g(|H|)\cdot\mathcal{O}(|V(G)|^{\omega})$ time, for some computable function $g$.
\end{enumerate}
\end{lemma}
\begin{proof}\textcolor{red}{TOPROVE 20}\end{proof}

As already argued, restricting the holant problem so that the number of vertices with signatures of type $\mathbb{T}[\infty]$ is upper-bounded by some parameter, renders the problem tractable. All of the above are formally stated in the following theorem.

\begin{lemma}\label{lem:restrictedHolant}
Let $\mathcal{S}$ be a finite set of signatures such that for all $s \in \mathcal{S}$, $s(0) \neq 0$. We assume that $\mathcal{S}$ is the disjoint union of a set $\mathcal{S}^e$ which is not of type $\mathbb{T}[\infty]$ and a set $\mathcal{S}^h$ that can be of any type. Let $\holantprob(\mathcal{S}^e;\mathcal{S}^h, r)$ be the restriction of $\holantprob(\mathcal{S})$ on instances where the number of vertices with signatures in $\mathcal{S}^h$ is at most $r$. 
\begin{enumerate} 
\item If $\mathcal{S}^e$ is of type $\mathbb{T}[\mathsf{Lin}]$ then $\holantprob(\mathcal{S}^e;\mathcal{S}^h, r)$ can be solved in FPT-near-linear time, that is, there exists a computable function $f$ such that $\holantprob(\mathcal{S}^e;\mathcal{S}^h,r)$ can be solved in $f(k,r)\cdot\tilde{\mathcal{O}}(|V(\Omega)| + |E(\Omega)|)$ time. 
\item If $\mathcal{S}^e$ is of type $\mathbb{T}[\omega]$, then $\holantprob(\mathcal{S}^e; \mathcal{S}^h, r)$ can be solved in FPT-matrix-multiplication time, that is, there exists a computable function $f$ such that $\holantprob(\mathcal{S}^e;\mathcal{S}^h,r)$ can be solved in $f(k, r)\cdot\mathcal{O}(|V(\Omega)|^{\omega})$ time.
\end{enumerate}
\end{lemma}
\begin{proof}\textcolor{red}{TOPROVE 21}\end{proof}

Now, we are ready to study the tractable cases of the classification.

\begin{lemma}\label{lem:reductionRestrictedHolant}
Let $\mathcal{S}$ be a finite set of signatures. Let $\mathcal{S}_0 = \{s \in \mathcal{S} \mid s(0) = 0\}$. Let $\Omega \in \holantprob(\mathcal{S})$ and set $n_0$ to be the number of vertices in $V(\Omega)$ with signatures in $\mathcal{S}_0$.
\begin{enumerate}
\item If $\mathcal{S}\setminus\mathcal{S}_0$ is of type $\mathbb{T}[\mathsf{Lin}]$, then $\holant(\Omega)$ can be computed in FPT-near-linear time, with respect to the parameters $k, n_0$, that is, there is a computable function $f$ such that $\holant(\Omega)$ can be computed in $f(k,n_0)\cdot\tilde{\mathcal{O}}(|V(\Omega)| + |E(\Omega)|)$ time.
\item If $\mathcal{S}\setminus\mathcal{S}_0$ is of type $\mathbb{T}[\omega]$, then $\holant(\Omega)$ can be computed in FPT-matrix-multiplication time with respect to the parameters $k, n_0$, that is, there is a computable function $f$ such that $\holant(\Omega)$ can be computed in $f(k, n_0)\cdot\mathcal{O}(|V(\Omega)|^{\omega})$ time.
\end{enumerate}
\end{lemma}
\begin{proof}\textcolor{red}{TOPROVE 22}\end{proof}

\subsection{A Trichotomy for $\holantprob(\mathcal{S})$ for all signatures}

With the upper bounds obtained above and the lower bounds implied directly by \Cref{main_thm}, we can fully classify $\holantprob(\mathcal{S})$ for all finite signature sets.

\begin{theorem}\label{thm:holant_trichotomy_zero}
Let $\mathcal{S}$ be a finite set of signatures. Let $\mathcal{S}_0 = \{s \in \mathcal{S} \mid s(0) = 0\}$.
\begin{enumerate}
\item If $\mathcal{S}\setminus\mathcal{S}_0$ is of type $\mathbb{T}[\mathsf{Lin}]$, then $\holantprob(\mathcal{S})$ can be solved in FPT-near-linear time, that is, there is a computable function $f$ such that $\holantprob(\mathcal{S})$ can be solved in $f(k)\cdot\tilde{\mathcal{O}}(|V(\Omega)| + |E(\Omega)|)$ time.
\item If $\mathcal{S}\setminus\mathcal{S}_0$ is of type $\mathbb{T}[\omega]$, then $\holantprob(\mathcal{S})$ can be solved in FPT-matrix-multiplication time, that is, there is a computable function $f$ such that $\holantprob(\mathcal{S})$ can be solved in $f(k)\cdot\mathcal{O}(|V(\Omega)|^{\omega})$ time. Moreover, $\textsc{p-Holant}(\mathcal{S})$ cannot be solved in time $f(k)\cdot \tilde{\mathcal{O}}(|V(\Omega)|+|E(\Omega)|)$ for any function $f$, unless the Triangle Conjecture fails.
\item Otherwise, that is, if $\mathcal{S}\setminus\mathcal{S}_0$ is of type $\mathbb{T}[\infty]$, $\holantprob(\mathcal{S})$ is $\#\mathrm{W}[1]$-complete. Moreover, $\textsc{p-Holant}(\mathcal{S})$ cannot be solved in time $f(k)\cdot |V(\Omega)|^{o(k/\log k)}$ for any function $f$, unless the Exponential Time Hypothesis fails.
\end{enumerate}
\end{theorem}
\begin{proof}\textcolor{red}{TOPROVE 23}\end{proof}

We obtain, as immediate consequence, the classification of the edge-coloured graph factor problem.
\begin{corollary}\label{cor:factor_classification_col}
    If $\mathcal{B}$ contains a set $\{0\} \subsetneq S \subsetneq \mathbb{N}$ then $\textsc{ColFactor}(\mathcal{B})$ is $\#\W[1]$-complete, and cannot be solved in time $f(k)\cdot n^{o(k/\log k)}$ for any function~$f$, unless the Exponential Time Hypothesis fails. Otherwise $\textsc{ColFactor}(\mathcal{B})$ is solvable FPT-near-linear time.
\end{corollary}
\begin{proof}\textcolor{red}{TOPROVE 24}\end{proof}

\section{Parameterised Uncoloured Holants}\label{sec:uncoloured}
We begin by recalling the definition of parameterised uncoloured holant problems; to avoid notational clutter, given a finite set $E$, we define $\binom{E}{k}:=\{A \subseteq E \mid |A| = k\}$.
\begin{definition}
    Let $\mathcal{S}$ be a finite set of signatures.
    An (uncoloured) \emph{signature grid} over $\mathcal{S}$ is a pair of a graph~$G$ and a collection of signatures $\{s_v\}_{v\in V(G)}$ from $\mathcal{S}$. Given a signature grid $\Omega=(G,\{s_v\}_{v\in V(G)})$, and a positive integer $k$, we set
    \[ \mathsf{UnColHolant}(\Omega,k) = \sum_{A \in \binom{E(G)}{k}} \prod_{v\in V(G)} s_v(|A \cap E(v)|) \,.\]
    The problem $\text{\sc{p-UnColHolant}}(\mathcal{S})$ expects as input a positive integer $k$ and a signature grid $\Omega=(G,\{s_v\}_{v\in V(G)})$ over $\mathcal{S}$. The output is $\mathsf{UnColHolant}(\Omega,k)$, and the problem is parameterised by $k$. 
\end{definition}

The goal of this section is to prove our main classification theorem for $\text{\sc{p-UnColHolant}}(\mathcal{S})$:

\begin{theorem}[Theorem~\ref{thm:main_uncol}, restated]\label{thm:main_uncol_restate}
    Let $\mathcal{S}$ be a finite set of signatures. 
    \begin{itemize}
        \item[(I)] If $\mathcal{S}$ is of type $\mathbb{T}[\mathsf{Lin}]$, then $\text{\sc{p-UnColHolant}}(\mathcal{S})$ can be solved in FPT-near-linear time.
        \item[(II)] Otherwise $\text{\sc{p-UnColHolant}}(\mathcal{S})$ is $\#\W[1]$-complete. If, additionally, $\mathcal{S}$ is of type $\mathbb{T}[\infty]$, then $\text{\sc{p-UnColHolant}}(\mathcal{S})$ cannot be solved in time $f(k)\cdot |V(\Omega)|^{o(k/\log k)}$, unless ETH fails. \qed
    \end{itemize}
\end{theorem}

Similarly as in the previous sections, we will translate the uncoloured Holants into a linear combination of homomorphism counts. To this end, given a finite set of signatures $\mathcal{S}$, we denote by $\mathcal{G}(\mathcal S)$ the set of all (isomorphism types of) $\mathcal{S}$-vertex-coloured graphs $(H,\nu)$, where $H$ does not contain isolated vertices, and $\nu:V(H)\to \mathcal{S}$ assigns each vertex of $H$ to a signature in $\mathcal{S}$. Two $\mathcal{S}$-vertex-coloured graphs $(H_1,\nu_1)$ and $(H_2,\nu_2)$ are isomorphic, denoted by $(H_1,\nu_1)\cong(H_2,\nu_2)$ if there is an isomorphism $\iota$ from $H_1$ to $H_2$ that preserves colours, that is, $\nu_2(\iota(v))=\nu_1(v)$ for all $v\in V(H_1)$. Moreover, an automorphism of $(H,\nu)$ is an isomorphism from $(H,\nu)$ to itself, and we denote by $\mathsf{Aut}(H,\nu)$ the set of all automorphisms of $(H,\nu)$. Similarly, a homomorphism from $(H,\nu)$ to $\Omega=(G,\{s_v\}_{v\in V(G)})$ is a homomorphism $\varphi$ from $H$ to $G$ such that $s_{\varphi(v)} = \nu(v)$ for all $v\in V(H)$. We write $\homs{(H,\nu)}{\Omega}$ for the set of all homomorphisms from $\varphi$ to $\Omega$. Embeddings from $(H,\nu)$ to $\Omega=(G,\{s_v\}_{v\in V(G)})$ are defined, likewise, as embeddings from $H$ to $G$ that preserve the signatures of the vertices, and we denote by $\embs{(H,\nu)}{\Omega}$ the set of all embeddings from $(H,\nu)$ to $\Omega$.

We begin by translating embeddings to homomorphisms. To this end, given $(H,\nu)$, a partition $\rho$ of $V(H)$ is called \emph{colour-consistent} if vertices in the same block must have the same colour w.r.t.\ $\nu$, and we denote the set of all colour-consistent partitions of $V(H)$ by $\mathsf{colPart}(H)$. For $\rho \in \mathsf{colPart}(H)$ we set $(H,\nu)/\rho = (H/\rho,\nu/\rho)$, where $(H/\rho)$ is the usual quotient graph, and $\nu/\rho$ assigns a vertex $v_B$ of $H/\rho$ to the signature of the vertices in the block $B$, which is well-defined since $\rho$ is colour-consistent.

We start by translating embeddings to homomorphisms; the proof is standard and deferred to \Cref{sec:app-uncol}.
\begin{lemma}\label{lem:colEmb_to_colHom_S-coloured}
    Let $\mathcal{S}$ be a finite set of signatures, let $(H,\nu)$ be an $\mathcal{S}$-vertex-coloured graph, and let $\Omega=(G,\{s_v\}_{v\in V(G)})$ be a signature grid over $\mathcal{S}$, we have
    \[ \#\embs{(H,\nu)}{\Omega} = \sum_{\rho \in \mathsf{colPart}(H)} \mu(\bot,\rho)\cdot \#\homs{(H,\nu)/\rho}{\Omega}\,,\]
    where $\mu(\bot,\rho)=\prod_{B\in \rho}(-1)^{|B|-1}(|B|-1)!$ is the (usual) M\"obius function of partitions.\qed
\end{lemma}

Now let $\mathcal{G}_k(\mathcal{S})$ denote the set of all (isomorphism types of) $\mathcal{S}$-vertex-coloured graphs $(H,\nu)$, where $H$ has exactly $k$ edges and does not contain isolated vertices, and $\nu:V(H)\to \mathcal{S}$ assigns each vertex of $H$ to a signature in $\mathcal{S}$.
\begin{lemma}\label{lem:uncoloured_hombasis_general}
    Let $\mathcal{S}=\{s_1,\dots,s_\ell\}$ be a finite set of signatures, and let
    $\Omega=(G,\{s_v\}_{v\in V(G)})$ be a signature grid over $\mathcal{S}$. Set $n_i=|\{v\in V(G)\mid s_v=s_i\}|$ for each $i\in [\ell]$, and let $k$ be a positive integer. We have
    \[ \mathsf{UnColHolant}(\Omega,k) = \prod_{i\in \ell}s_i(0)^{n_i} \cdot \sum_{(H,\nu)\in \mathcal{G}(\mathcal{S})} \zeta_{\mathcal{S},k}(H,\nu) \cdot \#\homs{(H,\nu)}{\Omega}\,, \]
    where 
    \[\zeta_{\mathcal{S},k}(H,\nu)= \sum_{(F,\nu_F)\in\mathcal{G}_k(\mathcal{S})} \frac{\prod_{v\in V(F)}{s_v(d_F(v))}/{s_v(0)}}{\#\mathsf{Aut}(F,\nu_F)} \cdot \sum_{\substack{\rho \in \mathsf{colPart}(F,\nu_F)\\(F,\nu_F)/\rho \cong (H,\nu)}}\mu(\bot,\rho)  \,.\]
\end{lemma}
\begin{proof}\textcolor{red}{TOPROVE 25}\end{proof}

Next, our goal is to establish complexity monotonicity for counting homomorphisms from $\mathcal{S}$-vertex-coloured graphs into signature grids over $\mathcal{S}$. The formal statement sufficient for our purposes is provided below and the proof is deferred to \Cref{sec:app-uncol}, since it is an easy consequence of previous works on linear combinations of homomorphism counts.

\begin{lemma}\label{lem:monotonicity_S-coloured}
    Let $\mathcal{S}$ be a finite set of signatures and let $\mathcal{C}$ be the class of all graphs $H$ for which there is a positive integer $k$ and a colouring $\nu:V(H)\to \mathcal{S}$ such that $\zeta_{\mathcal{S},k}(H,\nu)\neq 0$.
    \begin{itemize}
        \item[(1)] If all graphs in $\mathcal{C}$ are acyclic, then  $\text{\sc{p-UnColHolant}}(\mathcal{S})$ can be solved in FPT-near-linear time.
        \item[(2)] If $\mathcal{C}$ has unbounded treewidth, then $\text{\sc{p-UnColHolant}}(\mathcal{S})$ is $\#\W[1]$-complete.\qed
    \end{itemize}
\end{lemma}

The heart, and in fact the most challenging part, of our investigation of the uncoloured holant problem boils down to understanding the coefficient function $\zeta_{\mathcal{S},k}$. We provide the detailed analysis encapsulated in Section~\ref{sec:analysis_of_zeta}. In what follows, we present our key result on $\zeta_{\mathcal{S},k}$ and invoke it in the proof of Theorem~\ref{thm:main_uncol_restate}.

\begin{remark}[On $s(0)=1$]\label{rem:s_zero_equals_1}
Let $\mathcal{S}$ be a finite set of signatures $s$ such that $s(0) \neq 0$. Let $\Omega = (G, \{s_v\}_{v\in V(G)}) \in$ \text{\sc{p-UnColHolant}}$(\mathcal{S})$. Consider the signature grid $\Omega' = (G, \{s'_v\}_{v \in V(G)})$ obtained by replacing every signature $s_v$, $v \in V(G)$ with the signature $s_v' = s_v/s_v(0)$. For any $k$, it can be readily verified that 
\[\textup{\textsf{UnColHolant}}(\Omega', k) = \textup{\textsf{UnColHolant}}(\Omega, k) \cdot \prod_{v \in V(G)}s_v(0)^{-1}.\] 
Furthermore, for any algebraic complex number $q$, the signature sets $\{s\}$ and $\{q\cdot s\}$ have the same type, because the signatures $s$ and $q\cdot s$ have the same fingerprints. Hence, it suffices to classify \text{\sc{p-UnColHolant}}$(\mathcal{S})$ for finite signature sets such that, for each $s \in \mathcal{S}$, $s(0) = 1$.
\end{remark}

\begin{lemma}\label{lem:main_result_on_zeta}
    Let $\mathcal{S}$ be a finite set of signatures and let $\mathcal{C}$ be the class of all graphs $H$ for which there is a positive integer $k$ and a colouring $\nu:V(H)\to \mathcal{S}$ such that $\zeta_{\mathcal{S},k}(H,\nu)\neq 0$. If $\mathcal{S}$ is of type $\mathbb{T}[\mathsf{lin}]$, then all graphs in $\mathcal{C}$ are acyclic. Otherwise $\mathcal{C}$ has unbounded treewidth.\qed
\end{lemma}

Note that Theorem~\ref{thm:main_uncol_restate} almost immediately follows from the combination of Lemma~\ref{lem:monotonicity_S-coloured} and Lemma~\ref{lem:main_result_on_zeta}. The only missing part is the ETH-based lower bound for signatures of type $\mathbb{T}[\infty]$. Fortunately, for this case we can easily reduce from the coloured version:

\begin{lemma}\label{lem:col_to_uncol}
    Let $\mathcal{S}$ be a finite set of signatures. We have 
    \[ \holantprob(\mathcal{S}) \fptlinred \text{\sc{p-UnColHolant}}(\mathcal{S})\,.\]
\end{lemma}
\begin{proof}\textcolor{red}{TOPROVE 26}\end{proof}

\begin{proof}\textcolor{red}{TOPROVE 27}\end{proof}

\subsection{Analysis of $\zeta_{\mathcal{S},k}$ and the Proof of Lemma~\ref{lem:main_result_on_zeta}}\label{sec:analysis_of_zeta}

For reasons of accessibility, we analyze the coefficients $\zeta_{\mathcal{S}, k}(H, \nu_H)$ in multiple fashions. First, we assume that the vertex-colouring of $H$ is given by $\nu_H : V(H) \to \{s\}$, that is, all vertices of $H$ have been assigned the same signature $s$. Under this assumption, we first study the case where $|E(H)| = k$ and then lift the results to the case where $|E(H)| \leq k$. In each case, we show how to adapt the results above to the setting with multiple signatures. 

When we deal with graphs coloured by a single signature, we can simplify notation based on the following observation. 

\begin{remark}\label{rem:simplified_single_signature}
Let $(H, \nu_H) \in \mathcal{G}(\mathcal{S})$, where $\mathcal{S} = \{s\}$, for some signature $s$, such that $s(0) = 1$. We assume that $|E(H)| \leq k$, for some $k > 0$. Let $\mathcal{G}_k$ be the set of all uncoloured simple graphs with no isolated vertices and precisely $k$ edges and set
\[\zeta(H, k):= \sum_{\substack{F\in \mathcal{G}_k}} \frac{\prod_{v\in V(F)}{s(d_F(v))}}{\#\mathsf{Aut}(F)} \cdot \sum_{\substack{\rho \in \mathsf{Part}(F)\\F/\rho \cong H}}\mu(\bot,\rho)  \,.\]
Then, it can be readily verified that $\zeta_{\mathcal{S}, k}(H, \nu_H) = \zeta(H, k)$, which allows us to assume that, in the setting of $\mathcal{S}$ containing only one signature, all graphs we consider are uncoloured and thus simplify the notation.
\end{remark}

\begin{theorem} \label{Thm:special_case}
Let $(H, \nu_H)$ be a vertex-coloured graph without isolated vertices and precisely $k$ edges. Its vertex-colouring is given by $\nu_H : V(H) \to \mathcal{S}$, where $\mathcal{S} = \{s\}$, for some signature $s$ such that $s(0) = 1$. Then we have
\begin{equation}
    \zeta_{\mathcal{S}, k}(H,\nu_H)=\frac{1}{\#\mathsf{Aut}(H, \nu_H)}\prod_{v \in V(H)} \chi(d_H(v), s)
\end{equation}
\end{theorem}


By \Cref{rem:simplified_single_signature}, it suffices to analyze $\zeta(H, k)$ and show
\[\zeta(H, k) = \frac1{\#\mathsf{Aut}(H)}\prod_{v \in V(H)}\chi(d_H(v), s).\]

We introduce one auxiliary notion which is used in the proof:
\begin{definition}
Given graphs $F,H$, a homomorphism $\phi : F \to H$ is called \emph{edge preserving} if it induces a bijection $E(F) \to E(H)$. 
\end{definition}

\begin{proof}\textcolor{red}{TOPROVE 28}\end{proof}

\begin{theorem}\label{thm:k_edges_multiple_signatures}
Let $(H, \nu_H)$ be a vertex-coloured graph with no isolated vertices and precisely $k$ edges. Its vertex colouring is given by $\nu_H : V(G) \to \mathcal{S}$, where $\mathcal{S}$ is a finite set of signatures $s$ such that $s(0) = 1$. Then, we have
\[\zeta_{\mathcal{S}, k}(H, \nu_H) = \frac1{\#\mathsf{Aut}(H, \nu_H)}\prod_{v\in V(H)}\chi(d_H(v), \nu_H(v)).\]
\end{theorem}

\begin{proof}\textcolor{red}{TOPROVE 29}\end{proof}

For the general case, we introduce some additional notation:
\begin{itemize}
    \item A \emph{partition} $\lambda$ of a positive integer $d$ is a decomposition of $d$ into an (unordered) sum of positive integers (e.g. $\lambda=3+2+2+1$ is a partition of $d=8$). Sometimes these are written in exponential notation, e.g.
    \[
    4+3+3+3+2+1+1+1+1+1 = 4^1 3^3 2^1 1^5.
    \]
    \item We write $|\lambda|=d$ for the sum of all parts of the partition and $\mathsf{len}(\lambda)$ for the number of its summands. E.g. we have
    \[
    |3+2+2+1| = 8 \text{ and }\mathsf{len}(3+2+2+1)=4.
    \]
    \item Given two partitions $\lambda_1, \lambda_2$, their \emph{union} $\lambda_1 \cup \lambda_2$ is formed by combining the summands from each of the partitions:
    \[
    (3+2+2+1) \cup (2+1+1+1) = 3 + 2+2+2 + 1 + 1 + 1 + 1.
    \]
    Similarly, given a finite family $(\lambda_i)_{i \in I}$, we write $\bigcup_{i \in I} \lambda_i$ for their union.
    \item We write $\mathcal{P}_d$ for the set of all partitions of $d$ and $\mathcal{P} = \dot\bigcup_{d \geq 1} \mathcal{P}_d$ for the set of all partitions.
    \item Given a set-partition $\rho$ of $[d]$, its \emph{shape} is the partition of $d$ obtained from the size of the blocks $B \in \rho$, e.g.
    \[
    \mathsf{shape}(\{1,4\} \cup \{2,5,6\} \cup \{3\} \cup \{7\}) = 3 + 2 + 1 + 1.
    \]
    \item The set of \emph{automorphisms} of a partition $\lambda$ is the set of permutations of its summands leaving its shape unchanged. Its cardinality can be computed as the product of factorials from its exponential notation:
    \[
    \# \mathsf{Aut}(4^1 3^3 2^1 1^5) = 1! \cdot 3! \cdot 1! \cdot 5!.
    \]
    \item For an integer partition $\lambda$ of $d$, we define a multiplicity associated to $\lambda$ as
    \[
    \mathsf{mult}(\lambda) = \sum_{\substack{\rho \in P(d):\\\mathsf{shape}(\rho)=\lambda}} \mu(\{1\} \cup \{2\} \cup \ldots \cup \{d\}, \rho)\,,
    \]
    where $\mu$ is the Moebius function of the set partition lattice. This can be explicitly calculated as
    \begin{equation}
        \mathsf{mult}(\lambda) = \underbrace{\frac{1}{\# \mathsf{Aut}(\lambda)} \binom{d}{\lambda}}_{=\#\{\rho \in P(d): \mathsf{shape}(\rho)=\lambda\}} \cdot \underbrace{(-1)^{d-\mathsf{len}(\lambda)} \prod_{\lambda_i \in \lambda} (\lambda_i-1)!}_{=\mu(\{1\} \cup \{2\} \cup \ldots \cup \{d\}, \rho)}\,,
    \end{equation}
    where $\binom{d}{\lambda} = d! / \prod_{\lambda_i \in \lambda} \lambda_i!$ is the multinomial coefficient.
    \item We also define a generalization of the function $\chi(-, s)$ from above which takes partitions as arguments. Let $\lambda = \lambda_1 + \ldots + \lambda_\ell$ be an integer partition. Then we define
    \begin{equation} \label{eqn:chi_generalization}
        \chi(\lambda , s) := \sum_{\sigma\in P(\ell)} (-1)^{|\sigma|-1} (|\sigma|-1)! \cdot \prod_{B \in \sigma} s\left(\sum_{i \in B} \lambda_i \right)\,.
    \end{equation}
    One checks that for $\lambda = 1 + 1 + \ldots + 1 = 1^d$ we indeed have $\chi(\lambda, s)=\chi(d,s)$ in the notation above.
\end{itemize}

\begin{theorem}\label{thm:zeta_at_most_k_edges}
Let $H$ be a vertex-coloured graph without isolated vertices and at most $k \geq 0$ edges. Its vertex-colouring is given by $\nu_H : V(H) \to \mathcal{S}$, where $\mathcal{S} = \{s\}$, for some signature $s$ such that $s(0) = 1$. Then we have
\begin{equation} \label{eqn:Theorem_general1}
    \zeta_{\mathcal{S}, k}(H,\nu_H)=\frac{1}{\#\mathsf{Aut}(H, \nu_H)} \cdot \sum_{\substack{\lambda: E(H) \to \mathcal{P}\\\sum_e |\lambda(e)| = k }}\ \  \prod_{e \in E(H)} \frac{\mathsf{mult}(\lambda(e))}{|\lambda(e)|!}  \prod_{v \in V(H)} \chi(\mathsf{deg}(H, v, \lambda), s)\,,
\end{equation}
where the partitions $\mathsf{deg}(H, v, \lambda)$ are defined as
\[
\mathsf{deg}(H, v, \lambda) = \bigcup_{\substack{e \in E(H):\\ e \text{ incident to }v}} \lambda(e)\,. 
\]
\end{theorem}
\begin{proof}\textcolor{red}{TOPROVE 30}\end{proof}

\begin{theorem}\label{thm:general_coefficient}
Let $(H, \nu_H)$ be a vertex-coloured graph with no isolated vertices and at most $k \geq 0$ edges. Its vertex colouring is given by $\nu_H : V(H) \to \mathcal{S}$, where $\mathcal{S}$ is a finite set of signatures $s$ such that $s(0) = 1$. Then we have
\begin{equation} \label{eqn:Theorem_general}
    \zeta_{\mathcal{S},k}(H,\nu_H)=\frac{1}{\#\mathsf{Aut}(H, \nu_H)} \cdot \sum_{\substack{\lambda: E(H) \to \mathcal{P}\\\sum_e |\lambda(e)| = k }}\ \  \prod_{e \in E(H)} \frac{\mathsf{mult}(\lambda(e))}{|\lambda(e)|!}  \prod_{v \in V(H)} \chi(\mathsf{deg}(H, v, \lambda), \nu_H(v))\,,
\end{equation}
where the partitions $\mathsf{deg}(H, v, \lambda)$ are again defined as
\[
\mathsf{deg}(H, v, \lambda) = \bigcup_{\substack{e \in E(H):\\ e \text{ incident to }v}} \lambda(e)\,. 
\]    
\end{theorem}
\begin{proof}\textcolor{red}{TOPROVE 31}\end{proof}


\begin{lemma}\label{lem:linear_type_sigs_equivalence}

For a signature $s$ (with $s(0)=1$), the following are equivalent:
\begin{enumerate}
    \item[a)] $\{s\}$ is of type $\mathbb{T}[\mathsf{lin}]$
    \item[b)] $s(n) = s(1)^n$ for all $n\in \mathbb{N}_{>0}$
    \item[c)] $\chi(\lambda, s)=0$ for all partitions $\lambda$ with $\mathsf{len}(\lambda) \geq 2$
\end{enumerate}
\end{lemma}
\begin{proof}\textcolor{red}{TOPROVE 32}\end{proof}

For the following result, recall the simplification of the coefficient function $\zeta$ for the case of a unique signature $\mathcal{S}=\{s\}$ given in Remark~\ref{rem:simplified_single_signature}.
\begin{theorem} \label{Thm:Conj_2_false}
Let $\{s\}$ be of type $\mathbb{T}[\omega]$ and $d \geq 2$. Then for the complete graph $K_{d+1}$ on $d+1$ vertices, we have
\begin{equation} \label{eqn:zeta_complete_graph}
\zeta(K_{d+1}, d^2-1) = (-1)^{(d+1)(d-2)/2} \cdot A \cdot (s(2)-s(1)^2)^{(d+1)\cdot(d-1)} \text{ for some } A \in \mathbb{Q}_{>0}\,.
\end{equation}
In particular, $\zeta(K_{d+1}, d^2-1) \neq 0$ whenever $\{s\}$ is  of type $\mathbb{T}[\mathsf{\omega}]$. \qed
\end{theorem}

To prove the theorem, we need some more information about the function $\chi$. The first is the following recursive formula.

\begin{lemma} \label{Lem:chi_recursion}
Let $\lambda = a + \lambda_1 + \ldots + \lambda_m$ be a partition of length $m+1$ for $a \geq 2$ and consider a splitting $a  = a_1 + a_2$ with $a_1, a_2 \geq 1$. Then we have
\begin{equation}  \label{eqn:chi_recursion}
\chi(\lambda, s) = \chi(a_1 + a_2 + \lambda_1 + \ldots + \lambda_m, s) + \sum_{S \subseteq [m]} \chi(a_1 \cup \lambda_S, s) \cdot \chi(a_2 \cup \lambda_{S^c}, s)\,,
\end{equation}
where $S^c = [m] \setminus S$ is the complement of $S$ in $[m]=\{1, \ldots, m\}$ and
\[
\lambda_S = \sum_{i \in S} \lambda_i \text{ and } \lambda_{S^c} = \sum_{j \in S^c} \lambda_j\,.
\]
\end{lemma}
\begin{proof}\textcolor{red}{TOPROVE 33}\end{proof}


\begin{proposition} \label{Prop:chi_vanishing_property}
Let $\{s\}$ be of type $\mathbb{T}[\omega]$, then
we have
\begin{enumerate}
    \item[a)] $\chi(\lambda, s) = 0$ if $|\lambda| < 2\cdot  \mathsf{len}(\lambda)-2$,
    \item[b)] $\chi(\lambda, s) = a_\lambda \cdot (s(2)-s(1)^2)^{\mathsf{len}(\lambda)-1}$ for some $a_\lambda \in \mathbb{Z}_{>0}$ if $|\lambda| = 2 \cdot \mathsf{len}(\lambda)-2$.
\end{enumerate}
\end{proposition}
\begin{remark}
Computer experiments suggest that the formula of $a_\lambda$ is
\begin{equation}
a_\lambda = (d-2)! \cdot \prod_{\lambda_i \in \lambda} \lambda_i\,.
\end{equation}
We would be interested in seeing a proof of this, but content ourselves in proving the weaker statement (that $a_\lambda$ is a positive integer) below.
\end{remark}
\begin{proof}\textcolor{red}{TOPROVE 34}\end{proof}



\begin{proof}\textcolor{red}{TOPROVE 35}\end{proof}

We are finally able to prove Lemma~\ref{lem:main_result_on_zeta}, which we restate for convenience.
\begin{lemma}[Lemma~\ref{lem:main_result_on_zeta}, restated]
    Let $\mathcal{S}$ be a finite set of signatures and let $\mathcal{C}$ be the class of all graphs $H$ for which there is a positive integer $k$ and a colouring $\nu:V(H)\to \mathcal{S}$ such that $\zeta_{\mathcal{S},k}(H,\nu)\neq 0$. If $\mathcal{S}$ is of type $\mathbb{T}[\mathsf{lin}]$, then all graphs in $\mathcal{C}$ are acyclic. Otherwise $\mathcal{C}$ has unbounded treewidth.
\end{lemma}
\begin{proof}\textcolor{red}{TOPROVE 36}\end{proof}


\subsection{Extensions to Signatures Allowing $s(0) = 0$}
As in \Cref{sec:sig0}, we can lift the restriction on $s(0) \neq 0$ and establish the following dichotomy.

\begin{theorem}
Let $\mathcal{S}$ be a finite set of signatures. Let $\mathcal{S}_0 = \{s \in \mathcal{S} \mid s(0) = 0\}$. 
\begin{enumerate}
\item If $\mathcal{S}\,\backslash\, \mathcal{S}_0$ is of type $\mathbb{T}[\mathsf{Lin}]$, then $\text{\sc{p-UnColHolant}}(\mathcal{S})$ can be solved in FPT-near-linear time.
\item Otherwise $\text{\sc{p-UnColHolant}}(\mathcal{S})$ is $\#\mathrm{W}[1]$-hard. If, additionally, $\mathcal{S}\setminus \mathcal{S}_0$ is of type $\mathbb{T}[\infty]$, then $\text{\sc{p-UnColHolant}}(\mathcal{S})$ cannot be solved in time $f(k)\cdot |\Omega|^{o(k/\log k)}$, unless ETH fails.
\end{enumerate}
\end{theorem}
\begin{proof}\textcolor{red}{TOPROVE 37}\end{proof}

We obtain, as immediate consequence, the classification of the uncoloured graph factor problem.
\begin{corollary}\label{cor:factor_classification_uncol}
    If $\mathcal{B}$ contains a set $\{0\} \subsetneq S \subsetneq \mathbb{N}$ then $\textsc{Factor}(\mathcal{B})$ is $\#\W[1]$-complete, and cannot be solved in time $f(k)\cdot n^{o(k/\log k)}$ for any function~$f$, unless the Exponential Time Hypothesis fails. Otherwise $\textsc{Factor}(\mathcal{B})$ is solvable in FPT-near-linear time.
\end{corollary}
\begin{proof}\textcolor{red}{TOPROVE 38}\end{proof}

\bibliographystyle{plain}
\bibliography{references}

\newpage

\appendix

\section{Generating Signature Sets for each Type}

\begin{lemma}
    There are infinitely many signature sets of each type $\mathbb{T}[\mathsf{Lin}]$, $\mathbb{T}[\omega]$, and $\mathbb{T}[\infty]$. This remains true even if computation is done modulo a prime $p$, that is, if the fingerprint $\chi$ is evaluated modulo $p$, and $s(0)\neq 0 \mod p$ for all signatures.
\end{lemma}
\begin{proof}\textcolor{red}{TOPROVE 39}\end{proof}

\section{Omitted proofs from \Cref{sec:sig0}}\label{sec:appendix_fastMM}
\subsection{Proof of \Cref{lem:listHomsMatrix}}\label{lem:AppendixListHomsMatrix}
\begin{lemma}
Let $\mathcal{H}$ be a class of graphs of treewidth at most 2.
Then $\#\listhomsprob(\mathcal{H})$ can be solved in time $f(|H|)\cdot \mathcal{O}(|V(G)|^{\omega})$ for some computable function $f$. Here, $\omega$ is the matrix multiplication exponent.
\end{lemma}
\begin{proof}\textcolor{red}{TOPROVE 40}\end{proof}

\section{Omitted proofs from Section~\ref{sec:uncoloured}}\label{sec:app-uncol}
For the proofs of Lemma~\ref{lem:colEmb_to_colHom_S-coloured} and Lemma~\ref{lem:monotonicity_S-coloured}, it will be very convenient to consider $\mathcal{S}$-vertex-coloured graphs and signature grids over $\mathcal{S}$ as relational structures with unary predicates: For this section, fix a finite set of signatures $\mathcal{S}=\{s_1,\dots,s_\ell\}$. 

We define a \emph{vocabulary} $\tau=(E,P_1,\dots,P_\ell)$, where $E$ is a binary relation symbol, and each $P_i$ is a unary relation symbol. A $\tau$-\emph{structure} $\mathcal{H}$ consists of a finite set of vertices $V=V(\mathcal{H})$, a binary relation $E^\mathcal{H}$ over $V$, and unary relations $P_1^\mathcal{H},\dots,P_\ell^{\mathcal{H}}$ over $V$.
A \emph{homomorphism} from a $\tau$-structure $\mathcal{H}$ to a $\tau$-structure $\mathcal{G}$ is a mapping $\varphi: V(\mathcal{H}) \to V(\mathcal{G})$ such that
\begin{itemize}
    \item for all $(u,v)\in E^\mathcal{H}$ we have $(\varphi(u),\varphi(v))\in E^\mathcal{G}$, and
    \item for all $i\in[\ell]$ and for all $v\in P_i^\mathcal{H}$ we have $\varphi(v)\in P_i^\mathcal{G}$.
\end{itemize}

An \emph{embedding} from $\mathcal{H}$ to $\mathcal{G}$ is an injective homomorphism from~$\mathcal{H}$ to~$\mathcal{G}$, and an \emph{isomorphism} from~$\mathcal{H}$ to~$\mathcal{G}$ is a bijection $\iota:V(\mathcal{H})\to V(\mathcal{G})$ such that for all $u,v\in V(\mathcal{H})$ we have $(u,v)\in E^\mathcal{H} \Leftrightarrow (\iota(u),\iota(v))\in E^\mathcal{G}$ and, for each $i\in[\ell]$, $v\in P_i^\mathcal{H}\Leftrightarrow \iota(v)\in P_i^\mathcal{G}$. We write $\mathcal{H}\cong\mathcal{G}$ if an isomorphism exists. Finally, an \emph{automorphism} of $\mathcal{H}$ is an isomorphism from $\mathcal{H}$ to itself. We write $\homs{\mathcal{H}}{\mathcal{G}}$, $\embs{\mathcal{H}}{\mathcal{G}}$, and $\auts(\mathcal{H})$ for the sets of homomorphisms and embeddings from $\mathcal{H}$ to $\mathcal{G}$, and for the set of automorphisms of $\mathcal{H}$, respectively.

Now it is easy to see that both $\mathcal{S}$-vertex-coloured graph $(H,\xi)$ and signature grids $\Omega=(G,\{s_v\}_{v\in V(G)})$ over $\mathcal{S}$ correspond to $\tau$-structures where the binary relation symbol $E$ corresponds to the edges, and a vertex equipped with/coloured by signature $s_i$ is contained in the unary relation $P_i$. Moreover, it is also easy to see that the notions of homomorphisms, embeddings, isomorphisms, and automorphisms are identical when we them as $\tau$-structures.

We are now able to prove Lemma~\ref{lem:colEmb_to_colHom_S-coloured}, which we restate for convenience.
\begin{lemma}[Lemma~\ref{lem:colEmb_to_colHom_S-coloured}, restated]
      Let $\mathcal{S}$ be a finite set of signatures, let $(H,\nu)$ be an $\mathcal{S}$-vertex-coloured graph, and let $\Omega=(G,\{s_v\}_{v\in V(G)})$ be a signature grid over $\mathcal{S}$, we have
    \[ \#\embs{(H,\nu)}{\Omega} = \sum_{\rho \in \mathsf{colPart}(H)} \mu(\bot,\rho)\cdot \#\homs{(H,\nu)/\rho}{\Omega}\,,\]
    where $\mu(\bot,\rho)=\prod_{B\in \rho}(-1)^{|B|-1}(|B|-1)!$ is the (usual) M\"obius function of partitions.
\end{lemma}
\begin{proof}\textcolor{red}{TOPROVE 41}\end{proof}

We continue with proving Lemma~\ref{lem:monotonicity_S-coloured}, which we also restate for convenience.

\begin{lemma}[Lemma \ref{lem:monotonicity_S-coloured}, restated]
    Let $\mathcal{S}$ be a finite set of signatures and let $\mathcal{C}$ be the class of all graphs $H$ for which there is a positive integer $k$ and a colouring $\nu:V(H)\to \mathcal{S}$ such that $\zeta_{\mathcal{S},k}(H,\nu)\neq 0$.
    \begin{itemize}
        \item[(1)] If all graphs in $\mathcal{C}$ are acyclic, then  $\text{\sc{p-UnColHolant}}(\mathcal{S})$ can be solved in FPT-near-linear time.
        \item[(2)] If $\mathcal{C}$ has unbounded treewidth, then $\text{\sc{p-UnColHolant}}(\mathcal{S})$ is $\#\W[1]$-complete.\qed
    \end{itemize}
\end{lemma}
\begin{proof}\textcolor{red}{TOPROVE 42}\end{proof}




\end{document}
