\documentclass[11pt]{article}



\usepackage{amsmath, amssymb, amsthm} 

\usepackage{dsfont}

\usepackage{tikz}



\interfootnotelinepenalty=10000

\allowdisplaybreaks
\expandafter\let\expandafter\oldproof\csname\string\proof\endcsname
\let\oldendproof\endproof
\renewenvironment{proof}[1][\proofname]{\oldproof[\bf #1]}{\oldendproof}
\renewcommand{\qedsymbol}{$\blacksquare$}


\parindent 5mm
\parskip 0.2mm
\oddsidemargin  0pt \evensidemargin 0pt \marginparwidth 0pt
\marginparsep 0pt \topmargin 0pt \headsep 0pt \textheight 8.8in
\textwidth 6.6in
\renewcommand{\baselinestretch}{1.0}

\allowdisplaybreaks

\theoremstyle{plain}
\newtheorem{theorem}{Theorem}[section]
\newtheorem{lemma}[theorem]{Lemma}
\newtheorem{claim}[theorem]{Claim}
\newtheorem{proposition}[theorem]{Proposition}
\newtheorem{observation}[theorem]{Observation}
\newtheorem{corollary}[theorem]{Corollary}
\newtheorem{conjecture}[theorem]{Conjecture}
\newtheorem{property}[theorem]{Property}
\newtheorem{problem}[theorem]{Problem}
\newtheorem{strategy}[theorem]{Strategy}
\newtheorem{construction}[theorem]{Construction}
\newtheorem{condition}[theorem]{Condition}

\theoremstyle{definition}
\newtheorem{remark}[theorem]{Remark}
\newtheorem{definition}[theorem]{Definition}



\newcommand{\Bin}{\ensuremath{\textrm{Bin}}}
\newcommand{\whp}{w.h.p.}
\newcommand{\prob}{probability}
\newcommand{\rn}{random}
\newcommand{\rv}{random variable}
\newcommand{\hpg}{hypergraph}
\newcommand{\hpgs}{hypergraphs}
\newcommand{\subhpg}{subhypergraph}
\newcommand{\subhpgs}{subhypergraphs}
\newcommand{\wt}{\widetilde}
\newcommand{\D}{\mathcal D}
\newcommand{\C}{\mathcal C}
\newcommand{\V}{\mathcal V}
\newcommand{\M}{\mathcal M}
\newcommand{\K}{\mathcal K}
\newcommand{\Q}{\mathcal Q}
\newcommand{\R}{\mathcal R}
\newcommand{\Prop}{\mathcal P}
\newcommand{\Part}{\mathcal P}
\newcommand{\F}{\mathcal F}
\newcommand{\T}{\mathcal T}
\newcommand{\U}{\mathcal U}
\newcommand{\W}{\mathcal W}
\newcommand{\X}{\mathcal X}
\newcommand{\Y}{\mathcal Y}
\newcommand{\eps}{\varepsilon}
\newcommand{\Image}{\text{Im}}
\newcommand{\homleq}{\leq_{\hom}}
\newcommand{\poly}{\text{poly}}
\RequirePackage[normalem]{ulem} \RequirePackage{color}\definecolor{RED}{rgb}{1,0,0}\definecolor{BLUE}{rgb}{0,0,1} \providecommand{\DIFadd}[1]{{\protect\color{blue}\uwave{#1}}} \providecommand{\DIFdel}[1]{{\protect\color{red}\sout{#1}}}                      \providecommand{\DIFaddbegin}{} \providecommand{\DIFaddend}{} \providecommand{\DIFdelbegin}{} \providecommand{\DIFdelend}{} \providecommand{\DIFaddFL}[1]{\DIFadd{#1}} \providecommand{\DIFdelFL}[1]{\DIFdel{#1}} \providecommand{\DIFaddbeginFL}{} \providecommand{\DIFaddendFL}{} \providecommand{\DIFdelbeginFL}{} \providecommand{\DIFdelendFL}{} 

\newcommand{\ex}{\text{ex}}
\newcommand{\Aclosure}{\overline{A}}

\newcommand{\HH}{\hat{H}}
\newcommand{\dist}{\text{dist}}
\newcommand{\pathnum}{\text{spn}}
\newcommand{\girth}{\text{girth}}

\title{\vspace{-0.9cm}Induced subgraphs of $K_r$-free graphs and the Erd\H os--Rogers problem}

\author{
Lior Gishboliner\thanks{Department of Mathematics, University of Toronto, Canada. Email: lior.gishboliner@utoronto.ca}
\and 
Oliver Janzer\thanks{Department of Pure Mathematics and Mathematical Statistics, University of Cambridge, United Kingdom. Research supported by a fellowship at Trinity College. Email: oj224@cam.ac.uk}
\and 
Benny Sudakov\thanks{Department of Mathematics, ETH Z\"urich, Switzerland. Email: benjamin.sudakov@math.ethz.ch. Research supported in part by SNSF grant 200021-228014.}
}

\date{}

\begin{document}

\maketitle

\begin{abstract}
    For two graphs $F,H$ and a positive integer $n$, the function $f_{F,H}(n)$ denotes the largest $m$ such that every $H$-free graph on $n$ vertices contains an $F$-free induced subgraph on $m$ vertices. This function has been extensively studied in the last 60 years when $F$ and $H$ are cliques and became known as the Erd\H os--Rogers function. Recently, Balogh, Chen and Luo, and Mubayi and Verstra\"ete initiated the systematic study of this function in the case where $F$ is a general graph.
    
    Answering, in a strong form, a question of Mubayi and Verstra\"ete, we prove that for every positive integer $r$ and every $K_{r-1}$-free graph $F$, there exists some $\eps_F>0$ such that $f_{F,K_r}(n)=O(n^{1/2-\eps_F})$. This result is tight in two ways. Firstly, it is no longer true if $F$ contains $K_{r-1}$ as a subgraph. Secondly, we show that for all $r\geq 4$ and $\eps>0$, there exists a $K_{r-1}$-free graph $F$ for which $f_{F,K_r}(n)=\Omega(n^{1/2-\eps})$. Along the way of proving this, we show in particular that for every graph $F$ with minimum degree $t$, we have $f_{F,K_4}(n)=\Omega(n^{1/2-6/\sqrt{t}})$. This answers (in a strong form) another question of Mubayi and Verstra\"ete. Finally, we prove that there exist absolute constants $0<c<C$ such that for each $r\geq 4$, if $F$ is a bipartite graph with sufficiently large minimum degree, then $\Omega(n^{\frac{c}{\log r}})\leq f_{F,K_r}(n)\leq O(n^{\frac{C}{\log r}})$. This shows that for graphs $F$ with large minimum degree, the behaviour of $f_{F,K_r}(n)$ is drastically different from that of the corresponding off-diagonal Ramsey number $f_{K_2,K_r}(n)$.
\end{abstract}

\section{Introduction}

The Ramsey number $R(r,t)$ is the smallest $n$ such that every $n$-vertex graph contains a clique of size~$r$ or an independent set of size $t$. The study of this function is one of the most important problems in discrete mathematics. The instances that have  received the most attention are the ``diagonal case'' concerning $r=t$, and the case where $r$ is fixed and $t\rightarrow \infty$ (which is often called the ``off-diagonal case''). In this paper we will focus on the latter.

The first bound on this function was obtained by Erd\H os and Szekeres \cite{ESz35} in 1935, who proved that $R(r,t)=O(t^{r-1})$ for any fixed $r$ and $t\rightarrow \infty$.
Despite extensive research on the topic, the only (non-trivial) off-diagonal Ramsey number whose order of magnitude is known is $R(3,t)$. It was shown by Kim \cite{Kim95} in 1995 that $R(3,t)=\Omega(t^2/\log t)$, which matches an earlier upper bound by Ajtai, Koml\'os and Szemer\'edi \cite{AKSz80}.
Recently, a major breakthrough was obtained by Mattheus and Verstra\"ete \cite{Mattheus_Verstraete}, who proved that $R(4,t)\geq \Omega(t^{3}/(\log t)^4)$, matching the best known upper bound up to a polylogarithmic factor. Nevertheless, the problem of estimating $R(r,t)$ remains wide open for all $r\geq 5$, with the best bounds being $$c_1(r)\frac{t^{\frac{r+1}{2}}}{(\log t)^{\frac{r+1}{2}-\frac{1}{r-2}}}\leq R(r,t)\leq c_2(r) \frac{t^{r-1}}{(\log t)^{r-2}},$$
due to Bohman and Keevash \cite{BK10} and Ajtai, Koml\'os and Szemer\'edi \cite{AKSz80}, respectively.

In 1962, Erd\H os and Rogers \cite{ER62} considered the following generalization of the off-diagonal Ramsey problem. For positive integers $2\leq s<r$ and $n$, let $f_{s,r}(n)$ denote the largest $m$ such that every $K_r$-free graph on $n$ vertices contains a $K_s$-free induced subgraph on $m$ vertices. Note that the Ramsey problem is recovered as the special case $s=2$. The function $f_{s,r}$ has since become known as the Erd\H os--Rogers function and has attracted an extensive amount of research over the last 60 years (see, e.g. \cite{ER62,BH91,Kri94,Kri95,AK97,Sudakov,Sudakov_DRC,DR11,Dudek_Rodl,Wolfovitz,DM14,Dudek_Retter_Rodl,Gowers_Janzer,Janzer_Sudakov,MV_improved_bound}). 

In the last decade or so, there has been major progress towards finding the value of $f_{r-1,r}(n)$. Building on earlier work of Dudek and R\"odl \cite{DR11}, Wolfovitz \cite{Wolfovitz} proved that $f_{3,4}(n)\leq n^{1/2+o(1)}$, matching the easy lower bound $f_{3,4}(n)\geq n^{1/2}$ up to the $o(1)$ term. Later, it was shown by Dudek, Retter and R\"odl \cite{Dudek_Retter_Rodl} that for all $r\geq 4$, we have $f_{r-1,r}(n)\leq n^{1/2+o(1)}$, which is again tight up to the $o(1)$ term. Very recently, the upper bound was improved by Mubayi and Verstra\"ete \cite{MV_improved_bound}, who showed that $f_{r-1,r}(n)= O(n^{1/2}\log n)$, coming close to the best known lower bound $f_{r-1,r}(n)=\Omega\left(\frac{n^{1/2}(\log n)^{1/2}}{(\log \log n)^{1/2}}\right)$, observed in \cite{DM14}.

While the $s=r-1$ case is more or less settled, the next case $s=r-2$ is already open in general. For $r=4$, this problem is equivalent to determining the Ramsey numbers $R(4,k)$, and it follows from the recent breakthrough of Mattheus and Verstra\"ete \cite{Mattheus_Verstraete} that $f_{2,4}(n)\leq n^{1/3+o(1)}$, which is tight. Janzer and Sudakov \cite{Janzer_Sudakov} generalized this upper bound by proving that $f_{r-2,r}(n)\leq n^{\frac{1}{2}-\frac{1}{8r-26}+o(1)}$ holds for all $r\geq 4$. It is unknown whether this is tight for $r>4$; the best lower bound is $f_{r-2,r}(n)\geq n^{\frac{1}{2}-\frac{1}{6r-18}+o(1)}$, due to Sudakov \cite{Sudakov_DRC}.

Recently, Balogh, Chen and Luo \cite{Balogh_Chen_Luo} and Mubayi and Verstra\"ete \cite{MV_general_graphs} initiated a systematic study of the following generalization of the classical Erd\H os--Rogers function (see also \cite{HL24} for an earlier paper in this direction). For graphs $F$ and $H$ and a positive integer $n$, we write $f_{F,H}(n)$ for the largest $m$ such that every $H$-free graph on $n$ vertices contains an $F$-free induced subgraph on $m$ vertices. (Here $H$-free and $F$-free mean that they do not contain $H$ or $F$ as a not necessarily induced subgraph.) Both \cite{Balogh_Chen_Luo} and \cite{MV_general_graphs} are in fact mainly concerned with the case where $H$ is a clique, and this will be the focus of our paper as well. Note that this problem still closely resembles the original Ramsey problem; the only difference is that we are looking for a large $F$-free induced subgraph rather than a large independent set.

Among other results, Mubayi and Verstra\"ete \cite {MV_general_graphs} proved that for every (non-empty) triangle-free graph $F$, we have $f_{F,K_3}(n)=n^{1/2+o(1)}$, thereby resolving the case where $H$ is the triangle. Regarding the next case, namely that of $H=K_4$, they posed the following problem.

\begin{problem}[Mubayi--Verstra\"ete \cite{MV_general_graphs}] \label{problem:trianglefree vs K4}
    Is it true that for every triangle-free graph $F$, there exists $\eps=\eps_F>0$ such that $f_{F,K_4}(n)=O(n^{1/2-\eps})$?
\end{problem}


Our first result is an affirmative answer to Problem \ref{problem:trianglefree vs K4} in a more general form.

\begin{theorem} \label{thm:cliquefree vs clique}
    For every $r\geq 4$ and every $K_{r-1}$-free graph $F$, there exists $\eps=\eps_F>0$ such that $f_{F,K_r}(n)=O(n^{1/2-\eps})$.
\end{theorem}

The assumption that $F$ is $K_{r-1}$-free is necessary since if $F$ contains $K_{r-1}$, then $f_{F,K_r}(n)\geq f_{K_{r-1},K_r}(n)\geq n^{1/2+o(1)}$ for any $r$. Mubayi and Verstra\"ete also conjectured that the $1/2$ in the exponent in Problem \ref{problem:trianglefree vs K4} cannot be replaced by anything smaller, and that this is witnessed by taking $F=K_{t,t}$ for large enough $t$.

\begin{problem}[Mubayi--Verstra\"ete \cite{MV_general_graphs}] \label{problem:Ktt vs K4}
    Prove (or disprove) that for each $\eps>0$ there exists $t$ such that $f_{K_{t,t},K_4}(n)=\Omega(n^{1/2-\eps})$.
\end{problem}

We prove that this is indeed the case in the following more general form.

\begin{theorem} \label{thm:Ktt vs K4}
    For every $t$ and every graph $F$ with minimum degree $t$, $f_{F,K_4}(n)=\Omega(n^{1/2-6/\sqrt{t}})$.
\end{theorem}

In fact, using the same method, we prove that our Theorem \ref{thm:cliquefree vs clique} is tight for all $r$.

\begin{theorem} \label{thm:F vs Kr}
    For every $r\geq 4$ and $\eps>0$ there is a $K_{r-1}$-free graph $F$ such that $f_{F,K_r}(n)=\Omega(n^{1/2-\eps})$.
\end{theorem}

As a corollary of Theorem \ref{thm:Ktt vs K4}, we obtain the following result about graphs with large Tur\'an number.

\begin{corollary} \label{cor:large Turan}
    For every $\eps>0$ there exists some $\delta>0$ such that if a bipartite graph $F$ satisfies $\ex(m,F)=\Omega(m^{2-\delta})$, then $f_{F,K_4}(n)=\Omega(n^{1/2-\eps})$.
\end{corollary}

This shows that for bipartite graphs $F$ with large Tur\'an number, the exponent in $f_{F,K_4}(n)$ is close to $1/2$. This complements a result of Balogh, Chen and Luo \cite{Balogh_Chen_Luo} which states that if $\ex(m,F)=O(m^{1+\alpha})$ for some $\alpha\in [0,1/2)$, then $f_{F,K_4}(n)\leq n^{\frac{1}{3-2\alpha}+o(1)}$.

Motivated by Theorem \ref{thm:Ktt vs K4}, 
it is natural to ask what happens if instead of $K_4$ one considers the case of a general clique $K_r$. Our methods allow us to address this question as well, and we obtain the following rather accurate estimates on $f_{F,K_r}(n)$ when $F$ is a bipartite graph with large minimum degree.

\begin{theorem} \label{thm:Ktt vs Kr lower bound}
    For each $r\geq 4$ and $\varepsilon > 0$ there is $t_0$ such that for every $t \geq t_0$ and every graph $F$ with minimum degree $t$, we have
    $$
    f_{F,K_r}(n) = \Omega(n^{\frac{1}{\lceil \log_2 r \rceil} - \varepsilon}).
    $$
\end{theorem}
\begin{theorem}\label{thm:Ktt vs Kr upper bound}
      There is an absolute constant $C > 0$ such that for every $r$ and every bipartite graph~$F$, we have
    $$f_{F,K_r}(n)= O(n^{\frac{C}{\log r}}).$$
\end{theorem}

From the above two theorems, we see that there are absolute constants $c,C > 0$ such that for every $r \geq 4$ and every bipartite graph $F$ with large enough minimum degree (compared to $r$), we \nolinebreak have
$$\Omega(n^{\frac{c}{\log r}})\leq f_{F,K_r}(n)\leq O(n^{\frac{C}{\log r}}).$$
Note that this is in striking contrast with Ramsey numbers, for which we have $\Omega(n^{\frac{c}{r}})\leq f_{K_2,K_r}(n)\leq O(n^{\frac{C}{r}})$. We also point out that both Theorem \ref{thm:Ktt vs Kr lower bound} and Theorem \ref{thm:Ktt vs Kr upper bound} use methods that are rather novel in the study of Erd\H os--Rogers functions. 

As mentioned above, Mubayi and Verstra\"ete \cite{MV_general_graphs} proved that for all (non-empty) triangle-free graphs $F$, we have $f_{F,K_3}(n)=n^{1/2+o(1)}$. This shows that $f_{F,K_3}(n)$ is quite close to $f_{K_2,K_3}(n)$ for every triangle-free graph $F$. They asked to find an example where the two functions have different orders of magnitudes.

\begin{problem}[Mubayi--Verstra\"ete \cite{MV_general_graphs}] \label{problem:F-free triangle-free}
    Find a triangle-free $F$ for which $f_{F,K_3}(n)/f_{K_2,K_3}(n)\rightarrow \infty$.
\end{problem}

\noindent Note that by the celebrated result of Kim \cite{Kim95} on the Ramsey number $R(3,k)$, we have $f_{K_2,K_3}(n)=\Theta(\sqrt{n \log n})$. Problem \ref{problem:F-free triangle-free} remains open, but in Subsection \ref{sec:F vs triangle} we present a connection to the famous Zarankiewicz problem for $6$-cycles, similar to the connection between Ramsey numbers and the Zarankiewicz problem discussed in \cite{CMMV24}.

\paragraph{Organization of the paper.} In Section \ref{sec:Ktt-free}, we prove Theorems \ref{thm:Ktt vs K4}, \ref{thm:F vs Kr} and \ref{thm:Ktt vs Kr lower bound} and Corollary \ref{cor:large Turan}. In Section \ref{sec:construction}, we prove Theorems \ref{thm:cliquefree vs clique} and \ref{thm:Ktt vs Kr upper bound}. In this section we also discuss the problem of estimating $f_{F,K_3}(n)$ for an arbitrary triangle-free graph $F$, and reveal a connection to the Zarankiewicz problem for $C_6$. In Section \ref{sec:concluding}, we give some concluding remarks.

In Section \ref{sec:Ktt-free}, logarithms are in base $e$, while in Section \ref{sec:construction}, logarithms are in base $2$.

\section{Lower bounds}\label{sec:Ktt-free}
In this section we prove Theorems \ref{thm:Ktt vs K4}, \ref{thm:F vs Kr} and \ref{thm:Ktt vs Kr lower bound} and Corollary \ref{cor:large Turan}.
We denote by $\alpha_F(G)$ the largest order of an $F$-free induced subgraph of $G$.
The {\em $s$-domination number} $\gamma_s(F)$ of a graph $F$ is the minimum $k$ for which there is a set $A \subseteq V(F)$ with $|A| = k$ such that every $v \in V(F) \setminus A$ has at least $s$ neighbours in $A$. We will need the following lemma, showing that graphs of large minimum degree have small $s$-domination number.
\begin{lemma}\label{lem:domination}
Let $t\geq 2$ and let $F$ be a graph with minimum degree $t$. Let $\frac{6\log t}{t} \leq \delta \leq 1$ and set $s = \lfloor \frac{\delta t}{3} \rfloor$. Then $\gamma_s(F) \leq \delta \cdot v(F)$. 
\end{lemma}
\begin{proof}\textcolor{red}{TOPROVE 0}\end{proof}


The following lemma, which we think is of independent interest, is a key for the proof of Theorems \ref{thm:Ktt vs K4}, \ref{thm:F vs Kr} and \ref{thm:Ktt vs Kr lower bound}. Here and below, for $X \subseteq V(G)$, we let $N(X)$ denote the common neighbourhood of $X$. 
\begin{lemma}\label{claim:sampling}
		Let $0<\delta<\beta<1$, let $F$ be a graph, let $s \geq 1$, and
        suppose that $\gamma_s(F) \leq \delta \cdot v(F)$.
		Let $n$ be sufficiently large and let $G$ be an $n$-vertex graph with $\alpha_F(G) < 0.5n^{\beta-2\delta}$. Then there are at least $0.5 n^{(1-\beta+\delta)s}$ sets $X\subseteq V(G)$ of size $s$ with $|N(X)|\geq n^{1-\beta}$. 
\end{lemma}
\begin{proof}\textcolor{red}{TOPROVE 1}\end{proof}

\subsection{Proof of Theorems \ref{thm:Ktt vs K4} and \ref{thm:F vs Kr} and Corollary \ref{cor:large Turan}}

We will derive Theorems \ref{thm:Ktt vs K4} and \ref{thm:F vs Kr} from the following theorem.

\begin{theorem}\label{thm:F vs Kr domination}
    Let $r \geq 4$ and let $F$ be a graph which contains a copy of $K_{r-2}$. Let $\delta > 0$, and suppose that $\gamma_s(F) \leq \delta \cdot v(F)$ for $s = \lceil \frac{1}{\delta} \rceil$. Then $f_{F,K_r}(n) \geq 0.5n^{1/2 - 2\delta}$ holds for all sufficiently large~$n$.
\end{theorem}

\begin{proof}\textcolor{red}{TOPROVE 2}\end{proof}
By combining Theorem \ref{thm:F vs Kr domination} with Lemma \ref{lem:domination}, we get the following. 
\begin{theorem}\label{thm:F vs Kr main}
    Let $r \geq 4$ and let $F$ be a graph with minimum degree $t$ which contains a copy of $K_{r-2}$. Then $f_{F,K_r}(n) \geq 0.5n^{1/2 - 6/\sqrt{t}}$ holds for all sufficiently large $n$.
\end{theorem}
\begin{proof}\textcolor{red}{TOPROVE 3}\end{proof}
Taking $r=4$ in Theorem \ref{thm:F vs Kr main} immediately gives Theorem \ref{thm:Ktt vs K4}. Also, it is easy to see that there exists a graph $F$ which has arbitrarily large minimum degree and contains $K_{r-2}$ but not $K_{r-1}$. For example, we can take the complete $(r-2)$-partite graph with parts of size $t$, where $t$ is sufficiently large. Hence, Theorem~\ref{thm:F vs Kr main} implies Theorem \ref{thm:F vs Kr}.

To deduce Corollary \ref{cor:large Turan}, we will use the following result of Alon, Krivelevich and Sudakov.

\begin{theorem}[Alon--Krivelevich--Sudakov \cite{AKS03}] \label{thm:degeneracy Turan}
    If $F$ is a bipartite graph which does not contain a subgraph of minimum degree at least $t+1$, then $\ex(n,F)=O(n^{2-\frac{1}{4t}})$.
\end{theorem}

\begin{proof}\textcolor{red}{TOPROVE 4}\end{proof}

\subsection{Proof of Theorem \ref{thm:Ktt vs Kr lower bound}}
    We will derive Theorem \ref{thm:Ktt vs Kr lower bound} from the following theorem.
	
    \begin{theorem}\label{thm:clique vs Ktt} 
       Let $F$ be a graph, let $\delta > 0$ and suppose that $\gamma_s(F) \leq \delta \cdot v(F)$ for $s = \lceil \frac{1}{\delta^3}\rceil$.
       Then for every $k \geq 1$, 
       $f_{F,K_{2^k}}(n) \geq n^{\frac{1}{k} - 2^k\delta}$ for all sufficiently large~$n$.
	\end{theorem}
    Before proving Theorem \ref{thm:clique vs Ktt}, let us use it to prove Theorem \ref{thm:Ktt vs Kr lower bound}.
    \begin{proof}\textcolor{red}{TOPROVE 5}\end{proof}
    
    In the rest of this subsection, we prove Theorem \ref{thm:clique vs Ktt}.
	In the following lemma, $d(U,W)$ stands for the proportion of pairs $(u,w)\in U\times W$ for which $uw$ is an edge.
	\begin{lemma}\label{lem:main}
		Let $\delta,\varepsilon,\beta > 0$, Let $s \geq 2(\frac{\beta}{\varepsilon} + 1)$ be an integer, and let $F$ be a graph with $\gamma_s(F) \leq \delta \cdot v(F)$.
		Let $n$ be sufficiently large and let $G$ be an $n$-vertex graph with $\alpha_F(G) < 0.5n^{\beta-2\delta}$. Then there are 
		$U,W \subseteq V(G)$ with $|U| \geq \Omega(n^{1-\beta - \frac{2\beta(\beta+\varepsilon)}{\varepsilon s}})$, $|W| \geq n^{1-\beta}$ and $d(U,W) \geq n^{-\varepsilon}$. 
	\end{lemma} 
	\begin{proof}\textcolor{red}{TOPROVE 6}\end{proof}
	\begin{proof}\textcolor{red}{TOPROVE 7}\end{proof}



 

\section{Upper bound constructions} \label{sec:construction}

\subsection{Proof of Theorem \ref{thm:Ktt vs Kr upper bound}}
It suffices to prove Theorem \ref{thm:Ktt vs Kr upper bound} for $F = K_{t,t}$ (since every bipartite graph is contained in $K_{t,t}$ for a sufficiently large $t$). Hence, Theorem \ref{thm:Ktt vs Kr upper bound} follows from the following result.

 \begin{theorem}\label{thm:Ktt vs Kr construction}
     There is an absolute constant $C$ such that for every $k\geq 2$ and every $t$ we have $f_{K_{t,t},K_{2^k}}(n)=O(n^{C/k})$.
 \end{theorem}

     
	For graphs $G,H$, the {\em lexicographic product} $G \cdot H$ is the graph obtained from $G$ by substituting a copy of $H$ for each vertex of $G$ (and replacing edges of $G$ with complete bipartite graphs). It is easy to see that $\omega(G \cdot H) = \omega(G) \cdot \omega(H)$ and $\chi(G \cdot H) \leq \chi(G) \cdot \chi(H)$.
	\begin{lemma}\label{lem:substitution 2 graphs}
		For any positive integer $t$ and graphs $G$ and $H$, we have 
        $$\alpha_{K_{t,t}}(G \cdot H) \leq \alpha(G)\alpha_{K_{t,t}}(H) + (t-1)\alpha_{K_{t,t}}(G).$$ 
        In particular, $\alpha_{K_{t,t}}(G\cdot G)\leq t\alpha(G)\alpha_{K_{t,t}}(G)$.
	\end{lemma}
	\begin{proof}\textcolor{red}{TOPROVE 8}\end{proof}
	\noindent

    We construct $K_{2^k}$-free graphs with no large $K_{t,t}$-free set by induction on $k$. Roughly speaking, we start with a $K_{2^{k/2}}$-free graph $G_0$ with no large $K_{t,t}$-free set,
     take a union of it with a random graph on the same vertex set to obtain a graph $H$ (where the random graph ensures that $H$ has small independence number), and then consider $H\cdot H$. Then $H$ has no large $K_{t,t}$-free set by Lemma~\ref{lem:substitution 2 graphs}. Unfortunately, $H$ may contain a clique of size significantly greater than $2^{k/2}$, which means that $H\cdot H$ may contain a clique of size significantly greater than $2^k$.

     
	In order to overcome this issue, instead of considering the clique number, we consider the property of having no subgraph on $O(1)$ vertices with large chromatic number. This is more convenient because the chromatic number of the union of two graphs is at most the product of their chromatic numbers, whereas the clique number can be exponential in the clique numbers.

 \begin{definition}
     For an integer $r\geq 3$, let $S_r$ be the set of all $\rho\geq 0$ with the property that for all positive integers $t,s$ there is some $n_0=n_0(\rho,r,t,s)$ such that for all $n\geq n_0$ there exists an $n$-vertex graph $G$ in which every subgraph on $s$ vertices is $(r-1)$-colourable and which has $\alpha_{K_{t,t}}(G)\leq n^{\rho}$.

     Furthermore, let $\rho_r=\textrm{inf}(S_r)$.
 \end{definition}
\noindent 
Note that $1\in S_r$, so $\rho_r$ is well-defined and $\rho_r\leq 1$.

\begin{lemma} \label{lem:product chromatic}
    Let $H$ be a graph (on at least $s$ vertices) in which every subgraph on $s$ vertices is $r$-colourable. Then every subgraph of $H\cdot H$ on $s$ vertices is $r^2$-colourable.
\end{lemma}

\begin{proof}\textcolor{red}{TOPROVE 9}\end{proof}
 
	We also need the following well-known properties of random graphs, which can be easily proved using the union bound.
	\begin{lemma}\label{lem:random graph properties}
		Let $s$ and $r$ be fixed positive integers. Let $p=n^{-2/r}/\log n$. Then $G \sim G(n,p)$ satisfies the following properties.
		\begin{enumerate}
			\item Almost surely $\alpha(G) \leq n^{2/r}(\log n)^3$. 
			\item Almost surely every subgraph of $G$ on at most $s$ vertices has a vertex of degree at most $r-1$. Hence, every such subgraph is $r$-colorable. 
		\end{enumerate}
	\end{lemma}

The following lemma establishes a recursive inequality for the numbers $\rho_r$, which we will then use to prove Theorem \ref{thm:Ktt vs Kr construction}.
\begin{lemma}\label{lem:rho recursion}
	For every $1\leq i\leq k/2$, we have
	$$\rho_{2^k}\leq \frac{1}{2}\rho_{2^i}+2^{i-\lfloor k/2 \rfloor}.$$
\end{lemma}

\begin{proof}\textcolor{red}{TOPROVE 10}\end{proof}

\begin{proof}\textcolor{red}{TOPROVE 11}\end{proof}

\subsection{The proof of Theorem \ref{thm:cliquefree vs clique}}

In this subsection we prove Theorem \ref{thm:cliquefree vs clique} in the following more precise form.

\begin{theorem} \label{thm:cliquefree vs clique precise}
    For every $r\geq 4$ and every $K_{r-1}$-free graph $F$ on $s\geq 2$ vertices, we have $f_{F,K_r}(n)=O(n^{1/2-\frac{1}{8s-10}}(\log n)^3)$.
\end{theorem}

\begin{remark}
    In fact, our construction is such that every set of size roughly $n^{1/2-\frac{1}{8s-10}}(\log n)^3$ contains an \emph{induced} copy of $F$.
\end{remark}

The proof of Theorem \ref{thm:cliquefree vs clique precise} uses the method from \cite{Janzer_Sudakov} (which in turn built on \cite{Mattheus_Verstraete}), where this result was proved in the special case $F=K_s$, $r=s+2$. Similarly to those papers, the following graph provides the starting point in our construction.

\begin{proposition}[\cite{ONan72} or \cite{Mattheus_Verstraete}] \label{prop:algebraic graph}
    For every prime $q$, there is a bipartite graph $K$ with vertex sets $X$ and $Y$ such that the following hold.
    \begin{enumerate}
        \item $|X|=q^4-q^3+q^2$ and $|Y|=q^3+1$.
        \item $d_K(x)=q+1$ for every $x\in X$ and $d_K(y)=q^2$ for every $y\in Y$.
        \item $K$ is $C_4$-free. \label{prop:C4-free}
        \item $K$ does not contain the subdivision of $K_4$ as a subgraph with the part of size $4$ embedded to $X$. \label{prop:subdivisionfree}
    \end{enumerate}
\end{proposition}

Throughout this subsection, let $r\geq 4$ be a fixed positive integer and let $F$ be a fixed $K_{r-1}$-free graph on $s$ vertices. Let us identify the vertex set of $F$ with $[s]$. Let $q$ be a prime and let $K$ be the graph provided by Proposition~\ref{prop:algebraic graph}. We now construct a $K_r$-free graph $H$ on vertex set $X$ randomly as follows. 
For each $y\in Y$, partition $N_K(y)$ uniformly randomly as $A_1(y)\cup A_2(y)\cup \dots \cup A_s(y)$ and place a complete bipartite graph between $A_i(y)$ and $A_j(y)$ whenever $i$ and $j$ are adjacent in $F$. In other words, we place a blow-up of $F$ in $N_K(y)$ with parts $A_1(y),\dots,A_s(y)$. The following lemma, proved in \cite{Janzer_Sudakov}, combined with properties \ref{prop:C4-free} and \ref{prop:subdivisionfree} of Proposition \ref{prop:algebraic graph}, shows that $H$ is $K_r$-free with probability 1.

\begin{lemma}[{\cite[Lemma 2.2]{Janzer_Sudakov}}] \label{lem:clique partition}
    Assume that the edge set of a $K_r$ is partitioned into cliques $C_1,\dots,C_k$ of size at most $r-2$. Then there exist four vertices such that all six edges between them belong to different cliques $C_i$.
\end{lemma}

To see that Lemma \ref{lem:clique partition} implies that $H$ is $K_r$-free, assume that $H$ does contain a copy of $K_r$ on vertex set $R$. Note that by property \ref{prop:C4-free} of Proposition \ref{prop:algebraic graph}, for any edge $uv$ in the complete graph $H[R]$, there is a unique $y\in Y$ such that $u,v\in N_K(y)$. Hence, we can partition the edge set of $H[R]$ into cliques, one with vertex set $N_K(y)\cap R$ for each $y\in Y$ such that $|N_K(y)\cap R|\geq 2$. Moreover, any such clique has size at most $r-2$. (Indeed, $F$ is $K_{r-1}$-free, so if $|N_K(y)\cap R|\geq r-1$, then $N_K(y)\cap R$ must contain distinct vertices $u\in A_i(y)$ and $v\in A_j(y)$ such that $ij\not \in E(F)$ (or $i=j$), meaning that $uv$ is not an edge in $H$.) Hence, by Lemma \ref{lem:clique partition}, there are four vertices in $R$ such that for any two of them there is a different common neighbour in $Y$ in the graph $K$, contradicting property \ref{prop:subdivisionfree} of Proposition~\ref{prop:algebraic graph}.

Our key lemma, proved in Section \ref{sec:Ffree sets}, is as follows. Here and below we ignore floor and ceiling signs whenever they are not crucial.

\begin{lemma} \label{lem:few F-free}
    Let $q$ be a sufficiently large prime and let $t=q^{2-\frac{1}{s-1}}(\log q)^{3}$. Then with positive probability the number of sets $T\subset X$ of size $t$ for which $H[T]$ is $F$-free is at most $(q^{\frac{1}{s-1}})^t$.
\end{lemma}

It is easy to deduce Theorem \ref{thm:cliquefree vs clique precise} from this.

\begin{proof}\textcolor{red}{TOPROVE 12}\end{proof}

\subsubsection{The number of $F$-free sets} \label{sec:Ffree sets}

In this subsection we prove Lemma \ref{lem:few F-free}. While the proof is very similar to that of Lemma 2.3 in \cite{Janzer_Sudakov}, there are some small necessary changes, and we include a full proof for completeness. We will use the following lemma from \cite{Janzer_Sudakov}.

\begin{lemma}[{\cite[Lemma 2.4]{Janzer_Sudakov}}] \label{lem:thereisgoodscale}
    Assume that $q$ is sufficiently large. Then with positive probability, for every $U\subset X$ with $|U|\geq 500s^2q^2$ there exists some $\gamma\geq |U|/q^2$ such that the number of $y\in Y$ with $\gamma/(10s)\leq|A_i(y)\cap U|\leq \gamma$ for all $i\in [s]$ is at least $|U|q/(8(\log q)\gamma)$.
\end{lemma}

\begin{definition}
    Let us call an instance of $H$ \emph{nice} if it satisfies the conclusion of Lemma \ref{lem:thereisgoodscale}.
\end{definition}

Lemma \ref{lem:few F-free} can now be deduced from the following.

\begin{lemma} \label{lem:few F-free if nice}
    Let $q$ be sufficiently large and let $t=q^{2-1/(s-1)}(\log q)^{3}$. If $H$ is nice, then the number of sets $T\subset X$ of size $t$ for which $H[T]$ is $F$-free is at most $(q^{1/(s-1)})^t$.
\end{lemma}

In what follows, we will consider an $s$-uniform hypergraph on vertex set $X$ whose hyperedges correspond to the copies of $F$ in $H$. Then $F$-free subsets of $X$ will correspond to independent sets in this hypergraph, so to prove Lemma \ref{lem:few F-free if nice}, it suffices to bound the number of independent sets of certain size. This will be achieved using the hypergraph container method. For an $s$-uniform hypergraph $\mathcal{G}$ and some $\ell \in [s]$, we write $\Delta_{\ell}(\mathcal{G})$ for the maximum number of hyperedges in $\mathcal{G}$ containing the same set of $\ell$ vertices.

We use the following result from \cite{Janzer_Sudakov}.

\begin{lemma}[{\cite[Corollary 2.8]{Janzer_Sudakov}}] \label{lem:BMScontainer}
    For every positive integer $s\geq 2$ and positive reals $p$ and $\lambda$, the following holds. Suppose that $\mathcal{G}$ is an $s$-uniform hypergraph with at least two vertices such that $pv(\mathcal{G})$ and $v(\mathcal{G})/\lambda$ are integers, and for every $\ell\in [s]$,
    $$\Delta_{\ell}(\mathcal{G})\leq \lambda\cdot p^{\ell-1}\frac{e(\mathcal{G})}{v(\mathcal{G})}.$$

    Then there exists a collection $\mathcal{C}$ of at most $v(\mathcal{G})^{spv(\mathcal{G})}$ sets of size at most $(1-\delta \lambda^{-1})v(\mathcal{G})$ such that for every independent set $I$ in $\mathcal{G}$, there exists some $R\in \mathcal{C}$ with $I\subset R$, where $\delta=2^{-s(s+1)}$.
\end{lemma}

Let $\mathcal{H}$ be the $s$-uniform hypergraph on vertex set $X$ in which $s$ vertices form a hyperedge if they induce a copy of $F$ in $H$. The next lemma shows that if $H$ is nice, then a suitable subgraph of~$\mathcal{H}$ (chosen with the help of Lemma \ref{lem:thereisgoodscale}) satisfies the codegree conditions in Lemma \ref{lem:BMScontainer} with small values of $\lambda$ and $p$.

\begin{lemma} \label{lem:bounded degree}
    Assume that $H$ is nice. Then for each $U\subset X$ of size at least $500s^2q^2$ there exists a subgraph $\mathcal{G}$ of $\mathcal{H}[U]$ (on vertex set $U$) which satisfies
    \begin{equation}
        \Delta_{\ell}(\mathcal{G})\leq \lambda\cdot p^{\ell-1}\frac{e(\mathcal{G})}{v(\mathcal{G})} \label{eqn:bounded codegrees}
    \end{equation}
    for every $\ell\in [s]$ with $\lambda=O_s(\log q)$ and $p\leq |U|^{-1}q^{2-1/(s-1)}$.
\end{lemma}

\begin{proof}\textcolor{red}{TOPROVE 13}\end{proof}

Combining Lemma \ref{lem:BMScontainer} and Lemma \ref{lem:bounded degree}, we prove the following result.

\begin{lemma} \label{lem:container}
    Let $q$ be sufficiently large and assume that $H$ is nice. Let $U$ be a subset of $X$ of size at least $500s^2q^2$. Now there exists a collection $\mathcal{C}$ of at most $(q^4)^{sq^{2-1/(s-1)}}$ sets of size at most $(1-\Omega_s((\log q)^{-1}))|U|$ such that for any $F$-free (in $H$) set $T\subset U$ there exists some $R\in \mathcal{C}$ with $T\subset R$.
\end{lemma}

\begin{proof}\textcolor{red}{TOPROVE 14}\end{proof}

\begin{corollary} \label{cor:few sets}
    Let $q$ be sufficiently large and assume that $H$ is nice. Then there is a collection $\mathcal{C}$ of at most $(q^4)^{O_s(q^{2-1/(s-1)}(\log q)^2)}$ sets of size at most $500s^2q^2$ such that for any $F$-free (in $H$) set $T\subset X$ there exists some $R\in \mathcal{C}$ such that $T\subset R$.
\end{corollary}

\begin{proof}\textcolor{red}{TOPROVE 15}\end{proof}

Corollary \ref{cor:few sets} implies that if $q$ is sufficiently large and $H$ is nice, then the number of $F$-free sets of size $t=q^{2-1/(s-1)}(\log q)^{3}$ in $H$ is at most $$(q^4)^{O_s(q^{2-1/(s-1)}(\log q)^{2})}\binom{500s^2q^2}{t}\leq (q^4)^{O_s(q^{2-1/(s-1)}(\log q)^{2})}(q^{1/(s-1)}/\log q)^t\leq (q^{1/(s-1)})^t,$$
proving Lemma \ref{lem:few F-free if nice}.

\subsection{$F$-free induced subgraphs in triangle-free graphs} \label{sec:F vs triangle}

In this subsection, we observe a connection between Problem \ref{problem:F-free triangle-free} and the Zarankiewicz problem for $6$-cycles.

Let $z(n,m,\{C_4,C_6\})$ denote the maximum number of edges in a bipartite graph with $n+m$ vertices which does not contain $C_4$ or $C_6$ as a subgraph. An old result of de Caen and Sz\'ekely~\cite{DSz97} states that $z(n,m,\{C_4,C_6\})=O(n^{2/3}m^{2/3})$ for $n^{1/2}\leq m\leq n^2$. They observed that there are matching constructions for $m=n$, $m=n^{7/8}$, $m=n^{4/5}$ and $m=n^{1/2}$, but that there is some function $h(n)\rightarrow \infty$ such that $z(n,m,\{C_4,C_6\})=o(n^{2/3}m^{2/3})$ holds for $\omega(n^{1/2})\leq m\leq n^{1/2}h(n)$. We note that $h(n)$ comes from an application of the Ruzsa--Szemer\'edi $(6,3)$-theorem \cite{RSz78} and is of order $e^{\Theta(\log^*(n))}$, where $\log^*(n)$ is the iterated logarithm function.

Roughly speaking, we prove that if $z(n,m,\{C_4,C_6\})=\Theta(n^{2/3}m^{2/3})$ for $m\approx n^{1/2}(\log n)^{3/2}$, then $f_{F,K_3}(n)=\Theta_F(\sqrt{n \log n})$ for every triangle-free graph $F$. Note that this would also give a new proof of $R(3,t)=\Theta(t^2/\log t)$.

\begin{proposition}
	For every triangle-free graph $F$, if $c_F$ is sufficiently large, then the following holds. Let $m=c_F n^{1/2}(\log n)^{3/2}$. Assume that there is a $\{C_4,C_6\}$-free biregular bipartite graph with $n+m$ vertices and $\Omega((nm)^{2/3})$ edges. Then $f_{F,K_3}(n)\leq c_F\sqrt{n \log n}$.
\end{proposition}

\begin{remark}
	The biregularity assumption can be relaxed. Furthermore, any $C_6$-free graph can be made $C_4$-free by discarding at most half of its edges \cite{Gyori97}, so the same conclusion holds assuming the existence of a suitable $C_6$-free graph.
\end{remark}

\begin{proof}\textcolor{red}{TOPROVE 16}\end{proof}

\section{Concluding remarks} \label{sec:concluding}


\subsection{Remark about improving some lower bounds in \cite{MV_general_graphs}}
We outline an argument which is implicit in \cite{Sudakov_DRC} and can be used to improve some of the lower bounds for
$f_{F,K_4}(n)$ proved in \cite{MV_general_graphs}. 
The improvement comes from the fact that the proof in \cite{MV_general_graphs} uses that a $K_4$-free graph with average degree $d$ has independence number at least $\sqrt{d}$; this follows by considering a vertex of degree at least $d$ and using the fact that a triangle-free graph with $m$ vertices has independence number at least $\sqrt{m}$. The following proposition gives a better bound in the relevant range of $d$.
\begin{proposition}[\cite{Sudakov_DRC}]\label{prop:DRC}
    Every $n$-vertex $K_4$-free graph with average degree $d \geq n^{2/3}$ contains an independent set of size $\Omega( \frac{d}{n^{1/3}})$.
\end{proposition}
\noindent
Note that the bound $\frac{d}{n^{1/3}}$ beats the bound $\sqrt{d}$ whenever $d \gg n^{2/3}$. When trying to prove a lower bound of the form $f_{F,K_4}(n) \geq n^{1/3 + \varepsilon}$, one can assume that the average degree $d$ of the host graph $G$ is at most $n^{2/3+2\varepsilon}$ (because otherwise $\alpha(G) \geq \sqrt{d} \geq n^{1/3+\varepsilon}$). This is part of the proof in \cite{MV_general_graphs}. By instead using Proposition \ref{prop:DRC}, one obtains the stronger $d \leq n^{2/3 + \varepsilon}$, which immediately leads to improved bounds (with the rest of the proof in \cite{MV_general_graphs} remaining the same). For example, one can improve the constant $\frac{1}{100}$ in the bound $f_{C_k,K_4}(n) \geq n^{\frac{1}{3} + \frac{1}{100k}}$ proved in \cite{MV_general_graphs}. Since we think that this may be useful in future works on this topic, we decided to include Proposition \ref{prop:DRC} and its proof. The proof uses the dependent random choice method \cite{FoxSudakov}.

\begin{proof}\textcolor{red}{TOPROVE 17}\end{proof}

\subsection{Open problems}
\begin{itemize}
    \item We proved that $n^{1/2 - O(1/\sqrt{t})} \leq f_{K_{t,t},K_4}(n) \leq n^{1/2 - \Omega(1/t)}$, with the lower bound coming from Theorem \ref{thm:Ktt vs K4} and the upper bound from Theorem \ref{thm:cliquefree vs clique precise}. It might be interesting to determine for this problem the correct order of magnitude of the error term in the exponent. 
    \item Another natural question is to estimate $f_{K_{2,t},K_4}(n)$. By an argument along the lines of the proof of Theorem \ref{thm:Ktt vs K4}, using also Proposition \ref{prop:DRC}, one can show that $f_{K_{2,t},K_4}(n) \geq n^{\frac{8}{21}-o_t(1)}$. We believe that it would be interesting to decide whether
$f_{K_{2,t},K_4}(n) \leq O(n^{1/2 - c})$ for some $c > 0$ which is independent of~$t$.
    \item 
    The construction of Mattheus and Verstra\"ete \cite{Mattheus_Verstraete} shows that the bound in Proposition \ref{prop:DRC} is tight (up to polylogarithmic terms) for $d=\Theta(n^{2/3})$. Here an interesting problem is to prove a tight bound for the size of the largest independent set one can guarantee in every $n$-vertex $K_4$-free graph with average degree
    $d = \Theta(n^{\alpha})$ for $\frac{2}{3} < \alpha < 1$. In particular, is there any $\alpha > \frac{2}{3}$ for which the bound given by Proposition \ref{prop:DRC} is tight? 
\end{itemize}

\bibliographystyle{abbrv}
\bibliography{library}

\end{document}
