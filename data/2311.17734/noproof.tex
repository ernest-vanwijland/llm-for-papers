\documentclass[reqno, 11pt]{amsart}
\usepackage[letterpaper, margin=1in]{geometry}
\usepackage{setspace}


\usepackage[utf8]{inputenc} 
\usepackage[T1]{fontenc}

\usepackage[shortlabels]{enumitem}
\usepackage{amsmath}
\usepackage{amsthm} 
\usepackage{amssymb}
\usepackage{float}
\usepackage{todonotes}
\usepackage{xcolor}
\usepackage{graphicx}
\usepackage{tikz}
\usetikzlibrary{decorations.pathreplacing}
\usepackage{bm}
\usepackage{booktabs}
\usepackage{mathtools}

\usepackage[hidelinks,backref=page]{hyperref}


\renewcommand\backrefxxx[3]{\hyperlink{page.#1}{$\uparrow$#1}}





\usepackage[noabbrev,capitalize,nameinlink]{cleveref}
\crefformat{equation}{(#2#1#3)}
\crefmultiformat{equation}{(#2#1#3)}{ and~(#2#1#3)}{, (#2#1#3)}{ and~(#2#1#3)}

\newtheorem{theorem}{Theorem}[section]
\newtheorem{proposition}[theorem]{Proposition}
\newtheorem{lemma}[theorem]{Lemma}
\newtheorem{claim}[theorem]{Claim}
\newtheorem{corollary}[theorem]{Corollary}
\newtheorem{conjecture}[theorem]{Conjecture}
\newtheorem{fact}[theorem]{Fact}

\theoremstyle{definition}
\newtheorem{definition}[theorem]{Definition}
\newtheorem{construction}[theorem]{Construction}
\newtheorem{problem}[theorem]{Problem}
\newtheorem{open}[theorem]{Open Problem}

\newtheorem{question}[theorem]{Question}
\newtheorem{example}[theorem]{Example}
\newtheorem{setup}[theorem]{Setup}
\theoremstyle{remark}
\newtheorem{remark}[theorem]{Remark}


\newcommand{\abs}[1]{\left\lvert#1\right\rvert}
\newcommand{\abss}[1]{\lvert#1\rvert}
\newcommand{\norm}[1]{\left\lVert#1\right\rVert}
\newcommand{\snorm}[1]{\lVert#1\rVert}
\newcommand{\ang}[1]{\left\langle #1 \right\rangle}
\newcommand{\angs}[1]{\langle #1 \rangle}
\newcommand{\floor}[1]{\left\lfloor #1 \right\rfloor}
\newcommand{\ceil}[1]{\left\lceil #1 \right\rceil}
\newcommand{\paren}[1]{\left( #1 \right)}
\newcommand{\sqb}[1]{\left[ #1 \right]}
\newcommand{\set}[1]{\left\{ #1 \right\}}
\newcommand{\setcond}[2]{\left\{ #1 \;\middle\vert\; #2 \right\}}
\newcommand{\cond}[2]{\left( #1 \;\middle\vert\; #2 \right)}
\newcommand{\sqcond}[2]{\left[ #1 \;\middle\vert\; #2 \right]}
\newcommand{\one}{\mathbbm{1}}
\newcommand{\wt}{\widetilde}
\newcommand{\wh}{\widehat}



\newcommand{\CC}{\mathbb{C}}
\newcommand{\EE}{\mathbb{E}}
\newcommand{\FF}{\mathbb{F}}
\newcommand{\RR}{\mathbb{R}}
\newcommand{\PP}{\mathbb{P}}
\newcommand{\NN}{\mathbb{N}}
\newcommand{\ZZ}{\mathbb{Z}}
\newcommand{\QQ}{\mathbb{Q}}
\newcommand{\cC}{\mathcal{C}}


\newcommand{\edit}[1]{{\color{violet}\ttfamily\upshape\small[#1]\color{black}}}
\newcommand{\legendre}[2]{\left(\frac{#1}{#2}\right)}

\title{Uniacute Spherical Codes}
\author[Lepsveridze]{Saba Lepsveridze}
\author[Saatashvili]{Aleksandre Saatashvili}
\author[Zhao]{Yufei Zhao}

\thanks{Lepsveridze and Saatashvili were supported in part by MIT UROP. 
Zhao was supported in part by NSF CAREER award DMS-2044606.}

\address{Lepsveridze, Zhao: Massachusetts Institute of Technology, Cambridge, MA, USA}
\email{\{sabal,yufeiz\}@mit.edu}
\address{Saatashvili: Carnegie Mellon University, Pittsburgh, PA, USA}
\email{asaatash@andrew.cmu.edu}

\begin{document} 

\begin{abstract}
A spherical $L$-code, where $L \subseteq [-1,\infty)$,
consists of unit vectors in $\mathbb{R}^d$ whose pairwise inner products are contained in $L$. 
Determining the maximum cardinality $N_L(d)$ of an $L$-code in $\mathbb{R}^d$ is a fundamental question in discrete geometry and has been extensively investigated for various choices of $L$.
Our understanding in high dimensions is generally quite poor. 
Equiangular lines, corresponding to $L = \{-\alpha, \alpha\}$, is a rare and notable solved case.

Bukh studied an extension of equiangular lines and showed that $N_L(d) = O_L(d)$ for $L = [-1, -\beta] \cup \{\alpha\}$ with $\alpha,\beta > 0$ (we call such $L$-codes ``uniacute''), leaving open the question of determining the leading constant factor.
Balla, Dr\"{a}xler, Keevash, and Sudakov proved a ``uniform bound'' showing $\limsup_{d\to\infty} N_L(d)/d \le 2p$ for $L = [-1, -\beta] \cup \{\alpha\}$ and $p = \lfloor \alpha/\beta \rfloor + 1$.
For which $(\alpha,\beta)$ is this uniform bound tight?
We completely answer this question.

We develop a framework for studying uniacute codes, including a global structure theorem showing that the Gram matrix has an approximate $p$-block structure. 
We also formulate a notion of ``modular codes,'' which we conjecture to be optimal in high dimensions.
\end{abstract}

\maketitle

\section{Introduction}

A \emph{spherical $L$-code}, where $L \subseteq [-1,1)$, consists of unit vectors in $\RR^d$ whose pairwise inner products are contained in $L$.
Let $N_L(d)$ denote the size of the largest spherical $L$-code in $\RR^d$. 
First introduced by Delsarte, Goethals, and Seidel \cite{DGS77}, determining $N_L(d)$ is a central problem in discrete geometry. 

The classical spherical code problem of determining the maximum number of points on a sphere pairwise separated by angle at least $\theta$ is equivalent to $N_L(d)$ with $L = [-1, \cos\theta]$. When $0 < \theta < \pi/2$, we know that $N_L(d)$ grows exponentially in $d$, although there is an exponential gap between the best upper and lower bounds~\cite{KL78,JJP18}.

Another notable case is equiangular lines, which are lines through the origin with pairwise equal angles. For a given common angle $\theta$, this problem corresponds to $L = \{\pm \cos \theta\}$. The problem of determining $N_{\set{\pm \alpha}}(d)$ for a fixed $\alpha \in (0,1)$ and large $d$ was initiated by Lemmens and Seidel \cite{LS73} in 1973, which led to substantial work~\cite{Neu89,Buk16,BDKS18,JP20} that culminated in a recent solution by Jiang, Tidor, Yao, Zhang, and Zhao \cite{JTYZZ21}. 

Bukh~\cite{Buk16} proved that $N_L(d) = O_L(d)$ when $L = [-1, -\beta] \cup \{\alpha\}$ with $\alpha, \beta \in (0,1)$. Here the subscript in $O_L(\cdot)$ means that the hidden constant is allowed to depend on $L$.
Intuitively, Bukh's result says that the allowed acute angles matter more than the allowed obtuse angles.
When $L \subseteq [-1, -\beta] \cup \{\alpha\}$ with $\alpha,\beta >0$, we call spherical $L$-codes \emph{uniacute spherical codes}. Here the set of allowed angles contains exactly one acute angle.
Bukh's result prompts the determination of the growth rate of $N_L(d)$.

\begin{problem}[Rate of uniacute spherical codes] \label{prob:rate}
Given $L \subseteq [-1, -\beta] \cup \{\alpha\}$ with $\alpha, \beta \in (0,1)$, determine
$\lim_{d\to\infty} N_L(d)/d$.
\end{problem}

Bukh showed that $\limsup_{d\to\infty} N_L(d)/d \le 2^{O(\beta^{-2})}$.
Balla, Dr\"{a}xler,  Keevash, and Sudakov \cite{BDKS18} gave the following substantial improvement.
We call it the ``uniform bound.''

\begin{theorem}[Uniform bound~\cite{BDKS18}] \label{thm:uniform-lim}
Let $L = [-1, -\beta] \cup \{\alpha\}$ with $\alpha,\beta \in (0,1)$. Let
$p = \floor{\alpha/\beta} + 1$.
Then 
\[
\limsup_{d \to \infty} \frac{N_L(d)}{d} \le 2p.
\]
\end{theorem}

The parameter $p$ plays an important role in the problem, as we will see later in the global structure theorem. Also, as we will show, for each $p$, there is some $(\alpha,\beta)$ with $p = \floor{\alpha/\beta} + 1$ so that the above uniform bound is tight. 

A notable result from \cite{BDKS18} is an analogous uniform bound for equiangular lines:
\[
\limsup_{d \to \infty} \frac{N_{\set{\pm\alpha}}(d)}{d} \le 2,
\]
and equality occurs if and only if $\alpha = 1/3$.
The precise limit was later determined for all $\alpha$ in \cite{JTYZZ21}.

\cref{thm:uniform-lim} leads to the following natural question.

\begin{question}
    For which $(\alpha,\beta)$ is the uniform bound in \cref{thm:uniform-lim} tight?
\end{question}

We give a complete answer to this question. The main result is stated below.
More refined bounds will be given later in \cref{thm:detail}.

\begin{theorem} \label{thm:main}
Let $L = [-1, -\beta] \cup \{\alpha\}$ with $\alpha,\beta \in (0,1)$. 
Let 
\[
p = \floor{\frac{\alpha}{\beta}} + 1
\qquad 
\text{and} \qquad
\lambda = \frac{1-\alpha}{\alpha +\beta}.
\]
If any of the following holds:
\begin{enumerate}[(a)]
    \item $\alpha = \beta$ and $\lambda^2 \in \ZZ$,
    \item $\alpha/\beta \in \ZZ$, $\alpha \ge 2\beta$, and $\lambda \in \ZZ$,
    \item $\alpha/\beta \notin \ZZ$ and $\lambda \ge 1$,
\end{enumerate}
then 
\[
    N_L(d) = 2pd + O_L(1),
\]
and otherwise, there is some absolute constant $c > 0$ such that
\[
\limsup_{d \to \infty} \frac{N_L(d)}{d} \le 2p-c.
\]
\end{theorem}



\subsection{Methods}
Our work builds on methods for equiangular lines, where the main result is recalled below.

\begin{theorem}[Equiangular lines with a fixed angle~\cite{JTYZZ21}] \label{thm:equiangular}
    Let $ \alpha \in (0,1)$ and $\lambda = (1-\alpha)/\alpha$. Let $k = k(\lambda)$ denote the smallest $k$ such that there exists a $k$-vertex graph whose adjacency matrix has top eigenvalue exactly $\lambda$; set $k = \infty$ is none exists. 
    \begin{enumerate}[(a)]
        \item \label{itm:equiangular-a} If $k < \infty$, then $N_{\set{\pm \alpha}} (d) = \floor{k(d-1)/(k-1)}$ for all $d > d_0(\alpha)$ for some $d_0(\alpha)$;
        \item \label{itm:equiangular-b} If $k = \infty$, then $N_{\set{\pm\alpha}}(d) = d + o(d)$.
    \end{enumerate}
\end{theorem}

A natural generalization of equiangular lines is spherical two-distance sets, namely spherical $L$-codes for $L = \set{-\beta,\alpha}$ with $\alpha,\beta > 0$. 
The limiting value of $N_L(d)/d$ was conjectured in \cite{JTYZZ23} and proved whenever $\alpha < 2\beta$ or $(1-\alpha)/(\alpha+\beta) < \gamma^{1/2} + \gamma^{-1/2} \approx 2.02$ where $\gamma$ is the unique real solution to $\gamma^3 = \gamma + 1$ \cite{JP21+,JTYZZ23}.
In this paper, we propose an extension of the conjectured limit from spherical two-distance sets to more general uniacute codes. To motivate and state the conjecture, we need to first discuss the solution framework.

Given a spherical code $C$, define its \emph{Gram graph} $G_C$ to be the edge-weighted complete graph on vertex set $C$ and such that edge $uv$ has weight $\ang{u,v}$.
Let $G_C^-$ be the \emph{negative graph}, which is the simple graph on consisting of negative edges of $G_C$ (and then forgetting the edge weights).
Likewise let $G_C^+$ be the \emph{positive graph}, a simple graph consisting of positive edges of $G_C$.

First, we show that $G_C$ is close to a $p$-block ``template,'' a notion that we will make precise later on. See \cref{fig:template} for an illustration of the template.
This idea originates from \cite{BDKS18} and was further developed in \cite{JTYZZ21,JTYZZ23}. 
A refined statement is given below and proved in \cref{sec:global}.


\begin{figure}[h]
\definecolor{wwccff}{rgb}{0.4,0.8,1}
\begin{tikzpicture}[scale = 0.4]
\fill[fill=wwccff!30] (-3.9993347497973892,-0.07294901687515672) -- (-1.9364916731037085,3.5) -- (0,2) -- (-1.7320508075688779,-1) -- cycle; 
\node at (-2, 1) {\large$\gamma_{12}$};

\fill[fill=wwccff!30] (-1.7320508075688779,-1) -- (-2.0628430766936816,-3.427050983124842) -- (2.06284307669368,-3.427050983124843) -- (1.7320508075688772,-1) -- cycle;
\node at (0, -2.25) {\large$\gamma_{13}$};

\fill[fill=wwccff!30] (1.7320508075688772,-1) -- (3.9993347497973892,-0.07294901687515976) -- (1.9364916731037083,3.5) -- (0,2) -- cycle;
\node at (2, 1) {\large$\gamma_{23}$};

\draw [draw=none, fill=wwccff] (0, 4) circle (2cm) node {\large$\alpha$};
\node[above] at (0, 6) {$V_2$};
\draw [draw=none, fill=wwccff] (-3.464101615137755, -2) circle (2cm) node {\large$\alpha$};
\node[left] at (-5.4641016151377535, -2) {$V_1$};
\draw [draw=none, fill=wwccff] (3.4641016151377535, -2) circle (2cm) node {\large$\alpha$};
\node[right] at (5.4641016151377535, -2) {$V_3$};
\end{tikzpicture}

\caption{An example of a template with $p$ vertex parts $V_1, \dots, V_p$. Edges within each $V_i$ have weight $\alpha$. Edges between $V_i$ and $V_j$ have weight $\gamma_{ij} \in L \setminus \{\alpha\}$ for each $i \ne j$.} \label{fig:template}
\end{figure}


\begin{definition}
    An edge-weighted graph $H$ is a \textit{$\Delta$-modification} of another edge-weighted graph $G$ on the same vertex set if one can obtain $H$ from $G$ by changing edges and weights on a graph of maximum degree at most $\Delta$.
\end{definition}


\begin{theorem}[Global structure theorem]\label{thm:structure}
Fix $ \alpha, \beta \in (0,1)$ and let $L \subseteq [-1,-\beta]\cup\{\alpha\}$ and $p = \lfloor \alpha/ \beta \rfloor + 1$. There exist constants $\Delta_0$ and $K$ (both depending only on $\alpha$ and $\beta$) such for any spherical $L$-code $C$ and any integer $\Delta \geq \Delta_0$, we can remove at most $\Delta^2$ vectors from $C$ so that the Gram graph of remaining vectors is a $\Delta$-modification of a complete edge-weighted graph of the following form:
\begin{enumerate}[(a)]
	\item the vertex set can be partitioned as $V_1 \sqcup \cdots \sqcup V_p$ (some parts could be empty);
	\item for $1 \leq i \leq p$, all edges in $V_i$ have weight $\alpha$;
	\item for $1 \leq i  < j \leq p$, there is some $\gamma_{ij} \in L \setminus \{\alpha\}$ such that every edge between $V_i$ and $V_j$ has weight within $K \Delta^{-1/2}$ of $\gamma_{ij}$. \end{enumerate}
\end{theorem}

Informally, the global structure theorem says that after removing $O_L(1)$ vertices, and changing edge-weights on a subgraph of maximum degree $O_L(1)$, the resulting edge weights are very close to a $p$-block structure. \cref{thm:structure} is proved in \cref{sec:global}.

\begin{remark}[Necessity of vertex removal]
In \cref{thm:structure}, the removal of $O_L(1)$ vectors is necessary.
This is shown in the proof of \cref{thm:detail}\ref{itm:detail-non}.
In contrast, for equiangular lines, this step is unnecessary. (See \cite[Theorem 2.1]{JTYZZ21}; also see \cite{Bal21+} for quantitative improvements specifically in the context of equiangular lines. It should be noted that the improvements in \cite{Bal21+} do not seem to apply to spherical two-distance sets or more general uniacute spherical codes.)
\end{remark}

After applying the global structure theorem, the heart of the problem then lies in analyzing the bounded degree subgraph where weights were modified. 
For equiangular lines, to prove \cref{thm:equiangular}, we use that a connected bounded degree graph has sublinear eigenvalue multiplicity~\cite{JTYZZ21}.
In this paper, we will use a slight extension of the eigenvalue multiplicity result to edge-weighted graphs. The version we need is stated below. We omit the proof as it is identical to the proof of \cite[Theorem 2.1]{JTYZZ21}.

Given an $n$-vertex edge-weighted graph $G$, we define its weighted adjacency matrix $A_G$ as a matrix whose $(i,j)$-entry is the edge-weight on edge $ij$, and zero if $ij$ is not an edge. Label its eigenvalues by $\lambda_1(A_G) \ge \cdots \ge \lambda_n(A_G)$, and we refer to $\lambda_j(A_G)$ as the \emph{$j$th eigenvalue of $G$}.


\begin{theorem}[Sublinear eigenvalue multiplicity~\cite{JTYZZ21}]\label{thm:eigen-mult} 
Given a connected edge-weighted graph on $n$ vertices with maximum degree $\Delta$ and whose edge weights lie in the interval $[1, 1+\nu]$, its $j$th eigenvalue has multiplicity at most $O_{\Delta,j,\nu}(n/\log\log n)$.
\end{theorem}

\begin{remark}[Quantitatve bounds and error terms]
The $n/\log\log n$ bound in \cref{thm:eigen-mult} is the main source of the bottleneck in the quantitative bounds in \cref{thm:equiangular}. Specifically, the proof (together with later refinements by Balla~\cite{Bal21+}) gives $d_0(\alpha) = \exp(\alpha^{-Ck})$ in \cref{thm:equiangular}\ref{itm:equiangular-a} and $O(\log(1/\alpha) d/\log\log d)$ as the $o(d)$ error term in \cref{thm:equiangular}\ref{itm:equiangular-b}. See \cite{HSSZ22} lower bound constructions of connected bounded degree graphs with second eigenvalue multiplicity on the order of $\sqrt{n/\log n}$). Also, for \cref{thm:equiangular}\ref{itm:equiangular-b}, Schildkraut~\cite{Sch23+} exhibited infinitely many $\alpha$'s where the $o(d)$ error term cannot be replaced by $o(\log\log d)$.
\end{remark}


Prior to the discovery of \cref{thm:eigen-mult}, partial progress on the equiangular lines problem relied on ad hoc spectral graph theory arguments \cite{BDKS18,JP20}. Despite the success in addressing the equiangular lines problem, a significant challenge arises when attempting to extend the solution to other uniacute spherical code problems. 
A missing ingredient appears to be the lack of a useful generalization of \cref{thm:eigen-mult} to the types of graphs that arise in \cref{thm:structure}.
Indeed, for spherical two-distance sets \cite{JTYZZ23,JP21+}, various workarounds have been used to address certain cases of the problem.
The papers \cite{JP20,JP21+} solved certain forbidden subgraph characterization problems in graphs and signed graphs, which led to partial solutions of equiangular lines and spherical two-distance sets problems.

To show our main result, \cref{thm:main}, after applying the global structure theorem to decompose the Gram graph into $p$ parts, we deduce that unless the modification consists of a near perfect matching inside each of the $p$ parts, the size of the code must be much less than $2pd$ (\cref{sec:rank-bound}).
Once we deduce the matching structure within each part, we then analyze the modifications between parts (\cref{sec:upper}).


\begin{figure}[h]
\definecolor{wwccff}{rgb}{0.4,0.8,1}
\begin{tikzpicture}[scale = 0.8]
\fill[line width=0pt,color=wwccff,fill=wwccff!30] (-2.005278955937325,1.8055364510038712) -- (1,1.8) -- (1,-2.3) -- (-2,-2.3) -- cycle;

\fill[line width=1pt,color=wwccff,fill=wwccff] (-1,2) -- (-2,2) -- (-2,-2.5) -- (-1,-2.5) -- cycle;
\fill[line width=1pt,color=wwccff,fill=wwccff] (1,2) -- (0,2) -- (0,-2.5) -- (1,-2.5) -- cycle;



\draw [line width=1pt] (-1.5,1.5)-- (0.5,1);
\draw [line width=1pt] (0.5,1.5)-- (-1.5,1);
\draw [line width=1pt] (-1.5,1)-- (-1.5,1.5);
\draw [line width=1pt] (-1.5,1.5)-- (0.5,1.5);
\draw [line width=1pt] (0.5,1.5)-- (0.5,1);
\draw [line width=1pt] (0.5,1)-- (-1.5,1);
\draw [line width=1pt] (-1.5,0.5)-- (0.5,0.5);
\draw [line width=1pt] (0.5,0)-- (0.5,0.5);
\draw [line width=1pt] (-1.5,0)-- (-1.5,0.5);
\draw [line width=1pt] (-1.5,0)-- (0.5,0);
\draw [line width=1pt] (0.5,0)-- (-1.5,0.5);
\draw [line width=1pt] (-1.5,0)-- (0.5,0.5);
\draw [line width=1pt] (-1.5,-0.5)-- (0.5,-1);
\draw [line width=1pt] (0.5,-1)-- (0.5,-0.5);
\draw [line width=1pt] (0.5,-0.5)-- (-1.5,-1);
\draw [line width=1pt] (-1.5,-1)-- (-1.5,-0.5);
\draw [line width=1pt] (-1.5,-0.5)-- (0.5,-0.5);
\draw [line width=1pt] (0.5,-1)-- (-1.5,-1);
\draw [line width=1pt] (-1.5,-1.5)-- (0.5,-2);
\draw [line width=1pt] (0.5,-1.5)-- (-1.5,-2);
\draw [line width=1pt] (-1.5,-2)-- (-1.5,-1.5);
\draw [line width=1pt] (-1.5,-1.5)-- (0.5,-1.5);
\draw [line width=1pt] (0.5,-1.5)-- (0.5,-2);
\draw [line width=1pt] (0.5,-2)-- (-1.5,-2);

\draw [fill=black] (-1.5,1.5) circle (2pt);
\draw [fill=black] (-1.5,1) circle (2pt);
\draw [fill=black] (0.5,1.5) circle (2pt);
\draw [fill=black] (0.5,1) circle (2pt);
\draw [fill=black] (-1.5,0.5) circle (2pt);
\draw [fill=black] (-1.5,0) circle (2pt);
\draw [fill=black] (0.5,0.5) circle (2pt);
\draw [fill=black] (0.5,0) circle (2pt);
\draw [fill=black] (-1.5,-0.5) circle (2pt);
\draw [fill=black] (-1.5,-1) circle (2pt);
\draw [fill=black] (0.5,-0.5) circle (2pt);
\draw [fill=black] (0.5,-1) circle (2pt);
\draw [fill=black] (-1.5,-1.5) circle (2pt);
\draw [fill=black] (-1.5,-2) circle (2pt);
\draw [fill=black] (0.5,-1.5) circle (2pt);
\draw [fill=black] (0.5,-2) circle (2pt);
\end{tikzpicture}

\caption{An illustration of a modular code obtained by modifying a template on disjoint components.} \label{fig:modular}
\end{figure}

\subsection{Modularity conjecture}

One method for constructing uniacute spherical codes is to begin with the $p$-block template from the global structure theorem and then incorporate modifications that are ``modular.'' We introduce the concept of \emph{modular spherical $L$-codes}, or simply \emph{modular codes}, to describe this type of construction. 
These codes are modular in the sense that larger modular codes can be constructed by repeating small modular codes.
See \cref{fig:modular} for an illustration.

In all solved cases of the rate problem for uniacute spherical codes, such as for equiangular lines, the optimal rate can be achieved using modular codes. We hypothesize that a similar phenomenon holds for general uniacute codes. 
To formalize these notions, we formulate a \emph{modularity conjecture} in  \cref{sec:modularity}.
The modularity perspective was already helpful here in our work proving \cref{thm:main}.



\subsection{Detailed results}

We now state refined versions of \cref{thm:uniform-lim,thm:main}.

\begin{theorem}\label{thm:general-upper}
	Fix $\alpha, \beta \in (0,1)$. Let $p = \floor{\alpha/\beta} + 1$ and $\lambda = (1-\alpha)/(\alpha + \beta)$. Set $L = [-1,-\beta] \cup \{\alpha\}$. Then the maximum cardinality of an $L$-code in $\RR^d$ satisfies
\[
    N_L(d) \leq \begin{cases}
        2pd + O_L(1) & \text{ if } \lambda \geq 1,\\
        pd + o(d) & \text{ if } \lambda < 1.\\
    \end{cases}
\]

\end{theorem}

\begin{theorem} \label{thm:detail}
Let $L = [-1, -\beta] \cup \{\alpha\}$ with $\alpha,\beta \in (0,1)$. 
Let 
\[
p = \floor{\frac{\alpha}{\beta}} + 1
\qquad 
\text{and} \qquad
\lambda = \frac{1-\alpha}{\alpha +\beta}.
\]
\begin{enumerate}[(a)]
    \item \label{itm:detail-1}
    Suppose $\alpha/\beta = 1$. 
        \begin{enumerate}[(i)]
            \item \label{itm:detail-1int} If $\lambda^2 \in \ZZ$, then there exist positive integers $d_0 = d_0(\alpha)$ and $m = m(\alpha)$ such that for all $d \ge d_0$,
            \[
            N_L(d) \le 4(d-1),
            \]
            and, whenever $d-1$ is divisible by $m$,
            \[
            N_L(d) \ge 4(d-1).
            \]
            \item \label{itm:detail-1non} If $\lambda^2 \notin \ZZ$, then 
            \[N_L(d) \le 11d/3 + O_L(1).\]
        \end{enumerate}
    \item \label{itm:detail-2} Suppose $\alpha/\beta \in \ZZ$ and $\alpha/\beta \geq 2$.
        \begin{enumerate}[(i)]
            \item \label{itm:detail-2int}
                If $\lambda \in \ZZ$, then there exists a positive integer $m = m(\alpha, \beta)$ such that
                \[
                N_L(d) \le 2pd + O_L(1),
                \]
                and, whenever $d-p+1$ is divisible by $m$,
                \[
                N_L(d) \ge 2p(d-p+1).
                \]
            \item \label{itm:detail-2non} If $\lambda \notin \ZZ$, then 
            \[
            N_L(d) \le \left(2p - \frac{1}{2}\right)d + O_L(1).
            \]
        \end{enumerate}
    \item  \label{itm:detail-non} 
    Suppose $\alpha / \beta \notin \ZZ$. Then
    \[
        N_L(d) = \begin{cases}
                2pd + O_L(1) & \text{ if } \lambda \geq 1,\\
                pd + o(d) & \text{ if } \lambda < 1. \\
                 \end{cases}
    \]
    \item \label{itm:error-term} For all integers $p$ and $K$ there exists $\varepsilon$ such that whenever $p-\varepsilon < \alpha/\beta < p$, there exists $d = d_0(\alpha,\beta)$ such that for all $d \ge d_0$,
            \[
            N_L(d) \ge 2pd + K.
            \]
\end{enumerate}
\end{theorem}

The proofs are given as follows.

Upper bounds:
\begin{enumerate}
\item[\ref{itm:detail-1}]
\ref{itm:detail-1int} \cref{prop:upper} \quad 
\ref{itm:detail-1non} \cref{prop:algebraic-upper}\ref{itm:algebraic-upper-2}
\item [\ref{itm:detail-2}]
\ref{itm:detail-2int} \cref{thm:general-upper} \quad 
\ref{itm:detail-2non} \cref{prop:algebraic-upper}\ref{itm:algebraic-upper-p}
\item [\ref{itm:detail-non}]  \cref{thm:general-upper}
\end{enumerate}

Lower bounds:
\begin{enumerate}
\item[\ref{itm:detail-1}]
\ref{itm:detail-1int} \cref{prop:lower}\ref{itm:lower-2p}\ref{itm:lower-4}

\item [\ref{itm:detail-2}]
\ref{itm:detail-2int} \cref{prop:lower}\ref{itm:lower-2p}\ref{itm:lower-int-2p}

\item [\ref{itm:detail-non}]  \cref{prop:lower}\ref{itm:lower-nint-p} and \cref{prop:lower}\ref{itm:lower-2p}\ref{itm:lower-nint-2p}
\item [\ref{itm:error-term}] \cref{prop:lower-extra}.
\end{enumerate}






\begin{remark}
In \ref{itm:error-term}, we see that the $O_L(1)$ term in the uniform bound $N_L(d) \le 2pd + O_L(1)$ needs to be large when $\alpha/\beta$ falls in the upper end of its allowed interval $[p-1,p)$. 
The construction showing \ref{itm:error-term} also shows that in the global structure theorem, \cref{thm:structure}, it is sometimes necessary to remove a large number of vertices before partitioning the vertex set into $p$ parts.
\end{remark}

\subsection{Further directions.}
This paper makes progress on the rate problem for uniacute spherical codes (\cref{prob:rate}) by characterizing all $L = [-1,-\beta]\cup\set{\alpha}$ where the uniform bound \cref{thm:uniform-lim} is tight. 
It remains open to determine the rate in other cases. 


Our proposed modularity conjecture, discussed in \cref{sec:modularity}, provides a characterization of the rate.
Ideas from \cite{BDKS18,JP20,JP21+,JTYZZ23} may be helpful in establishing more cases, although a full solution may hinge on a proper generalization of \cref{thm:eigen-mult}.

Balla, Dr\"{a}xler, Keevash, and Sudakov \cite{BDKS18} showed that for $L = [-1, -\beta] \cup \{\alpha_1, \cdots, \alpha_k\}$ with $0 < \beta \leq 1$ and $0 \leq \alpha_1 < \cdots < \alpha_k < 1$,
\[
\limsup_{d \to \infty} \frac{N_L(d)}{d^k} \leq 2^k(k-1)!\paren{1 + \floor{\frac{\alpha_1}{\beta}}}
\]
and, for each $k$, there exists $L$ with a positive limsup.
There is much more work to do to better understand these ``$k$-acute spherical codes'' or ``multiacute codes.''

Another related direction concerns replacing real unit vectors by complex unit vectors. This is related to complex equiangular lines. 
Balla~\cite{Bal21+} recently showed that the maximum number of complex unit vectors in $\CC^d$ whose pairwise inner product has magnitude $\alpha$ is $O_\alpha(d)$. 
A related problem concerns configurations of subspaces with prescribed angles~\cite{LS73subspace,BS19}.




\section{Modularity}\label{sec:modularity}

To motivate the discussion in this section, we first consider two example constructions. Instead of dealing with spherical codes explicitly, \cref{lem:gram} will allow us to work with Gram matrices instead. 

\begin{lemma}\label{lem:gram}
	Let $L \subseteq [-1,1)$. There exists a spherical $L$-code of size $n$ in $\RR^d$ if and only if there exists an $n \times n$ positive semidefinite matrix of rank at most $d$ with all diagonal entries being $1$ and all non-diagonal entries belonging to $L$.
\end{lemma}

\begin{proof}\textcolor{red}{TOPROVE 0}\end{proof}

\begin{example}\label{ex:equiangular-construction}

Lemmens and Seidel~\cite{LS73} showed that $N_{\{\pm 1/3\}}(d) = 2(d-1)$ for all sufficiently large dimensions $d$. Our first example will be a tight lower bound construction for $N_{\{\pm 1/3\}}(d)$ for large $d$. 

	Let $L = \{\pm 1/3\}$. Observe that by \cref{lem:gram}, there is a spherical $L$-code of size two in $\RR^2$ with the Gram matrix 
\[
	\begin{pmatrix}
		1 &  -1/3 \\
		-1/3 & 1 \\
	\end{pmatrix} = \frac{1}{3}\begin{pmatrix}
						 1 &  1 \\
						1 & 1 
						\end{pmatrix} + \frac{2}{3}\begin{pmatrix}
													 \phantom{+}1 &  -1 \\
													-1 & \phantom{+}1 
													\end{pmatrix}.
\]
Note also that the matrix
\[
	P = \begin{pmatrix}
		 \phantom{+}1 &  -1 \\
		-1 & \phantom{+}1 
		\end{pmatrix}
\]
is positive semidefinite matrix of rank $1$. For $d \geq 2$ we can chain together $d-1$ copies of $P$ to form positive semidefinite matrix $Q$ of rank $(d-1)$ with block form  
\[
	Q = \frac{2}{3}\begin{pmatrix}
		  P &  &  &  \\
		   & P &  &  \\
		   &  & \ddots &  \\
		   &  &  & P \\
		    	\end{pmatrix}.
\]
Let $J$ be the $2(d-1) \times 2(d-1)$ all-ones matrix and 
\[
	T = \frac{1}{3} J  \quad \text{ and } \quad M = T + Q.
\]
Since $M$ is sum of positive semidefinite matrices, it is positive semidefinite. Moreover, by sub-additivity of rank, $M$ is a $2(d-1) \times 2(d-1)$  matrix of rank at most $d$. Entries of $M$ are
\[
M = \begin{pmatrix}
    \begin{array}{cccc}
       \begin{matrix}
 		 1 & -1/3 \\
  		-1/3 & 1
  \end{matrix} & \begin{matrix}
  					1/3 & \phantom{-}1/3 \\
  					1/3 & \phantom{-}1/3
  				\end{matrix} & \cdots & \begin{matrix}
 										 1/3 & \phantom{-}1/3 \\
  										1/3 & \phantom{-}1/3
 										 \end{matrix} \\
  \begin{matrix}
  1/3 & \phantom{-}1/3 \\
  1/3 & \phantom{-}1/3
  \end{matrix} & \begin{matrix}
 				 1 & -1/3 \\
 				 -1/3 & 1
 				 \end{matrix} & \cdots & \begin{matrix}
  										1/3 & \phantom{-}1/3 \\
  										1/3 & \phantom{-}1/3
  										\end{matrix} \\
  \vdots & \vdots & \ddots & \vdots \\
 
   \begin{matrix}
  1/3 & \phantom{-}1/3 \\
  1/3 & \phantom{-}1/3
  \end{matrix} & \begin{matrix}
 				 1/3 & \phantom{-}1/3 \\
 				 1/3 & \phantom{-}1/3
 				 \end{matrix} & \cdots & \begin{matrix}
  										1 & -1/3 \\
  										-1/3 & 1
  										\end{matrix} \\		
    \end{array}
\end{pmatrix}.
\]
The diagonal entries of $M$ are all $1$ and non-diagonal entries all lie in $\{\pm 1/3\}$. Hence, by \cref{lem:gram} there is a spherical $\{\pm1/3\}$-code of size $2(d-1)$ in $\RR^d$. 

\medskip

Observe that nearly all entries of $M$ coincide with those of the low rank matrix $T$. In this sense, one can think of $T$ as a template. Observe also that the difference of the Gram matrix and the template $M - T = Q$ is a positive semidefinite matrix. 
\end{example}

\begin{example}\label{ex:uniacute-construction}

\cref{thm:detail} implies that the maximum cardinality of a spherical $[-1, -1/3] \cup \{1/3\}$-code is exactly $4(d - 1)$ for all sufficiently large dimensions $d$. Our second example will be the sharp construction of spherical $[-1, -1/3] \cup \{1/3\}$-codes for large $d$.

	By \cref{lem:gram}, there is a spherical $[-1, -1/3] \cup \{1/3\}$-code of size 4 in $\RR^2$ with Gram matrix
	\[
		\begin{pmatrix}
		1 & -1/3 & -1 & 1/3 \\
		-1/3 & 1 & 1/3 & -1 \\
		-1 & 1/3 & 1 & -1/3 \\
		1/3 & -1 & -1/3 & 1
	\end{pmatrix}
		=
		\frac{1}{3}
			\begin{pmatrix}
			\phantom{+}1 & \phantom{+}1 & -1 & -1 \\
			\phantom{+}1 & \phantom{+}1 & -1 & -1 \\
			-1 & -1 & \phantom{+}1 & \phantom{+}1 \\
			-1 & -1 & \phantom{+}1 & \phantom{+}1
			\end{pmatrix}
		+
		\frac{2}{3}
			\begin{pmatrix}
			\phantom{+}1 & -1 & -1 & \phantom{+}1 \\
			-1 & \phantom{+}1 & \phantom{+}1 & -1 \\
			-1 & \phantom{+}1 & \phantom{+}1 & -1 \\
			\phantom{+}1 & -1 & -1 & \phantom{+}1
		\end{pmatrix}.
\]
Note also that the matrix 
\[
\begin{pmatrix}
	P_{11} & P_{12} \\
	P_{21} & P_{22}
\end{pmatrix}
		=
		\begin{pmatrix}
		\begin{array}{cc|cc}
			\phantom{+}1 & -1 & -1 & \phantom{+}1 \\
			-1 & \phantom{+}1 & \phantom{+}1 & -1 \\
			\hline
			-1 & \phantom{+}1 & \phantom{+}1 & -1 \\
			\phantom{+}1 & -1 & -1 & \phantom{+}1
		\end{array}
\end{pmatrix}
\]
is positive semidefinite matrix of rank $1$. For $d \geq 2$ we can chain together $(d-1)$ copies of the matrix to form positive semidefinite matrix $Q$ of rank $(d-1)$
\[
	Q = \frac{2}{3}\begin{pmatrix}
	\begin{array}{c|c}
		\begin{array}{ccc}
			P_{11} &  &    \\
			 &  \ddots &  \\
			 &   & P_{11}
		\end{array}
		&
		\begin{array}{ccc}
			P_{12} &   &  \\
			 &   \ddots &  \\
			 &    & P_{12}
		\end{array}
		\\
		\hline
		\begin{array}{ccc}
			P_{21} &    &  \\
			 &   \ddots &  \\
			 &   & P_{21}
		\end{array}
		&
		\begin{array}{ccc}
			P_{22} &  &    \\
			 &   \ddots &  \\
			 &    & P_{22}
		\end{array}
	\end{array}
\end{pmatrix}.
\]
Let  $J$ be $2(d-1)\times 2(d-1)$ all-ones matrix and define $T$ and $M$ by 
\[
	T = \frac{1}{3}\begin{pmatrix}
		\phantom{+}J & -J \\
		-J & \phantom{+}J \\
	\end{pmatrix} \quad \text{ and } \quad M = T + Q.
\]
Since $M$ is a sum of positive semidefinite matrices, it is positive semidefinite. Moreover, by subadditivity of rank, $M$ is a $4(d-1) \times 4(d-1)$ matrix of rank at most $d$. Finally, diagonal entries of $M$ are all $1$ and non-diagonal entries are in $\{-1, \pm 1/3\}$. Hence, by \cref{lem:gram}, there exists a spherical $[-1, -1/3]\cup \{1/3\}$-code of size $4(d-1)$ in $\RR^d$.

Observe also that the majority of entries of $M$ agree with those of the matrix $T$. Once again, one can think of the matrix $T$ as a template and note that the difference of the Gram matrix and the template $M -T = Q$ is a positive semidefinite matrix.
\end{example}

  Bearing these examples in mind, we will define modular codes.


\begin{definition}[Template]\label{def:template}
Let $\alpha, \beta \in (0,1)$ and $L \subseteq [-1,-\beta]\cup\{\alpha\}$ with $\alpha \in L$. We say that a square matrix $T$ is a \textit{template (with respect to $L$)} if it has a block form
	\[
		T = \begin{pmatrix}
    T_{11} & T_{12} & \dots & T_{1m} \\
    T_{21} & T_{22} & \dots & T_{2m} \\
    \vdots & \vdots & \ddots & \vdots \\
    T_{m1} & T_{m 2} & \dots & T_{mm}
		\end{pmatrix}
	\]
	satisfying the following properties:
	\begin{enumerate}[label=(\roman*)]
		\item $T$ is positive semidefinite;
		\item For each $1 \leq i \leq m$, all entries of $T_{ii}$ are equal to $\alpha$;
		\item For each $1 \leq i < j \leq m$, all entries of $T_{ij}$ are equal to $\gamma_{ij}$ for some $\gamma_{ij} \in L \setminus \{\alpha\}$.
	\end{enumerate}
\end{definition}

\begin{definition}[Modular code]\label{def:modular}
Let $\alpha, \beta \in (0,1)$ and $L \subseteq [-1,-\beta]\cup\{\alpha\}$ with $\alpha \in L$.
	We say that an $L$-code $C$ is \textit{modular} if there exists a template $T$ with respect to $L$ such that $M_C - T$ is positive semidefinite, where $M_C$ is the Gram matrix of possibly reordered $C$. 

 We denote $N_L^{mod}(d)$ to be the maximum size of a modular $L$-code in $\RR^d$.
\end{definition}

\cref{ex:equiangular-construction} is a construction of a modular $\{\pm 1/3\}$-code and \cref{ex:uniacute-construction} is a construction of a modular $[-1,-1/3]\cup\{1/3\}$-code. In fact, all uniacute spherical codes in high dimensions that are known to be tight arise from modular ones. Motivated by this, we form the following conjecture. 



\begin{conjecture}[Modularity]\label{conj:modularity}
Let $\alpha,\beta \in (0,1)$ and $L \subseteq [-1,-\beta]\cup\{\alpha\}$ with $\alpha \in L$. Then
\[
		N_L(d) = N_L^{mod}(d) + o(d) \text{ as } d \to \infty.
\]
\end{conjecture}

Our modularity conjecture is known to hold in many special cases, namely for equiangular lines $L = \set{\pm \alpha}$ \cite{JTYZZ21}, certain cases of the spherical two-distance set problem $L = \set{-\beta, \alpha}$ \cite{JTYZZ23, JP21+}, as well as for $L = [-1,-\beta] \cup \set{\alpha}$ when the uniform bound, \cref{thm:uniform-lim}, is tight, as we prove in this paper.

For equiangular lines, \cref{thm:equiangular}, in case \ref{itm:equiangular-a} where $k < \infty$, we have a stronger conclusion $N_L(d) = N_L^{mod}(d)$ for all sufficiently large $d$.
On the other hand, in case \ref{itm:equiangular-b} where $k = \infty$, the $N_L(d) - N_L^{mod}(d) = o(d)$ error term is $O(d/\log\log d)$, and Schildkraut \cite{Sch23+} proved that there exist infinitely many $\alpha$'s for which $N_{\{\pm \alpha\}}^{mod}(d) = d-1$ and $N_{\{\pm \alpha\}}(d) - N_{\{\pm \alpha\}}^{mod}(d) \ge  \Omega_\alpha(\log \log d)$. As a result, we cannot replace the $o(d)$ term in the modularity conjecture by $O_L(1)$.

In this paper, we prove the modularity conjecture for $L = [-1,-\beta]\cup\set{\alpha}$ in cases where the uniform bound (\cref{thm:uniform-lim}) is tight. In some instances, such as \cref{thm:detail}\ref{itm:detail-1}\ref{itm:detail-1int}, the error term is zero for infinitely many $d$. In other cases of \cref{thm:detail}, the error term is $O_L(1)$, but this error term can be arbitrarily large as a function of $L$, as seen in \cref{thm:detail}\ref{itm:error-term}.


\section{Lower Bound Constructions}\label{sec:lower}

In this section we will prove the lower bounds of \cref{thm:detail}.

\begin{proposition}\label{prop:lower}
    Fix $\alpha, \beta \in (0,1)$. Let $\lambda = (1-\alpha)/(\alpha + \beta)$ and $p = \lfloor \alpha/\beta \rfloor + 1$. Let $L = [-1,-\beta]\cup \{\alpha\}$. Then, there exist $m = m(\alpha, \beta)$ and $d_0 = d_0(\alpha, \beta)$ such that
    \begin{enumerate}[(a)]
        \item \label{itm:lower-nint-p} $N_L(d) \geq p(d-p+1)$ whenever $\alpha/\beta \notin \ZZ$ and $d \geq d_0$; 
        \item \label{itm:lower-2p}  $N_L(d) \geq 2p(d-p+1)$ if any of the following hold: 
        \begin{enumerate}[(i)]
            \item \label{itm:lower-nint-2p} $\alpha/\beta \notin \ZZ$ and $\lambda \geq 1$ and $d \geq d_0$,
            \item \label{itm:lower-int-2p}  $\alpha/\beta \in \ZZ$ and $\alpha/\beta \geq 2$ and $\lambda \in \ZZ$ and $d-p+1$ is divisible by $m$,
            \item \label{itm:lower-4} $\alpha/\beta = 1$ and $\lambda^2 \in \ZZ$ and $d-p+1$ is divisible by $m$.
        \end{enumerate}
    \end{enumerate}
\end{proposition}

Our constructions rely on the existence of certain orthogonal matrices.

\begin{proposition}\label{prop:matrix}
    For any positive integer $k$, there exists a square matrix $Q$ with $\{0, \pm 1\}$ entries such that $Q^\intercal Q = k I$. 
\end{proposition}

\begin{proof}\textcolor{red}{TOPROVE 1}\end{proof}

\begin{proposition}\label{prop:matrix-copies}
    For any positive integer $k$ and $p$, there exist square matrices $Q_1, \ldots, Q_p$ of the same size with $\{0, \pm 1\}$ entries, such that $Q_i^\intercal Q_i = k I$ for each $i$ and $Q_i^\intercal Q_j$ has $\{0, \pm 1\}$ entries whenever $i \neq j$.
\end{proposition}


The following lifting procedure is based on an idea of Wocjan and Beth \cite{WB05}. 

\begin{definition}[Net]\label{def:net}
A collection of $m^2 \times m$ matrices $V_1, \ldots, V_p$ with  $\{0,1\}$ entries is called a \emph{$(p,m)$-net} if
\[
    V_i^\intercal V_j = \begin{cases}
                            m I_m & \text{ if } i = j, \\
                            J_m   & \text{ if } i \neq j. \\
                         \end{cases}
\]
Here $I_m$ is $m\times m$ identity matrix and $J_m$ is $m\times m$ all ones matrix.
\end{definition}

\begin{proposition}\label{porp:net}
    For any positive integer $p$ there exists $m_0$ such that there exists a $(p,m)$-net for each $m \geq m_0$.
\end{proposition}
\begin{proof}\textcolor{red}{TOPROVE 2}\end{proof}

We will now use nets to lift matrix from \cref{prop:matrix} to prove \cref{prop:matrix-copies}.

\begin{proof}\textcolor{red}{TOPROVE 3}\end{proof}

\begin{lemma}\label{lem:matrix-copies}
    For any $p$ and $\varepsilon > 0$, there exists $m_0$ such that for all $m \geq m_0$ we can find $p$ orthogonal $m\times m$ matrices $R_1, \ldots, R_p$ such that  all entries of $R_i^\intercal R_j$ are at most $\varepsilon$ in absolute value for all $i \neq j$.
\end{lemma}
\begin{proof}\textcolor{red}{TOPROVE 4}\end{proof}

Let us introduce the templates we will be using in the construction.

\begin{definition}\label{def:uniform-template}
     Let $\alpha, \beta \in (0,1)$. Let $p, s_1, \ldots, s_p$ be positive integers. Write $s = (s_1,\ldots,s_p)$ and define $T_{\alpha, \beta}(p, s)$ by the following block form
         \[
        T_{\alpha, \beta}(p, s) = \begin{pmatrix}
            T_{11} & \cdots & T_{1p} \\
            \vdots & \ddots & \vdots \\
            T_{p1} & \cdots & T_{pp} \\
        \end{pmatrix},
    \]
    where $T_{ij}$ is the $s_i \times s_j$ matrix with all entries equal to $\alpha$ if $i = j$ and $-\beta$ if $i \neq j$. 
\end{definition}

\begin{lemma}\label{lem:psd-template}
    Let $\alpha, \beta \in (0,1)$ and $p$ be positive integer and $s$ be a $p$-tuple of positive integers. If $1 + \alpha/\beta \geq p$, then $T_{\alpha,\beta}(p,s)$ is a positive semidefinite matrix of rank at most $p$. 
    
    Furthermore, if $1 + \alpha/\beta = p$, then the rank of $T_{\alpha,\beta}(p,s)$ is at most $p-1$.
\end{lemma}

\begin{proof}\textcolor{red}{TOPROVE 5}\end{proof}

We are now ready to prove \cref{prop:lower}.

\begin{proof}\textcolor{red}{TOPROVE 6}\end{proof}
\medskip

\begin{proof}\textcolor{red}{TOPROVE 7}\end{proof}

\begin{proof}\textcolor{red}{TOPROVE 8}\end{proof}
\medskip

\begin{proof}\textcolor{red}{TOPROVE 9}\end{proof}

Next, we show that when $\alpha/\beta$ is near the upper end of its allowed interval $[p-1,p)$, one can add additional vectors to the $L$-code.

\begin{proposition}\label{prop:lower-extra}
     Let $L = [-1,-\beta]\cup \{\alpha\}$ with $\alpha, \beta \in (0,1)$ and $\lambda = (1-\alpha)/(\alpha + \beta)$. Assume $\lambda \ge 1$.
     Then, for any positive integers $p$ and $M$ we can find $\varepsilon > 0$ and $d_0$ such that if $p-\varepsilon < \alpha/\beta < p$, then for all $d \geq d_0$,
    \[
         N_L(d) \geq 2pd + M.
    \]
\end{proposition}

\begin{proof}\textcolor{red}{TOPROVE 10}\end{proof}

\section{Global Structure Theorem}\label{sec:global}

In this section, we will prove \cref{thm:structure}. 

Recall that given a spherical code $C$, its \emph{Gram graph} $G_C$ is the edge-weighted complete graph on vertex set $C$ such that edge $uv$ has weight $\ang{u,v}$. The \emph{negative graph} $G_C^-$ is the simple graph consisting of negative edges of $G_C$ (and then forgetting the edge weights). Likewise the \emph{positive graph} $G_C^+$ consisting of positive edges of $G_C$ (and then forgetting the edge weights).
We shall talk about the negative and positive degrees to refer to degrees in $G_C^-$ and $G_C^+$. For example, given $A \subseteq C$, the \emph{maximum positive degree in $A$} is the maximum degree of $G_A^+$. Given $A,B \subseteq C$, the \emph{maximum negative degree from $B$ to $A$} is the maximum number of neighbors that $b \in B$ can have in $A$ in the negative graph $G_C^-$.

\subsection{Forbidden local structures}

The next three lemmas use the positive semidefiniteness of the Gram matrix to deduce forbidden induced subgraphs in the Gram graph.

\begin{lemma}\label{lem:negative-clique}
The largest negative clique in any spherical $[-1,-\beta]\cup\{\alpha\}$ has at most $\beta^{-1} + 1$ vertices.
\end{lemma}

\begin{proof}\textcolor{red}{TOPROVE 11}\end{proof}



\begin{lemma}\label{lem:very-strong-positive-clique}
Let $ \alpha, \beta \in(0,1)$ and $L \subseteq [-1,-\beta]\cup\{\alpha\}$. Then for all $\Delta_0$, there exists  $K=K(\alpha,\beta,\Delta_0)$ such that the following holds for all $\Delta \geq 1$: 

Let $C = A \sqcup B \sqcup \{v\}$ be a spherical $L$-code such that for some $\gamma \in [-1,1]$,
\begin{itemize}
    \item[(a)] maximum negative degree from $B$ to $A$ is at most $\Delta_0$, and
    \item[(b)] all edges from $v$ to $A$ have weight at least $\gamma + K\Delta^{-1/2}$, and
    \item[(c)] all edges from $v$ to $B$ have weight at most $\gamma - K\Delta^{-1/2}$.
\end{itemize}
Then either $|A| < \Delta$ or $|B| < \Delta$.
\end{lemma}


\begin{figure}[h]
\definecolor{ffzzcc}{rgb}{1,0,1}
\definecolor{qqzzff}{rgb}{0,0.6,1}
\definecolor{wwccff}{rgb}{0.4,0.8,1}
\begin{tikzpicture}


\fill[line width=0pt,color=wwccff,fill=wwccff!30] 
(-1.5, 0.5) rectangle (-0.5, -0.5);
\node at (-1, 0) {\tiny $\alpha$};

\fill[line width=0pt,color=wwccff,fill=wwccff] 
(-1.5, 1.5) rectangle (-0.5, 0.5);
\node at (-1, 1) {\small $A$};


\fill[line width=0pt,color=wwccff,fill=wwccff] 
(-1.5, -1.5) rectangle (-0.5, -0.5);
\node at (-1, -1) {\small $B$};




\draw [fill=black] (1,0) circle (2pt)  node[below right] {\large$v$};

\fill[line width=0pt,color=wwccff,fill=wwccff!30]  (1, 0) -- (-0.5, 1.5) -- (-0.5, 0.5) -- cycle;
\node[rotate=-30] at (0,.6) {\tiny $> \gamma + \varepsilon$};

\fill[line width=0pt,color=wwccff,fill=wwccff!30]  (1, 0) -- (-0.5, -0.5) -- (-0.5, -1.5) -- cycle;
\node[rotate=30] at (0,-.6) {\tiny $< \gamma - \varepsilon$};

\end{tikzpicture}
\caption{Forbidden configuration from \cref{lem:very-strong-positive-clique}.}
\end{figure}

\begin{proof}\textcolor{red}{TOPROVE 12}\end{proof}

The next lemma roughly states that if there are $q$ not-too-small positive cliques, with mostly negative edges between each pair of cliques, then $q \le \floor{\alpha/\beta} + 1$.

\begin{lemma}\label{lem:many-cliques}
    Let $\alpha, \beta \in (0,1)$ and $L \subseteq [-1,-\beta]\cup\{\alpha\}$. For all $\Delta_0$ there exists $P = P(\alpha, \beta, \Delta_0)$ such that if $A_1 \sqcup \ldots \sqcup A_q$ is a spherical $L$-code satisfying 

    \begin{itemize}
        \item[(a)] for all $1 \leq i \leq q$, the $L$-code $A_i$ is a positive clique of size $P$;
    
        \item[(b)] for all $1 \leq i < j \leq q$, the maximum positive degree from $A_i$ to $A_j$ is at most $\Delta_0$,
    \end{itemize}
    then $q \leq \lfloor \alpha/\beta\rfloor + 1$.
\end{lemma}

\begin{figure}[h]
\definecolor{wwccff}{rgb}{0.4,0.8,1}
\begin{tikzpicture}[scale = 0.6]
\fill[color = wwccff!30] (-0.7524102376159736,2.3116585248889) -- (-2.3781601262366365,-0.5042228826015307) -- (-1.901151149655595,-0.7796241436364628) -- (-0.2754012610349321,2.036257263853968) -- cycle;
\node at (-1.3267806936357842,0.7660171906262186) {$-$};

\fill[color = wwccff!30] (0.2754012610349319,2.0362572638539675) -- (0.7524102376159734,2.3116585248888994) -- (2.378160126236635,-0.5042228826015329) -- (1.9011511496555942,-0.7796241436364647) -- cycle;
\node at (1.3267806936357835,0.7660171906262179) {$-$};

\fill[color = wwccff!30] (-1.6257498886206623,-1.2566331202175038) -- (-1.6257498886206623,-1.807435642287369) -- (1.6257498886206634,-1.8074356422873683) -- (1.6257498886206634,-1.2566331202175054) -- cycle;
\node at (0,-1.5320343812524366) {$-$};


\fill[color = wwccff!30] (-1.6257498886206623,-1.2566331202175038) -- (-0.6948350198094596,-0.7191691701167311) -- (-0.9702362808443916,-0.24216019353568968) -- (-1.901151149655595,-0.7796241436364628) -- cycle;
\node at (-1.3267806936357842,-0.766017190626218) {$-$};

\fill[color = wwccff!30] (-0.2754012610349321,0.9613293636524214) -- (0.2754012610349319,0.9613293636524213) -- (0.2754012610349319,2.0362572638539675) -- (-0.2754012610349321,2.036257263853968) -- cycle;
\node at (0,1.5320343812524366) {$-$};

\fill[color = wwccff!30] (1.9011511496555942,-0.7796241436364647) -- (1.6257498886206634,-1.2566331202175054) -- (0.6948350198094596,-0.7191691701167312) -- (0.9702362808443913,-0.24216019353569118) -- cycle;
\node at (1.3267806936357835,-0.7660171906262188) {$-$};


\fill [color = wwccff] (0,0) circle (1.064068762504873cm);
\node at (0,0) {$+$};


\fill [color = wwccff]  (0,3.0640687625048733) circle (1.0640687625048733cm);
\node at (0,3.0640687625048733) {$+$};


\fill [color = wwccff]  (-2.6535613872715684,-1.532034381252436) circle (1.0640687625048733cm);
\node at (-2.6535613872715684,-1.532034381252436) {$+$};


\fill [color = wwccff]  (2.653561387271567,-1.5320343812524375) circle (1.0640687625048733cm);
\node at (2.653561387271567,-1.5320343812524375) {$+$};
\end{tikzpicture}

\caption{Forbidden structure from \cref{lem:many-cliques}.}
\end{figure}

\begin{proof}\textcolor{red}{TOPROVE 13}\end{proof}

\subsection{Deducing global structure}
In the remainder of this section, we deduce the global structure theorem, \cref{thm:structure}.


\begin{lemma}\label{lem:strong-positive-clique}
Let $\alpha, \beta \in (0,1)$ and $L \subseteq [-1,-\beta]\cup\{\alpha\}$.
For all $\Delta_0$ there exists $\Delta$ such that if $C = A \sqcup B \sqcup \{v\}$ is an $L$-code satisfying 
\begin{itemize}
    \item[(a)] the maximum negative degree from $B$ to $A$ is at most $\Delta_0$, and
    \item[(b)] all edges from $v$ to $A$ have positive weight, and
    \item[(c)] all edges from $v$ to $B$ have negative weight.
\end{itemize}
Then either $|A| \leq \Delta$ or $|B| \leq \Delta$.
\end{lemma}

\begin{figure}[h]
\definecolor{ffzzcc}{rgb}{1,0,1}
\definecolor{qqzzff}{rgb}{0,0.6,1}
\definecolor{wwccff}{rgb}{0.4,0.8,1}
\begin{tikzpicture}
\fill[line width=0pt,color=wwccff,fill=wwccff] 
(-1.5, 1.5) rectangle (-0.5, 0.5);
\node at (-1, 1) {\small $A$};


\fill[line width=0pt,color=wwccff,fill=wwccff] 
(-1.5, -1.5) rectangle (-0.5, -0.5);
\node at (-1, -1) {\small $B$};

\fill[line width=0pt,color=wwccff,fill=wwccff!30] 
(-1.5, 0.5) rectangle (-0.5, -0.5);
\node at (-1, 0) {$+$};


\draw [fill=black] (1,0) circle (2pt)  node[below right] {\large$v$};

\fill[line width=0pt,color=wwccff,fill=wwccff!30]  (1, 0) -- (-0.5, 1.5) -- (-0.5, 0.5) -- cycle;
\node[rotate=-30] at (0,.6) { $+ $};

\fill[line width=0pt,color=wwccff,fill=wwccff!30]  (1, 0) -- (-0.5, -0.5) -- (-0.5, -1.5) -- cycle;
\node[rotate=30] at (0,-.6) {$-$};

\end{tikzpicture}
\caption{Forbidden configuration from \cref{lem:strong-positive-clique}.}
\end{figure}

\begin{proof}\textcolor{red}{TOPROVE 14}\end{proof}


The next lemma states roughly that when given a large positive clique, any other vertex is connected to the clique either mostly via positive edges or mostly via negative edges.

\begin{lemma}\label{lem:positive-clique}
Let $\alpha, \beta \in (0,1)$ and $L \subseteq [-1,-\beta]\cup\{\alpha\}$. For all $\Delta_0$ there exists $\Delta$ such that if $S\sqcup\{v\}$ is an $L$-code where maximum negative degree in $S$ is at most $\Delta_0$, then either
\begin{enumerate}
    \item[(a)] the negative degree from $v$ to $S$ is at  most $\Delta$, or
    \item[(b)]  the positive degree from $v$ to $S$ is at  most $\Delta$.
\end{enumerate}
\end{lemma}

\begin{proof}\textcolor{red}{TOPROVE 15}\end{proof}

The following lemma is a more general version of \cref{lem:positive-clique}. It states that most edges from a vertex to a large positive clique have weights concentrated around a single value.

\begin{lemma}\label{lem:edges-to-positive-clique}
Let $ \alpha, \beta \in (0,1)$ and $L \subseteq [-1,-\beta]\cup\{\alpha\}$. For all $\Delta_0$ there exists $K = K(L, \Delta_0)$ such that the following holds for all $\Delta \geq 1$:

If $S\sqcup\{v\}$ is an $L$-code where the maximum negative degree in $S$ is at most $\Delta_0$, then there exists $\gamma \in \RR$ such that for all but at most $2\Delta$ many $s \in S$
\[
	|\gamma - \langle s, v \rangle| \leq K \Delta^{-1/2}.
\]
\end{lemma}

\begin{proof}\textcolor{red}{TOPROVE 16}\end{proof}


The next lemma says that, given two mostly positive cliques, most edges between them have roughly equal weights.

\begin{lemma}\label{lem:two-positive-cliques}
Let $ \alpha, \beta \in (0,1)$ and $L \subseteq [-1,-\beta]\cup\{\alpha\}$. Then, for every $\Delta_0$ there exists a constant $K = K(L, \Delta_0) $ such that the following holds for all $\Delta \geq K$: 

Suppose $A \sqcup B$ is an $L$-code with $\min\set{|A|,|B|} \geq \Delta^2$ such that the maximum negative degrees in $A$ and in $B$ are at most $\Delta_0$. Then, there exist some $\gamma \in \RR$ and $R \subseteq A \cup B$  of size at most $\Delta$ such that
\begin{enumerate}
	\item for each $a \in A \setminus R$ and all but at most $\Delta$ many $b \in B$,
	\[
		|\gamma - \langle a,b\rangle| \leq K\Delta^{-1/2};
	\]
	\item for each $b \in B \setminus R$ and all but at most $\Delta$ many $a \in B$,
	\[
		|\gamma - \langle a,b\rangle| \leq K\Delta^{-1/2}.
	\]
\end{enumerate}
\end{lemma}	

\begin{proof}\textcolor{red}{TOPROVE 17}\end{proof}

\begin{figure}\label{fig:two-positive-cliques-2}
   \definecolor{ffzzcc}{rgb}{1,0,1}
   \definecolor{wwccff}{rgb}{0.4,0.8,1}
\begin{tikzpicture}
\fill[line width=0pt,color=black,fill=wwccff] (1.5,1.5) -- (0.5,1.5) -- (0.5,-1.5) -- (1.5,-1.5) -- cycle;
\fill[line width=0pt,color=wwccff,fill=wwccff] (-0.5,1.5) -- (-1.5,1.5) -- (-1.5,-1.5) -- (-0.5,-1.5) -- cycle;
\fill[line width=0pt,color=wwccff,fill=wwccff!40] (0.5,1.5) -- (-0.5,1.5) -- (-0.5,-1.5) -- (0.5,-1.5) -- cycle;

\fill[line width=0pt,color=black,fill=ffzzcc!30] (-0.5,1.5) -- (-0.5,-1) -- (0.5,1) -- (1.5,1) -- (1.5,1.5) -- cycle;
\fill[line width=0pt,color=ffzzcc,fill=ffzzcc!] (-1.5,1.5) -- (-0.5,1.5) -- (1,0) -- (-0.5,1) -- (-1.5,1) -- cycle;



\fill[line width=0pt,color=ffzzcc,fill=ffzzcc] (0.5,1.5) -- (1.5,1.5) -- (1.5,1) -- (0.5,1) -- cycle;
\fill[line width=0pt,color=ffzzcc,fill=ffzzcc]  (0.5,-1.5) -- (1.5,-1.5) -- (1.5,-1) -- (0.5,-1) -- cycle;
\fill[line width=0pt,color=ffzzcc,fill=ffzzcc] (-0.5,1.5) -- (-1.5,1.5) -- (-1.5,1) -- (-0.5,1) -- cycle;
\fill[line width=0pt,color=ffzzcc,fill=ffzzcc]  (-0.5,-1.5) -- (-1.5,-1.5) -- (-1.5,-1) -- (-0.5,-1) -- cycle;

\draw [line width=1pt] (1,0)-- (-1,0);
\draw [fill=black] (-1,0) circle (2pt) node[below left] {\large$x$};
\draw [fill=black] (1,0) circle (2pt) node[below right] {\large$y$};

\draw [line width = 0.75pt, decorate, decoration={brace, amplitude=5pt}] (-1.55, -1)--(-1.55,1.5);

\draw [line width = 0.75pt, decorate, decoration={brace, amplitude=5pt}] (1.55,1.5)--(1.55, -1);

\node[left] at (-1.65, 0.25) {$A'$};
\node[right] at (1.65, 0.25) {$B'$};


\end{tikzpicture}
\caption{Illustration of the argument that $\gamma_A \approx \gamma_B$ in \cref{lem:two-positive-cliques}. We find $y \in B'$ and then $x \in A'$ such that $\gamma_A \approx \gamma_y \approx \ang{x,y} \approx \gamma_y \approx \gamma_B$.}
\end{figure}
	
	
\begin{proof}\textcolor{red}{TOPROVE 18}\end{proof}

The following corollary of \cref{thm:structure} will be used towards proving \cref{thm:main}.
A similar result appeared in the argument by Balla, Dr\"{a}xler, Keevash, Sudakov \cite{BDKS18}.


\begin{corollary}\label{cor:global} 
Fix $\alpha, \beta \in (0,1) $ and $L \subseteq [-1,-\beta]\cup\{\alpha\}$ and $p = \lfloor \alpha/ \beta \rfloor + 1$. There exists $\Delta = \Delta(\alpha,\beta)$  such that from any $L$-code, we can remove at most $\Delta$ vectors so that the negative graph of the remaining vertices is a $\Delta$-modification of complete $p$-partite graph.
\end{corollary}



\section{Rank Bound Within Each Part}\label{sec:rank-bound}

The global structure theorem partitions the $L$-code (minus $O_L(1)$ vectors) into at most $\floor{\alpha/\beta} + 1$ parts $V_1, \dots, V_q$.
Let us now examine the structure inside each $V_i$. 
We already know from the global structure theorem that the maximum negative degree inside each $V_i$ is $O_L(1)$. 
Our goal in this section is to show that, in order for the uniform bound to be tight, the negative graph in each $V_i$ must essentially be a perfect matching. 
For the following statements, one should think about vectors inside a single $V_i$.

Now we define a property enjoyed by negative components inside $V_i$ for an optimal modular code.

\begin{definition}\label{def:a-singular}
Let $ \alpha, \beta \in (0,1)$ and $L \subseteq [-1,-\beta] \cup \{\alpha\}$ with $\alpha \in L$. 
We say that a spherical $L$-code $C$ is \textit{$\alpha$-singular} if $M_C- \alpha J$ is a singular positive semidefinite matrix.
\end{definition}

\begin{proposition}\label{prop:a-singular}
Let $\alpha, \beta \in (0,1)$ and $L = [-1,-\beta] \cup \{\alpha\}$ with $\alpha \in L$.
Let $\lambda = (1-\alpha)/(\alpha + \beta)$. 
There exists an $\alpha$-singular $L$-code if and only if $\lambda \ge 1$. Furthermore, when $\lambda \ge 1$, there exists a unique $\alpha$-singular $L$-code of size two, namely the code with Gram matrix
\[
P = \begin{pmatrix}
            1 & 2\alpha - 1 \\
            2\alpha - 1 & 1
        \end{pmatrix}.
\]
\end{proposition}

\begin{proof}\textcolor{red}{TOPROVE 19}\end{proof}


\begin{proposition}[Rank bound]\label{prop:rank-bound}
Let $\alpha, \beta \in (0,1)$ and $L \subseteq [-1,-\beta] \cup \{\alpha\}$ with $\alpha \in L$.
Suppose $V$ is an $L$-code in $\RR^d$ with maximum negative degree at most $\Delta$.
Also suppose that among the connected components of its negative graph, exactly $r$ are $\alpha$-singular. 
Then
\[
\abs{V} \leq \begin{cases}
					d + O_{L,\Delta}\paren{\frac{d}{\log \log d}} & \text{ if } r = 0, \\
					d + r - 1 & \text{ if } r \neq 0.
				\end{cases}	
\]
\end{proposition}


\begin{proof}\textcolor{red}{TOPROVE 20}\end{proof}

Roughly speaking, the rank bound shows that when the size of $V$ is large, there must be many $\alpha$-singular components present. The following corollary provides an upper bound on the size of certain $L$-codes in terms of the number of $\alpha$-singular components of size two that they contain. 

   
\begin{corollary}\label{cor:rank-appl}
Let $\alpha, \beta \in (0,1)$ and $L \subseteq [-1,-\beta] \cup \{\alpha\}$. For all positive integers $p$ and $\Delta$ there exists $d_0 = d_0(L,p,\Delta)$ such that for all $d \geq d_0$ the following holds.

Suppose $C = V_1 \sqcup \cdots \sqcup V_p$ is an $L$-code in $\RR^d$ such that the maximum negative degree in $V_i$ is at most $\Delta$ for each $i \in [p]$. For each $i \in [p]$, let $r_i$ be the number of connected components of $G_{V_i}^-$ that form $\alpha$-singular $L$-codes of size two. Then
    \[
      |C| \leq \frac{3p(d-1)}{2} + \frac{(r_1 + \cdots + r_p)}{2}.
    \]
\end{corollary}

\begin{proof}\textcolor{red}{TOPROVE 21}\end{proof}

\section{Upper Bounds on Code Size}\label{sec:upper}

In this section we prove the upper bounds in the main results, \cref{thm:general-upper,thm:detail}. 

\begin{proof}\textcolor{red}{TOPROVE 22}\end{proof}

In the remainder of this section, we prove the upper bound of \cref{thm:detail}. Let us first make the following observation about how $\alpha$-singular codes of size two interact with each other.


\begin{lemma}\label{lem:a-singular-edges} Let $\alpha, \beta \in (0,1)$ and $L = [-1, -\beta] \cup \{\alpha\}$. Suppose $C$ is an $L$-code such whose negative graph is a perfect matching, consisting of edges edges $a_ia_i'$ for $i \in [r]$, with $r = \abs{C}/2$, and such that each $\set{a_i,a_i'}$ is an $\alpha$-singular $L$-codes of size two. Then, 
\[
    \frac{ a_i + a'_i}{2} =  \frac{ a_j + a'_j}{2} \quad \text{ for any} \quad  i, j \in [r].
\]
Moreover, writing their common midpoint as $ a = (a_i + a'_i) / 2$,
\[
    \left\{ \frac{ a_1 -  a}{\sqrt{1-\alpha}}, \cdots,  \frac{ a_r -  a}{\sqrt{1-\alpha}}\right\}
\]
is orthonormal and all orthogonal to $a$.   
\end{lemma}

\begin{proof}\textcolor{red}{TOPROVE 23}\end{proof}

The next result proves \cref{thm:detail}\ref{itm:detail-1}\ref{itm:detail-1non} and \cref{thm:detail}\ref{itm:detail-2}\ref{itm:detail-2non}.

\begin{proposition}\label{prop:algebraic-upper}
    Let $ \alpha, \beta \in (0,1)$ and $\lambda = (1-\alpha)/(\alpha + \beta)$ and $L = [-1, -\beta]\cup\{\alpha\}$. Suppose $\alpha/\beta + 1 = p \in \ZZ$.  
    \begin{enumerate}[(a)]
        \item\label{itm:algebraic-upper-2} If $p = 2$ and $\lambda^2 \notin \ZZ$, then 
        \[
            N_L(d) \leq \frac{11d}{3} + O_L(1),
        \]
        \item\label{itm:algebraic-upper-p} If $p \geq 3$ and $\lambda \notin \ZZ$, then 
        \[
            N_L(d) \leq \left(2p-\frac{1}{2}\right)d + O_L(1).
        \]
    \end{enumerate}
    \end{proposition}
   
\begin{proof}\textcolor{red}{TOPROVE 24}\end{proof}

The next result proves the upper bound in \cref{thm:detail}\ref{itm:detail-1}\ref{itm:detail-1int}.

\begin{proposition}\label{prop:upper}
    Let $ \alpha \in (0,1)$ and $L = [-1,-\alpha]\cup\{\alpha\}$. Then, there exists $d_0 = d_0(\alpha)$ such that for all $d \geq d_0$,
    \[
        N_L(d) \leq 4(d-1).
    \]
\end{proposition}

\begin{proof}\textcolor{red}{TOPROVE 25}\end{proof}


\bibliographystyle{amsplain0}
\bibliography{bibliography}

\end{document}