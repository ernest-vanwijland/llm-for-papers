The proof of this lemma is very similar to that of the analogous claim
in~\cite[Lemma~3]{AKS19:jcss}, where it was established for $d=2$.
The main difference is to 
use $A^d=I$ rather than $A^2=I$. For the sake of completeness we give the proof here.

\noindent\begin{proof}
%
{\it Finite-dimensional case.}
Suppose that the conditions of the lemma hold and $\vc Ar$ are pairwise
  commuting operators such that the equations $Q_1=\dots=Q_m=0$ are
  true when these matrices are assigned to $\vc xr$. Then, since $\vc Ar$ are
  normal and commute, by Theorem~\ref{the:SST} there is a unitary matrix $U$
  such that $E_i=UA_iU^{-1}$ is a diagonal matrix. Then, $E_i^d=I$, because $A_i^d=I$. Therefore, every diagonal entry $E_i(jj)$ belongs to $U_d$. For every equation $Q_\ell$ we have $Q_\ell(\vc Ar)=0$ implying
  $Q_\ell(\vc Er)=UQ_\ell(\vc Ar)U^{-1}=0$. Since every $E_i$ is diagonal, for every $j$ it also holds $Q_\ell(E_1(jj)\zd E_r(jj))=0$. By the conditions of the lemma we also have $Q((E_1(jj)\zd E_r(jj))=0$, and $Q(\vc Ar)=U^{-1}Q(\vc Er)U=0$.
%

\smallskip

{\it Infinite-dimensional case.}
Suppose that the conditions of the lemma hold and $\vc Ar$ are pairwise commuting  operators such that the equations $Q_1=\dots=Q_m=0$ are true when these operators are assigned to $\vc Xr$. Then, since $\vc Ar$ are normal and commute, by  Theorem~\ref{the:general-sst} there exist a measure space $(\Omega,\cM,\mu)$, a unitary map $U:\cH\to L^2(\Omega,\mu)$, and functions $\vc ar\in L^\infty(\Omega,\mu)$ such that, for the multiplication operators $E_i=T_{a_i}$ of $L^2(\Omega,\mu)$, the equalities $A_i=U^{-1}E_iU$ hold for $i\in[r]$. This implies $UA_iU^{-1}=E_i$. As $A_i^d=I$ we also have $E_i^d=I$. Therefore $a_i(\omega)^d=1$ for almost all $\omega\in\Omega$, or more formally $\mu(\{\omega\in\Omega\mid (a_i(\om)^d\ne1\})=0$. Hence $a_i(\omega)\in U_d$ for almost all $\omega\in\Omega$. For every $\ell\in[m]$ we have $Q_\ell(\vc Er)=UQ_\ell(\vc Ar)U^{-1}$, implying by the conditions of the lemma that $Q_\ell(\vc Er)=0$. Since $E_i$ is the multiplication operator given by the function $a_i$, it holds that $Q_\ell(a_1(\om)\zd a_r(\om))=0$ for almost all $\om\in\Omega$. As for almost all $\om\in\Omega$, it holds that the value $a_i(\om)\in U_d$ for each $i\in[r]$, we have $Q(a_1(\om)\zd a_r(\om))=0$ for almost all $\om\in\Omega$. Therefore $Q(\vc Er)=0$ and this implies $Q(\vc Ar)=0$ as before.
%
\end{proof}

