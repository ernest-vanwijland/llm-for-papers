In this section we prove that the so-called primitive positive definitions, a key tool in the
algebraic approach to CSPs~\cite{Bulatov05:classifying}, not only give rise to
(polynomial-time) reductions that preserve satisfiability over $U_d$ but also
preserve satisfiability via operators. This was
established for the special case of Boolean domains (i.e., for $d=2$)
in~\cite{AKS19:jcss} and the same idea works for larger domains. We will need
this result later in Section~\ref{sec:gap} to prove that certain CSPs of 
unbounded width admit a satisfiability gap.

Let $\Gm$ be a constraint language over $U_d$, let $r$ be an integer, and
let $x_1,\ldots,x_r$ be variables ranging over the domain $U_d$. A primitive
positive formula (\emph{pp-formula}) over $\Gm$ is a formula of the form
%
\begin{equation}\label{eq:pp-formula}
  \phi(x_1,\ldots,x_r)=\exists y_1\cdots\exists y_s(\rel_1(\bz_1)\wedge\cdots\wedge
  \rel_m(\bz_m)),
\end{equation}
%
where $\rel_i\in\Gm$ is either the binary equality relation on $U_d$ or a relation over $U_d$ of arity $r_i$ and each $\bz_i$ is an $r_i$-tuple of variables from
$\{x_1,\ldots,x_r\}\cup\{y_1,\ldots,y_s\}$.
%
A relation $\rel\subseteq U_d^r$ is primitive positive definable
(\emph{pp-definable}) from $\Gm$ if there
exists a pp-formula $\phi(x_1,\ldots,x_r)$ over $\Gm$ such that $\rel$ is equal to the set of
models of $\phi$, that is, the set of $r$-tuples $(a_1,\ldots,a_r)\in U_d^r$
that make the formula $\phi$ true over $U_d$ if $a_i$ is
substituted for $x_i$ in $\phi$ for every $i\in [r]$.

Let $\rel_T=U_d^2$ denote the full binary relation on $U_d$.
%
Our goal in this section is to prove the following result.

\begin{theorem}\label{the:pp}
  Let $\Gm$ be a constraint language over $U_d$ and let $R$ be pp-definable from
  $\Gm$.
  \begin{enumerate}
    \item If $\CSP(\Gm\cup\{R\})$ has a satisfiability gap of the first or second kind then so does $\CSP(\Gm)$.
    \item If $\CSP(\Gm\cup\{R\})$ has a satisfiability gap of the third kind then so does $\CSP(\Gm\cup\{R_T\})$.
  \end{enumerate}
\end{theorem}

Let $\rel\subseteq U_d^r$ be a pp-definable formula
over $\Gm$ via the pp-formula $\phi(x_1,\ldots,x_r)$ as
in~(\ref{eq:pp-formula}). Let $=_d$ denote the equality relation on $U_d$. 
Given an instance
$\cP\in\CSP(\Gm\cup\{\rel\})$ we describe a construction of an instance
$\cP'\in\CSP(\Gm\cup\{=_d\})$, and then transform $\cP'$ into an instance $\cP^=$ of $\CSP(\Gm)$ that is, in some sense, equivalent to $\cP$. 
%
We start with the instance $\cP$. For every constraint $\ang{\bu,\rel}$ of $\cP$ with
$\bu=(u_1,\ldots,u_r)$, we
introduce $s$ fresh new variables $t_1,\ldots,t_s$ for the quantified
variables in~(\ref{eq:pp-formula}); furthermore, we replace $\ang{\bu,\rel}$ by $m$ constraints
$\ang{\bw_i,\rel_i}$, $i\in [m]$, where $\bw_i$ is the tuple of variables obtained
from $\bz_i$ in~(\ref{eq:pp-formula}) by replacing $x_j$ by $u_j$, $j\in [r]$,
and by replacing $y_j$ by $t_j$, $j\in [s]$.
The collection of variables $u_1,\ldots,u_r,t_1,\ldots,t_s$ is called the
\emph{block} of the constraint $\ang{\bu,\rel}$ in $\cP'$.
%
This construction is known as the \emph{gadget construction} in the CSP
literature and it is known that $\cP$ has a solution over $U_d$
if and only if $\cP'$ has a solution over $U_d$~\cite{Bulatov05:classifying,BKW17}.

Next, suppose that $\cP'$ is an instance of $\CSP(\Gm\cup\{=_d\})$ with the set of variables $\{\vc xn\}$. The instance $\cP^=$ is obtained by identifying variables $x_i,x_j$ whenever there is the constraint $=_d(x_,x_j)$ in $\cP'$. More formally, let $\vr'$ denote the equivalence relation on $[n]$ given by $(i,j)\in\vr'$ if and only if $\cP'$ contains the constraint $=_d(x_i,x_j)$, and let $\vr$ be the symmetric-transitive closure of $\vr'$. Select a representative of every equivalence class of $\vr$; without loss of generality assume $[m]$ is the set of these representatives. Let $\vf:[n]\to[m]$ be the mapping that maps every $i\in[n]$ to the representative of its equivalence class. Then $\cP^=$ is the instance of $\CSP(\Gm)$ with the set of variables $\{\vc ym\}$ that, for every constraint $\ang{(x_{i_1}\zd x_{i_r}),\rel}$, where $\rel$ is not the equality relation, contains the constraint $\ang{(y_{\vf(i_1)}\zd y_{\vf(i_r)}),\rel}$.
%
Thus, in order to prove Theorem~\ref{the:pp}\,(1), it suffices to show the following lemma.

\begin{lemma}\label{lem:lift}
  Let $\Gm$ be a constraint language over $U_d$ and let $\rel$ be pp-definable
  from $\Gm$. Furthermore, let $\cP\in\CSP(\Gm\cup\{\rel\})$ and let
  $\cP'\in\CSP(\Gm\cup\{=_d\})$, $\cP^=\in\CSP(\Gm)$ be the gadget construction replacing constraints involving
  $\rel$ in $\cP$, and the instance obtained from $\cP'$ by identifying the variable related by the equality relation. If there is a satisfying operator assignment for $\cP$ then there is a satisfying operator assignment for $\cP^=$. 
\end{lemma}

Indeed, if $\CSP(\Gm\cup\{\rel\})$ has a satisfiability gap of the first or second kind then there is an unsatisfiable instance $\cP\in\CSP(\Gm\cup\{\rel\})$ that has a satisfying operator assignment. By the results in~\cite{Bulatov05:classifying} (cf.~also~\cite{BKW17}), $\cP',\cP^=$ are unsatisfiable. By~Lemma~\ref{lem:lift}, $\cP^=$ has a satisfying operator assignment. Hence $\cP^=$ establishes that $\CSP(\Gm)$ has a satisfiability gap of the same kind, as required to prove Theorem~\ref{the:pp}\,(1).

We will frequently use (below and also in Section~\ref{sec:gap}) the following observations. We give a proof of them for the sake of completeness.

\begin{lemma}\label{lem:matrix-polys}
Let $f$ be a polynomial and $A,B$ operators on a Hilbert space.\\[2mm]
(1) If $A,B$ commute, then so do $f(A),f(B)$.\\
(2) If $A$ is normal, then so is $f(A)$.\\
(3) If $U$ is a unitary operator, then $Uf(A)U^{-1}=f(UAU^{-1})$.\\[1mm]
(4) Let $g(x_1,\dots,x_r)$ be a multi-variate polynomial and $A_1,\dots,A_r$
  operators on a Hilbert space. If $U$ is a unitary operator, then 
\[
Ug(A_1,\dots,A_r)U^{-1}=g(UA_1U^{-1},\dots,UA_rU^{-1}).
\]
\end{lemma}

\begin{proof}
Let $f(x)=\sum_{i=0}^k\al_ix^i$.\\[2mm]
(1) We have
\begin{align*}
f(A)f(B) &=\left(\sum_{i=0}^k\al_iA^i\right)\left(\sum_{i=0}^k\al_iB^i\right)=\sum_{i,j=0}^k\al_i\al_jA^iB^j\\
&=\sum_{i,j=0}^k\al_j\al_iB^jA^i=\left(\sum_{i=0}^k\al_iB^i\right)\left(\sum_{i=0}^k\al_iA^i\right)=f(B)f(A).
\end{align*}

\noindent
(2) The operator $A$ is normal if it commutes with its 
adjoint $A^*$. Since for any operators $B,C$ it holds that $(B+C)^*=B^*+C^*$ and $(B^k)^*=(B^*)^k$, we obtain $f(A^*)=f(A)^*$. The result then follows from item (1).

\noindent
(3) We have
\begin{align*}
Uf(A)U^{-1} &=U\left(\sum_{i=0}^k\al_iA^i\right)U^{-1} =\sum_{i=0}^kU\al_iA^iU^{-1} =\sum_{i=0}^k\al_i(UAU^{-1})(UAU^{-1})\dots(UAU^{-1})\\ 
&=\sum_{i=0}^k\al_i(UAU^{-1})^i=f(UAU^{-1}).
\end{align*}

\noindent
(4) In this case the proof is similar to that of (3). Let
\[
g(x_1,\dots,x_r)=\sum_{i=(i_1,\dots,i_r)\in\cal I}\al_ix_1^{i_1}\dots x_r^{i_r},
\]
where $\cI$ is a finite subset of $(\zN\cup\{0\})^r$. We have
\begin{align*}
Ug(A_1,\dots,A_r)U^{-1} &=U\left(\sum_{i=(i_1,\dots,i_r)\in\cal I}\al_iA_1^{i_1}\dots A_r^{i_r}\right)U^{-1}\\
&=\sum_{i=(i_1,\dots,i_r)\in\cal I}U\al_iA_1^{i_1}\dots A_r^{i_r}U^{-1}\\ 
&=\sum_{i=(i_1,\dots,i_r)\in\cal I}\al_i(UA_1^{i_1}U^{-1})\dots(UA_r^{i_r}U^{-1})\\ 
&=\sum_{i=(i_1,\dots,i_r)\in\cal I}\al_i(UA_1U^{-1})^{i_1}\dots(UA_rU^{-1})^{i_r}\\
&=g(UA_1U^{-1},\dots,UA_rU^{-1}).
\end{align*}
\end{proof}




%
\begin{proof}[Proof of Lemma~\ref{lem:lift}] 
We prove the lemma in two steps. First we show that $\cP'$ has a satisfying assignment if and only $\cP'$ does and then prove the same connection for $\cP'$ and $\cP^=$. 

{\it Finite-dimensional case.}
Let $\cP=(V,U_d,\cC)$ and let $\{A_v\}_{v\in V}$ be an
  operator assignment that is satisfying for $\cP$ on a finite-dimensional
  Hilbert space $\cH$. We may assume that $\cH=\zC^p$ for some positive integer
  $p$
  %
  and thus $\{A_v\}_{v\in V}$ are $p\times p$ matrices. We will
  construct an operator assignment that is satisfying for $\cP'$; it will be an
  operator assignment on the same space and an extension of the original
  assignment.
 
  Given a constraint $\ang{(u_1,\ldots,u_r),\rel}\in\cC$, the operators
  $\{A_{u_i}\}_{1\leq i\leq r}$ pairwise commute by assumption. Since they are
  also normal, by Theorem~\ref{the:SST} there is a unitary matrix $U$ such that
  $E_i=UA_{u_i}U^{-1}$ is a diagonal matrix for each $i\in [r]$. Since $A_{u_i}^d=I$,
  we have $E_{u_i}^d=I$. Thus every diagonal entry $E_{u_i}(jj)$ belongs to
  $U_d$. Since $P_\rel(A_{u_1},\ldots,A_{u_r})=I$, we have $P_\rel(E_{u_1},\ldots,E_{u_r})=UP_\rel(A_{u_1},\ldots,A_{u_r})U^{-1}=I$ by Lemma~\ref{lem:matrix-polys}. As
  $E_{u_i}$ is a diagonal matrix, we have that
  $P_\rel(E_{u_1}(jj),\ldots,E_{u_r}(jj))=1=\lambda_0$ for every $j\in [p]$.
  Thus the tuple $(E_{u_1}(jj),\ldots,E_{u_r}(jj))\in\rel$ for every $j\in [p]$. 
  For each $j\in [p]$, let $b(j)=(b_1(j),\ldots,b_s(j))\in U_d^s$ be a tuple of
  witnesses to the existentially quantified variables $y_1,\ldots,y_s$ in the
  formula pp-defining $\rel$ in~(\ref{eq:pp-formula}) when $E_{u_i}(jj)$ is
  substituted for $x_i$. For $i\in [s]$, let $B_i$ be the diagonal matrix with
  $B_i(jj)=b_i(j)$ for every $j\in [d]$. Now let $C_i=U^{-1}B_iU$. Since $U$ is
  unitary we have $U^{-1}=U^*$ and thus each $C_i$ is 
  normal:
  $C_iC_i^*=U^*B_iU(U^*B_iU)^*=U^*B_iUU^*B_i^*U
  =U^*B_iB_i^*U
  =U^*B_i^*B_iU
  =U^*B_i^*UU^*B_iU
  =(U^*B_iU)^*U^*B_iU=C_i^*C_i$.
  Since $b_i(j)\in U_d$, we have $C_i^d=I$. Also
  $E_1,\ldots,E_r,B_1,\ldots,B_s$ pairwise commute since they are all diagonal
  matrices. Hence, $A_i,\ldots,A_r,C_1,\ldots,C_s$ also pairwise commute since
  they all are simultaneously similar via $U$.
  %
  As each conjunct in~(\ref{eq:pp-formula}) is satisfied by the assignment
  sending $x_i$ to $E_{u_i}(jj)$ and $y_i$ to $b_i(j)$ for all $j\in [d]$, we
  can conclude that the matrices that are assigned to the variables in the
  conjuncts make the corresponding polynomial evaluate to $I$. But
  this means that the assignment to the variables in the block of the constraint
  $\ang{(u_1,\ldots,u_r),\rel}$ makes a satisfying operator assignment for the
  constraint of $\cP'$ that has come from the conjunct. As different constraints involving $\rel$ in $\cP$
  produce their own sets of fresh variables, the operator assignments do not
  affect each other.

\smallskip

{\it Infinite-dimensional case.}
Let $\cP=(V,U_d,\cC)$ and let $\{A_v\}_{v\in V}$ be a collection of bounded
  normal linear operators on a Hilbert space $\cH$ that is satisfying for $\cP$. We need to define bounded
normal linear operators for the new variables of $\cP'$ that were introduced in
  the gadget construction. We define these operators
simultaneously for all variables $t_1,\dots,t_s$ that come from the same constraint $\ang{(u_1,\ldots,u_r),\rel}\in\cC$ of $\cP$.

By the commutativity condition of satisfying operator assignments, the operators
  $A_1,\dots,A_r$ pairwise commute. Each $A_i$ is normal, and so the General Strong Spectral Theorem (cf. Theorem~\ref{the:general-sst}) applies.
Thus, there exist a measure space $(\Omega,\cM,\mu)$, a unitary map $U:\cH\to L^2(\Omega,\mu)$, and functions $a_1,\dots,a_r\in L^\infty(\Omega,\mu)$ such
that, for the multiplication operators $E_i = T_{a_i}$ of $L^2(\Omega,\mu)$, the relations $A_i = U^{-1} E_iU$ hold for each $i\in [r]$. Equivalently,
$U A_iU^{-1} = E_i$. From $A_i^d=I$ we conclude $E_i^d=I$. Hence, $a_i(\omega)^d = 1$ for almost all $\omega\in\Omega$; i.e., formally $\mu(\{\omega\in\Omega\mid a_i(\omega)^d\ne1\}) = 0$. Thus, $a_i(\omega)\in U_d$ for almost all $\omega\in\Omega$. The conditions of Lemma~\ref{lem:matrix-polys} apply, thus $P_R(E_1,\dots, E_r)=U^{-1}P_R (A_1,\dots, A_r)U=1$. Now, $E_i$ is the multiplication operator given by $a_i$, and $a_i(\omega)\in U_d$ for almost all $\omega\in\Omega$, so $P_R (a_1(\omega),\dots,a_r(\omega)) = 1$ for almost all $\omega\in\Omega$. Thus the tuple $a(\omega) = (a_1(\omega),\dots,a_r(\omega))$ belongs to the relation
$R$ for almost all $\omega\in\Omega$. Now we are ready to define the operators for the variables $t_1,\dots,t_s$.

For each $\omega\in\Omega$ for which the tuple $a(\omega)$ belongs to $R$, let $b(\omega) = (b_1(\omega),\dots, b_s(\omega))\in U_d^s$ be the lexicographically
smallest tuple of witnesses to the existentially quantified variables in $\phi(a_1(\omega),\dots, a_r(\omega))$; such a vector of witnesses
must exist since $\phi$ defines $R$, and the lexicographically smallest exists because $R$ is finite. For every other $\omega\in\Omega$, define
$b(\omega) = (b_1(\omega),\dots, b_s(\omega)) = (0,..., 0)$.

Note that each function $b_k:\Omega\to\zC$ is bounded since its range is in $U_d\cup\{0\}$. We claim that such functions of witnesses
$b_k$ are also measurable functions of $(\Omega,\cM,\mu)$. This will follow from the fact that $a_1,\dots,a_r$ are measurable functions themselves, the fact that $R$ is a finite relation, and the choice of a definite tuple of witnesses of each $\omega\in\Omega$: the lexicographically
smallest if $a(\omega)$ is in $R$, or the all-zero tuple otherwise. 

Since $R$ is finite, the event $Q = \{\omega\in\Omega\mid b_k(\omega) =\sigma\}$, for fixed $\sigma\in U_d\cup\{0\}$, can be expressed as a finite Boolean
combination of events of the form $Q_{i,\tau} = \{\omega\in\Omega\mid a_i(\omega) = \tau\}$, where $i\in[r]$ and $\tau\in U_d$. Here is how: If $\sigma\ne0$, then
\[
Q=\bigcup_{a\in R:b(a)_k=\sigma}\left(\bigcap_{i\in r} Q_{i,a_i}\right),
\]
where $b(a)$ denotes the lexicographically smallest tuple of witnesses in $U_d^s$ for the quantified variables in $\phi(a_1,\dots,a_r)$. If $\sigma = 0$, then $Q$ is the complement of this set. Each $Q_{i,\tau}$ is a measurable set in the measure space $(\Omega,\cM,\mu)$ since
$a_i$ is a measurable function and $Q_{i,\tau} = a^{-1}_i(B_{1/2d}(\tau))$, where $B_{1/2d}(\tau)$ denotes the complex open ball of radius $1/2d$ centered
at $\tau$, which is a Borel set in the standard topology of $\zC$. Since the range of $b_k$ is in the finite set $U_d\cup\{0\}$, the preimage
$b^{-1}_k(S)$ of each Borel subset $S$ of $\zC$ is expressed as a finite Boolean combination of measurable sets, and is thus measurable
in $(\Omega,\cM,\mu)$.

We just proved that each $b_k$ is bounded and measurable, so its equivalence class under almost everywhere equality
is represented in $L^\infty(\Omega,\mu)$. We may assume without loss of generality that $b_k$ is its own representative; else modify it
on a set of measure zero in order to achieve so. Let $F_k = T_{b_k}$ be the multiplication operator given by $b_k$ and let $t_k$ be
assigned the linear operator $B_k = U^{-1} F_kU$, which is bounded because $b_k$ is bounded and $U$ is unitary. Also because $U$
is unitary, each such operator is normal and $B_k^d=I$ since $b_k(\omega)\in U_d$ for almost all $\omega\in\Omega$. Moreover, $E_1,\dots, E_r, F_1,\dots, F_s$ pairwise commute since they are multiplication operators; thus $A_1,\dots, A_r, B_1,\dots, B_s$ pairwise
commute since they are simultaneously similar via $U$. Moreover, as each atomic formula in the quantifier-free part of $\phi$
is satisfied by the mapping that sends $u_i\mapsto a_i(\omega)$ and $t_i\mapsto b_i(\omega)$ for almost all $\omega\in\Omega$, another application of Lemma~\ref{lem:matrix-polys}
shows that the operators that are assigned to the variables of this atomic formula make the corresponding indicator polynomial evaluate to $I$. This means that the assignment to the variables $u_1,\dots, u_r,t_1,\dots,t_s$ makes a satisfying operator assignment for
the constraints of $\cP'$ that come from the constraint $\ang{(u_1,\dots, u_r), R}$ in $\cP$. As different constraints from $\cP$ produce their own sets of
variables $t_1,\dots,t_s$, these definitions of assignments are not in conflict with each other, and the proof of the lemma is complete.

\smallskip

Let $P_=$ be the polynomial interpolating the equality relation, and let $W$ be the set of variables of $\cP'$. As is easily seen, to prove the second step it suffices to prove that for any satisfying operator assignment $\{A_w\}_{w\in W}$ for $\cP'$ if $\ang{(u,v),=_d}$ is a constraint in $\cP'$ then $A_u=A_v$. In the finite-dimensional case let $\{A_w\}_{w\in W}$ be an operator assignment over a $p$-dimensional Hilbert space, $\ang{(u,v),=_d}$ a constraint in $\cP'$, and let $E_u=UA_uU^{-1}, E_v=UA_vU^{-1}$ for some unitary operator $U$ be diagonal operators. Then as before $P_=(E_u(jj),E_v(jj))=1$ for all $j\in[p]$ implying $E_u=E_v$, and hence $A_u=A_v$.

The infinite-dimensional case is similar, except we use the General Strong Spectral Theorem to find a measure space $(\Omega,\cM,\mu)$, a unitary map $U:\cH\to L^2(\Omega,\mu)$, and functions $a,b\in L^\infty(\Omega,\mu)$ such
that, for the multiplication operators $E_u = T_a, E_v=T_b$ of $L^2(\Omega,\mu)$, the relations 
$U A_uU^{-1} = E_u, UA_vU^{-1}=E_v$ hold. Since $P_=(A_u,A_v)=I$, we conclude that $a(\omega)=b(\omega)$ for almost all $\omega\in\Omega$ implying $E_u=E_v$, and therefore $A_u=A_v$. 
\end{proof}

We do not know whether the converse of Lemma~\ref{lem:lift} holds; this is not known even in the case of $d=2$~\cite{AKS19:jcss}.
%
The obvious idea would be to take the restriction of the operator assignment
that is satisfying for $\cP'$ but it is not clear why this should be satisfying
for $\cP$, because there is no guarantee that the operators assigned to variable in the scope of a constraint of $\cP$ of the form $\ang{\bs, \rel}$ commute. However, under a slight technical assumption on $\Gm$ --- namely,
that it includes the full binary relation $\rel_T$ on $U_d$\footnote{This is a special case of the
so-called \emph{commutativity gadget}~\cite{AKS19:jcss}; the ``T'' stands for
trivial.\\ Also, although the full binary relation can be pp-defined from the equality relation, it does not help to prove Theorem~\ref{the:pp}\,(2) as the full binary relation is needed exactly to deal with pp-definitions.} --- one can
enforce commutativity within a constraint scope and thus project an operator
assignment. 

For an instance $\cP'$ as defined above (and in the statement of Lemma~\ref{lem:lift}), we denote by $\cP''$ the instance obtained from $\cP'$ by adding, for every constraint $\ang{(\vc ur),\rel}$ of $\cP$, constraints of the form $\ang{(u_i,u_j),\rel_T}$ for every $i\neq j\in [r]$.
%
To prove Theorem~\ref{the:pp}\,(2), it suffices to show the following lemma.

\begin{lemma}\label{lem:proj}
  Let $\Gm$ be a constraint language over $U_d$ with $\rel_T\in\Gm$ and let $\rel$ be pp-definable
  from $\Gm$. Furthermore, let $\cP\in\CSP(\Gm\cup\{\rel\})$ and let
  $\cP''\in\CSP(\Gm)$ be defined as above.  Then, we have the following:\\[2mm]
  %
  (1) For every satisfying operator assignemnt for $\cP$ on a Hilbert
  space $\cH$ there is an extension that is a satisfying operator
  assignment for $\cP''$ on $\cH$.\\[2mm]
  %
  (2) For every satisfying operator assignment for $\cP''$ on $\cH$, the
  restriction of it onto the variables of $\cP$ is a satisfying operator
  assignment for $\cP$ on $\cH$.
  %
\end{lemma}
%
Indeed, if $\CSP(\Gm\cup\{\rel\})$ has a satisfiability gap of the
third kind then there is an instance $\cP\in\CSP(\Gm\cup\{\rel\})$ that is not
satisfiable via finite-dimensional operators but has a satisfying
infinite-dimensional operator assignment. By~Lemma~\ref{lem:proj}, $\cP''$ is
not satisfiable via finite-dimensional operators but has a satisfying
infinite-dimensional operator assignment. Hence $\cP''$ establishes that
$\CSP(\Gm\cup\{\rel_T\})$ has a satisfiability gap of the third kind, as required to prove Theorem~\ref{the:pp}\,(2).
%
\begin{proof}[Proof of Lemma~\ref{lem:proj}]
%
(1) follows from Lemma~\ref{lem:lift}: The satisfying operator assignment for $\cP'$ constructed in the proof of Lemma~\ref{lem:lift} is also a satisfying operator assignment or $\cP''$. Indeed, the constraints already present in $\cP'$ are by assumption satisfied in $\cP''$. Regarding the extra constraints in $\cP''$ not present in $\cP'$, each such constraint involves the $\rel_T$ relation and the two variables in the scope of the constraint come from the block of some constraint in $\cP'$. The proof of Lemma~\ref{lem:lift} established that the constructed operators pairwise commute on these variables. Also, as $P_{\rel_T}(a,b)=1$ for any $a,b\in U_d$, the polynomial constraints are satisfied as they evaluate to $I$.

For~(2), take a satisfying operator assignment for $\cP''$ and consider its restriction $\{A_v\}_{v\in V}$ onto the variables of $\cP$. Any two operators whose variables appear within the scope of some constraint of $\cP$ necessarily commute since the two variables are in the block of some constraint in $\cP''$. It remains to show that the polynomial constraints of $\cP$ are satisfied, that is, that $P_\rel(A_{u_1},\ldots,A_{u_r})=I$ for every constraint $\ang{(u_1,\ldots,u_r),\rel}$ of $\cP$.
%
For this, we use Lemma~\ref{lem:lemma-3}.
Let $\phi$ be the formula pp-defining $\rel$ as in~(\ref{eq:pp-formula}). We
  define several polynomials over variables $x_1,\ldots,x_r,y_1,\ldots,y_s$ that
  correspond to the variables in~(\ref{eq:pp-formula}). For every $i\in [m]$,
  let $Q_i$ be the polynomial $P_{\rel_i}(\bz_i)-1$ so that the equation $Q_i=0$
  ensures $P_{\rel_i}(\bz_i)=1$, where $P_{\rel_i}$ is the characteristic
  polynomial of $\rel_i$, and $\bz_i$ is the tuple of variables from
  $x_1,\ldots,x_r,y_1,\ldots,y_s$ that correspond to the variables of the same
  name that appear in the conjunct $\rel_i(\bz_i)$ of~(\ref{eq:pp-formula}).
Let $Q$ be the polynomial $P_\rel(x_1,\ldots,x_r)-1$, where $P_\rel$ is the characteristic polynomial of $\rel$. By the construction of the polynomials and the choice of $\phi$, every assignment over $U_d$ that satisfies all equations $Q_1=\cdots=Q_m=0$ also satisfies $Q=0$. By Lemma~\ref{lem:lemma-3}, we get $P_\rel(A_{u_1},\ldots,A_{u_r})-I=0$, as required.
%
\end{proof}

