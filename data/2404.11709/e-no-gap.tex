In this section we prove the first part of our main result.

\begin{theorem}\label{the:no-gap}
Let $\Gm$ be a constraint language over $U_d$. If $\CSP(\Gm)$ has bounded
  width then it has no satisfiability gap of any kind.
\end{theorem}

The main idea behind the proof
of Theorem~\ref{the:no-gap} is to simulate the inference provided by SLAC 
by inference in polynomial equations. Let $\cS$ be a SLAC-program solving $\CSP(\Gm)$.
%
In order to prove Theorem~\ref{the:no-gap} we take an instance $\cP=(V,U_d,\cC)$
of $\CSP(\Gm)$ that has no solution, and therefore is not SLAC-consistent, as
$\CSP(\Gm)$ has bounded width, and prove that it also has no satisfying operator
assignment. This rules out a gap of the second kind, and thus also a gap of any
kind (cf. the discussion in Section~\ref{sec:operator-CSP}). We will prove it by contradiction, assuming $\cP$ has a satisfying operator assignment $\{A_v\}$ and then using the rules of a SLAC-program solving $\CSP(\Gm)$ to infer stronger and stronger conditions on $\{A_v\}$ that eventually lead to a contradiction. We start with a series of lemmas that will help to express the restrictions on $\{A_v\}$.

The following lemma introduces a restriction that is satisfied by any operator assignment. 

\begin{lemma}\label{lem:whole-domain-poly}
Let $\cP=(V,U_d,\cC)\in\CSP(\Gm)$. For any operator assignment $\{A_v\}$
for $\cP$ we have
\[
\prod_{k=0}^{d-1}(\ld_k I-A_v)=0.
\]
\end{lemma}
%
\begin{proof}
Note that the equation $\prod_{k=0}^{d-1}(\ld_k-x)=0$ is true for any
$x\in U_d$, that is, it follows from the empty set of equations. By
Lemma~\ref{lem:lemma-3}, it also holds for any fully commuting operator 
assignment. However, as the equation contains only one variable any 
operator assignment is fully commuting, and the result follows.
\end{proof}

Recall that every rule of a SLAC-program has the form $(x\in S)\meet\rel(x,y,\vc zr))\to (y\in S')$ for some variables $x,y\in V$, a constraint $\ang{(x,y,\vc zr),\rel}$, and sets $S,S'\sse U_d$. Therefore, we need to show how to encode unary relations and rules of a SLAC-program through polynomials. For any $S\sse U_d$, we represent the unary constraint restricting the domain of a variable
$x$ to the set $S$ by the polynomial
\[
Dom_S(x)=\prod_{k\in S}(\ld_k-x)+1.\footnote{This is not the representation of $S$ as in the beginning of Section~\ref{sec:operator-CSP}, as $Dom_S(a)$ is not necessarily equal to $\ld_1$ for $a\not\in S$. However, it suffices for our purposes, because we only need the property that $Dom_S(a)=1$ if and only if $a\in S$.}
\]
%
Similarly, the rule $(x\in S)\meet\rel(x,y,\vc zr))\to (y\in S')$ of the SLAC 
program is represented by
\[
Rule_{S,\rel,S'}(x,y,\vc zr)=(Dom_{\ov S}(x)-1)(P_\rel(x,y,\vc zr)-\ld_1)
(Dom_{S'}(y)-1).
\]
As the next lemma shows, any operator assignment is a zero of $Rule_{S,\rel,S'}$.

\begin{lemma}\label{lem:rule-poly}
Let $\cP=(V,U_d,\cC)\in\CSP(\Gm)$. For any operator assignment $\{A_v\}$
for $\cP$ and any rule $(x\in S)\meet\rel(x,y,\vc zr))\to (y\in S')$ of the SLAC program for $\CSP(\Gm)$ we have
\begin{eqnarray*}
\lefteqn{Rule_{S,\rel,S'}(A_x,A_y,A_{z_1}\zd A_{z_r}) }\\
&=& (Dom_{\ov S}(A_x)-I)
(P_\rel(A_x,A_y,A_{z_1}\zd A_{z_r})-\ld_1I)(Dom_{S'}(A_y)-I)=0.
\end{eqnarray*}
\end{lemma}
%
\begin{proof}
Note that the equation $Rule_{S,\rel,S'}(x,y,\vc zr)=0$ is true for any
$x,y,\vc zr\in U_d$, that is, it follows from the empty set of equations. By
Lemma~\ref{lem:lemma-3}, it also holds for any fully commuting operator 
assignment. However, as all the variables $x,y,\vc zr$ belong to the scope
of the same constraint, the operators $A_x,A_y,A_{z_1}\zd A_{z_r}$ 
pairwise commute.  The result follows.
\end{proof}

Now, assume $\cP=(V,U_d,\cC)$ is not SLAC-consistent and $D_v$ denote the domain of $v\in V$ obtained after establishing SLAC-consistency. This means that for 
some $v\in V$ there is a derivation of $D_v=\eps$ using only facts
$\rel(\bs)$ for $\ang{\bs,\rel}\in\cC$ and 
$T_B(x_i)\meet\rel(\vc xk)\to T_C(x_j)$ for the rules of the SLAC-program. Moreover, this derivation can be subdivided into sections
of the form $(v=a)\to(v\ne a)$, each of which is linear. The latter 
condition means that each such section looks like a chain
$(v=a)\to(v_1\in S_1)\to\dots\to(v_\ell\in S_\ell)\to(v\in D_v-\{a\})$,
where each step is by a rule of the form 
$((v_i\in S_i)\meet\rel_i(v_i,v_{i+1},\vc ur))\to(v_{i+1}\in S_{i+1})$.

\begin{lemma}\label{lem:derivation-poly}
For any satisfying operator assignment $\{A_v\}$ for $\cP$ and any rule $(x\in S)\meet\rel(x,y,\vc zr))\to (y\in S')$ of the SLAC program for $\CSP(\Gm)$ if 
$\rel(x,y,\vc zr)\in\cC$ then 
\[
(Dom_{\ov S}(A_x)-I)(Dom_{S'}(A_y)-I)=I.
\]
\end{lemma}
%
\begin{proof}
By Lemma~\ref{lem:rule-poly}, the equation 
\[
(Dom_{\ov S}(A_x)-I)(P_\rel(A_x,A_y,A_{z_1}\zd A_{z_r})-\ld_1I)
(Dom_{S'}(A_y)-I)=0
\]
holds as well as the equation 
\[
P_\rel(A_x,A_y,A_{z_1}\zd A_{z_r})-I=0.
\]
Multiplying the latter one by $(Dom_{\ov S}(A_x)-I)$ on the left, and 
by $(Dom_{S'}(A_y)-I)$ on the right and subtracting it from the first equation
we obtain
\[
-(Dom_{\ov S}(A_x)-I)(1-\ld_1)I(Dom_{S'}(A_y)-I)=0.
\]
The result follows.
\end{proof}

\begin{lemma}\label{lem:transitive-poly}
Let $(v_1\in S_1)\to\dots\to(v_\ell\in S_\ell)$ be a derivation in the SLAC-program $\cS$ and $\{A_v\}$ a satisfying operator assignment for $\cP$. 
%
Then for each $i=2\zd\ell$
\[
(Dom_{\ov S_1}(A_{v_1})-I)(Dom_{S_i}(A_{v_i})-I)=0.
\]
\end{lemma}
%
\begin{proof}
We proceed by induction on $i$. For $i=2$ the equation holds by 
Lemma~\ref{lem:derivation-poly}.
%
In the induction case we have equations
\begin{equation}\label{equ:equation1}
(Dom_{\ov S_1}(A_{v_1})-I)(Dom_{S_i}(A_{v_i})-I)=0,
\end{equation}
and 
\begin{equation}\label{equ:equation2}
(Dom_{\ov S_i}(A_{v_i})-I)(Dom_{S_{i+1}}(A_{v_{i+1}})-I)=0.
\end{equation}
The idea is to multiply (\ref{equ:equation1}) by 
$(Dom_{S_{i+1}}(A_{v_{i+1}})-I)$ on the right, multiply (\ref{equ:equation2})
by $(Dom_{\ov S_1}(A_{v_1})-I)$ on the left and subtract. The problem
is, however, that 
\[
Dom_{S_i}(A_{v_i})-Dom_{\ov S_i}(A_{v_i})
\]
is not a constant polynomial. So, we also need to prove that any polynomial
of the form 
\[
Dom_S(x)-Dom_{\ov S}(x)
\]
is invertible modulo $x^d-1$. The polynomial has the form
\[
p(x)=\prod_{k\in S}(x-\ld_k)-\prod_{k\not\in S}(x-\ld_k).
\]
As is easily seen, assuming that the product of an empty set of factors equals~1, $\ld_k$ is not a
root of $p(x)$ for any $\ld_k\in U_d$. Therefore the greatest common divisor of $p(x)$ and $x^d-1$ has degree 0, and hence there exists $q(x)$ such that 
\[
p(x)q(x)=c+r(x)(x^d-1).
\]

Thus before subtracting equations (\ref{equ:equation1}) and 
(\ref{equ:equation2}) we also multiply them by $q(A_{v_i})$. Then we get
\begin{eqnarray*}
(Dom_{\ov S_1}(A_{v_1})-I)(Dom_{S_i}(A_{v_i})q(A_{v_i})
-q(A_{v_i})Dom_{\ov S_i}(A_{v_i}))(Dom_{S_{i+1}}(A_{v_{i+1}})-I) &=& 0\\
(Dom_{\ov S_1}(A_{v_1})-I)q(A_{v_i})(Dom_{S_i}(A_{v_i})
-Dom_{\ov S_i}(A_{v_i}))(Dom_{S_{i+1}}(A_{v_{i+1}})-I) &=& 0\\
c(Dom_{\ov S_1}(A_{v_1})-I)(Dom_{S_{i+1}}(A_{v_{i+1}})-I) &=& 0.
\end{eqnarray*}
The first transformation uses the fact that $A_{v_i}$ commutes with itself, while the 
second one uses the property $A_{v_i}^d=I$.
The result follows.
\end{proof}

\begin{proof}[Proof of Theorem~\ref{the:no-gap}]
To complete the proof of Theorem~\ref{the:no-gap} note that the lack of
SLAC-consistency means that for some $v\in V$ the statement 
$(v=\ld_k)\to(v\ne\ld_k)$ can be derived from $\cP$ for every $\ld_k\in U_d$.
By Lemma~\ref{lem:transitive-poly}, for any operator assignment $\{A_w\}$
and any $\ld_k\in U_d$ the operator $A_v$ satisfies the equation
\[
\prod_{j\ne k}(A_v-\ld_jI)=0.
\]
We show that for any $S\sse U_d$ these equations imply 
\[
\prod_{j\in S}(A_v-\ld_jI)=0.
\]
Then for $S=\eps$ we get $I=0$, witnessing that $\cP$ has no satisfying 
operator assignment.

We proceed by (reverse) induction on the size of $S$. Suppose the statement is true 
for all sets of size $r$ and let $S\sse U_d$ be such that $|S|=r-1$. Without loss fo
generality, assume that
$S=\{\ld_0\zd\ld_{r-1}\}$. Let $S_1=S\cup\{\ld_r\}, S_2=S\cup\{\ld_{r+1}\}$.
Consider
\begin{equation}\label{equ:equation3}
\prod_{j\in S_1}(A_v-\ld_jI)=0
\end{equation}
and
\begin{equation}\label{equ:equation4}
\prod_{j\in S_2}(A_v-\ld_jI)=0.
\end{equation}
Subtracting (\ref{equ:equation4}) from (\ref{equ:equation3}) we obtain
\[
(A_v-\ld_rI-A_v+\ld_{r+1}I)\prod_{j=0}^{r-1}(A_v-\ld_jI)=
(\ld_{r+1}-\ld_r)I\prod_{j=0}^{r-1}(A_v-\ld_jI)=0,
\]
implying the equation for $S$.
\end{proof}
