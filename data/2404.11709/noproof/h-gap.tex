In this section we prove the second part of our main result: we show that one
can ``implement'' linear equations over an Abelian group of prime order $p$
in $\CSP(\Gm)$ provided that $\CSP(\Gm)$ does not have bounded width. The rest
of Theorem~\ref{thm:main-informal}\,(2b) will then follow from the existence of a
satisfiability gap of the second kind~\cite{SZ24:personal} and, for $p=2$, the
existence of a satisfiability gap of the first
kind~\cite{Mermin1990simple,Mermin1993hidden,Peres1990incompatible} and the
third kind~\cite{Slofstra20:jams}, as all these gap instances are just systems
of linear equations. Moreover, for NP-hard CSPs one can ``implement'' linear
equations over $\mathbb{Z}_2$, and thus obtains satisfiabiilty gaps of all three
kinds, as claimed in Theorem~\ref{thm:main-informal}\,(1).

We achieve our goal in several steps via a chain of reductions that has been
used since the inception of the algebraic method to the
CSP~\cite{Bulatov05:classifying}. While more direct constructions have been
developed later, see, e.g., \cite{BKW17}, we find this original approach to be
better suited for operator CSPs.


%%%%%%%%%%%%%%%%%%%%%%%%%%%%%%%%%%%%%
\subsection{Bounded width and Abelian groups}

We start by introducing several definitions. 

A constraint language $\Gm$ over $U_d$ is said to be a \emph{core language} if
its every endomorphism is a permutation.
This term comes from finite model theory where it is used for relational structures that do not have endomorphisms (homomorphisms to themselves) that are not automorphisms. Such structures, and therefore languages, have a number of useful properties that we will exploit later. The standard way to convert a constraint language $\Gm$ into a core language is to repeat the following procedure until the resulting language is a core language: Pick an endomorphism $\vr$ of $\Gm$ that is not a permutation and set 
\[
\vr(\Gm)=\{\vr(\rel)\mid \rel\in\Gm\}, \quad\text{where } \vr(\rel)=\{(\vr(a_1)\zd \vr(a_n))\mid  (\vc an)\in\rel\}.
\]
There always exists an endomorphism $\vr$ of $\Gm$ such that $\vr(\Gm)$ is
core~\cite{Bulatov05:classifying,BKW17} and $\vr$ is \emph{idempotent}, that is, $\vr\circ\vr=\vr$. We will denote this core language by $\core(\Gm)$, as it (up to an isomorphism) does not depend on the choice of $\vr$. Note that the fact that $\vr$ is idempotent implies that it acts as identity on its image.

The language $\Gm$ is called \emph{idempotent} if it contains all the \emph{constant} relations, that is, relations of the form $C_a=\{(a)\}$, $a\in U_d$. For an arbitrary language $\Gm$ over $U_d$ we use $\Gm^*=\Gm\cup\{C_a\mid a\in U_d\}$.  A unary relation (a set) $B\sse U_d$ pp-definable in $\Gm$ is called a \emph{subalgebra} of $\Gm$. For a subalgebra $B$ we introduce the \emph{restriction} $\Gm\red B$ of $\Gm$ to $B$ defined as follows
 \[
\Gm\red B=\{\rel\cap B^{ar(\rel)}\mid \rel\in\Gm\}.
\]


An equivalence relation $\th$ pp-definable in $\Gm$ is said to be a \emph{congruence} of $\Gm$. The equivalence class of $\th$ containing $a\in U_d$ will be denoted by $a\fac\th$, and the set of all equivalence classes, the \emph{factor-set}, by $U_d\fac\th$. Congruences of a constraint language allow one to define a \emph{factor-language} as follows. For a congruence $\th$ of the language $\Gm$ the factor language $\Gm\fac\th$ is the language over $U_d\fac\th$ given by 
\[
\Gm\fac\th=\{\rel\fac\th\mid \rel\in\Gm\}, \quad\text{where } \rel\fac\th=\{(a_1\fac\th\zd a_n\fac\th)\mid  (\vc an)\in\rel\}.
\]

In order to fit core languages, subalgebras, and factor-languages in our framework where the domain is the set of roots of unity, we let $e=|\vr(U_d)|$, $e=|B|$ or $e=|U_d\fac\th|$, respectively, arbitrarily choose a bijection $\pi:\vr(U_d)\to U_e$, $\pi:B\to U_e$, and $\pi:U_d\fac\th\to U_e$, and replace $\vr(\Gm)$, $\Gm\red B$, and $\Gm\fac\th$ with $\pi(\vr(\Gm))$, $\pi(\Gm\red B)$, and $\pi(\Gm\fac\th)$, respectively. 

All the languages above are connected with each other in terms of the reducibility of the corresponding CSPs, as Figure~\ref{fig:reductions} and the following statements indicate.

\begin{prop}[\cite{Bulatov05:classifying,BKW17}]\label{pro:reductions}
Let $\Gm$ be a constraint language over $U_d$. Then
\begin{itemize}
\item[(1)]
  $\CSP(\Gm)$ and $\CSP(\core(\Gm))$ are polynomial-time interreducible.
\item[(2)]
If $\Gm$ is a core language, $\CSP(\Gm)$ and $\CSP(\Gm^*)$ are polynomial-time interreducible.
\item[(3)]
If $B$ is a subalgebra of $\Gm$ then $\CSP(\Gm\red B)$ is polynomial-time reducible to $\CSP(\Gm)$.
\item[(4)]
If $\th$ is a congruence of $\Gm$ then $\CSP(\Gm\fac\th)$ is polynomial-time reducible to $\CSP(\Gm)$.
\end{itemize}
\end{prop}

\begin{figure}
\[
\CSP(\Gm)\ \leftrightarrow\ \CSP(\core(\Gm)) \leftrightarrow\ \CSP(\core(\Gm)^*)\ \leftarrow\ \CSP(\core(\Gm)^*\red B) \ \leftarrow\ \CSP(\core(\Gm)^*\red B\fac\th)
\]
\caption{Reductions between CSPs corresponding to derivative languages}\label{fig:reductions}
\end{figure}

Finally, to relate the reductions above with bounded width we apply the
following result that can be extracted from the known results on the algebraic
approach to CSPs~\cite{Bulatov09:width,Barto14:local,BKW17}. Also, the dichotomy
conjecture of Bulatov and Zhuk~\cite{Bulatov17:focs,Zhuk20:jacm} implies that
any NP-hard CSP can implement linear equations over $\mathbb{Z}_2$.

\begin{prop}[\cite{Bulatov09:width,Barto14:local,BKW17}]\label{pro:abelian}
For a constraint language $\Gm$ over $U_d$, $\CSP(\Gm)$ does not have bounded
  width if and only there exists a language $\Dl$ pp-definable in $\Gm$, a
  subalgebra $B$ of $\core(\Dl)^*$, a congruence $\th$ of $\core(\Dl)^*\red B$,
  and an Abelian group $\zA$ of prime order $p$ such that $\Gamma'=\core(\Dl)^*\red B\fac\th$ contains relations $\rel_{3,a},\rel_{p+2}$ for every $a\in\zA$ given by 
\[
\rel_{3,a}=\{(x,y,z)\mid x+y+z=a\},\quad\text{and}\quad \rel_{p+2}=\{(\vc a{p+2})\mid a_1+\dots+a_{p+2}=0\}.\footnote{The relations $\rel_{3,a},\rel_{p+2}$ are chosen here because they are needed for our purpose. In fact, they can be replaced with any relations expressible by linear equations over $\zA$.}
\]
  Moreover, if $\CSP(\Gm)$ is NP-hard then one can take $p=2$.
\end{prop}

We now have everything to formally state our second main result.

\begin{theorem}\label{the:hsp-gap}
  Let $\Gm$ be a constraint language over $U_d$ such that $\CSP(\Gm)$ does not
  have bounded width. Furthermore, let $\Gm'$ be the language guaranteed by
  Proposition~\ref{pro:abelian}. Then, if $\CSP(\Gm')$ has a satisfiability gap
  of the first or the second kind then so does $\CSP(\Gm)$.  Finally, if
  $\CSP(\Gm')$ has a satisfiability gap of the third kind then so does
  $\CSP(\Gm\cup\{\rel_T\})$.
\end{theorem}


%%%%%%%%%%%%%%%%%%%%%%%%%%%%%%%%%%%%%
\subsection{Proof of Theorem~\ref{the:hsp-gap}}

In this section we prove that the connections shown in
Figure~\ref{fig:reductions} hold in terms of satisfiability gaps, thus proving
Theorem~\ref{the:hsp-gap}. We start with a helpful observation. As any mapping on a finite set of complex numbers can be interpolated by a polynomial, we may apply such mappings to operators as well (we assume that such an interpolating polynomial is of the lowest degree possible, and so is unique). A polynomial $\vr$ is said to \emph{interpolate} a set $B\sse U_d$ if $\vr(\ld)=1$ if $\ld\in B$ and $\vr(\ld)=0$ if $\ld\in U_d-B$.

\begin{lemma}[Finite-dimensional case]\label{lem:mapping}
Let $d,e\in\nat$, $e\le d$. 
Let $\pi:U_e\to U_d$ be an injective mapping and $\vr$ a unary polynomial that interpolates $B=\Im(\pi)$.  Let $\vc Ak$ be pairwise commuting normal operators of order $e$ on a finite dimensional Hilbert space. Then $C_i=\pi(A_i)$, $i\in[k]$, are pairwise commuting normal operators of order $d$, and $\vr(C_i)=I$. Conversely, let $\vc Ck$ be pairwise commuting normal operators of order $d$ such that $\vr(C_i)=I$. Then for $A_i=\pi^{-1}(C_i)$, $i\in[k]$, it holds that the $A_i$'s are pairwise commuting normal operators of order $e$, and the eigenvalues of the $C_i$'s belong to $B$.
\end{lemma}

\begin{proof}\textcolor{red}{TOPROVE 0}\end{proof}

\begin{lemma}\label{lem:polynomial-image}
Let $\vr$ be a polynomial, $(\Omega,\cM,\mu)$ a measure space, and $c\in L^\infty(\Omega,\mu)$ with a finite range. Then $\vr(c)$ is bounded and measurable.
\end{lemma}

\begin{proof}\textcolor{red}{TOPROVE 1}\end{proof}

\begin{lemma}[Infinite-dimensional case]\label{lem:mapping-infinite}
Let $d,e\in\nat$, $e\le d$. 
Let $\pi:U_e\to U_d$ be an injective mapping and $\vr$ a unary polynomial that interpolates $B=\Im(\pi)$.  Let $\vc Ak$ be pairwise commuting bounded normal operators of order $e$ on a Hilbert space. Then $C_i=\pi(A_i)$, $i\in[k]$, are pairwise commuting bounded normal operators of order $d$, and $\vr(C_i)=I$. Conversely, let $\vc Ck$ be pairwise commuting bounded normal operators of order $d$ such that $\vr(C_i)=I$. Then for $A_i=\pi^{-1}(C_i)$, $i\in[k]$, it holds that the $A_i$'s are pairwise commuting bounded normal operators of order $e$, and there exist a measure space $(\Omega,\cM,\mu)$, a unitary map $U:\cH\to L^2(\Omega,\mu)$ and functions $c_1,\dots,c_k\in L^\infty(\Omega,\mu)$ such that $C_i = U^{-1} T_{c_i}U$ for $i\in [k]$ and the multiplication operators $T_{c_i}$ of $L^2(\Omega,\mu)$, and $\mu(\{\omega\in\Omega\mid c_i(\omega)\not\in B\}) = 0$ for $i\in[k]$.
\end{lemma}

\begin{proof}\textcolor{red}{TOPROVE 2}\end{proof}


Next we consider the four connections from Figure~\ref{fig:reductions} one by
one and prove that each of them preserves the satisfiability gap. For Steps
2--4, which rely on pp-definitions, a gap of the third kind is only preserved up
to the addition of $\rel_T$ to the language, as in the statement of
Theorem~\ref{the:hsp-gap}.

\smallskip

{\bf Step 1 (Reductions to a core).}
Let $\Gm$ be a constraint language over the set $U_d$ and $\vr:U_d\to U_d$ an idempotent
endomorphism of $\Gm$. Then $\CSP(\Gm)$ has a satisfiability gap if and only if $\CSP(\vr(\Gm))$ does.

\smallskip

Suppose that $|\Im(\vr)|=e$, let $\pi':\Im(\vr)\to U_e$ be any bijection between $U_e$ and $\Im(\vr)$, and $\pi=\pi'\circ\vr$. Let $\Dl=\{\pi(\rel)\mid \rel\in\Gm\}$, we show that $\CSP(\Dl)$ has a satisfiability gap if and only if $\CSP(\Gm)$ does.

Let $\cP=(V,U_e,\cC)$ be a gap instance of $\CSP(\Dl)$, and let $\cP^\pi=(V,U_d,\cC^\pi)$ be the corresponding instance of $\CSP(\Gm)$, where for each $\ang{\bs,\rel}\in\cC$ the set $\cC^\pi$ includes $\ang{\bs,\relo}$ with $\relo\in\Gm$ and $\pi(\relo)=\rel$. As is easily seen, $\cP^\pi$  has no solution, because for any solution $\vf$ of $\cP^\pi$ the mapping $\pi\circ\vf$ is a solution of $\cP$. 

{\it Finite-dimensional case.}
Let $\{A_v\mid v\in V\}$ be an $\ell$-dimensional satisfying operator assignment for $\cP$. We prove that $\{\pi'^{-1}(A_v)\mid v\in V\}$ is a satisfying operator assignment for $\cP^\pi$. Let $C_v=\pi'^{-1}(A_v)$.

By Lemma~\ref{lem:mapping}, the $C_v$'s are normal, $C_v^d=I$, $v\in V$,
and for any constraint $\ang{\bs,\rel}\in\cC$ and any $v,w\in\bs$, $C_v,C_w$ commute.
Now, let $\ang{\bs,\rel}\in\cC$, $\bs=(\vc vk)$, $\ang{\bs,\relo}$ be the
corresponding constraint of $\cP^\pi$, and $f_\rel,f_\relo$ be the polynomials representing $\rel,\relo$ over $U_e,U_d$ respectively. We have $f_\rel(A_{v_1}\zd A_{v_k})=I$, and we need to show that $f_\relo(C_{v_1}\zd C_{v_k})=I$. Let $U$ diagonalize $A_{v_1}\zd A_{v_k}$ and
\[
UA_{v_i}U^{-1}=\left(\begin{array}{ccc}\mu_{i1}&\dots&0\\ &\ddots&\\ 0&\dots&\mu_{i\ell}\end{array}\right).
\]
Then $(\mu_{1j}\zd\mu_{kj})\in\rel$ for $j\in[\ell]$, because
\[
I=f_\rel(A_{v_1}\zd A_{v_k})=f_\rel(UA_{v_1}U^{-1}\zd UA_{v_k}U^{-1})= \left(\begin{array}{ccc}f_\rel(\mu_{11}\zd\mu_{k1})&\dots&0\\ &\ddots&\\ 0&\dots&f_\rel(\mu_{1\ell}\zd\mu_{k\ell})\end{array}\right).
\]
Therefore, $(\pi'^{-1}(\mu_{1j})\zd\pi'^{-1}(\mu_{kj}))\in\vr(\relo)$ for $j\in[\ell]$, and, as $\vr$ is an endomorphism, $(\pi'^{-1}(\mu_{1j})\zd\pi'^{-1}(\mu_{kj}))\in\relo$ and so $f_\relo(\pi'^{-1}(\mu_{1j})\zd\pi'^{-1}(\mu_{kj}))=\ld_0=1$. Since 
\[
UC_{v_i}U^{-1}=U\pi'^{-1}(A_{v_i})U^{-1}=\pi'^{-1}(UA_{v_i}U^{-1})=\left(\begin{array}{ccc}\pi'^{-1}(\mu_{i1})&\dots&0\\ &\ddots&\\ 0&\dots&\pi'^{-1}(\mu_{i\ell})\end{array}\right),
\]
we also have $f_\relo(C_{v_1}\zd C_{v_k})=\ld_0I=I$.

{\it Infinite-dimensional case.}
Let $\{A_v\mid v\in V\}$ be an (infinite-dimensional) satisfying operator assignment for $\cP$. We prove that $\{\pi'^{-1}(A_v)\mid v\in V\}$ is a satisfying operator assignment for $\cP^\pi$. Let $C_v=\pi'^{-1}(A_v)$.

By Lemma~\ref{lem:mapping-infinite}, the $C_v$'s are normal, $C_v^d=I$, $v\in V$,
and for any constraint $\ang{\bs,\rel}\in\cC$ and any $v,w\in\bs$, $C_v,C_w$ commute.
Now, let $\ang{\bs,\rel}\in\cC$, $\bs=(\vc vk)$, $\ang{\bs,\relo}$ be the
corresponding constraint of $\cP^\pi$, and $f_\rel,f_\relo$ be the polynomials representing $\rel,\relo$ over $U_e,U_d$ respectively. We have $f_\rel(A_{v_1}\zd A_{v_k})=I$, and we need to show that $f_\relo(C_{v_1}\zd C_{v_k})=I$. By the General Strong Spectral Theorem (cf. Theorem~\ref{the:general-sst}) there exist a measure space $(\Omega,\cM,\mu)$, a unitary map $U:\cH\to L^2(\Omega,\mu)$ and functions $a_1,\dots,a_k\in L^\infty(\Omega,\mu)$ such that, for the multiplication operators $E_i = T_{a_i}$ of $L^2(\Omega,\mu)$, the relations $A_{v_i} = U^{-1} E_iU$ hold for each $i\in [k]$. As $A_{v_i}^e=I$, it holds that $a_i(\omega)\in U_e$ for almost all $\omega\in\Omega$. Choosing a different representative of the equivalence class of $a_i$ we may assume that $a_i(\omega)\in U_e$ for all $\omega\in\Omega$. Then $(a_1(\omega)\zd a_k(\omega))\in\rel$ for almost all $\omega\in\Omega$, because
\[
I=Uf_\rel(A_{v_1}\zd A_{v_k})U^{-1}=f_\rel(UA_{v_1}U^{-1}\zd UA_{v_k}U^{-1})= f_\rel(T_{a_i}\zd T_{a_k})=T_{f_\rel(a_i\zd a_k)},
\]
implying $f_\rel(a_i\zd a_k)(\omega)=1$ for almost all $\omega\in\Omega$. Therefore, $(\pi'^{-1}(a_1(\omega))\zd \pi'^{-1}(a_k(\omega)))\in\vr(\relo)$ for almost all $\omega\in\Omega$, and, as $\vr$ is an endomorphism, $(\pi'^{-1}(a_1(\omega))\zd \pi'^{-1}(a_k(\omega)))\in\relo$ and so $f_\relo(\pi'^{-1}(a_1(\omega))\zd \pi'^{-1}(a_k(\omega)))=\ld_0=1$ for almost all $\omega\in\Omega$. Since 
\[
UC_{v_i}U^{-1}=U\pi'^{-1}(A_{v_i})U^{-1}=\pi'^{-1}(UA_{v_i}U^{-1})=\pi'^{-1}(T_{a_i})=T_{\pi'^{-1}(a_i)},
\]
we also have $f_\relo(C_{v_1}\zd C_{v_k})=\ld_0I=I$. Moreover, $\pi'^{-1}(a_i)\in L^\infty(\Omega,\mu)$ by Lemma~\ref{lem:polynomial-image}.

\smallskip

Now, let $\cP^\pi=(V,U_d,\cC^\pi)$ be a gap instance of $\CSP(\Gm)$ and $\cP=(V,U_e,\cC)$, $\cC=\{\ang{\bs,\rel}\mid\ang{\bs,\relo}\in\cC^\pi,\rel=\pi(\relo)\}$, the corresponding instance of $\CSP(\Dl)$. Then again $\cP$ has no solution over $U_e$.

{\it Finite-dimensional case.}
Let $\{C_v\mid v\in V\}$ be an $\ell$-dimensional satisfying operator assignment for $\cP^\pi$. We need to prove that $\{A_v\mid v\in V\}$, $A_v=\pi(C_v)$ is a satisfying operator assignment for $\cP=(V,\cC)$. For $\ang{\bs,\rel}\in\cC$, $\bs=(\vc vk)$, let $f_\rel,f_\relo$ be polynomials representing $\rel$ and $\relo\in\Gm$ with $\pi(\relo)=\rel$, respectively. 

First, we show that the $C_v$'s can be replaced with $\vr(C_v)$. Since for any
$\ang{\bs,\relo}\in\cC^\pi$, $\bs=(\vc vk)$, and any $\vc ak\in U_d$, we have
that $f_\relo(\vr(a_1)\zd\vr(a_k))=1$ whenever $f_\relo(\vc ak)=1$, by Lemma~\ref{lem:lemma-3} $f_\relo(\vr(C_{v_1})\zd\vr(C_{v_k}))=I$ whenever $f_\relo(C_{v_1}\zd C_{v_k})=I$. Thus, we assume $A_v=\pi'(C_v)$ for $v\in V$.

That $A_v$ is normal, $A_v^e=I$, $v\in V$, and the $A_v$'s locally commute follows from Lemma~\ref{lem:mapping}. Let $f^\pi_\relo(\vc xk)=f_\relo(\pi'^{-1}(x_1)\zd\pi'^{-1}(x_k))$. As is easily seen, for any $\vc ak\in U_e$, if $f^\pi_\relo(\vc ak)=1$ then $f_\rel(\vc ak)=1$. Hence, by Lemma~\ref{lem:lemma-3} $f_\rel(A_{v_1}\zd A_{v_k})=I$ whenever $f^\pi_\relo(A_{v_1}\zd A_{v_k})=I$. Finally, we prove that $\pi'^{-1}(A_v)=C_v$; this implies that $f^\pi_\relo(A_{v_1}\zd A_{v_k})=I$ completing the proof. Let $U$ diagonalize $C_v$ and 
\[
UC_vU^{-1}=\left(\begin{array}{ccc}\mu_1&\dots&0\\ &\ddots&\\ 0&\dots&\mu_\ell\end{array}\right).
\]
Then
\begin{eqnarray*}
\pi'^{-1}(A_v) &=& \pi'^{-1}(\pi'(C_v))=U^{-1}U\pi'^{-1}(\pi'(C_v))U^{-1}U=U^{-1}\pi'^{-1}(\pi'(UC_vU^{-1}))U\\
&=& U^{-1}\left(\begin{array}{ccc}\pi'^{-1}(\pi'(\mu_1))&\dots&0\\ &\ddots&\\ 0&\dots&\pi'^{-1}(\pi'(\mu_\ell))\end{array}\right)U
= U^{-1}\left(\begin{array}{ccc}\mu_1&\dots&0\\ &\ddots&\\ 0&\dots&\mu_\ell\end{array}\right)U=C_v.
\end{eqnarray*}

{\it Infinite-dimensional case.}
Let $\{C_v\mid v\in V\}$ be a (infinite-dimensional) satisfying operator assignment for $\cP^\pi$. We need to prove that $\{A_v\mid v\in V\}$, $A_v=\pi(C_v)$ is a satisfying operator assignment for $\cP=(V,\cC)$. For $\ang{\bs,\rel}\in\cC$, $\bs=(\vc vk)$, let $f_\rel,f_\relo$ be polynomials representing $\rel$ and $\relo\in\Gm$ with $\pi(\relo)=\rel$, respectively. As in the finite-dimensional case it can be shown that the $C_v$'s can be replaced with $\vr(C_v)$, and therefore we assume $A_v=\pi'(C_v)$ for $v\in V$.

That $A_v$ is normal, $A_v^e=I$, $v\in V$, and the $A_v$'s locally commute follows from Lemma~\ref{lem:mapping-infinite}. As before, for any $\vc ak\in U_e$, if $f^\pi_\relo(\vc ak)=1$ then $f_\rel(\vc ak)=1$, implying by Lemma~\ref{lem:lemma-3} that $f_\rel(A_{v_1}\zd A_{v_k})=I$ whenever $f^\pi_\relo(A_{v_1}\zd A_{v_k})=I$. Finally, we prove that $\pi'^{-1}(A_v)=C_v$; this implies that $f^\pi_\relo(A_{v_1}\zd A_{v_k})=I$ completing the proof. By the General Strong Spectral Theorem (cf. Theorem~\ref{the:general-sst}) there exist a measure space $(\Omega,\cM,\mu)$, a unitary map $U:\cH\to L^2(\Omega,\mu)$ and functions $c_1,\dots,c_k\in L^\infty(\Omega,\mu)$ such that, for the multiplication operators $D_i = T_{c_i}$ of $L^2(\Omega,\mu)$, the relations $C_{v_i} = U^{-1} D_iU$ hold for each $i\in [k]$. As before, the $c_i$'s can be chosen such that $c_i(\omega)\in U_d$ for all $\omega\in\Omega$. Then by Lemma~\ref{lem:polynomial-image} $\pi'^{-1}(\pi'(c_i))\in L^\infty(\Omega,\mu)$ and 
\begin{align*}
\pi'^{-1}(A_v) &= \pi'^{-1}(\pi'(C_v))=U^{-1}U\pi'^{-1}(\pi'(C_v))U^{-1}U\\
&=U^{-1}\pi'^{-1}(\pi'(UC_vU^{-1}))U=U^{-1}\pi'^{-1}(\pi'(T_{c_i}))U=U^{-1}T_{\pi'^{-1}(\pi'(c_i))}U.
\end{align*}
Since $\pi'^{-1}(\pi'(c_i(\omega)))=c_i(\omega)$ for almost all $\omega\in\Omega$, we obtain $U^{-1}T_{\pi'^{-1}(\pi'(c_i))}U=C_{v_i}$.


\medskip

{\bf Step 2  (Adding constant relations).}
Let $\Gm$ be a core language. Then $\CSP(\Gm)$ has a satisfiability gap if and only if $\CSP(\Gm^*)$ does.

\smallskip

Since $\Gm\sse\Gm^*$, if $\CSP(\Gm)$ has a satisfiability gap, so does
$\CSP(\Gm^*)$. We prove that if $\CSP(\Gm^*)$ has a satisfiability gap then
$\CSP(\Gm)$ has a satisfiability gap.

We will use the following relation $\rel_\Gm$ that is known to be pp-definable in $\Gm$ \cite{Jeavons99:expressive}: 
\[
\rel_\Gm=\{(\vr(\ld_0)\zd\vr(\ld_{d-1}))\mid \text{$\vr$ is an endomorphism of $\Gm$}\}.
\]
As $\Gm$ is a core language, for any $(\vc ad)\in\rel_\Gm$ it holds that $\{\vc
ad\}=U_d$. By Theorem~\ref{the:pp} we may assume that $\rel_\Gm\in\Gm$.

Let $\cP=(V,U_d,\cC)$ be a gap instance of $\CSP(\Gm^*)$. We construct an instance $\cP'=(V',U_d,\cC')$ of $\CSP(\Gm)$ as follows. 
\begin{itemize}
    \item 
    $V'= V \cup \{v_a | a \in U_d \}$;
    \item
    $\cC'$ consists of three parts: $\{C=\ang{\bs,\rel}\in\cC\mid \rel\in\Gm\}$, $\{\ang{(v_{a_1}\zd v_{a_n}),\rel_\Gm}\}$, and  $\{\ang{(v,v_a),=_d}\mid \ang{(v),C_a}\in\cC\}$, where $=_d$ denotes the equality relation on $U_d$.
\end{itemize}

It is known \cite{Jeavons99:expressive} that $\cP'$ has a classic solution if and only if $\cP$ has one. If $\vf:V\to U_d$ is a solution of $\cP$ then we can extend it to a solution of $\cP'$ by mapping $v_a$ to $a$. Conversely, let $\vf:V'\to U_d$ be a solution of $\cP'$. The restriction of $\vf$ to $V$ may not be a solution of $\cP$, because for some constraint $\ang{(v),C_a}\in\cC$ it may be the case that $\vf(v)=\vf(v_a)\ne a$. This can be fixed as follows.  As $\vf$ is a solution, $(\vf(v_{\ld_0})\zd \vf(v_{\ld_{d-1}}))\in\rel_\Gm$, hence, the mapping $\vr:U_d\to U_d$ given by $\vr(a)=\vf(v_a)$ is an endomorphism of $\Gm$.  As $\Gm$ is a core language, $\vr$ is a permutation on $U_d$ and $\vr^s$ is the identity permutation for some $s$. Then $\vr^{s-1}$ is the inverse $\vr^{-1}$ of $\vr$ and is also an endomorphism of $\Gm$. Therefore $\vf'=\vr^{-1}\circ\vf$ is also a solution of $\cP'$ and $\vf'(v_a)=a$ for $a\in U_d$. Thus, $\vf'\red V$ is a solution of $\cP$.

\smallskip

{\it Finite-dimensional case.}
Suppose that $\{A_v\mid v\in V\}$ is an $\ell$-dimensional satisfying operator assignment for $\cP$. First, we observe that if $\cC$ contains a constraint $\ang{(v),C_a}$ then $A_v$ is the scalar operator $aI$. Indeed, let $f_a(x)$ be a polynomial representing $C_a$, that is,  $f_a(a)=\ld_0$ and $f_a(b)=\ld_1$ for $b\in U_d-\{a\}$. Let also $U$ be a unitary operator that diagonalizes $A_v$ and $\vc\mu\ell$ the eigenvalues of $A_v$. Then, as $f_a(A_v)=I$ we obtain
\[
I=UIU^{-1}=Uf_a(A_v)U^{-1}=f_a(UA_vU^{-1})=
\left(\begin{array}{ccc}f_a(\mu_1))&\dots&0\\ &\ddots&\\ 0&\dots&f_a(\mu_\ell))\end{array}\right)
\]
implying that $f_a(\mu_i)=\ld_0$ for $i\in[\ell]$. Thus, $a$ is the only eigenvalue of $A_v$ and 
\[
A_v=U^{-1}aIU=aI.
\]
All such operators pairwise commute regardless of the value of $a$. Therefore $v_a$ can be assigned $aI$ for $a\in U_d$, and the resulting assignment is a satisfying operator assignment for $\cP'$.

\smallskip

{\it Infinite-dimensional case.}
Suppose that $\{A_v\mid v\in V\}$ is a (infinite-dimensional) satisfying operator assignment for $\cP$. As before, we observe that if $\cC$ contains a constraint $\ang{(v),C_a}$ then $A_v$ is the scalar operator $aI$. Let $f_a(x)$ be a polynomial representing $C_a$, that is,  $f_a(a)=\ld_0$ and $f_a(b)=\ld_1$ for $b\in U_d-\{a\}$. By the General Strong Spectral Theorem (cf. Theorem~\ref{the:general-sst}) there exist a measure space $(\Omega,\cM,\mu)$, a unitary map $U:\cH\to L^2(\Omega,\mu)$ and a function $c\in L^\infty(\Omega,\mu)$ such that, for the multiplication operator $E = T_c$ of $L^2(\Omega,\mu)$, the relation $A_v = U^{-1} EU$ holds. By Lemma~\ref{lem:lemma-3} 
\[
f_a(E)=f_a(UA_vU^{-1})=Uf_a(A_v)U^{-1}=UIU^{-1},
\]
implying $f_a(c(\omega))=1$ for almost all $\omega\in\Omega$. Therefore, $A_v=U^{-1}EU=U^{-1}aIU=aI$. We then complete the proof as in the finite-dimensional case.

\medskip

{\bf Step 3 (Satisfiability gap from subalgebras).}
Let $\Gm$ be a constraint language over the set $U_d$ and let $B$ be its subalgebra.
Then if $\CSP(\Gm\red B)$ has a satisfiability gap then so does $\CSP(\Gm)$.

\smallskip

Let $\Dl=\Gm\red B$. Then by Theorem~\ref{the:pp} we may assume $\Dl\sse\Gm$ and $B\in\Gm$. Let $e=|B|$ and $\pi:U_e\to U_d$ a bijection between $U_e$ and $B$. 

Let $\cP=(V,U_e,\cC)$ be a gap instance of $\CSP(\pi^{-1}(\Dl))$ and the instance $\cP^\pi=(V,U_d,\cC^\pi)$ constructed as follows. For every $\ang{\bs,\rel}\in\cC$ the instance $\cP^\pi$ contains $\ang{\bs,\pi(\rel)}$. As is easily seen, $\cP^\pi$ has no classic solution. Therefore, it suffices to show that for any finite- or infinite-dimensional satisfying operator assignment $\{A_v\mid v\in V\}$ for $\cP$, the assignment $C_v=\pi(A_v)$ is a satisfying operator assignment for $\cP^\pi$.

By Lemmas~\ref{lem:mapping},~\ref{lem:mapping-infinite}, the $C_v$'s are normal, satisfy the condition $C_v^d=I$, locally commute, and bounded in the infinite-dimensional case. For $\ang{\bs,\rel}\in\cC$, $\bs=(\vc vk)$, let $f^\pi_\rel(\vc xk)=f_\rel(\pi^{-1}(x_1)\zd\pi^{-1}(x_k))$. As in \textbf{Step 1}, it can be shown that $\pi^{-1}(C_v)=A_v$, and therefore $f^\pi_\rel(C_{v_1}\zd C_{v_k})=I$. For any $\vc ak\in U_d$, if $(\vc ak)\in\pi(\rel)$ then $\vc ak\in B$. Therefore, if $f^\pi_\rel(\vc ak)=1$ then $f_{\pi(\rel)}(\vc ak)=1$. By Lemma~\ref{lem:lemma-3} this implies $f_{\pi(\rel)}(C_{v_1}\zd C_{v_k})=I$.

\smallskip

{\bf Step 4 (Satisfiability gap from homomorphic images).}
Let $\Gm$ be a constraint language over the set $U_d$ and $\th$ a congruence of
$\Gm$.  Then if $\CSP(\Gm\fac\th)$ has a satisfiability gap then so does $\CSP(\Gm)$.

\smallskip

Let $\vr:U_d\to U_d\fac\th$ be the natural mapping $a\mapsto a\fac\th$ and $\pi':U_d\fac\th\to U_e$, where $e=|U_d\fac\th|$, a bijection. Finally, let $\pi=\pi'\circ\vr$ and $\Dl=\pi(\Gm)$. For $\rel\in\Dl$ let $\pi^{-1}(\rel)$ be the full preimage of $\rel$ under $\pi$. Since $\th$ is pp-definable in $\Gm$, so is $\pi^{-1}(\rel)$ for any $\rel\in\Dl$. Indeed, if $\rel=\pi(\relo)$ for some $\relo\in\Gm$, then 
\[
\pi^{-1}(\rel)(\vc xk)=\exists\vc yk\ \ \relo(\vc yk)\wedge\bigwedge_{i\in[k]}\th(x_i,y_i).
\]
Using Theorem~\ref{the:pp} we may assume that $\pi^{-1}(\rel)\in\Gm$ for $\rel\in\Dl$. Let $\pi^*:U_e\to U_d$ assign to $a\in U_e$ a representative of the $\th$-class $\pi'^{-1}(a)$. Thus, in a certain sense, $\pi^*$ is an inverse of $\pi$.  

Suppose that $\cP=(V,U_e,\cC)$ is a gap instance of $\CSP(\Dl)$ and let $\cP^\pi=(V,U_d,\cC^\pi)$ be given by $\cC^\pi=\{\ang{\bs,\pi^{-1}(\rel)}\mid \ang{\bs,\rel}\in\cC\}$. We prove that $\cP^\pi$ is a gap instance of $\CSP(\Gm)$. Firstly, observe that $\cP^\pi$ has no classic solution, because for any solution $\vf$ of $\cP^\pi$ the mapping $\pi\circ\vf$ is a solution of $\cP$. Let $\{A_v\mid v\in V\}$ be a satisfying operator assignment for $\cP$. We set $C_v=\pi^*(A_v)$ and prove that $\{C_v\mid v\in V\}$ is a satisfying operator assignment for $\cP^\pi$. By Lemma~\ref{lem:mapping},~\ref{lem:mapping-infinite}, the $C_v$'s are normal, satisfy the condition $C_v^d=I$, bounded, and locally commute. For $\ang{\bs,\rel}\in\cC$, $\bs=(\vc vk)$, let $f^\pi_\rel(\vc xk)=f_\rel(\pi(x_1)\zd\pi(x_k))$. As before, it is easy to see that $\pi(C_v)=A_v$ for $v\in V$, and therefore $f^\pi_\rel(C_{v_1}\zd C_{v_k})=I$. Finally, for any $(\vc ak)\in U_d$, if $f^\pi_\rel(\vc ak)=f_\rel(\pi(a_1)\zd\pi(a_k))=1$, then $(\pi(a_1)\zd\pi(a_k))\in\rel$, and so $(\vc ak)\in\pi^{-1}(\rel)$ and $f_{\pi^{-1}(\rel)}(\vc ak)=1$. By Lemma~\ref{lem:lemma-3} this implies that $f_{\pi^{-1}(\rel)}(C_{v_1}\zd C_{v_k})=I$.


