Symmetry leads to efficient computation. This phenomenon has
manifested itself in several research areas that have one aspect in common, namely a
model of computation with local constraints that restrict the solution space of
the problem of interest. An elegant way to describe such problems is in the
framework of Constraint Satisfaction Problems (CSPs).
%
CSPs have driven some of the most influential
developments in theoretical computer science, from NP-completeness to the PCP
theorem to semidefinite programming algorithms to the Unique Games Conjecture.
The mathematical structure of tractable decision
CSPs~\cite{Bulatov17:focs,Zhuk20:jacm}, infinite-domain CSPs~\cite{Bodirsky2021complexity,Bodirsky23:jacm},  optimization CSPs~\cite{tz16:jacm}, as well as approximable
CSPs~\cite{Raghavendra08:everycsp,Brown-CohenRaghavendra15}, is now known to be linked to certain
forms of higher-order symmetries of the solution spaces.
%
A recently emerging research direction links 
CSPs with foundational
topics in physics and quantum
computation~\cite{Cleve14:icalp,Cleve17:jmp-perfect,AKS19:jcss,Paddock23:arxiv,Mancinska20:focs}.

\paragraph{Constraint Satisfaction Problems}
%
CSPs capture some of the most fundamental
computational problems, including graph and hypergraph colorings, linear
equations, and variants and generalizations of satisfiability. Informally, one
is given a set of variables and a set of constraints, each depending only on
constantly many variables. Given a CSP instance, the goal is to find an assignment
of values to all the variables so that all constraints are satisfied. For
example, if the domain is $\{r,g,b\}$ and the constraints are of the form
$R(x,y)$, where $R=\{(r,g),(g,r),(g,b),(b,g),(r,b),(b,r)\}$ is the binary disequality relation on $\{r,g,b\}$, we obtain
the classic graph \textsc{3-Colorability}
problem. If the domain is $\{r,g,b,o\}$ and the
constraints are of the form $R(x,y)$, where
$R=\{(r,g),(g,r),(g,b),(b,g),(b,o),(o,b),(o,r),(r,o)\}$,
we obtain a variant of
the graph \textsc{4-Colorability} problem in which adjacent vertices must be
assigned different colors and, additionally, red and blue vertices must not be adjacent
and green and orange vertices must not be adjacent. 

Back in 1978, Schaefer famously classified all Boolean CSPs as solvable in
polynomial time or NP-hard~\cite{Schaefer78:stoc}.\footnote{We call a CSP
\emph{Boolean} if the domain of all variables is of size two. Some papers call
such CSPs binary. We use the term \emph{binary} for a relation of arity two and
for CSPs whose constraints involve binary relations.} The
tractable cases are the standard textbook problems, namely 
\textsc{2-SAT}, \textsc{Horn-SAT}, \textsc{dual
Horn-SAT}, and system of linear equations on a two-element set.\footnote{There
are also two trivial, uninteresting cases called 0-valid and 1-valid.} 
Hell and Ne\v{s}et\v{r}il studied a special case of CSPs known
as graph homomorphisms~\cite{hell2004graphs}. These are CSPs in which all constraints
involve the same binary symmetric relation, i.e., a graph. The above-mentioned 
\textsc{3-Colorability} problem is the homomorphism problem to $K_3$,
the undirected clique on three vertices, say $\{r,g,b\}$. The above-mentioned
variant of \textsc{4-Colorability} is the homomorphism problem to $C_4$, the
undirected cycle on four vertices, say $\{r,g,b,o\}$. Generalizing greatly
the classic result of Karp that \textsc{$k$-Colorability} is solvable in polynomial time for $k\leq 2$ and NP-hard
for $k\geq 3$ and other concrete problems such as the (tractable) variant of
\textsc{4-Colorability} above, Hell and Ne\v{s}et\v{r}il obtained in
1990 a complete classification of such CSPs~\cite{HellN90}.
%
Motivated by these two results and the quest to identify the largest subclass of
NP that could exhibit a dichotomy and thus avoid NP-intermediate cases, Feder
and Vardi famously conjectured that \emph{all} CSPs on finite domains admit a
dichotomy; i.e., are either solvable in polynomial time or are
NP-hard~\cite{Feder98:monotone}.

Following the so-called algebraic approach to CSPs, pioneered by Jeavons, Cohen
and Gyssens~\cite{Jeavons97:closure} and Bulatov, Jeavons, and
Krokhin~\cite{Bulatov05:classifying}, the conjecture was resolved in the
affirmative in 2017 by Bulatov~\cite{Bulatov17:focs} and, independently,
by Zhuk~\cite{Zhuk17:focs,Zhuk20:jacm}. The algebraic approach allows for a very clean and
precise characterization of what makes certain CSPs computationally tractable
--- this is captured by the notion of polymorphisms, which can be thought of as
multivariate symmetries of solutions spaces of CSPs.
%
Along the way to the resolution of the Feder-Vardi dichotomy conjecture, the
algebraic approach has been successfully used to
establish other results about CSPs, 
e.g., characterizing the power of local consistency
algorithms for CSPs~\cite{Bulatov09:width,Barto14:local}, characterizing robustly solvable
CSPs~\cite{Guruswami12:tght,Dalmau13:toct,Barto16:sicomp}, classifying the
complexity of exact optimization CSPs~\cite{tz16:jacm,Kolmogorov17:sicomp}, the
tremendous progress on classifying the complexity of CSPs on infinite
domains~\cite{Bodirsky2021complexity}, and very recently using SDPs for robustly
solving certain promise CSPs~\cite{BGS23:stoc}.

\paragraph{Operator Constraint Satisfaction Problems}

Consider the following instance of a Boolean CSP, consisting in nine variables $x_1,\ldots,x_9$ over the Boolean domain $\{-1,+1\}$ and the following six
constraints:

\begin{equation}\label{eq:magic}
  \begin{split}
    x_1 x_2 x_3 = +1, \qquad x_1 x_4 x_7 = +1,\\
    x_4 x_5 x_6 = +1, \qquad x_2 x_5 x_8 = +1,\\
    x_7 x_8 x_9 = +1, \qquad x_3 x_6 x_9 = -1.
  \end{split}
\end{equation}

Graphically, this system of equations can be represented by a square, where each
equation on the left of~(\ref{eq:magic}) comes from a row, and each equation on the right
of~(\ref{eq:magic}) comes from a column.

\begin{center}
{\large
\begin{tabular}{|c|c|c|}
 \hline
  $x_1$ & $x_2$ & $x_3$ \\\hline 
  $x_4$ & $x_5$ & $x_6$ \\\hline 
  $x_7$ & $x_8$ & $x_9$ \\\hline
\end{tabular}
}
\begin{tabular}{c}
    $+1$\\
    $+1$\\
    $+1$
\end{tabular}
\\[-0.7em]
\begin{tabular}{ccc}
  $+1$ & $+1$ & $-1$
\end{tabular}
\phantom{\begin{tabular}{c} $+1$\\ $+1$\\ $+1$ \end{tabular}}
\end{center}
\vspace*{-1.2em}
%
The system of equations~(\ref{eq:magic}) has no solution in the Boolean domain
$\{-1,+1\}$: By multiplying the left-hand sides of all equations we get $+1$
because every variable occurs twice in the system and $x_i^2=+1$ for every
$1\leq i\leq 9$. However, by multiplying the right-hand sides of all equations,
we get $-1$. Note that this argument used implicitly the assumption that the
variables commute pairwise even if they do not appear in the same equation,
which is true over $\{-1,+1\}$.
Thus, this argument does not hold if one assumes that 
only variables occurring
in the same equation commute pairwise. In fact, Mermin famously established that
the system~(\ref{eq:magic}) has a solution consisting of linear operators on a
Hilbert space of dimension four~\cite{Mermin1990simple,Mermin1993hidden} and the
construction is now know as the Mermin-Peres magic
square~\cite{Peres1990incompatible}. This construction proves the
Bell-Kochen-Specker theorem on the impossibility to explain quantum
mechanics via hidden variables~\cite{Bell1966problem,Kochen67}. 

Any Boolean CSP instance, just like the one above, can be associated with a certain non-local game with two
players, say Alice and Bob, who are unable to communicate while the game is in
progress. Alice is given a constraint at
random and must return a satisfying assignment to the variables in the
constraint. Bob is given a variable from the constraint and must return an
assignment to the variable. The two players win if they assign the same value to
the chosen variable. With shared randomness, Alice and Bob can play the game
perfectly if and only if the instance is satisfiable~\cite{Cleve14:icalp}.
%
The Mermin-Peres
construction~\cite{Mermin1990simple,Peres1990incompatible}
was the first example of an instance where the players can play perfectly by
sharing an entangled quantum state although the instance is not satisfiable. We
note that~\cite{Mermin1990simple,Peres1990incompatible} were looking at quantum contextuality scenarios rather than
non-local games and it was Aravind who reformulated the construction in the above-described
game setting~\cite{aravind2002bell}, cf. also~\cite{Cleve04:ccc}. The game was introduced for any Boolean
CSP by Cleve and Mittal~\cite{Cleve14:icalp}
and further studied by Cleve, Liu, and Slofstra~\cite{Cleve17:jmp-perfect}.

Every Boolean relation can be identified with its characteristic function, which
has a unique representation as a multilinear polynomial via the Fourier
transform. The multilinear polynomial representation of Boolean relations (and
thus also Boolean CSPs) makes it possible to consider \emph{relaxations} of
satisfiability in which the variables take values in a suitable space,
rather than in $\{-1,+1\}$. Such relaxations of satisfiability have been
considered in the  foundations of physics long ago, playing an important role in
our understanding of the differences between classical and quantum theories.
%
In detail, given a Boolean CSP instance, a classical assignment assigns to every
variable a value from $\{-1,+1\}$. An operator assignment assigns to every
variable a linear operator $A$ on a Hilbert space so that $A^2=I$
and each $A$ is self-adjoint, i.e., $A=A^*$ and thus in particular $A$ is
normal, meaning that $AA^*=A^*A$.\footnote{For finite-dimensional Hilbert
spaces, a linear operator is just, up to a choice of basis, a matrix with
complex entries, and the adjoint $A^*$ of $A$ is the conjugate transpose of $A$,
also denoted by $A^*$. For
infinite-dimensional Hilbert spaces, the notions are more involved, cf.
Section~\ref{sec:preliminaries} for all details.}
Furthermore, it is required that operators
assigned to variables from the scope of some constraint should pairwise commute. 

Ji showed that for Boolean CSPs corresponding to \textsc{2-SAT} there is no difference between (classical)
satisfiability and satisfiability via operators~\cite{Ji13:arxiv-binary}.
Later, Atserias, Kolaitis, and Severini established a complete classification for all Boolean CSPs parameterized by the set of
allowed constraint relations. In particular, using the
substitution method~\cite{Cleve14:icalp} they showed that
that only CSPs whose
relations are 0-valid, 1-valid, or come from \textsc{2-SAT}, \textsc{Horn-SAT}, or \textsc{Dual
Horn-SAT} have ``no satisfiability gap'' in the sense that (classic)
satisfiability is equivalent to operator satisfiability (over both finite- and
infinite-dimensional Hilbert spaces)~\cite{AKS19:jcss}. 
For all other Boolean CSPs, Atserias et al.~\cite{AKS19:jcss} showed that there are
satisfiability gaps in the
following sense: There are instances that are not (classically) satisfiable but are
satisfiable via operators on finite-dimensional Hilbert spaces, relying on the Mermin-Peres magic
square --- gaps of the \emph{first kind}.  This immediately implies that
there are instances that are not (classically) satisfiable but are satisfiable
via operators on infinite-dimensional Hilbert spaces --- gaps
of the \emph{second kind}. Finally, there are instances that are not satisfiable via
operators on finite-dimensional Hilbert spaces but are satisfiable via
operators on infinite-dimensional Hilbert spaces, relying on the 
breakthrough result of Slofstra who proved this for linear
equations~\cite{Slofstra20:jams} --- gaps of the \emph{third kind}.
%
The result is
established in~\cite{AKS19:jcss} by showing that reductions between CSPs based on primitive positive
formulas, which preserve complexity and were used to establish Schaefer's
classification of Boolean CSPs~\cite{Schaefer78:stoc}, preserve satisfiability gaps.

\paragraph{Contributions}

As our main contribution, we generalize the result of Atserias et
al.~\cite{AKS19:jcss} from Boolean CSPs to CSPs on \emph{arbitrary finite}
domains.
%
As has been done in, e.g,~\cite{Culf23:arxiv,Goemans04:jcss,Qassim20:jpa}, we represent a
finite domain of size $d$ by the $d$-th roots of unity, and require that each
operator $A$ in an operator assignment should be normal (i.e. $AA^*=A^*A$) and
should satisfy $A^d=I$. The representation of non-Boolean CSPs relies on
multi-dimensional Fourier transform. Our main result is a partial classification
of satisfiability gaps for CSPs on arbitrary finite domains, with the remaining
cases being related to a well-known open problem in the area, which is the
existence of gap instances of the first kind for linear equations over
$\mathbb{Z}_p$ with $p>2$. 

In detail, as our first result we prove that 
\emph{NP-hard} CSPs do have gaps of all three kinds. Second, we show that CSPs of \emph{bounded width} (on arbitrary finite domains) do not have a
satisfiability gap (of any kind), meaning that classical satisfiability is
equivalent to satisfiability via operators on finite- and infinite-dimensional
Hilbert spaces (Theorem~\ref{the:no-gap}). Third, all other CSPs (i.e.,
tractable CSPs of unbounded width) can simulate, in a
precise technical sense, linear equations over an Abelian group of prime order
$p$. We prove that these simulations preserve
satisfiability gaps (Theorem~\ref{the:hsp-gap}).
Thus, we obtain a satisfiability
gap of the second kind by the result from an upcoming paper by Slofstra and
Zhang~\cite{SZ24:personal}.
Moreover, for $p=2$, we obtain a satisfiability
gap of the first kind (and thus also of the second kind) from the Mermin-Peres magic
square~\cite{Mermin1990simple,Mermin1993hidden,Peres1990incompatible}, and a 
satisfiability gap of the third kind from Slofstra's
result~\cite{Slofstra20:jams}. Finally, the reductions will also establish the
first claim, gaps of all three kinds for NP-hard CSPs
via~\cite{Mermin1990simple,Mermin1993hidden,Peres1990incompatible,Slofstra20:jams}.

\begin{theorem}[Main result, informal statement]\label{thm:main-informal}
  Let $\Gm$ be an arbitrary finite set of relations on a finite domain. 
  \begin{enumerate}
  \item If $\CSP(\Gm)$ is NP-hard then $\CSP(\Gm)$ has gaps of all three kinds.
  \item Otherwise, $\CSP(\Gm)$ is solvable in polynomial time and
  \begin{enumerate}
  \item either $\Gm$ has bounded width and $\CSP(\Gm)$ has no gaps of any kind,
  \item or $\CSP(\Gm)$ does not have bounded width and $\CSP(\Gm)$ can simulate linear equations over an Abelian
    group of prime order $p$ in a satisfiability-gap-preserving fashion. In this
    case, $\CSP(\Gm)$ has a gap of the second kind. Furthermore, if $p=2$ then
    $\CSP(\Gm)$ has gaps of all three kinds. 
  \end{enumerate} 
  \end{enumerate}
\end{theorem}

We note that Theorem~\ref{thm:main-informal} implies a classification of Boolean
CSPs, thus recovering the result by Atserias et al.~\cite{AKS19:jcss}. Indeed,
a Boolean language $\Gm$ that is 0-valid, 1-valid, or comes from
\textsc{2-SAT}, \textsc{Horn-SAT}, or \textsc{Dual Horn-SAT} has bounded
width; also, for a Boolean $\Gm$ one gets $p=2$ in
Theorem~\ref{thm:main-informal}\,(2b).

Theorem~\ref{thm:main-informal} also implies
a classification for CSPs on graphs (such CSPs are also known as $H$-Coloring problems): If $\Gm=\{\rel\}$, where $\rel$ is a binary
symmetric relation, then either $R$ is bipartite or not. In the former case,
$\CSP(\Gm)$ has bounded width~\cite{Bulatov18:survey} and thus has no gaps of any kind. In the latter
case, $\CSP(\Gm)$ is NP-hard~\cite{HellN90,Bulatov05:tcs} and thus has gaps of
all three kinds.

\begin{corollary}\label{cor:graphs}
Let $G=(V,E)$ be a graph. If $G$ is bipartite, $\CSP(\{E\})$ has no satisfiability gap of any kind. Otherwise it has satisfiability gaps of all three kinds.
\end{corollary}

We note that the remaining cases not covered by Theorem~\ref{thm:main-informal}
(i.e., $\CSP(\Gm)$ that are tractable but have unbounded width and can simulate
linear equations over an Abelian group of prime order $p>2$ but not of order
$2$) relate to the known difficulties of establishing satisfiability gaps of the
first kind for linear equations over $\mathbb{Z}_p$ with $p>2$, e.g., there are
results showing that such an instance cannot be obtained from generalized Pauli
matrices~\cite{Qassim20:jpa} and beyond~\cite{Frembs22:arxiv}.\footnote{We
remark that our results show that one can obtain gap instances from homomorphic
preimages of already established gap instance (cf. Step~4 in
Section~\ref{sec:gap}).}

\medskip
The proof of Theorem~\ref{thm:main-informal} relies on several ingredients. Firstly, we observe that
results establishing that primitive positive definability preserve
satisfiability gaps~\cite{AKS19:jcss} can be lifted from Boolean to arbitrary finite
domains.\footnote{\cite[Theorem~6.4]{Mastel24:stoc} considers subdivisions, a
special case of conjunctive definitions, which in turn are a special case of
primitive positive definitions (without existential quantifiers). Our result
shows that~\cite[Theorem~6.4]{Mastel24:stoc} extends to conjunctive
definitions.}
%
Secondly, for CSPs of bounded width we show that there is no difference
between classical and operator satisfiability by simulating the inference by the
so-called Singleton Linear Arc Consistency (SLAC) algorithm in polynomial
equations. We note that while there are several (seemingly stronger) algorithms
for CSPs of bounded width, our proof relies crucially on the special structure of SLAC and
the breakthrough result of Kozik that SLAC solves all CSPs of bounded
width~\cite{Kozik21:sicomp}. Thirdly, to prove that NP-hard CSPs and certain CSPs of unbounded width have
satisfiability gaps we use the algebraic approach to CSPs, namely, we show that not
only primitive positive definitions but also other reductions,
namely going to the core, adding constants, and restrictions to subalgebras and
factors, preserve satisfiability gaps.

We note that that there is a significant hurdle to go from Boolean CSPs to CSPs
over arbitrary finite domains. While any non-Boolean CSP can be Booleanized via
indicator variables and extra constraints, such constructions do not
immediately imply classifications of non-Boolean CSPs as the ``encoding
constraints'' are intended to be used in only a particular way. Indeed, while
the complexity of Boolean CSPs was established by Schaefer in
1978~\cite{Schaefer78:stoc} and the complexity of CSPs on three-element domains
was established by Bulatov in 2002~\cite{Bulatov02:focs,Bulatov06:jacm}, the
dichotomy for all finite domains was only established in
2017~\cite{Bulatov17:focs,Zhuk17:focs,Zhuk20:jacm}.
%
Similarly for other variants of CSPs and different notions of tractability,
results on Boolean domains, including the work of Atserias et
al.~\cite{AKS19:jcss}, rely crucially on the explicit knowledge of the structure
of relations on Boolean domains (established by Post~\cite{Post41}), which is not known for non-Boolean CSPs.
%
Indeed, on the tractability side, the structure of relations in tractable Boolean CSPs is simple and very well understood; on the intractability side, reductions based only on primitive positive definitions suffice for a complete classification of Boolean CSPs. Neither of these two facts is true for non-Boolean CSPs.

We find it fascinating that bounded width might be the borderline for
satisfiability gaps for CSPs, thus linking a notion coming from a
natural combinatorial algorithm for CSPs with a foundational topic in quantum
computation. 
%
Indeed, if satisfiability gaps of the first kind exist for linear equations over
$\mathbb{Z}_p$ with $p>2$ then our main result implies that all CSPs of
unbounded width admit satisfiability gaps of all three kinds.
% 
This is yet another result indicating the fundamental nature of of bounded
width, which captures not only the power of the local consistency
algorithm~\cite{LaroseZadori07:au,Maroti2008existence,Bulatov09:width,Barto14:local}
as conjectured in~\cite{Feder98:monotone} with links to Datalog, pebble games,
and logic~\cite{Feder98:monotone,Kolaitis00:jcss}, but also robust solvability
of CSPs~\cite{Barto16:sicomp}, exact solvability of valued CSPs by
LP~\cite{tz17:sicomp} and SDP~\cite{tz18} relaxations, and now also
possibly satisfiability via operators.





\paragraph{Related work}
%
Paddock and Slofstra~\cite{Paddock23:arxiv} 
streamlined the results from~\cite{AKS19:jcss}, and also gave 
an overview of other notions of
satisfiability and their relationship, including the celebrated MIP$^*$=RE
result of Ji et al.~\cite{Ji20:arxiv,Ji22:cacm}.
%
Further, Culf, Mousavi, and Spirig recently studied approximability of operator
CSPs~\cite{Culf23:arxiv}.
%
While we used the magic square to construct a gap of the first kind, other
systems based on linear equations could be used instead, e.g. the pentagram and
the BCS in~\cite{Slofstra20:jams}, cf. also~\cite{Zhang24:arxiv}.
%
Finally, we note that the notion of quantum homomorphism from the work of Man\v{c}inska and
Roberson~\cite{Mancinska16:jctb}, as well as the recent work of
Ciardo~\cite{Ciardo24:lics},
is different from ours as it requires that the
operators $A$ should be idempotent, i.e., $A^2=A$.
(E.g., Ciardo shows~\cite{Ciardo24:lics} that CSPs of
bounded width have no quantum advantage (over finite-dimensional spaces) by establishing a so-called minion
homomorphism --- relying crucially on idempotency --- to the minion capturing the basic SDP relaxation, which is known
to solve CSPs of bounded width by the work of Barto and Kozik~\cite{Barto16:sicomp}. The methods used
in~\cite{Ciardo24:lics} rely, unlike methods and results in this paper, on being in the finite-dimensional case.)

