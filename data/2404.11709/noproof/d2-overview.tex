In this section, we give an overview of how our main result is proved. All
definitions and details are provided in the main part of the paper comprizing of
Sections~\ref{sec:no-gap}--\ref{sec:gap}.

\paragraph{Bounded width}

One part of our main result is the following.

\begin{theorem}\label{the:no-gap-o}
Let $\Gm$ be a constraint language over $U_d$. If $\CSP(\Gm)$ has bounded
width then it has no satisfiability gap of any kind.
\end{theorem}

The main idea behind the proof of Theorem~\ref{the:no-gap-o} 
is to simulate the inference provided by SLAC (see Fig.~\ref{fig:slac}) 
by inference in polynomial equations. Let $\cS$ be a SLAC-program solving $\CSP(\Gm)$.
%
In order to prove Theorem~\ref{the:no-gap-o} we take an instance $\cP=(V,U_d,\cC)$ of $\CSP(\Gm)$ that has no solution, and therefore  is not SLAC-consistent, as $\CSP(\Gm)$ has bounded width, and prove that it also has no operator solution. We will prove it by contradiction, assuming $\cP$ has an operator solution $\{A_v\}$ and then using the rules of a SLAC-program solving $\CSP(\Gm)$ to infer stronger and stronger conditions on $\{A_v\}$ that eventually lead to a contradiction. 

Recall that every rule of a SLAC-program has the form $(x\in S)\meet\rel(x,y,\vc zr))\to (y\in S')$ for some variables $x,y\in V$, a constraint $\ang{(x,y,\vc zr),\rel}$, and sets $S,S'\sse U_d$. Therefore, we need to show how to encode unary relations and rules of a SLAC-program through polynomials. For any $S\sse U_d$, the unary constraint restricting the domain of a variable
$x$ to the set $S$ is represented by the polynomial
\[
Dom_S(x)=\prod_{k\in S}(\ld_k-x)+1.\footnote{This is not the representation of $S$ as in the beginning of Section~\ref{sec:operator-CSP}, as $Dom_S(a)$ is not necessarily equal to $\ld_1$ for $A\not\in S$. However, it suffices for our purposes, because we only need the property that $Dom_S(a)=1$ if and only if $a\in S$.}
\]
%
Similarly, the rule $(x\in S)\meet\rel(x,y,\vc zr))\to (y\in S')$ of the SLAC 
program is represented by
\[
Rule_{S,\rel,S'}(x,y,\vc zr)=(Dom_{\ov S}(x)-1)(P_\rel(x,y,\vc zr)-\ld_1)
(Dom_{S'}(y)-1).
\]
%
To give an idea of how Theorem~\ref{the:no-gap-o} is proved, we sketch the proof of the following.
%
\begin{lemma}\label{lem:transitive-poly-o}
Let $(v_1\in S_1)\to\dots\to(v_\ell\in S_\ell)$ be a derivation in the SLAC-program $\cS$ and $\{A_v\}$ an operator assignment for $\cP$. 
%
Then for each $i=2\zd\ell$
\[
(Dom_{\ov S_1}(A_{v_1})-I)(Dom_{S_i}(A_{v_i})-I)=0.
\]
\end{lemma}
%
\begin{proof}\textcolor{red}{TOPROVE 0}\end{proof}

To complete the proof of Theorem~\ref{the:no-gap-o} note that the lack of
SLAC-consistency means that for some $v\in V$ the statement 
$(v=\ld_k)\to(v\ne\ld_k)$ can be derived from $\cP$ for every $\ld_k\in U_d$.
By Lemma~\ref{lem:transitive-poly-o}, for any operator assignment $\{A_w\}$
and any $\ld_k\in U_d$ the operator $A_v$ satisfies the equation
$\prod_{j\ne k}(A_v-\ld_jI)=0$.
By reverse induction on the size of $S$, one can show that for any $S\sse U_d$ these equations imply 
$\prod_{j\in S}(A_v-\ld_jI)=0$.
Then for $S=\eps$ we get $I=0$, witnessing that $\cP$ has no satisfying 
operator assignment. 

\paragraph{NP-hard CSPs and unbounded width}

The second part of our main result boils down to the following.

\begin{theorem}\label{the:hsp-gap-o}
Let $\Gm$ be a constraint language over $U_d$. 
If $\CSP(\Gm)$ does not have bounded width then $\CSP(\Gm)$ can simulate linear
  equations over an Abelian group of prime order $p$ in a
  satisfiability-gap-preserving fashion. Moreover, if $\CSP(\Gm)$ is NP-hard
  then one can take $p=2$.
\end{theorem}


Assume that $\CSP(\Gm)$ does not have bounded width. The ``implementation'' of
linear equations in $\CSP(\Gm)$ that preserves gaps will be achieved
in several steps via a 
chain of reductions that lies at the heart of the algebraic approach to CSPs~\cite{Bulatov05:classifying}. 
%
The reductions are shown in Figure~\ref{fig:reductions-o}. They are
known to preserve satisfiability; we show that they also preserve satisfiability
gaps. 

The most basic reduction (used in the chain in several places) is that of
primitive positive definitions. 
%
Let $\Gm$ be a constraint language over $U_d$, let $r$ be an integer, and
let $x_1,\ldots,x_r$ be variables ranging over the domain $U_d$. A primitive
positive formula (\emph{pp-formula}) over $\Gm$ is a formula of the form
%
\begin{equation}\label{eq:pp-formula-o}
  \phi(x_1,\ldots,x_r)=\exists y_1\cdots\exists y_s(\rel_1(\bz_1)\wedge\cdots\wedge
  \rel_m(\bz_m)),
\end{equation}
%
where $\rel_i$ is either the binary equality relation on $U_d$ or $\rel_i\in\Gm$ is a relation over $U_d$ of arity $r_i$, and each $\bz_i$ is an $r_i$-tuple of variables from
$\{x_1,\ldots,x_r\}\cup\{y_1,\ldots,y_s\}$.
%
A relation $\rel\subseteq U_d^r$ is primitive positive definable
(\emph{pp-definable}) from $\Gm$ if there
exists a pp-formula $\phi(x_1,\ldots,x_r)$ over $\Gm$ such that $\rel$ is equal to the set of
models of $\phi$, that is, the set of $r$-tuples $(a_1,\ldots,a_r)\in U_d^r$
that make the formula $\phi$ true over $U_d$ if $a_i$ is
substituted for $x_i$ in $\phi$ for every $i\in [r]$.

\begin{figure}[t!]
\[
\CSP(\Gm)\ \leftrightarrow\ \CSP(\core(\Gm)) \leftrightarrow\ \CSP(\core(\Gm)^*)\ \leftarrow\ \CSP(\core(\Gm)^*\red B) \ \leftarrow\ \CSP(\core(\Gm)^*\red B\fac\th)
\]
\caption{Reductions between CSPs corresponding to derivative languages.}
\label{fig:reductions-o}
\end{figure}


\begin{theorem}\label{the:pp-o}
  Let $\Gm$ be a constraint language over $U_d$ and let $R$ be pp-definable from
  $\Gm$. Then, if $\CSP(\Gm\cup\{R\})$ has a satisfiability gap of the first (second, third) kind then so does
  $\CSP(\Gm)$.
\end{theorem}

Weaker forms of Theorem~\ref{the:pp-o} have appeared in the literature. This includes for example subdivisions from Theorem~6.4 of~\cite{Mastel24:stoc}.

Let $\rel\subseteq U_d^r$ be a pp-definable formula
over $\Gm$ via the pp-formula $\phi(x_1,\ldots,x_r)$ as in~(\ref{eq:pp-formula-o}).
Given an instance
$\cP\in\CSP(\Gm\cup\{\rel\})$, one can turn it into an instance
$\cP'\in\CSP(\Gm)$ that is equivalent to $\cP$. 
% 
Intuitively, each constraint $\ang{\bu,\rel}$ of $\cP$ is replaced with
constraints 
from its pp-definition in~(\ref{eq:pp-formula-o})
over fresh new variables (and similarly for the binary equality relation, cf. Section~\ref{sec:operator-pp} for details).
%
This construction is known as the \emph{gadget construction} in the CSP
literature and it is known that $\cP$ has a solution over $U_d$
if and only if $\cP'$ has a solution over $U_d$~\cite{Bulatov05:classifying,BKW17}.
%
Thus, to prove Theorem~\ref{the:pp-o}, it suffices to show the following;
the proof is a simple generalization of the $d=2$ case
proved in~\cite{AKS19:jcss}.

\begin{lemma}\label{lem:lift-o}
  Let $\Gm$ be a constraint language over $U_d$ and let $\rel$ be pp-definable
  from $\Gm$. Furthermore, let $\cP\in\CSP(\Gm\cup\{\rel\})$ and let
  $\cP'\in\CSP(\Gm)$ be the gadget construction replacing constraints involving
  $\rel$ in $\cP$. If there is a (finite or infinite dimensional) satisfying operator assignment for $\cP$ then there is a (respectively, finite or infinite dimensional) satisfying operator assignment for $\cP'$. 
\end{lemma}

If $\CSP(\Gm\cup\{\rel\})$ has a satisfiability gap of the first kind then there is
an unsatisfiable instance $\cP\in\CSP(\Gm\cup\{\rel\})$ with a satisfying
operator assignment. By~\cite{Bulatov05:classifying} (cf.
also~\cite{BKW17}), $\cP'$ is unsatisfiable. By~Lemma~\ref{lem:lift-o}, $\cP'$ has a satisfying operator assignment. 
Hence $\cP'$ establishes that $\CSP(\Gm)$ has a satisfiability gap, and
Theorem~\ref{the:pp-o} is proved. This argument also extends to gaps of the second and third kind.

\smallskip

Pp-definitions are the starting point of the algebraic approach to
CSPs~\cite{Bulatov05:classifying} and suffice for dealing with Boolean CSPs,
not only in~\cite{AKS19:jcss} but also in all papers on Boolean (variants of) CSPs. For
CSPs over larger domains, more tools are needed.

A constraint language $\Gm$ is a \emph{core} language if all its endomorphisms
are permutations; that is, $\Gm$ has no endomorphisms that are not
automorphisms. There always exists an endomorphism $\vr$ of $\Gm$ such that $\vr(\Gm)$ is
core and $\vr\circ\vr=\vr$~\cite{Bulatov05:classifying}. We will denote this core language by $\core(\Gm)$, as it (up to an isomorphism) does not depend on the choice of $\vr$.

A constraint language $\Gm$ is called \emph{idempotent} if it contains all the
\emph{constant} relations, that is, relations of the form $C_a=\{(a)\}$, $a\in
U_d$. For an arbitrary language $\Gm$ over $U_d$ we use $\Gm^*=\Gm\cup\{C_a\mid
a\in U_d\}$.  A unary relation (a set) $B\sse U_d$ pp-definable in $\Gm$ is
called a \emph{subalgebra} of $\Gm$. For a subalgebra $B$ we introduce the
\emph{restriction} $\Gm\red B$ of $\Gm$ to $B$ defined as $\Gm\red B=\{\rel\cap
B^{ar(\rel)}\mid \rel\in\Gm\}$.

An equivalence relation $\th$ pp-definable in $\Gm$ is said to be a \emph{congruence} of $\Gm$. The equivalence class of $\th$ containing $a\in U_d$ will be denoted by $a\fac\th$, and the set of all equivalence classes, the \emph{factor-set}, by $U_d\fac\th$. Congruences of a constraint language allow one to define a \emph{factor-language} as follows. For a congruence $\th$ of the language $\Gm$ the factor language $\Gm\fac\th$ is the language over $U_d\fac\th$ given by $\Gm\fac\th=\{\rel\fac\th\mid \rel\in\Gm\}$, where $\rel\fac\th=\{(a_1\fac\th\zd a_n\fac\th)\mid  (\vc an)\in\rel\}$.

All the languages above are related to each other 
by the reducibility of the corresponding CSPs, as Figure~\ref{fig:reductions-o}
indicates (cf. Proposition~\ref{pro:reductions}).
%
In Section~\ref{sec:gap}, we show that all arrows in
Figure~\ref{fig:reductions-o} preserve satisfiability gaps.
%
To relate these reductions with bounded width and magic squares we use the
following result.

\begin{prop}[\cite{Bulatov09:width,Barto14:local,BKW17}]\label{pro:abelian-o}
For a constraint language $\Gm$ over $U_d$, $\CSP(\Gm)$ does not have bounded
  width if and only there exists a language $\Dl$ pp-definable in $\Gm$, a
  subalgebra $B$ of $\core(\Dl)^*$, a congruence $\th$ of $\core(\Dl)^*\red B$,
  and an Abelian group $\zA$ of prime order $p$ such that $\Gm'=\core(\Dl)^*\red B\fac\th$ contains relations $\rel_{3,a},\rel_{p+2}$ for every $a\in\zA$ given by $\rel_{3,a}=\{(x,y,z)\mid x+y+z=a\}$ and $\rel_{p+2}=\{(\vc a{p+2})\mid a_1+\dots+a_{p+2}=0\}$. If $\CSP(\Gm)$ is NP-hard, then $\Dl$ can be chosen to contain $\rel_{3,a},\rel_{p+2}$ for $p=2$.\footnote{The relations $\rel_{3,a},\rel_{p+2}$ are chosen here because they are needed for our purpose. In fact, they can be replaced with any relations expressible by linear equations over $\zA$.}
\end{prop}

To prove Theorem~\ref{the:hsp-gap-o}, suppose that $\CSP(\Gm)$ does not have
bounded width. 
%
By Proposition~\ref{pro:abelian-o} there is 
a language $\Dl$ pp-definable in $\Gm$, a subalgebra $B$ of $\core(\Dl)^*$, a congruence
$\th$ of $\core(\Dl)^*\red B$, and an Abelian group $\zA$ of prime order $p$
such that $\Gm'=\core(\Dl)^*\red B\fac\th$ contains relations $\rel_{3,a},\rel_{p+2}$
for every $a\in\zA$.
%
Our goal is to show that if $\CSP(\Gm')$ has a satisfiability gap
then so does $\CSP(\Gm)$. 


To give an idea of the gap-preservation proofs, we sketch how a satisfiability
gap is preserved.
%
Let $\Gm$ be a constraint language over the set $U_d$ and let $B$ be its subalgebra.
We show that if $\CSP(\Gm\red B)$ has a satisfiability gap then so does
$\CSP(\Gm)$. We show this for a gap of the first kind (over finite-dimensional Hilbert spaces); the infinite-dimensional case is more complicated, cf. Section~\ref{sec:gap} for details.
%
Let $\Dl=\Gm\red B$. Then by Theorem~\ref{the:pp-o} we may assume $\Dl\sse\Gm$ and $B\in\Gm$. Let $e=|B|$ and $\pi:U_e\to U_d$ a bijection between $U_e$ and $B$. 
%
Let $\cP=(V,U_e,\cC)$ be a gap instance of $\CSP(\pi^{-1}(\Dl))$ and the instance $\cP^\pi=(V,U_d,\cC^\pi)$ constructed as follows: For every $\ang{\bs,\rel}\in\cC$ the instance $\cP^\pi$ contains $\ang{\bs,\pi(\rel)}$. As is easily seen, $\cP^\pi$ has no classic solution. Therefore, it suffices to show that for any satisfying operator assignment $\{A_v\mid v\in V\}$ for $\cP$, the assignment $C_v=\pi(A_v)$ is a satisfying operator assignment for $\cP^\pi$.

By a technical lemma that shows that injective maps on finite sets that are interpolated
by polynomials preserve normal operators that pairwise commute (cf.
Lemma~\ref{lem:mapping}), 
the $C_v$'s are normal, satisfy the condition $C_v^d=I$, and locally commute. For $\ang{\bs,\rel}\in\cC$, $\bs=(\vc vk)$, let $f^\pi_\rel(\vc xk)=f_\rel(\pi^{-1}(x_1)\zd\pi^{-1}(x_k))$. 
It can be shown that $\pi^{-1}(C_v)=A_v$, and therefore $f^\pi_\rel(C_{v_1}\zd C_{v_k})=I$. For any $\vc ak\in U_d$, if $(\vc ak)\in\pi(\rel)$ then $\vc ak\in B$. Therefore, $f^\pi_\rel(\vc ak)=1$ then $f_{\pi(\rel)}(\vc ak)=1$. By Lemma~\ref{lem:lemma-3} this implies $f_{\pi(\rel)}(C_{v_1}\zd C_{v_k})=I$.


The other reductions use similar ideas, carefully relying on the spectral
theorem given in Theorem~\ref{the:SST} to simultaneously diagonalize the
restriction of an operator assignment to the scope of a constraint,
Lemma~\ref{lem:lemma-3} that relates polynomial equations over $U_d$ and
operators, and the above mentioned result on
preservation of operator assignments by certain polynomials (Lemma~\ref{lem:mapping}).
%
The infinite-dimensional case is more delicate, relying on the General Strong
Spectral Theorem (Theorem~\ref{the:general-sst}).

\smallskip
To finish the proof of Theorem~\ref{thm:main-informal}, assume that
$\CSP(\Gm)$ does not have bounded width and $\Gm'$ from
Proposition~\ref{pro:abelian-o} is over a group of prime order $p$. As reductions preserve satisfiability gaps, it suffices that
$\CSP(\Dl_p)$, where $\Dl_p=\{\rel_{3,a}\mid a\in\zA\}\cup\{\rel_{p+2}\}$, has a
satisfiability gap. The result of
Slofstra and Zhang~\cite{SZ24:personal} provides a gap instance of $\CSP(\{\rel_{3,1},\rel_{3,-1}\})$ of the 
second kind for any prime $p$, and thus the same holds for $\CSP(\Gm)$.
For $p=2$ the Mermin-Peres magic square from~\cite{Mermin1990simple}
provides a gap instance of $\CSP(\{\rel_{3,1},\rel_{3,-1}\})$ of the first kind,
and same holds for $\CSP(\Gm)$.
Finally, the result by Slofstra~\cite{Slofstra20:jams} provides a gap instance of
$\CSP(\{\rel_{3,1},\rel_{3,-1}\})$ of the third kind, and we get the same for
$\CSP(\Gm\cup\{\rel_T\})$, where $\rel_T$ is the full binary relation on $U_d$ (cf.
Theorem~\ref{the:pp} in Section~\ref{sec:operator-pp} and the discussion
therein). Finally, if $\CSP(\Gm)$ is NP-hard then $\Gm$ can simulate any
constraint language~\cite{Bulatov05:classifying}, and thus in particular $\Delta_2$.


