In this section we prove that the so-called primitive positive definitions, a key tool in the
algebraic approach to CSPs~\cite{Bulatov05:classifying}, not only give rise to
(polynomial-time) reductions that preserve satisfiability over $U_d$ but also
preserve satisfiability via operators. This was
established for the special case of Boolean domains (i.e., for $d=2$)
in~\cite{AKS19:jcss} and the same idea works for larger domains. We will need
this result later in Section~\ref{sec:gap} to prove that certain CSPs of 
unbounded width admit a satisfiability gap.

Let $\Gm$ be a constraint language over $U_d$, let $r$ be an integer, and
let $x_1,\ldots,x_r$ be variables ranging over the domain $U_d$. A primitive
positive formula (\emph{pp-formula}) over $\Gm$ is a formula of the form
%
\begin{equation}\label{eq:pp-formula}
  \phi(x_1,\ldots,x_r)=\exists y_1\cdots\exists y_s(\rel_1(\bz_1)\wedge\cdots\wedge
  \rel_m(\bz_m)),
\end{equation}
%
where $\rel_i\in\Gm$ is either the binary equality relation on $U_d$ or a relation over $U_d$ of arity $r_i$ and each $\bz_i$ is an $r_i$-tuple of variables from
$\{x_1,\ldots,x_r\}\cup\{y_1,\ldots,y_s\}$.
%
A relation $\rel\subseteq U_d^r$ is primitive positive definable
(\emph{pp-definable}) from $\Gm$ if there
exists a pp-formula $\phi(x_1,\ldots,x_r)$ over $\Gm$ such that $\rel$ is equal to the set of
models of $\phi$, that is, the set of $r$-tuples $(a_1,\ldots,a_r)\in U_d^r$
that make the formula $\phi$ true over $U_d$ if $a_i$ is
substituted for $x_i$ in $\phi$ for every $i\in [r]$.

Let $\rel_T=U_d^2$ denote the full binary relation on $U_d$.
%
Our goal in this section is to prove the following result.

\begin{theorem}\label{the:pp}
  Let $\Gm$ be a constraint language over $U_d$ and let $R$ be pp-definable from
  $\Gm$.
  \begin{enumerate}
    \item If $\CSP(\Gm\cup\{R\})$ has a satisfiability gap of the first or second kind then so does $\CSP(\Gm)$.
    \item If $\CSP(\Gm\cup\{R\})$ has a satisfiability gap of the third kind then so does $\CSP(\Gm\cup\{R_T\})$.
  \end{enumerate}
\end{theorem}

Let $\rel\subseteq U_d^r$ be a pp-definable formula
over $\Gm$ via the pp-formula $\phi(x_1,\ldots,x_r)$ as
in~(\ref{eq:pp-formula}). Let $=_d$ denote the equality relation on $U_d$. 
Given an instance
$\cP\in\CSP(\Gm\cup\{\rel\})$ we describe a construction of an instance
$\cP'\in\CSP(\Gm\cup\{=_d\})$, and then transform $\cP'$ into an instance $\cP^=$ of $\CSP(\Gm)$ that is, in some sense, equivalent to $\cP$. 
%
We start with the instance $\cP$. For every constraint $\ang{\bu,\rel}$ of $\cP$ with
$\bu=(u_1,\ldots,u_r)$, we
introduce $s$ fresh new variables $t_1,\ldots,t_s$ for the quantified
variables in~(\ref{eq:pp-formula}); furthermore, we replace $\ang{\bu,\rel}$ by $m$ constraints
$\ang{\bw_i,\rel_i}$, $i\in [m]$, where $\bw_i$ is the tuple of variables obtained
from $\bz_i$ in~(\ref{eq:pp-formula}) by replacing $x_j$ by $u_j$, $j\in [r]$,
and by replacing $y_j$ by $t_j$, $j\in [s]$.
The collection of variables $u_1,\ldots,u_r,t_1,\ldots,t_s$ is called the
\emph{block} of the constraint $\ang{\bu,\rel}$ in $\cP'$.
%
This construction is known as the \emph{gadget construction} in the CSP
literature and it is known that $\cP$ has a solution over $U_d$
if and only if $\cP'$ has a solution over $U_d$~\cite{Bulatov05:classifying,BKW17}.

Next, suppose that $\cP'$ is an instance of $\CSP(\Gm\cup\{=_d\})$ with the set of variables $\{\vc xn\}$. The instance $\cP^=$ is obtained by identifying variables $x_i,x_j$ whenever there is the constraint $=_d(x_,x_j)$ in $\cP'$. More formally, let $\vr'$ denote the equivalence relation on $[n]$ given by $(i,j)\in\vr'$ if and only if $\cP'$ contains the constraint $=_d(x_i,x_j)$, and let $\vr$ be the symmetric-transitive closure of $\vr'$. Select a representative of every equivalence class of $\vr$; without loss of generality assume $[m]$ is the set of these representatives. Let $\vf:[n]\to[m]$ be the mapping that maps every $i\in[n]$ to the representative of its equivalence class. Then $\cP^=$ is the instance of $\CSP(\Gm)$ with the set of variables $\{\vc ym\}$ that, for every constraint $\ang{(x_{i_1}\zd x_{i_r}),\rel}$, where $\rel$ is not the equality relation, contains the constraint $\ang{(y_{\vf(i_1)}\zd y_{\vf(i_r)}),\rel}$.
%
Thus, in order to prove Theorem~\ref{the:pp}\,(1), it suffices to show the following lemma.

\begin{lemma}\label{lem:lift}
  Let $\Gm$ be a constraint language over $U_d$ and let $\rel$ be pp-definable
  from $\Gm$. Furthermore, let $\cP\in\CSP(\Gm\cup\{\rel\})$ and let
  $\cP'\in\CSP(\Gm\cup\{=_d\})$, $\cP^=\in\CSP(\Gm)$ be the gadget construction replacing constraints involving
  $\rel$ in $\cP$, and the instance obtained from $\cP'$ by identifying the variable related by the equality relation. If there is a satisfying operator assignment for $\cP$ then there is a satisfying operator assignment for $\cP^=$. 
\end{lemma}

Indeed, if $\CSP(\Gm\cup\{\rel\})$ has a satisfiability gap of the first or second kind then there is an unsatisfiable instance $\cP\in\CSP(\Gm\cup\{\rel\})$ that has a satisfying operator assignment. By the results in~\cite{Bulatov05:classifying} (cf.~also~\cite{BKW17}), $\cP',\cP^=$ are unsatisfiable. By~Lemma~\ref{lem:lift}, $\cP^=$ has a satisfying operator assignment. Hence $\cP^=$ establishes that $\CSP(\Gm)$ has a satisfiability gap of the same kind, as required to prove Theorem~\ref{the:pp}\,(1).

We will frequently use (below and also in Section~\ref{sec:gap}) the following observations. We give a proof of them for the sake of completeness.

\begin{lemma}\label{lem:matrix-polys}
Let $f$ be a polynomial and $A,B$ operators on a Hilbert space.\\[2mm]
(1) If $A,B$ commute, then so do $f(A),f(B)$.\\
(2) If $A$ is normal, then so is $f(A)$.\\
(3) If $U$ is a unitary operator, then $Uf(A)U^{-1}=f(UAU^{-1})$.\\[1mm]
(4) Let $g(x_1,\dots,x_r)$ be a multi-variate polynomial and $A_1,\dots,A_r$
  operators on a Hilbert space. If $U$ is a unitary operator, then 
\[
Ug(A_1,\dots,A_r)U^{-1}=g(UA_1U^{-1},\dots,UA_rU^{-1}).
\]
\end{lemma}

\begin{proof}\textcolor{red}{TOPROVE 0}\end{proof}




%
\begin{proof}\textcolor{red}{TOPROVE 1}\end{proof}

We do not know whether the converse of Lemma~\ref{lem:lift} holds; this is not known even in the case of $d=2$~\cite{AKS19:jcss}.
%
The obvious idea would be to take the restriction of the operator assignment
that is satisfying for $\cP'$ but it is not clear why this should be satisfying
for $\cP$, because there is no guarantee that the operators assigned to variable in the scope of a constraint of $\cP$ of the form $\ang{\bs, \rel}$ commute. However, under a slight technical assumption on $\Gm$ --- namely,
that it includes the full binary relation $\rel_T$ on $U_d$\footnote{This is a special case of the
so-called \emph{commutativity gadget}~\cite{AKS19:jcss}; the ``T'' stands for
trivial.\\ Also, although the full binary relation can be pp-defined from the equality relation, it does not help to prove Theorem~\ref{the:pp}\,(2) as the full binary relation is needed exactly to deal with pp-definitions.} --- one can
enforce commutativity within a constraint scope and thus project an operator
assignment. 

For an instance $\cP'$ as defined above (and in the statement of Lemma~\ref{lem:lift}), we denote by $\cP''$ the instance obtained from $\cP'$ by adding, for every constraint $\ang{(\vc ur),\rel}$ of $\cP$, constraints of the form $\ang{(u_i,u_j),\rel_T}$ for every $i\neq j\in [r]$.
%
To prove Theorem~\ref{the:pp}\,(2), it suffices to show the following lemma.

\begin{lemma}\label{lem:proj}
  Let $\Gm$ be a constraint language over $U_d$ with $\rel_T\in\Gm$ and let $\rel$ be pp-definable
  from $\Gm$. Furthermore, let $\cP\in\CSP(\Gm\cup\{\rel\})$ and let
  $\cP''\in\CSP(\Gm)$ be defined as above.  Then, we have the following:\\[2mm]
  %
  (1) For every satisfying operator assignemnt for $\cP$ on a Hilbert
  space $\cH$ there is an extension that is a satisfying operator
  assignment for $\cP''$ on $\cH$.\\[2mm]
  %
  (2) For every satisfying operator assignment for $\cP''$ on $\cH$, the
  restriction of it onto the variables of $\cP$ is a satisfying operator
  assignment for $\cP$ on $\cH$.
  %
\end{lemma}
%
Indeed, if $\CSP(\Gm\cup\{\rel\})$ has a satisfiability gap of the
third kind then there is an instance $\cP\in\CSP(\Gm\cup\{\rel\})$ that is not
satisfiable via finite-dimensional operators but has a satisfying
infinite-dimensional operator assignment. By~Lemma~\ref{lem:proj}, $\cP''$ is
not satisfiable via finite-dimensional operators but has a satisfying
infinite-dimensional operator assignment. Hence $\cP''$ establishes that
$\CSP(\Gm\cup\{\rel_T\})$ has a satisfiability gap of the third kind, as required to prove Theorem~\ref{the:pp}\,(2).
%
\begin{proof}\textcolor{red}{TOPROVE 2}\end{proof}

