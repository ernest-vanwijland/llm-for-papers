In this section we prove the first part of our main result.

\begin{theorem}\label{the:no-gap}
Let $\Gm$ be a constraint language over $U_d$. If $\CSP(\Gm)$ has bounded
  width then it has no satisfiability gap of any kind.
\end{theorem}

The main idea behind the proof
of Theorem~\ref{the:no-gap} is to simulate the inference provided by SLAC 
by inference in polynomial equations. Let $\cS$ be a SLAC-program solving $\CSP(\Gm)$.
%
In order to prove Theorem~\ref{the:no-gap} we take an instance $\cP=(V,U_d,\cC)$
of $\CSP(\Gm)$ that has no solution, and therefore is not SLAC-consistent, as
$\CSP(\Gm)$ has bounded width, and prove that it also has no satisfying operator
assignment. This rules out a gap of the second kind, and thus also a gap of any
kind (cf. the discussion in Section~\ref{sec:operator-CSP}). We will prove it by contradiction, assuming $\cP$ has a satisfying operator assignment $\{A_v\}$ and then using the rules of a SLAC-program solving $\CSP(\Gm)$ to infer stronger and stronger conditions on $\{A_v\}$ that eventually lead to a contradiction. We start with a series of lemmas that will help to express the restrictions on $\{A_v\}$.

The following lemma introduces a restriction that is satisfied by any operator assignment. 

\begin{lemma}\label{lem:whole-domain-poly}
Let $\cP=(V,U_d,\cC)\in\CSP(\Gm)$. For any operator assignment $\{A_v\}$
for $\cP$ we have
\[
\prod_{k=0}^{d-1}(\ld_k I-A_v)=0.
\]
\end{lemma}
%
\begin{proof}\textcolor{red}{TOPROVE 0}\end{proof}

Recall that every rule of a SLAC-program has the form $(x\in S)\meet\rel(x,y,\vc zr))\to (y\in S')$ for some variables $x,y\in V$, a constraint $\ang{(x,y,\vc zr),\rel}$, and sets $S,S'\sse U_d$. Therefore, we need to show how to encode unary relations and rules of a SLAC-program through polynomials. For any $S\sse U_d$, we represent the unary constraint restricting the domain of a variable
$x$ to the set $S$ by the polynomial
\[
Dom_S(x)=\prod_{k\in S}(\ld_k-x)+1.\footnote{This is not the representation of $S$ as in the beginning of Section~\ref{sec:operator-CSP}, as $Dom_S(a)$ is not necessarily equal to $\ld_1$ for $a\not\in S$. However, it suffices for our purposes, because we only need the property that $Dom_S(a)=1$ if and only if $a\in S$.}
\]
%
Similarly, the rule $(x\in S)\meet\rel(x,y,\vc zr))\to (y\in S')$ of the SLAC 
program is represented by
\[
Rule_{S,\rel,S'}(x,y,\vc zr)=(Dom_{\ov S}(x)-1)(P_\rel(x,y,\vc zr)-\ld_1)
(Dom_{S'}(y)-1).
\]
As the next lemma shows, any operator assignment is a zero of $Rule_{S,\rel,S'}$.

\begin{lemma}\label{lem:rule-poly}
Let $\cP=(V,U_d,\cC)\in\CSP(\Gm)$. For any operator assignment $\{A_v\}$
for $\cP$ and any rule $(x\in S)\meet\rel(x,y,\vc zr))\to (y\in S')$ of the SLAC program for $\CSP(\Gm)$ we have
\begin{eqnarray*}
\lefteqn{Rule_{S,\rel,S'}(A_x,A_y,A_{z_1}\zd A_{z_r}) }\\
&=& (Dom_{\ov S}(A_x)-I)
(P_\rel(A_x,A_y,A_{z_1}\zd A_{z_r})-\ld_1I)(Dom_{S'}(A_y)-I)=0.
\end{eqnarray*}
\end{lemma}
%
\begin{proof}\textcolor{red}{TOPROVE 1}\end{proof}

Now, assume $\cP=(V,U_d,\cC)$ is not SLAC-consistent and $D_v$ denote the domain of $v\in V$ obtained after establishing SLAC-consistency. This means that for 
some $v\in V$ there is a derivation of $D_v=\eps$ using only facts
$\rel(\bs)$ for $\ang{\bs,\rel}\in\cC$ and 
$T_B(x_i)\meet\rel(\vc xk)\to T_C(x_j)$ for the rules of the SLAC-program. Moreover, this derivation can be subdivided into sections
of the form $(v=a)\to(v\ne a)$, each of which is linear. The latter 
condition means that each such section looks like a chain
$(v=a)\to(v_1\in S_1)\to\dots\to(v_\ell\in S_\ell)\to(v\in D_v-\{a\})$,
where each step is by a rule of the form 
$((v_i\in S_i)\meet\rel_i(v_i,v_{i+1},\vc ur))\to(v_{i+1}\in S_{i+1})$.

\begin{lemma}\label{lem:derivation-poly}
For any satisfying operator assignment $\{A_v\}$ for $\cP$ and any rule $(x\in S)\meet\rel(x,y,\vc zr))\to (y\in S')$ of the SLAC program for $\CSP(\Gm)$ if 
$\rel(x,y,\vc zr)\in\cC$ then 
\[
(Dom_{\ov S}(A_x)-I)(Dom_{S'}(A_y)-I)=I.
\]
\end{lemma}
%
\begin{proof}\textcolor{red}{TOPROVE 2}\end{proof}

\begin{lemma}\label{lem:transitive-poly}
Let $(v_1\in S_1)\to\dots\to(v_\ell\in S_\ell)$ be a derivation in the SLAC-program $\cS$ and $\{A_v\}$ a satisfying operator assignment for $\cP$. 
%
Then for each $i=2\zd\ell$
\[
(Dom_{\ov S_1}(A_{v_1})-I)(Dom_{S_i}(A_{v_i})-I)=0.
\]
\end{lemma}
%
\begin{proof}\textcolor{red}{TOPROVE 3}\end{proof}

\begin{proof}\textcolor{red}{TOPROVE 4}\end{proof}
