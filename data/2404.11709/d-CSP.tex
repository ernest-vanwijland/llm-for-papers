In order to relax the notion of satisfiability, we first consider CSPs on $U_d$
for some $d$ and represent CSPs via polynomials. Let $\Gm$ be a constraint 
language over $U_d$. Every constraint $\ang{\bs,\rel}$ of an instance 
$\cP=(V,U_d,\cC)$ of $\CSP(\Gm)$ is represented by a polynomial 
$P_\rel(\bs)$ that represents the characteristic function $f_\rel$ of
$\rel$:
%
\[
f_\rel(\bs)=\left\{\begin{array}{ll}
  \ld_0, & \text{if $\rel(\bs)$ is true,}\\
  \ld_1, & \text{otherwise}.
\end{array}\right.
\]
%
We note that our choice of the polynomial representation is somewhat arbitrary
but other choices lead to the same results. (For instance, \cite{AKS19:jcss}
studied the $d=2$ case and used $\ld_1$ to represent true.) 

An operator $A$ on a Hilbert space $\cH$ is a \emph{normal operator of order
$d$} if $A$ is normal and $A^d=I$. Operator assignment to the instance $\cP$ on a Hilbert space $\cH$ is a mapping 
that assigns to every variable from $V$ an operator $A_v$ on $\cH$ such that
\begin{itemize}
\item[(a)]
$A_v$ is a normal operator of order $d$ for every $v\in V$;
\item[(b)]
the operators $A_{v_1}\zd A_{v_r}$ pairwise commute
for every constraint  $\ang{(\vc vr),\rel}\in\cC$.
\end{itemize}

\noindent
We call an operator assignment $\{A_v\}$ \emph{satisfying for $\cP$} if 
$P_\rel(A_{v_1}\zd A_{v_r})=I$ for every constraint $\ang{(\vc vr),\rel}\in\cC$.
%
Let $\cP$ be a CSP instance. 
Following the terminology of Atserias et al.~\cite{AKS19:jcss}, we say that
$\cP$ has a \emph{satisfiability gap of the first kind} if $\cP$ is not
satisfiable over $U_d$ but is satisfiable by an operator assignment on 
a finite-dimensional Hilbert space. Similarly, we say that 
$\cP$ has a \emph{satisfiability gap of the second kind} if $\cP$ is not
satisfiable over $U_d$ but is satisfiable by an operator assignemnt on an
infinite-dimensional Hilbert space. Finally, we say that 
$\cP$ has a \emph{satisfiability gap of the third kind} if $\cP$
is not satisfiable on finite-dimensional Hilbert spaces but is satisfiable by an
operator assignment on 
an infinite-dimensional Hilbert space.
%
We say that $\CSP(\Gm)$ has a satisfiablity gap of the $i$-th
kind, $i=1,2,3$, if there is at least one instance $\cP\in\CSP(\Gm)$ witnessing
such a gap.
%
By definition, a gap of the first kind implies a gap of the second kind. Also, a
gap of the third kind implies a gap of the second kind. Put differently, if
$\Gm$ has no gap of the second kind then it has no gap of the first or third
kind either.

We shall repeatedly use the following simple lemma, proved in Appendix~\ref{app:lemma}.

\begin{lemma}\label{lem:lemma-3}
Let $\vc xr$ be variables, let $\vc Qm,Q$ be polynomials in $\zC[\vc xr]$,
and let $\cH$ be a Hilbert space. If every assignment over $U_d$ that 
satisfies the equations $Q_1=\dots=Q_m=0$ also satisfies the equation
$Q=0$, then every fully commuting operator assignment on $\cH$ that 
satisfies the equations $Q_1=\dots=Q_m=0$ also satisfies the equation
$Q=0$.
\end{lemma}

The proof of this lemma is very similar to that of the analogous claim
in~\cite[Lemma~3]{AKS19:jcss}, where it was established for $d=2$.
The main difference is to 
use $A^d=I$ rather than $A^2=I$. For the sake of completeness we give the proof here.

\noindent\begin{proof}
%
{\it Finite-dimensional case.}
Suppose that the conditions of the lemma hold and $\vc Ar$ are pairwise
  commuting operators such that the equations $Q_1=\dots=Q_m=0$ are
  true when these matrices are assigned to $\vc xr$. Then, since $\vc Ar$ are
  normal and commute, by Theorem~\ref{the:SST} there is a unitary matrix $U$
  such that $E_i=UA_iU^{-1}$ is a diagonal matrix. Then, $E_i^d=I$, because $A_i^d=I$. Therefore, every diagonal entry $E_i(jj)$ belongs to $U_d$. For every equation $Q_\ell$ we have $Q_\ell(\vc Ar)=0$ implying
  $Q_\ell(\vc Er)=UQ_\ell(\vc Ar)U^{-1}=0$. Since every $E_i$ is diagonal, for every $j$ it also holds $Q_\ell(E_1(jj)\zd E_r(jj))=0$. By the conditions of the lemma we also have $Q((E_1(jj)\zd E_r(jj))=0$, and $Q(\vc Ar)=U^{-1}Q(\vc Er)U=0$.
%

\smallskip

{\it Infinite-dimensional case.}
Suppose that the conditions of the lemma hold and $\vc Ar$ are pairwise commuting  operators such that the equations $Q_1=\dots=Q_m=0$ are true when these operators are assigned to $\vc Xr$. Then, since $\vc Ar$ are normal and commute, by  Theorem~\ref{the:general-sst} there exist a measure space $(\Omega,\cM,\mu)$, a unitary map $U:\cH\to L^2(\Omega,\mu)$, and functions $\vc ar\in L^\infty(\Omega,\mu)$ such that, for the multiplication operators $E_i=T_{a_i}$ of $L^2(\Omega,\mu)$, the equalities $A_i=U^{-1}E_iU$ hold for $i\in[r]$. This implies $UA_iU^{-1}=E_i$. As $A_i^d=I$ we also have $E_i^d=I$. Therefore $a_i(\omega)^d=1$ for almost all $\omega\in\Omega$, or more formally $\mu(\{\omega\in\Omega\mid (a_i(\om)^d\ne1\})=0$. Hence $a_i(\omega)\in U_d$ for almost all $\omega\in\Omega$. For every $\ell\in[m]$ we have $Q_\ell(\vc Er)=UQ_\ell(\vc Ar)U^{-1}$, implying by the conditions of the lemma that $Q_\ell(\vc Er)=0$. Since $E_i$ is the multiplication operator given by the function $a_i$, it holds that $Q_\ell(a_1(\om)\zd a_r(\om))=0$ for almost all $\om\in\Omega$. As for almost all $\om\in\Omega$, it holds that the value $a_i(\om)\in U_d$ for each $i\in[r]$, we have $Q(a_1(\om)\zd a_r(\om))=0$ for almost all $\om\in\Omega$. Therefore $Q(\vc Er)=0$ and this implies $Q(\vc Ar)=0$ as before.
%
\end{proof}



