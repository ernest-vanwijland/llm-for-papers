\documentclass[12pt,reqno]{amsart}
\pdfoutput=1

\usepackage[T1]{fontenc}

\usepackage{amssymb,amsmath,amsthm,mathtools,tabularx,bbm,url,microtype}
\usepackage{float,graphicx} \usepackage{mathabx}\changenotsign        \usepackage{dsfont}
\usepackage[shortlabels]{enumitem}
\usepackage[utf8]{inputenc}
\allowdisplaybreaks

\usepackage[dvipsnames]{xcolor} 

\usepackage[pagebackref=true]{hyperref} 
\hypersetup{
	colorlinks,
	linkcolor={blue!60!black},
	citecolor={green!60!black},
	urlcolor={blue!60!black}
}


\linespread{1.3}
\usepackage{geometry}
\geometry{left=25mm,right=25mm, top=25mm, bottom=25mm}




















\makeatletter
\def\moverlay{\mathpalette\mov@rlay}
\def\mov@rlay#1#2{\leavevmode\vtop{   \baselineskip\z@skip \lineskiplimit-\maxdimen
		\ialign{\hfil$\m@th#1##$\hfil\cr#2\crcr}}}
\newcommand{\charfusion}[3][\mathord]{
	#1{\ifx#1\mathop\vphantom{#2}\fi
		\mathpalette\mov@rlay{#2\cr#3}
	}
	\ifx#1\mathop\expandafter\displaylimits\fi}
\makeatother

\usepackage{tikz}
\usepackage{pgfplots}
\pgfplotsset{compat=1.18}
\usepackage{mathrsfs}
\usetikzlibrary{arrows}

\usepackage{tkz-euclide}
\tkzSetUpPoint[fill=black, size = 3pt]
\tkzSetUpLine[color=black, line width=0.6pt]




\definecolor{mygreen}{RGB}{106, 168, 79}
\definecolor{myblue}{RGB}{98, 160, 234}
\definecolor{myred}{RGB}{246,97,81}
\definecolor{black}{rgb}{0,0,0}









\usepackage{caption}
\captionsetup{font=footnotesize}
\usepackage{subcaption}
\renewcommand{\thesubfigure}{(\upshape \alph{subfigure})}







\usepackage[mathlines]{lineno}
\usepackage{etoolbox} 

\newcommand*\linenomathpatch[1]{\expandafter\pretocmd\csname #1\endcsname {\linenomath}{}{}\expandafter\pretocmd\csname #1*\endcsname{\linenomath}{}{}\expandafter\apptocmd\csname end#1\endcsname {\endlinenomath}{}{}\expandafter\apptocmd\csname end#1*\endcsname{\endlinenomath}{}{}}
\newcommand*\linenomathpatchAMS[1]{\expandafter\pretocmd\csname #1\endcsname {\linenomathAMS}{}{}\expandafter\pretocmd\csname #1*\endcsname{\linenomathAMS}{}{}\expandafter\apptocmd\csname end#1\endcsname {\endlinenomath}{}{}\expandafter\apptocmd\csname end#1*\endcsname{\endlinenomath}{}{}}

\expandafter\ifx\linenomath\linenomathWithnumbers
\let\linenomathAMS\linenomathWithnumbers
\patchcmd\linenomathAMS{\advance\postdisplaypenalty\linenopenalty}{}{}{}
\else
\let\linenomathAMS\linenomathNonumbers
\fi

\linenomathpatchAMS{gather}
\linenomathpatchAMS{multline}
\linenomathpatchAMS{align}
\linenomathpatchAMS{alignat}
\linenomathpatchAMS{flalign}
\linenomathpatch{equation}






\usepackage{cleveref}[capitalise, compress, nameinlink, noabbrev]

\theoremstyle{plain}
\newtheorem{theorem}{Theorem}[section]
\crefname{theorem}{Theorem}{Theorems}

\newtheorem{proposition}[theorem]{Proposition}
\crefname{proposition}{Proposition}{Propositions}

\newtheorem{corollary}[theorem]{Corollary}
\crefname{corollary}{Corollary}{Corollaries}

\newtheorem{lemma}[theorem]{Lemma}
\crefname{lemma}{Lemma}{Lemmata}


\newtheorem{conjecture}[theorem]{Conjecture}
\crefname{conjecture}{Conjecture}{Conjectures}

\newtheorem{problem}[theorem]{Problem}
\crefname{problem}{Problem}{Problems}

\newtheorem{claim}[theorem]{Claim}
\crefname{claim}{Claim}{Claims}

\newtheorem{observation}[theorem]{Observation}
\crefname{observation}{Observation}{Observations}

\newtheorem{setup}[theorem]{Setup}
\crefname{setup}{Setup}{Setups}

\newtheorem{fact}[theorem]{Fact}
\crefname{fact}{Fact}{Facts}

\newtheorem{algorithm}[theorem]{Algorithm}
\crefname{algorithm}{Algorithm}{Algorithms}

\newtheorem{remark}[theorem]{Remark}
\crefname{remark}{Remark}{Remarks}

\newtheorem{example}[theorem]{Example}
\crefname{example}{Example}{Examples}

\theoremstyle{definition}
\newtheorem{definition}[theorem]{Definition}
\crefname{definition}{Definition}{Definitions}

\newtheorem{construction}[theorem]{Construction}
\crefname{construction}{Construction}{Constructions}

\newtheorem{question}[theorem]{Question}
\crefname{question}{Question}{Questions}

\numberwithin{equation}{section}

\crefname{section}{Section}{Sections}
\crefname{appendix}{Appendix}{Appendix}

\crefname{figure}{Figure}{Figures}



\newcommand{\fw}[1]{\textcolor{black}{#1}}


\newcommand{\Prob}{\mathbb{P}}
\newcommand{\Exp}{\mathbb{E}}

\DeclarePairedDelimiter{\abs}{\lvert}{\rvert}
\DeclarePairedDelimiter{\ceil}{\lceil}{\rceil}
\DeclarePairedDelimiter{\floor}{\lfloor}{\rfloor}
\DeclarePairedDelimiter{\set}{\{}{\}}

\renewcommand{\epsilon}{\varepsilon}
\renewcommand{\rho}{\varrho}
\renewcommand{\ge}{\geqslant}
\renewcommand{\le}{\leqslant}
\renewcommand{\geq}{\geqslant}
\renewcommand{\leq}{\leqslant}
\renewcommand{\emptyset}{\varnothing}
\renewcommand{\setminus}{\smallsetminus}
\NewCommandCopy{\subs}{\subset}
\renewcommand{\subset}{\subseteq}




\newcommand{\defn}[1]{\textcolor{Maroon}{\emph{#1}}}
\newcommand{\eps}{\varepsilon}

\newcommand{\comin}{\delta^{\ast}}
\newcommand{\bR}{\mathbb{R}}
\newcommand{\bS}{\mathbb{S}}
\newcommand{\PG}[3]{{P^{(#3)}}(#1;#2)}

\newcommand\restr[2]{{\left.\kern-\nulldelimiterspace #1 \vphantom{\big|} \right|_{#2} }}




\newcommand{\cA}{\mathcal{A}}
\newcommand{\cB}{\mathcal{B}}
\newcommand{\cC}{\mathcal{C}}
\newcommand{\cD}{\mathcal{D}}
\newcommand{\cE}{\mathcal{E}}
\newcommand{\cF}{\mathcal{F}}
\newcommand{\cG}{\mathcal{G}}
\newcommand{\cH}{\mathcal{H}}
\newcommand{\cI}{\mathcal{I}}
\newcommand{\cJ}{\mathcal{J}}
\newcommand{\cK}{\mathcal{K}}
\newcommand{\cL}{\mathcal{L}}
\newcommand{\cM}{\mathcal{M}}
\newcommand{\cN}{\mathcal{N}}
\newcommand{\cO}{\mathcal{O}}
\newcommand{\cP}{\mathcal{P}}
\newcommand{\cQ}{\mathcal{Q}}
\newcommand{\cR}{\mathcal{R}}
\newcommand{\cS}{\mathcal{S}}
\newcommand{\cT}{\mathcal{T}}
\newcommand{\cU}{\mathcal{U}}
\newcommand{\cV}{\mathcal{V}}
\newcommand{\cW}{\mathcal{W}}
\newcommand{\cX}{\mathcal{X}}
\newcommand{\cY}{\mathcal{Y}}
\newcommand{\cZ}{\mathcal{Z}}

\title{Spanning spheres in Dirac hypergraphs}
\date{\today}

\author[F.~Illingworth]{Freddie Illingworth}
\author[R.~Lang]{Richard Lang}
\author[A.~M\"{u}yesser]{Alp M\"{u}yesser}
\author[O.~Parczyk]{Olaf Parczyk}
\author[A.~Sgueglia]{Amedeo Sgueglia}

\address[Illingworth]
{
	Department of Mathematics,
	University College London,
	London, UK
}
\email{f.illingworth@ucl.ac.uk}

\address[Lang]
{
	Departament de Matemàtiques,
	Universitat Politècnica de Catalunya,
	Barcelona, Spain
}
\email{richard.lang@upc.edu}



\address[M\"uyesser]
{
	New College,
	University of Oxford,
	Oxford, UK
}
\email{alp.muyesser@new.ox.ac.uk}



\address[Parczyk]
{
	Zuse Institute Berlin, 
	Department for AI in Society, Science, and Technology,
	Berlin, Germany
}
\email{parczyk@zib.de}

\address[Sgueglia]
{
	Fakult\"at f\"ur Informatik und Mathematik, 
	Universit\"at Passau, 
	Germany.
}
\email{amedeo.sgueglia@uni-passau.de}

\def\deltabar{{\mathchar '26\mkern -9.2mu \delta}}

\thanks{FI was supported by EPSRC grant EP/V521917/1 and the Heilbronn Institute for Mathematical Research.
	RL was supported by the EU Horizon 2020 programme MSCA (101018431).
	OP was funded by the DFG (EXC-2046/1, project ID: 390685689).
	While conducting this research, AS was affiliated with University College London and supported by the Royal Society.}

\begin{document}
	
	\begin{abstract}
		We show that a $k$-uniform hypergraph on $n$ vertices has a spanning subgraph homeomorphic to the $(k - 1)$-dimensional sphere provided that $H$ has no isolated vertices and each set of $k - 1$ vertices supported by an edge is contained in at least $n/2 + o(n)$ edges.
		This gives a topological extension of Dirac's theorem and asymptotically confirms a conjecture of Georgakopoulos, Haslegrave, Montgomery, and Narayanan.
		
		Unlike typical results in the area, our proof does not rely on the Absorption Method, the Regularity Lemma or the Blow-up Lemma.
		Instead, we use a recently introduced framework that is based on covering the vertex set of the host graph with a family of complete blow-ups.
	\end{abstract}
	
	\maketitle
	


	\section{Introduction}
	
	A seminal theorem of Dirac~\cite{Dirac} determines the best possible minimum degree condition for a graph to contain a Hamilton cycle.
	Over the past decades, a great number of generalisations of this result have been obtained~\cite{SS19}, often  in parallel with the development of new proof techniques such as the modern form of the Absorption Method~\cite{RRS08a} as well as the Blow-up Lemma~\cite{KSS98}.
	In many of these Dirac-type results, the embedded cyclical structures are organised along a linear ordering of the vertex set, an inherently $1$-dimensional concept.
	We study a different, topological direction of this problem with early traces in the work of Brown, Erdős, and Sós~\cite{SEB73} that concerns the existence of subgraphs with rich, high-dimensional structure in hypergraphs with good minimum degree conditions.
	
	To motivate this area, we first observe that a Hamilton cycle in a $2$-graph is a set of edges such that the simplicial complex induced by these edges is homeomorphic to the $1$-dimensional sphere, $\bS^1$, with the additional property that the $0$-skeleton of the complex comprises the entire vertex-set of the host graph.
	This topological viewpoint led Gowers and, independently, Conlon (see~\cite{georgakopoulos2022spanning}) to ask which degree conditions force a $3$-graph to contain a spanning copy of the $2$-dimensional sphere $\mathbb{S}^2$.
	To clarify, a \defn{copy of a $(k - 1)$-sphere} in a $k$-graph~$H$ is a set of edges that induce a homogeneous simplicial complex homeomorphic to the $(k - 1)$-sphere,~$\mathbb{S}^{k - 1}$, and such a copy is called \defn{spanning} if the $0$-skeleton of the simplicial complex is the entire vertex-set of $H$.
	
	The original question of Gowers and Conlon was stated in terms of codegrees.
	Formally, the \defn{minimum codegree} of a $k$-graph is the maximum integer $m$ such that every set of $k-1$ vertices is contained in at least $m$ edges.
	For $k=3$, Georgakopoulos, Haslegrave, Montgomery, and Narayanan~\cite{georgakopoulos2022spanning} gave an asymptotically optimal bound by proving that every $n$-vertex $3$-graph with minimum codegree at least $n/3 + o(n)$ contains a spanning copy of a $2$-sphere.
	It is an open conjecture whether this result can be extended to higher uniformities, see also \cref{sec:open-problems}.
	
	One of the main obstacles to combinatorial embeddings of spheres is topological connectivity.
	We capture this notion as follows.
	For a $k$-graph $H$, let $\hat H$ be the \defn{line graph} on the vertex set $E(H)$ with an edge $ef$ whenever $\abs{e \cap f} = k - 1$.
	A subgraph of $H$ is \defn{tightly connected} if it has no isolated vertices and its edges induce a connected subgraph in $\hat{H}$.
	Moreover, we refer to edge maximal tightly connected subgraphs as \defn{tight components}.
	It is easy to see that a $(k - 1)$-sphere is tightly connected.
	Thus, if a $k$-graph $H$ contains a vertex spanning sphere, then $H$ must have a spanning tight component.
	We can use this observation to obtain lower bounds for extremal problems.
	For instance, there are simple $3$-uniform constructions with codegree $n/3 - 1$ and no spanning tight component, and so the above result is sharp.
	
	In light of this, it is natural to ask which codegree condition that is locally supported by a tight component forces a spanning sphere.
	Formally, the \defn{minimum supported co-degree}, $\comin(H)$, of a non-empty $k$-graph $H$, is the maximum integer $m$ such that every $(k - 1)$-set contained in one edge of $H$ is contained in at least $m$ edges of $H$.
	A plausible first guess is that every tightly connected $3$-graph $H$ with $\delta^\ast(H)\geq n/3+o(n)$ contains a spanning copy of a $2$-sphere.
	However, constructions show that in fact $\delta^\ast(H)\geq n/2$ is necessary and this extends to all uniformities $k\geq 2$
	(see \cref{sec:constructions}).
	Given this, Georgakopoulos et al.~\cite{georgakopoulos2022spanning} posed the following conjecture.
	
	\begin{conjecture}\label{conj:mainconj}
		Every tightly connected $k$-graph $H$ of order $n> k \geq 2$ with $\comin(H) \ge n/2$ contains a spanning copy of $\bS^{k - 1}$.
	\end{conjecture}
	
	We remark that $2$-uniform case of the conjecture corresponds to Dirac's theorem.
	Our main result confirms \cref{conj:mainconj} asymptotically.
	
	\begin{theorem}\label{thm:maintheorem}
		For every $k \geq 2$ and $\eps>0$, there exists $n_0$ such that every tightly connected $k$-graph $H$ of order $n \geq n_0$ with $\comin(H) \ge (1/2+\eps)n$ contains a spanning copy of $\bS^{k - 1}$.
	\end{theorem}
	
	Spheres have been studied before in other combinatorial settings.
	Historically, the notion can be traced back to the work of Brown, Erdős, and Sós~\cite{SEB73} in the seventies.
	More recently, Luria and Tessler~\cite{LT19} studied spanning spheres in random complexes.
	In the graph setting, Kühn, Osthus, and Taraz~\cite{KOT05} gave asymptotically optimal minimum degree conditions for the existence of a $2$-dimensional sphere in graphs, later generalised by the Bandwidth theorem of B\"{o}ttcher, Schacht, and Taraz~\cite{BST09}.
	A sharper version of the former result was obtained by Kühn and Osthus~\cite{KO05}.
	Finally, the notion of minimum supported degrees was recently studied in the context of Tur\'an-type problems~\cite{Bal21,Pik23}.
	
	We remark that unlike in the graph setting it is not possible to derive optimal minimum supported degree conditions for tight spheres from a bandwidth theorem.
	This is because in the context of hypergraphs, a bandwidth theorem requires structure (namely a tight Hamilton path) with higher degree thresholds than those needed for spheres.
	See \cref{sec:open-problems} for a discussion of tight Hamilton cycles and paths.
	
	Nevertheless, our approach benefits from ideas that have been developed to tackle the problem of embedding hypergraphs of bounded bandwidth.
	In particular, \cref{lem:findblowupchain} is an instance of a forthcoming more general setup~\cite{LS24}, which has been tailored to the requirements of embedding spanning spheres.
	
	\subsection{Proof outline}
	
	There are two well-known approaches to Dirac-type problems in graphs and hypergraphs.
	The first approach consists of a combination of Szemer\'{e}di's Regularity Lemma and the Blow-up Lemma due to Koml\'{o}s, S\'{a}rk\"{o}zy, and Szemer\'{e}di~\cite{KSS98}.
	The other approach is based on the Absorption Method, whose modern form was introduced by R\"{o}dl, Ruci\'{n}ski, and Szemer\'{e}di~\cite{RRS08a}, often also in conjunction with a (hypergraph) Regularity Lemma.
	Georgakopoulos et al.~\cite{georgakopoulos2022spanning} followed the second approach when embedding spanning $2$-spheres in $3$-uniform hypergraphs using a $2$-uniform version of the Regularity Lemma.
	One of the main difficulties in extending their method to higher uniformities is that one needs to invoke a stronger ($k$-uniform) version of the Regularity Lemma.
	While this is presumably not prohibitive, it leads to a quite technical setup with many additional challenges (see for instance~\cite{LS22}).
	In order to avoid such complications, our proof is based on an upcoming framework for embedding large structures into hypergraphs of Lang and Sanhueza-Matamala~\cite{LS24}, which has a precursor in the setting of perfect tilings~\cite{lang2023tiling}.
	
	\subsubsection*{Classic approach}
	This new approach is conceptually inspired by the classic combination the Regularity Lemma and the Blow-up Lemma, which we briefly recapitulate in the following.
	Broadly speaking, the Regularity Lemma allows us to approximate a given host-graph $H$ with a quasirandom blow-up of a \emph{reduced graph} $R$ of constant order, meaning that vertices of $R$ are replaced with (disjoint, linear sized) vertex sets of $H$ and the edges of~$R$ are replaced by quasirandom partite graphs.
	Importantly, the Regularity Lemma guarantees that $R$ approximately inherits degree conditions.
	The Blow-up Lemma then tells us that one can effectively assume that these (quasirandom) partite graphs are complete.
	Ignoring a number of technicalities, this reduces the problem of embedding a large structure $S$ into $H$ to the problem of embedding a large structure $S$ into a complete blow-up of $R$.
	This last step is known as the \emph{allocation} of $S$, and there is now a broad set of techniques available to facilitate this step. Therefore, the reduction to the allocation problem often presents considerable step towards the successful embedding of $S$ into $H$.
	
	\subsubsection*{Our approach}
	In comparison to this, our approach proceeds as follows.
	Instead of approximating the structure of the host graph $H$ with a single quasirandom blow-up, we cover its vertex set $V(H)$ with a family of complete blow-ups $R_1^\ast,\dots,R_\ell^\ast$ contained in $H$, whose reduced graphs $R_1,\dots,R_\ell$ inherit the degree conditions of $H$.
	Importantly, the blow-ups are interlocked in a path-like fashion.
	So, in order to embed a given guest graph $S$ into $H$, we first find $S_1,\dots,S_\ell$ such that $\bigcup S_i = S$, and satisfying that $S_i$ and $S_j$ are fully disjoint unless $j=i+1$. Subsequently, we embed each $S_i$ into the blow-up of $R_i$ such that the connections between $S_i$ and $S_{i+1}$ are in accordance.
	This effectively reduces the problem of embedding $S$ into $H$ to embedding each $S_i$ into a complete blow-up of $R_i$, which is analogous to the above detailed allocation problem.
	
	\subsubsection*{Allocation}
	To find a sphere $S_i$ that covers the blow-up $R_i^\ast$ of $R_i$, we first construct a preliminary sphere $S_i'$ that has (for simplicity of the sketch) a suitable ``entry facet'' $F_e \in E(R_i^\ast)$ within the blow-up of every edge $e \in E(R_i)$.
	This is possible since $R_i$ is tightly connected.
	Notably, $S_i'$ can be taken to be very sparse, as the order of $R_i$ is much smaller than the blow-up part sizes.
	This sparsity allows us to ignore the perturbation created by reserving~$S_i'$.
	We then extend $S_i'$ by gluing additional spheres $B_e$ onto the faces $F_e$, where each $B_e$ is contained in the blow-up of the edges $e \in E(R_i)$.
	Importantly, the spheres $B_e$ are disjoint from each other and from the sphere $S_i'$ (apart from the designated entry facets).
	The main difficulty here is to compute the appropriate sizes of the spheres $B_e$, which we achieve by solving a $2$-dimensional matching problem using Dirac's theorem.
	Together, this leads to the desired spanning sphere $S_i$ in $R^\ast_i$.
	In order to connect the spheres $S_1,\dots,S_t$, we carefully place additional facets (at intersections between each $R_i^\ast$ and $R^\ast_{i+1}$) such that the spanning sphere $S$ can be obtained by gluing together the spheres $S_1,\dots,S_t$ along these facets.
	This is possible since the blow-ups $R_1^\ast, \dots, R_\ell^\ast$ are interlocked in a path-like fashion.
	
	
	
	\subsection{Open problems}\label{sec:open-problems}
	
	Our work leads to a series of interesting open problems, which we discuss in the following.
	
	\subsubsection*{Connectivity}
	
	Georgakopoulos et al.~\cite{georgakopoulos2022spanning} conjectured that any $n$-vertex $k$-graph with minimum codegree at least $n/k$ contains a spanning $(k - 1)$-sphere and confirmed the case $k=3$ approximately.
	We believe that our framework is suitable for attacking the general conjecture.
	A first step towards this would be to show that such a $k$-graph contains a spanning tight component, which is an interesting problem on its own.
	
	\begin{conjecture}
		Every $k$-graph on $n$ vertices with minimum codegree at least  $n/k$ contains a vertex spanning tight component.
	\end{conjecture}
	
	We remark that this minimum degree is best possible due to the constructions for the original conjecture~\cite{georgakopoulos2022spanning}.
	
	Another question in this direction raised in ~\cite{georgakopoulos2022spanning} is how large a sphere can be found in a $3$-graph with a given minimum codegree below $n/3$. As noted in~\cite{georgakopoulos2022spanning}, it is possible here that tight components are far from spanning, but in this situation pairs supported within small tight components will have large codegree relative to the size of the component, and hence our \cref{thm:maintheorem} could potentially help in resolving this problem.
	
	
	\subsubsection*{Tight cycles}
	
	Given our main result, it is natural to ask about the minimum supported codegree threshold for tight Hamilton cycles.
	Formally, a $k$-uniform \defn{tight cycle} comes with a cyclical ordering of its vertices such that every $k$ consecutive vertices form an edge.
	A classic result of R\"{o}dl, Ruci\'{n}ski, and Szemer\'{e}di~\cite{RRS08a} states that an $n$-vertex graph with minimum codegree at least $n/2 + o(n)$ contains a tight Hamilton cycle.
	As it turns out, this is no longer true for supported degree.
	
	Indeed, consider an $n$-vertex $k$-graph $G$ with $n$ divisible by $k$ and a partition $X \cup Y$ of its vertex-set with $\abs{Y} = n/k + 1$ such that $G$ contains all edges with at least $k - 1$ vertices in $X$.
	Note that any matching in $G$ misses at least one vertex of $Y$, so $G$ does not have a perfect matching and by extension no tight Hamilton cycle.
	On the other hand, $G$ satisfies $\delta^\ast(G) = (1 - 1/k)n - (k - 1)$.
	
	We believe that these constructions are optimal.
	
	\begin{conjecture}
		For every $k \geq 2$, there is some $n_0$ such that every $k$-uniform tightly connected $G$ on $n \geq n_0$ vertices with $\comin(G) \geq (1 - 1/k) n$ contains a tight Hamilton cycle.
	\end{conjecture}

    Very recently, this conjecture has been confirmed approximately by Mycroft and Zárate-Guerén~\cite{MZ25}.
	
	\subsubsection*{Exact bounds}
	Our main result gives an asymptotic solution to \cref{conj:mainconj}.
	To prove the exact conjecture, further work is required possibly including a stability analysis.
	It is conceivable that our embedding setup, namely \cref{lem:findblowupchain}, can be extended to account for this by using tools from the theory of property testing (see \cite{lang2023tiling} for more details).
	On the other hand, removing the error term in \cref{thm:maintheorem}, would also require a more precise analysis of the host graph structure, which could pose a significant challenge.
	We note that even in the setting of tight Hamilton cycles in $k$-graphs, which has been much more heavily investigated than the topological variant considered here, exact results are only known for $k \leq 3$~\cite{Dirac,rodl2011dirac}.
	
	\subsubsection*{Vertex degree}
	
	Since the $3$-uniform minimum codegree threshold for spanning $2$-spheres is well-understood, at least in the approximate sense, it is natural to ask what happens for $2$-spheres under minimum vertex-degrees.
	The following construction gives a lower bound.
	
	Consider a $3$-graph $G$ whose vertices are partitioned by sets $X$, $Y$, and $Z$ each of size $n/3$ and whose edges are composed of all edges of type $XXY$, $YYZ$, and $ZZX$ as well as all edges inside each of $X$, $Y$, and $Z$.
	It is not hard to see that $G$ does not have a spanning tight component and hence no spanning $2$-sphere.
	On the other hand, a simple calculation shows that every vertex of $G$ is on at least $(4/9 - o(1)) \binom{n}{2}$ edges.
	
	It is plausible, that $3$-graphs above this degree contain spanning $2$-spheres.
	To formalise this, we denote by $\delta_1(G)$ the maximum $m$ such that every vertex of a $3$-graph $G$ is contained in at least $m$ edges.
	
	\begin{conjecture}
		For every $k \geq 2$ and $\eps > 0$, there is $n_0$ such that any $k$-graph $G$ on $n\geq n_0$ vertices with $\delta_1(G) \geq (4/9 + \eps ) \binom{n}{2}$ contains a spanning $2$-sphere.
	\end{conjecture}
	
	Note that this conjecture implicitly asserts that the minimum (relative) vertex degree threshold for a spanning tight component is $4/9$, which is itself an interesting problem.
	On the other hand, it can be shown that the threshold for spanning spheres is at most $5/9$.
	This follows from a combination of the facts that $5/9$ is the threshold for tight Hamilton paths~\cite{RRR19}, certain blow-ups of tight paths containing spanning spheres (\cref{lem:thin-path}) and an upcoming hypergraph bandwidth theorem~\cite{LS24}.
	
	
	\subsection{Organisation of the paper}
	
	In \cref{sec:preliminaries} we collect notation and definitions which we will use throughout the paper.
	In the same section, we state our two key lemmata (\cref{lem:allocation,lem:findblowupchain}) and then prove our main result (\cref{thm:maintheorem}) assuming them.
	\cref{sec:tools} collects known tools, needed for our proofs.
	Finally, we prove \cref{lem:findblowupchain} in \cref{sec:blow-up} and \cref{lem:allocation} in \cref{sec:geometric_obs,sec:allocation}.
	
	\subsection*{Acknowledgements} We would like to thank an anonymous referee for a careful reading of the paper, and in particular for pointing us to an inaccuracy in the proof of Lemma~\ref{lem:findblowupchain} in an earlier version of this manuscript.
	
	
	\section{Preliminaries, key lemmata, and proof of the main result}
	\label{sec:preliminaries}
	
	\subsection{Notation and definitions}
	
	A \defn{$k$-uniform hypergraph} $H$ (or $k$-graph for short) consists of a set of \defn{vertices} $V(H)$ and a set of \defn{edges} $E(H)$, where each edge is a set of $k$ vertices.
	For a subset $S \subset V(H)$, we denote by  $\deg_H(S)$ the number of edges $e\in E(H)$ such that $S\subseteq e$.
	The \defn{minimum codegree} of $H$, denoted by $\delta(H)$, is the maximum integer $m$ such that every $(k - 1)$-set has degree at least $m$ in $H$.
	A set $S$ is \defn{supported} in $H$ if it has positive degree.
	We write \defn{$\partial H$} for the set of supported edges in $H$.
	
	A \defn{blow-up} of a $k$-graph $F$ is obtained by replacing each each vertex $x \in V(F)$ by a non-empty vertex set $V_x$ and each edge $e = x_1 \dotsc x_k \in E(F)$ by a complete $k$-partite $k$-graph on parts $V_{x_1}, V_{x_2}, \dotsc, V_{x_k}$. If $F^\ast$ is a blow-up of $F$, then there is a \defn{projection} map $\phi \colon V(F^\ast) \to V(F)$ defined by $\phi(v) = x$ for $v \in V_x$.
	Moreover, we write $\phi(U) = \set{\phi(v)\colon v \in U}$ for $U\subset V(F^\ast)$.
	
	
	We call $F^\ast$ a \defn{$(\gamma, m)$-regular blow-up} of $F$ if $F^\ast$ is a blow-up of $F$ where each part has size in the interval $[(1 - \gamma)m, (1 + \gamma)m]$.
	We abbreviate to \defn{$m$-regular blow-up} if $\gamma = 0$.
	Finally,~$F^\ast$ is called a \defn{$(\gamma,m)$-nearly-regular blow-up} if $F^\ast$ is a blow-up of $F$ where all but at most one part has size in the interval $[(1 - \gamma)m, (1 + \gamma)m]$, and the size of the (at most one) remaining exceptional part is exactly one. Note that every $(\gamma,m)$-regular blow-up is, in particular, a $(\gamma,m)$-nearly-regular blow-up.
	
	\subsection{Key lemmata}
	As anticipated in the introduction, our strategy is as follows: we would like to partition the graph into \emph{well-behaved} pieces, meaning that each piece can be covered with a sphere and that it is possible to glue all such spheres together into a spanning one.
	We now give more details and state the two key lemmata.
	
	Let $H$ be a $k$-graph with no isolated vertices and $\comin(H) \ge (1/2+o(1))n$.
	Methods from \cite{lang2023tiling} (see \cref{sec:property_graph}) can already cover all but at most $o(n)$ vertices of $H$ with a family $\cF$ of pairwise vertex-disjoint $k$-graphs with $\abs{\cF}=O(1)$ such that each $k$-graph $F^\ast \in \cF$ is the blow-up of a graph $F$ with no isolated vertices and $\comin(F) \ge (1/2+o(1))\abs{V(F)}$.
	However, this alone, is not sufficient for us.
	Indeed we want to cover the leftover as well and make sure that the parts are well-connected in a path-like fashion as described above.
	
	In order to achieve that, we modify the graphs in $\cF$ to get a sequence of $k$-graphs $F_1^\ast, \dotsc, F_{\ell}^\ast$ with $\ell=O(1)$ satisfying the following properties.
	Each $F_i^\ast$ is the nearly-regular blow-up of a $k$-graph $F_i$ with no isolated vertices and $\comin(F_i) \ge (1/2+o(1))\abs{V(F_i)}$; each vertex of $H$ is covered by at least one $F_i^\ast$; each $F_i^\ast$ can share vertices only with the $k$-graph coming before and the one coming after in the sequence.
	More precisely, $F_i^\ast$ and $F_{i+1}^\ast$ share exactly $k$ vertices which induce an edge in both $F_i^\ast$ and $F_{i+1}^\ast$.
	The formal statement of the first key lemma is as follows.
	\begin{lemma}[Blow-up chain]\label{lem:findblowupchain}
		Let $1/n\ll 1/m_2\ll 1/m_1 \ll 1/s, \gamma \ll \eps, 1/k\leq 1$, and let $H$ be an $n$-vertex $k$-graph with $\comin(H) \ge (1/2+\eps)n$ and no isolated vertices. Then, there exists a sequence of $s$-vertex $k$-graphs $F_1,\dotsc, F_\ell$ and a sequence of subgraphs $F_1^\ast,\dotsc, F_\ell^\ast\subseteq H$ such that the following properties hold for each $i \in [\ell]$ and $j \in [\ell-1]$:
		\begin{enumerate}[label = \textup{(}\arabic*\textup{)}]
			\item \label{blowup_1} $F_i$ has no isolated vertices and $\comin(F_i) \ge (1/2+\eps/2)\abs{V(F_i)}$,
			\item \label{blowup_2} there exists an $m_i^\ast\in [m_1,m_2]$ such that $F_i^\ast$ is a $(\gamma, m_i^\ast)$-nearly-regular blow-up of $F_i$,
			\item \label{blowup_3} $V(F_1^\ast) \cup \dots \cup V(F_\ell^\ast)=V(H)$,
			\item \label{blowup_4} $V(F_i^\ast)\cap V(F_j^\ast)=\emptyset$ if $\abs{i - j}\geq 2$, and
			\item \label{blowup_5} $V(F_j^\ast)\cap V(F_{j+1}^\ast)$ has size $k$ and induces an edge in $F_j^\ast$ and $F_{j+1}^\ast$ which is disjoint with the singleton parts of $F_j^\ast$ and $F_{j+1}^\ast$ \textup{(}if they exist\textup{)}.
		\end{enumerate}
	\end{lemma}
	
	Suppose we have a sequence $F_1^\ast, \dotsc, F_\ell^\ast$ as in the statement of \cref{lem:findblowupchain}.
	We would be done if we could cover each $F_i^\ast$ with a spanning sphere, while making sure that the edges induced by $V(F_{i-1}^\ast)\cap V(F_{i}^\ast)$ and $V(F_{i}^\ast)\cap V(F_{i+1}^\ast)$ are facets.
	In fact, then we would simply glue these spheres along the common facet and get a spanning sphere of $H$.
	This will be done with our second key lemma.
	
	\begin{lemma}[Allocation]\label{lem:allocation}
		Let $1/m\ll 1/s\ll \gamma \ll \eps, 1/k\leq 1/3$, and let $R$ be an $s$-vertex $k$-graph without isolated vertices and with $\comin(R) \ge (1/2 + \eps)s$. Let $R^\ast$ be a $(\gamma, m)$-nearly-regular blow-up of $R$ and let $f_1, f_2 \in E(R^\ast)$ such that $\phi(f_1)$, $\phi(f_2)$ and the vertex in $R$ corresponding to the singleton part of $R^\ast$ \textup{(}if it exists\textup{)} are all disjoint. Then, $R^\ast$ contains a spanning copy of $\bS^{k - 1}$ where $f_1$ and $f_2$ are facets.
	\end{lemma}
	
	
	\subsection{Proof of main result}
	
	Assuming \cref{lem:findblowupchain,lem:allocation}, we can easily prove our main result.
	
	\begin{proof}\textcolor{red}{TOPROVE 0}\end{proof}
	
	\subsection{Lower bound constructions}\label{sec:constructions}
	
	The following construction due to Georgakopoulos et al.~\cite{georgakopoulos2022spanning} shows that the degree condition in \cref{conj:mainconj} is tight for $k=3$.
	Let $H$ be an $n$-vertex $3$-graph where $V(H) = \set{u,v}\cup X \cup Y$ where $\abs{X} = \abs{Y} = (n-2)/2$ and $E(H)$ consists of all possible $3$-edges with the exception of those meeting both $X$ and $Y$. It is easy to check that $\comin(H) = n/2-O(1)$ and, furthermore, $H$ does not contain a spanning copy of a $2$-sphere. Indeed, if it did, removing every edge that contains both $u$ and $v$ from this copy splits the copy of the sphere into two tight components, which gives a contradiction using elementary topological arguments.
	
	This construction naturally generalises to $k$-graphs by replacing $\set{u,v}$ with a set of $k - 1$ vertices.
	{Indeed, let $T = \set{u_1, \dotsc, u_{k - 1}}$ be the set of $k - 1$ vertices that replaces $\set{u, v}$ and suppose there is a spanning $(k - 1)$-sphere $S$ in $H$. Note that at most two edges of $S$ contain~$T$. These edges form a $(k - 1)$-ball with no vertices in its interior. Since a $(k - 1)$-sphere is $(k - 2)$-connected in the topological sense, the removal of these edges does not disconnect $S$. However, removing all edges that contain $T$ from $H$ splits $H$ into two tight components.}
	
	\section{Tools}\label{sec:tools}
	
	\subsection{Connectivity}
	
	A $k$-uniform \defn{tight walk} $W$ in a $k$-graph $G$ comes with an ordered multiset of vertices such that the edges of $W$ are precisely the $k$-sets of consecutive vertices. A \defn{tight path} is a tight walk that does not repeat vertices.
	The following result~\cite[Prop.~5.1]{LS23} ties tight connectivity (recall this definition from the introduction) and tight walks. We include the proof for completeness.
	
	\begin{lemma}\label{lem:tight-connectivity-walk}
		Let $H$ be a $k$-graph. Then $H$ is tightly connected if and only if $H$ has no isolated vertices and contains a tight walk that contains every edge of $H$ as a subwalk.
	\end{lemma}
	
	\begin{proof}\textcolor{red}{TOPROVE 1}\end{proof}
	
	The \defn{order} of a tight walk $W$ is the number of vertices in the multiset. We now bound the order of the tight walk given by the previous lemma.
	
	\begin{lemma}\label{lem:tight-connectivity-bounded-tight-walk}
		Let $H$ be an $n$-vertex tightly connected $k$-graph.
		Then $H$ contains a tight walk of order at most $n^{2k}$ that contains each edge of $H$ as a subwalk.
	\end{lemma}
	
	\begin{proof}\textcolor{red}{TOPROVE 2}\end{proof}
	
	
	\begin{lemma}[Tightly connected]\label{lem:dirac-to-tightly-connected}
		Every $n$-vertex $k$-graph with $\comin(H) \geq \floor{(n - k + 1)/2}$ and no isolated vertices is tightly connected.
	\end{lemma}
	
	\begin{proof}\textcolor{red}{TOPROVE 3}\end{proof}
	
	\subsection{Minimum supported \texorpdfstring{$d$}{d}-degree}
	
	For $1 \leq d < k$ and a $k$-graph $H$, the \defn{minimum $d$-degree} of $H$, denoted by $\delta_d(H)$, is the maximum integer $m$ such that every set of $d$ vertices has degree at least $m$ in $H$.
	We write $\partial_d H$ for the set of supported $d$-sets in $H$.
	For non-empty~$H$, the \defn{minimum supported $d$-degree}, denoted by $\comin_d(H)$, is the maximum integer $m$ such that every supported $d$-set has degree at least $m$ in $H$.
	If $H$ is empty, we set $\comin_d(H)=0$.
	We remark that $\delta_{k - 1}(H) = \delta(H)$ and $\comin_{k - 1}(H)=\comin(H)$.
	
	Analogously to the usual minimum $d$-degree, the notion of minimum supported $d$-degree is stronger for larger $d$.
	This is formalised in the following fact.
	
	\begin{fact}\label{claim:supported_minimum_degree}
		Let $H$ be a $k$-graph and $d$ be an integer with $1 \le d < k - 1$.
		Then $\comin_d(H) \ge \frac{\comin(H)}{k-d} \cdot \comin_{d+1}(H)$.
	\end{fact}
	
	\begin{proof}\textcolor{red}{TOPROVE 4}\end{proof}
	
	\subsection{Concentration}
	
	We use the following standard concentration bound.
	\begin{lemma}[{\cite[Cor.~2.2]{concentration}}]\label{lem:concentration}
		Let $V$ be an $n$-set with a function $h$ from the $s$-sets of $V$ to $\bR$.
		Suppose that there exists $K \geq 0$ such that $\abs{h(S)-h(S')} \le K$ for any $s$-sets $S, S' \subset V$ with $\abs{S \cap S'} = s-1$.
		Let $S \subset V$ be an $s$-set chosen uniformly at random.
		Then, for any $\ell >0$,
		\begin{equation*}
			\Prob(\abs{h(S) - \Exp [h(S)]} \geq \ell) \leq 2 \exp\biggl(-\frac{2 \ell^2}{\min\set{s, n-s} K^2}\biggr).
		\end{equation*}
	\end{lemma}
	
	\subsection{Property graph} \label{sec:property_graph}
	
	Given a property $\cP$ and a graph $H$ satisfying $\cP$, we are interested in which subgraphs of $H$ inherit the property $\cP$. Following~\cite{lang2023tiling}, this is formalised in terms of the property graph.
	
	\begin{definition}[Property graph] \label{def:property-graph}
		For an $m$-graph $H$ and a family of $s$-vertex $m$-graphs $\cP$, the \defn{property graph}, denoted by $\PG{H}{\cP}{s}$, is the $s$-graph on vertex set $V(H)$ with an edge $S \subset V(H)$ whenever the induced subgraph $H[S]$ \defn{satisfies}~$\cP$, that is $H[S] \in \cP$.
	\end{definition}
	
	We use the following result, which appeared in the context of hypergraph tilings~\cite[Lemma~4.4]{lang2023tiling}.
	For sake of completeness, its proof can be found in \cref{lem:almost-blow-up-cover}.
	
	\begin{lemma}[Almost perfect blow-up-tiling] \label{lem:covering-with-blow-ups}
		For all $2\leq k \leq s$, $m\geq 1$ and $\mu,\eta >0$, there is an $n_0>0$ such that the following holds for every $s$-vertex $k$-graph property $\cP$ and $k$-graph $H$ on $n \geq n_0$ vertices  with
		\begin{equation*}
			\delta_1 \bigl(\PG{H}{ \cP }{s}\bigr) \geq   (1 - 1/s + \mu) \tbinom{n-1}{s-1}.
		\end{equation*}
		All but at most $\eta n$ vertices of $H$ may be covered with pairwise vertex-disjoint $m$-regular blow-ups of members of $\cP$.
	\end{lemma}
	
	We define \defn{$\cP(\eps, k)$} to be the family of $k$-graphs $H$ with $\comin(H) \geq (1/2 + \eps)\abs{V(H)}$ and without any isolated vertices.
	The next lemma shows that $\cP(\eps,k)$ satisfies a local inheritance principle.
	
	\begin{lemma}\label{lem:robustmoregeneral}
		Let $1/n\ll 1/s \ll 1/k, 1/t, \eps$. Let $H$ be an $n$-vertex $k$-graph such that $H\in \cP(\eps,k)$, and let $T\subseteq V(H)$ be a $t$-subset. Let $S$ be an $(s-t)$-subset of $V(H) \setminus T$ chosen uniformly at random.
		Then, with probability at least $1-e^{-\sqrt{s}}$, we have that $H[T\cup S]\in \cP(\eps/2,k)$.
	\end{lemma}
	
	\begin{proof}\textcolor{red}{TOPROVE 5}\end{proof}
	
	The $t=1$ case of \cref{lem:robustmoregeneral} has the following important corollary.
	
	\begin{corollary}[Property graph is robust]\label{cor:propgraphrobust}
		Let $1/n\ll 1/s \ll 1/k, \eps$. Let $H$ be an $n$-vertex $k$-graph such that $H\in \cP(\eps,k)$. Then,
		\begin{equation*}
			\delta_1 \bigl(\PG{H}{ \cP(\eps/2,k) }{s} \bigr) \geq \bigl(1-1/s^2 \bigr) \tbinom{n-1}{s-1}.
		\end{equation*}
	\end{corollary}
	
	
	\section{Blow-up chains}\label{sec:blow-up}
	
	This section is dedicated to the proof of \cref{lem:findblowupchain}, for which we need a few preliminary results concerning blow-ups. The first one is a well-known insight of Erd\H{o}s~\cite{Erdos1964hypextremal}, stating that the Tur\'an density of $K_s^{(s)}(b)$ is zero, where \defn{$K_s^{(s)}(b)$} denotes the complete $s$-partite $s$-graph with each part of size $b$.
	
	\begin{theorem}
		\label{thm:erd64}
		For all $s\geq 2$, $b \geq 1$ and $\gamma > 0$, there is $n_0>0$ such that every $s$-graph $P$ on $n\geq n_0$ vertices with $e(P) \geq \gamma n^s$ contains a copy of $K_s^{(s)}(b)$.
	\end{theorem}
	
	Next, we present two applications of \cref{thm:erd64}.
	For a partition $\cV = \set{V_1, \dotsc, V_s}$ of a ground set $V$, a subset $A \subset V$ is called \defn{$\cV$-partite} if it has at most one element in each part.
	A $k$-graph $H$ on vertex set $V$ is \defn{$\cV$-partite} if all its edges are $\cV$-partite.
	We often do not explicitly mention the partition $\cV$ and just speak of an \defn{$s$-partite} graph $H$.
	Finally a blow-up $F^\ast \subset H$ of a $k$-graph $F$ is called \defn{consistent} in the $\cV$-partite $H$ if the parts of $F^\ast$ can be written $U_1, \dotsc, U_s$ such that $U_i \subset V_i$ for each $i \in [s]$.
	
	\begin{lemma}\label{lem:pigeonhole}
		Let $1/m_2 \ll 1/m_1, 1/s, 1/k$.
		Let $\cP$ be a family of $s$-vertex $k$-graphs. Let $\cA$ be an $s$-partite $k$-graph with parts $A_1, \dotsc, A_s$ where $\abs{A_i} = m_2$ for all $i\in[s]$. Suppose that every $\set{A_i}_{i \in [s]}$-partite $s$-set induces a member of $\cP$ in $\cA$. Then, there exists some $P \in \cP$ so that $\mathcal{A}$ contains a consistent $m_1$-regular blow-up of $P$.
	\end{lemma}
	
	\begin{proof}\textcolor{red}{TOPROVE 6}\end{proof}
	
	\begin{lemma}[Rooted blow-ups]\label{lem:rooted-blow-ups}
		Let $1/n \ll 1/m_2 \ll 1/m_1, 1/s,1/k$, $0 \leq q < s$ and $\mu>0$.
		Let~$G$ be an $n$-vertex $k$-graph.
		Let $\cV = \set{V_i}_{i \in [q]}$ be a family of pairwise disjoint $m_2$-sets in $V(G)$.
		Let $\cP$ be a family of $s$-vertex $k$-graphs.
		Suppose that, for every $\cV$-partite $q$-set $X$, there are at least $\mu n^{s - q}$ subsets $S \subseteq V(G) \setminus \big(X \cup V_1 \cup \dots \cup V_q\big)$ of size $s - q$ such that $G[S \cup X] \in \cP$.
		Then there is a family $\cU = \set{U_i}_{i \in [s - q]}$ of pairwise disjoint $m_1$-sets disjoint with $\bigcup_{i\in[q]} V_i$ such that the $s$-partite $k$-graph induced by $\cV \cup \cU$ contains a consistent $m_1$-regular blow-up of some $P\in \cP$.
	\end{lemma}
	
	\begin{proof}\textcolor{red}{TOPROVE 7}\end{proof}
	
	
	We are now ready to prove the main result of this section, \cref{lem:findblowupchain}.
	
	\begin{proof}\textcolor{red}{TOPROVE 8}\end{proof}
	
	
	\section{Geometric observations}
	\label{sec:geometric_obs}
	
	Recall that a \defn{simplicial $k$-sphere} is a homogeneous simplicial $k$-complex that is homeomorphic to $\bS^k$. In this section we will first introduce some ways to make new simplicial spheres from old ones and then show that certain $k$-graphs contain spanning copies of $\bS^{k - 1}$ (see \cref{lem:partitesphere,lem:tightpathsphere}).
	
	The first operation that makes new simplicial spheres is gluing along a facet.
	
	\begin{remark}\label{rmk:gluecommonface}
		Let $\cK$ and $\cK'$ both be simplicial $k$-spheres whose intersection is a $k$-simplex $F$. Let $\cK''$ be the simplicial complex whose vertex-set is $V(\cK) \cup V(\cK')$ and whose simplices are those of $\cK$ and $\cK'$ except $F$. This operation glues $\cK$ and $\cK'$ on $F$ and so $\cK''$ is a simplicial $k$-sphere.
	\end{remark}
	
	Given a topological space $X$, the \defn{suspension} of $X$ is obtained by taking the cylinder $X \times [0, 1]$, contracting $X \times \set{0}$ to a single point, and contracting $X \times \set{1}$ to a single point. Crucially, the suspension of $\bS^{k - 1}$ is $\bS^k$ (for $k \geq 1$). We now define the analogous operation for simplicial complexes.
	
	\begin{definition}[Suspension]
		Let $\cK$ be a homogeneous simplicial $(k - 1)$-complex. The \defn{suspension} $\cK'$ of $\cK$ is a homogeneous simplicial $k$-complex whose vertex-set is $\set{u, v} \cup V(\cK)$, where $u$ and $v$ are new vertices, and whose set of simplices is
		\begin{equation*}
			\set{\set{u} \cup F \colon F \in E(\cK)} \cup \set{\set{v} \cup F \colon F \in E(\cK)}.
		\end{equation*}
	\end{definition}
	
	For example, the suspension of a 4-cycle is an octahedron.
	
	\begin{lemma}\label{lem:suspensions}
		The suspension of a simplicial $(k - 1)$-sphere is a simplicial $k$-sphere.
	\end{lemma}
	
	\begin{proof}\textcolor{red}{TOPROVE 9}\end{proof}
	
	We now show that certain $k$-partite $k$-graphs contain spanning copies of $\bS^{k - 1}$.
	Given positive integers $k$ and $a_1, \dotsc, a_k \ge 1$, we denote by \defn{$K_k^{(k)}(a_1, \dotsc, a_k)$} the complete $k$-partite $k$-graph with parts of size $a_1, \dotsc, a_k$.
	
	\begin{lemma}\label{lem:partitesphere}
		For any $k \geq 2$, the following $k$-partite $k$-graphs contain spanning copies of $\bS^{k - 1}$\textup{:}
		\begin{enumerate}[label = \textup{(}\alph*\textup{)}]
			\item \label{itm:partitesphere-a} $K_k^{(k)}(2, \dotsc, 2, 2, \ell, \ell)$ for any $\ell \geq 2$,
			\item \label{itm:partitesphere-b} $K_k^{(k)}(2, \dotsc, 2, 3, \ell, \ell)$ for any $\ell \geq 3$.
		\end{enumerate}
	\end{lemma}
	
	\begin{proof}\textcolor{red}{TOPROVE 10}\end{proof}
	
	We now show that certain blow-ups of tight paths contain large copies of $\bS^{k - 1}$ with special properties. We say that a copy $S$ of $\bS^{k - 1}$ in a blow-up of a tight path $P$ is \defn{doubly edge-covering} if there are two families $\set{f_e \colon e \in E(P)}$, $\set{f'_e \colon e \in E(P)}$ of facets of $S$ such that each family is vertex-disjoint, and, for each $e \in E(P)$, $f_e \neq f'_e$ and $\phi(f_e) = \phi(f'_e) = e$. For positive integers $a_1, \dotsc, a_\ell$ (where $\ell \geq k \geq 2$), \defn{$P_\ell^{(k)}(a_1, \dotsc, a_\ell)$} denotes a blow-up of the $k$-uniform tight path on $\ell$ vertices where the $i$th vertex has been blown-up by $a_i$.
	
	\begin{lemma}\label{lem:thin-path}
		For $k\geq 2$, the blow-up $P_{k + 1}^{(k)}(1, 2, 2, \dotsc, 2, 2, 1)$ contains a spanning copy of $\bS^{k - 1}$ that is doubly edge-covering.
	\end{lemma}
	\begin{proof}\textcolor{red}{TOPROVE 11}\end{proof}
	
	\begin{lemma}\label{lem:growing-path}
		Let $\ell - 1 \geq k \geq 2$.
		If the blow-up  $P_\ell^{(k)}(a_1, \dotsc, a_\ell)$ contains a spanning copy of~$\bS^{k - 1}$ that is doubly edge-covering, then so does $P_{\ell + 1}^{(k)}(1, a_1 + 1, \dotsc, a_{k - 1} + 1, a_k, \dotsc, a_\ell)$.
	\end{lemma}
	\begin{proof}\textcolor{red}{TOPROVE 12}\end{proof}
	
	\begin{lemma}\label{lem:tightpathsphere}
		For $\ell - 1 \geq k \geq 2$, the blow-up $P_\ell^{(k)}(k, \dotsc, k)$ contains a \textup{(}not necessarily spanning\textup{)} copy $S$ of $\bS^{k - 1}$ that is doubly edge-covering.
	\end{lemma}
	
	\begin{proof}\textcolor{red}{TOPROVE 13}\end{proof}
	
	
	
	
	
	\section{Allocation}
	\label{sec:allocation}
	
	This section is dedicated to the proof of \cref{lem:allocation}.
	
	\subsection{Filling the blow-ups with spheres}
	
	The following result allows us to cover the vertices in a suitable blow-up with pairwise vertex-disjoint simplicial spheres.
	For a $k$-graph $R$, we define \defn{$\partial_2 R$} to be the set of pairs of vertices of $R$ which are contained in at least one edge of $R$.
	
	\begin{lemma}\label{lem:filling}
		Let $1/m \ll 1/s \ll \gamma \ll \eps, 1/k \le 1/3$ with $k$, $s$, and $m$ positive integers. Let $R$ be an $s$-vertex $k$-graph without isolated vertices and with $\comin(R) \ge (1/2 + \eps)s$.
		Let $R^\ast$ be {$(\gamma, m)$-regular blow-up} of $R$.
		Let $\set{f_e \colon e \in E(R)}$ be a collection of pairwise vertex-disjoint edges of $R^\ast$ with $\phi(f_e) = e$ for all $e \in E(R)$.
		
		Then there exists a family of copies of $\bS^{k - 1}$, $\set{S_e \colon e \in E(R)}$, such that $f_e$ is a facet of $S_e$ for each $e \in E(R)$ and $\set{V(S_e) \colon e \in E(R)}$ is a partition of $V(R^\ast)$.
	\end{lemma}
	
	\begin{proof}\textcolor{red}{TOPROVE 14}\end{proof}
	
	
	\subsection{Connecting spheres}
	
	Now we are ready to finish the proof of \cref{lem:allocation}.
	
	\begin{proof}\textcolor{red}{TOPROVE 15}\end{proof}
	
	{
		\fontsize{11pt}{12pt}
		\selectfont
		
		\hypersetup{linkcolor={red!70!black}}
		\setlength{\parskip}{2pt plus 0.3ex minus 0.3ex}
		
		\newcommand{\etalchar}[1]{$^{#1}$}
		\begin{thebibliography}{GHMN22}
			\providecommand{\url}[1]{\texttt{#1}}
			\providecommand{\urlprefix}{\textsc{url:} }
			\expandafter\ifx\csname urlstyle\endcsname\relax
			\providecommand{\doi}[1]{doi:\discretionary{}{}{}#1}\else
			\providecommand{\doi}{doi:\discretionary{}{}{}\begingroup \urlstyle{rm}\Url}\fi
			
			\bibitem[BLP21]{Bal21}
			\textsc{Jozsef Balogh}, \textsc{Nathan Lemons}, and \textsc{Cory Palmer} (2021).
			\newblock \href{https://doi.org/10.1137/20M1336989}{Maximum size intersecting families of bounded minimum positive co-degree}.
			\newblock \emph{SIAM Journal on Discrete Mathematics} \textbf{35}(3), 1525--1535.
			
			\bibitem[BST09]{BST09}
			\textsc{Julia B\"{o}ttcher}, \textsc{Mathias Schacht}, and \textsc{Anusch Taraz} (2009).
			\newblock \href{https://doi.org/10.1007/s00208-008-0268-6}{Proof of the bandwidth conjecture of {B}ollob{\'a}s and {K}oml{\'o}s}.
			\newblock \emph{Mathematische Annalen} \textbf{343}(1), 175--205.
			
			\bibitem[DH81]{DH81}
			\textsc{David~E. Daykin} and \textsc{Roland H\"{a}ggkvist} (1981).
			\newblock \href{https://doi.org/10.1017/S0004972700006924}{Degrees giving independent edges in a hypergraph}.
			\newblock \emph{Bull. Austral. Math. Soc.} \textbf{23}(1), 103--109.
			
			\bibitem[Dir52]{Dirac}
			\textsc{Gabriel~A. Dirac} (1952).
			\newblock \href{https://doi.org/https://doi.org/10.1112/plms/s3-2.1.69}{Some theorems on abstract graphs}.
			\newblock \emph{Proceedings of the London Mathematical Society} \textbf{s3-2}(1), 69--81.
			
			\bibitem[Erd64]{Erdos1964hypextremal}
			\textsc{Paul Erd\H{o}s} (Sep. 1964).
			\newblock \href{https://doi.org/10.1007/BF02759942}{On extremal problems of graphs and generalized graphs}.
			\newblock \emph{Israel Journal of Mathematics} \textbf{2}(3), 183--190.
			
			\bibitem[GHMN22]{georgakopoulos2022spanning}
			\textsc{Agelos Georgakopoulos}, \textsc{John Haslegrave}, \textsc{Richard Montgomery}, and \textsc{Bhargav Narayanan} (2022).
			\newblock \href{https://doi.org/10.4171/JEMS/1101}{Spanning surfaces in 3-graphs}.
			\newblock \emph{Journal of the European Mathematical Society} \textbf{24}(1), 303--339.
			
			\bibitem[GIKM17]{concentration}
			\textsc{Catherine Greenhill}, \textsc{Mikhail Isaev}, \textsc{Matthew Kwan}, and \textsc{Brendan~D. McKay} (Jun. 2017).
			\newblock \href{https://doi.org/10.1016/j.ejc.2017.02.003}{The average number of spanning trees in sparse graphs with given degrees}.
			\newblock \emph{European Journal of Combinatorics} \textbf{63}, 6--25.
			
			\bibitem[KO05]{KO05}
			\textsc{Daniela K{\"u}hn} and \textsc{Deryk Osthus} (2005).
			\newblock \href{https://doi.org/10.1002/jgt.20076}{Spanning triangulations in graphs}.
			\newblock \emph{Journal of Graph Theory} \textbf{49}(3), 205--233.
			
			\bibitem[KOT05]{KOT05}
			\textsc{Daniela K\"{u}hn}, \textsc{Deryk Osthus}, and \textsc{Anusch Taraz} (2005).
			\newblock \href{https://doi.org/10.1016/j.jctb.2005.04.004}{Large planar subgraphs in dense graphs}.
			\newblock \emph{Journal of Combinatorial Theory, Series B} \textbf{95}(2), 263--282.
			
			\bibitem[KSS98]{KSS98}
			\textsc{J\'anos Koml\'{o}s}, \textsc{G\'abor~N. S\'{a}rk\"{o}zy}, and \textsc{Endre Szemer{\'e}di} (1998).
			\newblock \href{https://doi.org/10.1002/(SICI)1097-0118(199811)29:3<167::AID-JGT4>3.0.CO;2-O}{On the {P}\'{o}sa-{S}eymour conjecture}.
			\newblock \emph{Journal of Graph Theory} \textbf{29}(3), 167--176.
			
			\bibitem[Lan23]{lang2023tiling}
			\textsc{Richard Lang} (Aug. 2023).
			\newblock \href{http://arxiv.org/abs/2308.12281}{Tiling dense hypergraphs}.
			\newblock arXiv:2308.12281.
			
			\bibitem[LSM22]{LS22}
			\textsc{Richard Lang} and \textsc{Nicol\'{a}s Sanhueza-Matamala} (2022).
			\newblock \href{https://doi.org/10.1112/jlms.12561}{Minimum degree conditions for tight {H}amilton cycles}.
			\newblock \emph{Journal of the London Mathematical Society} \textbf{105}(4), 2249--2323.
			
			\bibitem[LSM23]{LS23}
			\textsc{Richard Lang} and \textsc{Nicol\'{a}s Sanhueza-Matamala} (2023).
			\newblock \href{https://doi.org/10.1112/plms.12552}{On sufficient conditions for spanning structures in dense graphs}.
			\newblock \emph{Proceedings of the London Mathematical Society} \textbf{127}(3), 709--791.
			
			\bibitem[LSM24a]{langsanhueza2024dirac}
			\textsc{Richard Lang} and \textsc{Nicol\'{a}s Sanhueza-Matamala} (Dec. 2024).
			\newblock \href{http://arxiv.org/abs/2412.19912}{Blowing up {D}irac's theorem}.
			\newblock arXiv:2412.19912.
			
			\bibitem[LSM24b]{LS24}
			\textsc{Richard Lang} and \textsc{Nicol\'{a}s Sanhueza-Matamala} (Dec. 2024).
			\newblock \href{http://arxiv.org/abs/2412.14891}{A hypergraph bandwidth theorem}.
			\newblock arXiv:2412.14891.
			
			\bibitem[LT19]{LT19}
			\textsc{Zur Luria} and \textsc{Ran~J. Tessler} (2019).
			\newblock \href{https://doi.org/10.1112/plms.12247}{A sharp threshold for spanning 2-spheres in random 2-complexes}.
			\newblock \emph{Proceedings of the London Mathematical Society} \textbf{119}(3), 733--780.
			
			\bibitem[MZG25]{MZ25}
			\textsc{Richard Mycroft} and \textsc{Camila Z{\'a}rate-Guer{\'e}n} (May 2025).
			\newblock \href{http://arxiv.org/abs/2505.11400}{Positive codegree thresholds for {H}amilton cycles in hypergraphs}.
			\newblock arXiv:2505.11400.
			
			\bibitem[Pik23]{Pik23}
			\textsc{Oleg Pikhurko} (2023).
			\newblock \href{https://doi.org/10.37236/11912}{On the limit of the positive $\ell$-degree {T}ur{\'a}n problem}.
			\newblock \emph{Electronic Journal of Combinatorics} P3.25.
			
			\bibitem[RRR{\etalchar{+}}19]{RRR19}
			\textsc{Christian Reiher}, \textsc{Vojt{\v{e}}ch R{\"o}dl}, \textsc{Andrzej Ruci{\'n}ski}, \textsc{Mathias Schacht}, and \textsc{Endre Szemer{\'e}di} (2019).
			\newblock \href{https://doi.org/10.1112/plms.12235}{Minimum vertex degree condition for tight {H}amiltonian cycles in $3$-uniform hypergraphs}.
			\newblock \emph{Proceedings of the London Mathematical Society} \textbf{119}(2), 409--439.
			
			\bibitem[RRS08]{RRS08a}
			\textsc{Vojt{\v{e}}ch R{\"o}dl}, \textsc{Andrzej Ruci{\'n}ski}, and \textsc{Endre Szemer{\'e}di} (2008).
			\newblock \href{https://doi.org/10.1007/s00493-008-2295-z}{An approximate {D}irac-type theorem for $k$-uniform hypergraphs}.
			\newblock \emph{Combinatorica} \textbf{28}, 229--260.
			
			\bibitem[RRS11]{rodl2011dirac}
			\textsc{Vojt{\v{e}}ch R{\"o}dl}, \textsc{Andrzej Ruci{\'n}ski}, and \textsc{Endre Szemer{\'e}di} (Jun. 2011).
			\newblock \href{https://doi.org/10.1016/j.aim.2011.03.007}{Dirac-type conditions for {H}amiltonian paths and cycles in 3-uniform hypergraphs}.
			\newblock \emph{Advances in Mathematics} \textbf{227}(3), 1225--1299.
			
			\bibitem[SEB73]{SEB73}
			\textsc{Vera~T. S{\'o}s}, \textsc{Paul Erd{\H{o}}s}, and \textsc{William~G. Brown} (1973).
			\newblock \href{https://doi.org/10.1007/BF02018585}{On the existence of triangulated spheres in 3-graphs, and related problems}.
			\newblock \emph{Periodica Mathematica Hungarica} \textbf{3}(3-4), 221--228.
			
			\bibitem[SS19]{SS19}
			\textsc{Mikl\'os Simonovits} and \textsc{Endre Szemer{\'e}di} (2019).
			\newblock \href{https://doi.org/10.1007/978-3-662-59204-5_14}{Embedding {G}raphs into {L}arger {G}raphs: {R}esults, {M}ethods, and {P}roblems}.
			\newblock \emph{Building Bridges II: Mathematics of L\'aszl\'o Lov\'asz}, 445--592.
			
		\end{thebibliography}
	}
	
	\appendix
	
	\section{Covering most vertices with blow-ups}\label{lem:almost-blow-up-cover}
	
	In this section, we show \cref{lem:covering-with-blow-ups} following the original exposition~\cite[Lemma~4.4]{lang2023tiling}.
	We require a simple Dirac-type bound for hypergraph matchings due to Daykin and H\"{a}ggkvist~\cite{DH81} (for a short proof, see~\cite[Lemma~B.2]{langsanhueza2024dirac}).
	
	\begin{lemma}\label{lem:matching}
		Let $ 1/n \ll 1/s, \mu$ with $n$ divisible by $s$.
		Let $P$ be an $n$-vertex $s$-graph with $\delta_1(P) \geq (1 - 1/s + \mu) \tbinom{n-1}{s-1}$.
		Then $P$ has a perfect matching.
	\end{lemma}
	
	The proof of \cref{lem:covering-with-blow-ups} is then based on the next fact, which tiles a $k$-graph $P$ almost perfectly with constant-sized blow-ups of edges, provided that the minimum degree of $P$ forces a perfect matching (as in \cref{lem:matching}).
	For a $k$-graph $H$ and a family of $k$-graphs $\cR$, an \defn{$\cR$-tiling} is a set of pairwise vertex-disjoint $k$-graphs $R_1,\dots,R_\ell \subset H$ with $R_1,\dots,R_\ell \in \cR$.
	
	\begin{lemma}\label{lem:blow-up-matching}
		Let $1/n \ll \mu,\, 1/b,\, 1/s$.
		Then every $n$-vertex $s$-graph $P$ with $\delta_1(P) \geq (1-1/s+\mu) \binom{n-1}{s-1}$ contains a $K_s^{(s)}(b)$-tiling missing at most $\mu n$ vertices.
	\end{lemma}
	
	To derive \cref{lem:covering-with-blow-ups}, we first apply \cref{lem:blow-up-matching} to cover most vertices of the property $s$-graph with complete partite graphs.
	Then we use \cref{thm:erd64} to distil most vertices of each of the partite graphs into the desired blow-ups.
	
	\begin{proof}\textcolor{red}{TOPROVE 16}\end{proof}
	
	It remains to show \cref{lem:blow-up-matching}.
	A first proof follows from a straightforward application of Szemerédi's (Weak Hypergraph) Regularity Lemma.
	Here we give an alternative (possibly simpler) argument.
	More precisely, we derive \cref{lem:blow-up-matching} by iteratively applying the following result.
	In each step, we turn a tiling $\cB$ with `big' tiles $B$ into a tiling $\cL$ of `little' tiles $L$ covering additional $(\mu/8)^2n$ vertices.
	So we arrive at \cref{lem:blow-up-matching} after about $(8/\mu)^2$ steps.
	
	\begin{lemma}\label{lem:larger-matching}
		Let	$1/n \ll 1/m \ll  1/\ell \ll 1/s, \mu$, and let $P$ be an $s$-graph on $n$ vertices with $\delta_1(P) \geq \left(1-1/s+ \mu \right) \binom{n-1}{s-1}$.
		Set $B = K_{s}^{(s)}(m)$ and $L= K_{s}^{(s)}(\ell)$.
		Suppose that $P$ contains a $B$-tiling $\cB$ on $\lambda n$ vertices with $\lambda \leq 1-\mu/8$.
		Then $P$ contains an $L$-tiling on at least $(\lambda + 2^{-6}\mu^2 ) n$ vertices.
	\end{lemma}
	
	\begin{proof}\textcolor{red}{TOPROVE 17}\end{proof}
	
	We remark that the bounds on the order $n$ in the above argument are quite large (tower-type).
	A more economical approach is discussed in the conclusion of \cite{lang2023tiling} and implemented, for instance, in \cite[Lemma~B.3]{langsanhueza2024dirac}.
	
	
	
	
	
\end{document}