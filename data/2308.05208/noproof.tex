\documentclass[11pt]{amsart}

\setlength{\textwidth}{\paperwidth}
\addtolength{\textwidth}{-2in}
\setlength{\textheight}{\paperheight}
\addtolength{\textheight}{-2in}
\calclayout

\usepackage{mathtools}
\usepackage{amssymb}
\usepackage{amsthm}
\usepackage[T1]{fontenc}
\usepackage[utf8]{inputenc}
\usepackage{lmodern}
\usepackage{inconsolata}
\usepackage{microtype}
\usepackage{enumitem}
\usepackage[colorlinks,unicode]{hyperref}
\usepackage[noabbrev,capitalise,nameinlink]{cleveref}
\usepackage{mleftright} \mleftright
\usepackage{xcolor}

\definecolor{lavender}{rgb}{0.75,0,0.8}
\definecolor{darkblue}{rgb}{0,0,0.85}

\hypersetup{colorlinks=true, citecolor=lavender,linkcolor=lavender,urlcolor=lavender}

\usepackage[bb=boondox]{mathalfa}

\usepackage[backend=biber,style=zbsalph,maxnames=9,backref=true]{biblatex}
\defbibheading{bibliography}[\refname]{\section*{#1}}
\renewcommand*{\bibfont}{\small}
\addbibresource{refs.bib}

\newtheorem{theorem}{Theorem}[section]
\newtheorem{proposition}[theorem]{Proposition}
\newtheorem{corollary}[theorem]{Corollary}
\newtheorem{conjecture}[theorem]{Conjecture}
\newtheorem{question}[theorem]{Question}
\newtheorem{problem}[theorem]{Problem}
\newtheorem{lemma}[theorem]{Lemma}

\theoremstyle{definition}
\newtheorem{definition}[theorem]{Definition}
\newtheorem{remark}[theorem]{Remark}
\newtheorem{example}[theorem]{Example}

\DeclarePairedDelimiter{\paren}{(}{)}
\DeclarePairedDelimiter{\brac}{[}{]}
\DeclarePairedDelimiter{\set}{\{}{\}}
\DeclarePairedDelimiter{\abs}{\lvert}{\rvert}
\DeclarePairedDelimiter{\Abs}{\lVert}{\rVert}
\DeclarePairedDelimiter{\floor}{\lfloor}{\rfloor}
\DeclarePairedDelimiter{\ceil}{\lceil}{\rceil}


\DeclareMathOperator{\sgn}{sgn}

\let\Vec\mathbf
\newcommand{\perm}{\psi^{\mathrm{max}}}
\newcommand{\calf}{\mathcal{F}}
\newcommand{\eps}{\varepsilon}
\newcommand{\RR}{\mathbb{R}}
\newcommand{\VV}{\mathbf{V}}

\newcommand{\dfn}[1]{\textcolor{darkblue}{\emph{#1}}}

\begin{document}

\title{Ordering Candidates via Vantage Points} 

\author{Noga Alon}
\address{Department of Mathematics, Princeton University, Princeton, NJ 08544, USA and Schools of Mathematical Sciences and Computer Science, Tel Aviv University, Tel Aviv, Israel}
\email{nalon@math.princeton.edu}
\author{Colin Defant}
\address{Department of Mathematics, Harvard University, Cambridge, MA 02139, USA}
\email{colindefant@gmail.com}
\author{Noah Kravitz}
\address{Department of Mathematics, Princeton University, Princeton, NJ 08540, USA}
\email{nkravitz@princeton.edu}
\author{Daniel G. Zhu}
\address{Department of Mathematics, Massachusetts Institute of Technology, Cambridge, MA 02139, USA}
\email{zhd@princeton.edu}

\begin{abstract}
Given an $n$-element set $C\subseteq\RR^d$ and a (sufficiently generic) $k$-element multiset $V\subseteq\RR^d$, we can order the points in $C$ by ranking each point $c\in C$ according to the sum of the distances from $c$ to the points of $V$. Let $\Psi_k(C)$ denote the set of orderings of $C$ that can be obtained in this manner as $V$ varies, and let $\perm_{d,k}(n)$ be the maximum of $\lvert\Psi_k(C)\rvert$ as $C$ ranges over all $n$-element subsets of $\RR^d$. We prove that $\perm_{d,k}(n)=\Theta_{d,k}(n^{2dk})$ when $d \geq 2$ and that $\perm_{1,k}(n)=\Theta_k(n^{4\ceil{k/2}-1})$. As a step toward proving this result, we establish a bound on the number of sign patterns determined by a collection of functions that are sums of radicals of nonnegative polynomials; this can be understood as an analogue of a classical theorem of Warren. We also prove several results about the set $\Psi(C)=\bigcup_{k\geq 1}\Psi_k(C)$; this includes an exact description of $\Psi(C)$ when $d=1$ and when $C$ is the set of vertices of a vertex-transitive polytope. 
\end{abstract}

\maketitle

\section{Introduction}\label{sec:intro}

Let $d \geq 1$ and $n, k \geq 0$ be integers, and consider a set $C=\set{c_1,\ldots,c_n}$ of \dfn{candidate points} in $\RR^d$. Given a multiset $V=\set{v_1,\ldots,v_k}$ of $k$ \dfn{vantage points} in $\RR^d$,
define the function $D_V\colon \RR^d \to \RR$ by $D_V(x) = \sum_{i\in[k]} \Abs{x - v_i}$, where $\Abs{-}$ denotes the Euclidean distance and $[k]\coloneqq\set{1,\ldots,k}$. We say $V$ \dfn{distinguishes} the points in $C$ if the values $D_V(c_1),\ldots,D_V(c_n)$ are distinct. If $V$ distinguishes the points in $C$, then there is a unique permutation $\sigma$ of $[n]$ such that $D_V(c_{\sigma(1)})<\cdots<D_V(c_{\sigma(n)})$; in this case, we say $V$ \dfn{witnesses} the tuple $(c_{\sigma(1)},\ldots,c_{\sigma(n)})$, and we denote this tuple by $\Sigma_V^C$. Throughout this paper, we will identify this tuple with the function $\Sigma_V^C \colon [n] \to C$ that sends $i$ to $c_{\sigma(i)}$, as these two objects clearly contain equivalent information.

Let $\Psi_k(C)$ be the set of tuples $\Sigma_V^C$ witnessed by $k$-element multisets of $\RR^d$ that distinguish the points in $C$, and let $\psi_k(C)=\abs{\Psi_k(C)}$. In other words, $\psi_k(C)$ counts the possible rankings of $C$, where the ranking of a point is determined by the sum of its distances from $k$ vantage points.

The quantity $\psi_1(C)$ was first studied by Good and Tideman \cite{Good1977}, who viewed the points in $C$ as political candidates and the single vantage point $v_1$ as a voter who ranks the candidates based on how far away they are in the Euclidean metric. They proved that \[\psi_1(C) \leq s(n,n)+s(n,n-1)+\cdots+s(n,n-d)\] for every set $C$, where $s(n,r)$ denotes an unsigned Stirling number of the first kind; moreover, they showed that this upper bound is tight. Zaslavsky \cite{Zaslavsky2002} provided a different proof of this inequality using hyperplane arrangements. Carbonero, Castellano, Gordon, Kulick, Ohlinger, and Schmitz \cite{Carbonero2021} continued this line of work by showing that the minimum possible value of $\psi_1(C)$ is $2n-2$; this minimum is independent of the dimension $d$ because it is attained when the points in $C$ are arranged on a line. They also constructed additional point configurations $C$ for which $\psi_1(C)$ attains other values, and they initiated the investigation of $\psi_k(C)$ for larger values of $k$ (with a focus on the case $k=2$). 

Let $\perm_{d,k}(n)$ be the maximum value of $\psi_k(C)$ as $C$ ranges over all $n$-element subsets of $\RR^d$. One of our main results is the following theorem. 
The $k=2$ case asymptotically settles a problem raised in \cite{Carbonero2021}.

\begin{theorem}\label{thm:main}
If $d \geq 2$ and $k \geq 1$ are fixed, then $\perm_{d,k}(n) = \Theta_{d, k}(n^{2dk})$.  If $d=1$ and $k \geq 1$ is fixed, then $\perm_{1,k}(n)=\Theta_k(n^{4\ceil{k/2}-2})$.
\end{theorem}

Given a tuple $\calf=(f_1, \ldots, f_m)$ of real-valued functions on $\RR^N$ and a point $x \in \RR^N$, we obtain the \dfn{sign pattern} $\eps_{\calf}(x)=(\eps_1, \ldots,\eps_m)$, where $\eps_i=\sgn(f_i(x))\in\set{0,1,-1}$.  A sign pattern is \dfn{proper} if its entries are all nonzero. When $\calf$ is a tuple of polynomials of bounded degree, a classical result of Warren \cite{Warren1968} provides an upper bound for the number of distinct proper sign patterns of the form $\eps_{\calf}(x)$ as $x$ varies over $\RR^d$. In \cref{sec:sign_patterns}, we prove an analogue of Warren's theorem which may be of independent interest. This theorem (\cref{thm:sign-patterns-fractional}) bounds the number of proper sign pattern of the form $\eps_{\calf}(x)$ when the functions in $\calf$ are sums of radicals of nonnegative polynomial functions. We then apply this theorem in \cref{sec:upper} to prove the upper bounds $\perm_{d,k}(n)=O_{d,k}(n^{2dk})$ and $\perm_{1,k}(n)=O_k(n^{4\ceil{k/2}-2})$ in \cref{thm:main}. \cref{sec:lower,sec:lowerd1,sec:lowerdg2} are devoted to proving the lower bounds $\perm_{d,k}(n)=\Omega_{d,k}(n^{2dk})$ and $\perm_{1,k}(n)=\Omega_k(n^{4\ceil{k/2}-2})$ in \cref{thm:main}. Our constructions are delicate and technical and proceed by induction on the number of vantage points. 

In \cref{sec:unlimited}, we turn our attention to the set $\Psi(C)=\bigcup_{k\geq 1}\Psi_k(C)$. This is the collection of orderings of $C$ that are witnessed by arbitrarily large (finite) multisets that distinguish the points in $C$. We say a tuple $(x_1,\ldots,x_m)$ of points in $\RR^d$ is \dfn{protrusive} if for every $i\in[m-1]$, the point $x_{i+1}$ is not in the convex hull of $x_1,\ldots,x_i$. A simple argument involving the triangle inequality shows that every tuple in $\Psi(C)$ is protrusive. We show that $\Psi(C)$ is exactly equal to the set of protrusive orderings of the points in $C$ when $d=1$ (\cref{thm:protrusive_d_1}), when $n\leq 4$ (\cref{thm:4-points}), and when $C$ is the set of vertices of a vertex-transitive polytope (\cref{thm:polytope}). In a different direction, we construct a $6$-element set $C\subseteq\RR^2$ such that some protrusive orderings of $C$ are not in $\Psi(C)$. We leave open the problem of determining whether a similar construction exists with only $5$ points. 



\section{Sign patterns of sums of radicals}\label{sec:sign_patterns}

It is natural to try to estimate the number of proper sign patterns arising from a tuple $\calf=(f_1, \ldots, f_m)$ of real-valued functions on $\RR^N$. A classical result of Warren gives an upper bound for the case where $f_1, \ldots, f_m$ are polynomials.

\begin{theorem}[{\cite[Theorem~3]{Warren1968}}]\label{thm:warren}
Let $N,m,\Delta$ be positive integers, and let $\calf=(f_1, \ldots, f_m)$, where each $f_i$ is a polynomial in $\RR[x_1,\ldots, x_N]$ of degree at most $\Delta$. Then the number of distinct proper sign patterns of the form $\eps_{\calf}(x)$ for $x\in\RR^N$ is at most
$2(2\Delta)^N\sum_{\ell=0}^N 2^\ell \binom{m}{\ell}$.
\end{theorem}

This theorem has many combinatorial applications (see, e.g., \cite{Alon1995} for some early applications).  We will prove an analogue of Warren's theorem for functions that are sums of radicals of nonnegative polynomials. 

\begin{theorem}\label{thm:sign-patterns-fractional}
Let $N,m,\Delta, r,s$ be positive integers with $r \geq 2$, and let $\calf=(f_1, \ldots, f_m)$, where each $f_i$ is of the form
$f_i=\sum_{j=1}^{r_i} a_{i,j} g_{i,j}^{1/s}$
with $r_i \leq r$ a positive integer, each $a_{i,j}$ a real number, and each $g_{i,j}$ a polynomial in $\RR[x_1, \ldots, x_N]$ of degree at most $\Delta$ such that $g_{i,j}(x) \geq 0$ for all $x \in \RR^N$. Then the number of distinct proper sign patterns of the form $\eps_{\calf}(x)$ for $x\in\RR^N$ is at most
$2(2s^{r-2}\Delta)^N\sum_{\ell=0}^N 2^\ell \binom{m}{\ell}$.
\end{theorem}

Warren deduced \cref{thm:warren} from a topological statement about the connected components of the complement of a real algebraic variety.  Our proof of \cref{thm:sign-patterns-fractional} will follow the same strategy, and we will use Warren's topological statement as a black box.  For a function $p\colon\RR^N \to \RR$, let $\VV(p)\coloneqq\set{x \in \RR^N: p(x)=0}$ denote its zero set.

\begin{lemma}[{\cite[Theorem~2]{Warren1968}}]\label{lem:warren}
Let $N,m,\Delta$ be positive integers, and let $f_1, \ldots, f_m \in \RR[x_1,\ldots, x_N]$ be polynomials of degree at most $\Delta$. Then the set $\RR^N \setminus \bigcup_{i=1}^m \VV(f_i)$ has at most 
$2(2\Delta)^N\sum_{\ell=0}^N 2^\ell \binom{m}{\ell}$
connected components.
\end{lemma}

We will also require the following basic fact about products of ``Galois conjugates.''

\begin{lemma}\label{lem:galois}
Let $r,s \geq 2$ be integers, let $\omega=e^{2\pi i/s}$, and let $\xi_1,\ldots,\xi_r$ be variables.  Then
\[\prod_{0 \leq t_2,\ldots, t_r \leq s-1} (\xi_1+\omega^{t_2} \xi_2+\cdots+\omega^{t_r}\xi_r)\] is a polynomial in $\xi_1^s, \ldots, \xi_r^s$.
\end{lemma}

\begin{proof}\textcolor{red}{TOPROVE 0}\end{proof}

We can now prove \cref{thm:sign-patterns-fractional}.

\begin{proof}\textcolor{red}{TOPROVE 1}\end{proof}
We remark that the same proof yields a similar bound in the more general case where the fractional power appearing in the definition
of the function $f_i$ is $s_i$, and the integers $s_i$ are not necessarily all equal. We omit the details since this more general statement is not needed here.

The following example shows that the functions $f_i$ considered in \Cref{thm:sign-patterns-fractional} can have many sign changes even when $N=1$ and $\Delta=s=2$ are fixed.

\begin{proposition}
\label{prop:noga}
Let $\ell$ be a positive integer, and let $0<\delta<2/\ell$ be a real number.  For $\Vec a=(a_1, \ldots, a_\ell) \in \set{-1,1}^\ell$, define $f_{\Vec a}\colon\RR\to\RR$ by
\[
f_{\Vec a}(x)\coloneqq\sum_{i=1}^\ell a_i \paren*{\sqrt{(x-i)^2 +\delta^2} - \sqrt{(x-i)^2}}.
\]
Then $\sgn(f_{\Vec a}(j))=a_j$ for all $j\in[\ell]$. 
\end{proposition}

\begin{proof}\textcolor{red}{TOPROVE 2}\end{proof}
In particular, if we let $\ell = 2^m$ for some positive integer $m$ and choose $\Vec a_1,\ldots,\Vec a_m$ appropriately, then the tuple of functions $(f_{\Vec a_1},\ldots,f_{\Vec a_m})$ can take on all $2^m$ proper sign patterns of length $m$. Thus we can achieve $2^m$ proper sign patterns even while fixing $N = 1$ and $\Delta = s = 2$, at the expense of letting $r = 2^{m+1}$ grow exponentially. Somewhat informally, this implies that the bound in \cref{thm:sign-patterns-fractional} must have at least a linear dependence on $r$ if it is to depend polynomially on $m$. This is very far from our bound (which is exponential in $r$), and it would be interesting to close this gap.

A simpler, though still interesting, problem in this direction concerns the case where we fix $N = m = 1$ and $\Delta = s = 2$ and consider a function $f \colon \RR \to \RR$ which is a linear combination of $r$ square roots of everywhere-positive quadratic polynomials. Instead of considering the number of proper sign patterns of $f$, which is obviously bounded above by $2$, we instead consider the number of connected components of $\RR \setminus \VV(f)$. Using a similar ``multiplication by conjugates'' trick, this can be bounded above by $2^{r-1} + 1$. On the other hand, we can achieve $2r-1$ connected components by letting
\[f(x) = 1 + \sum_{i=1}^{r-1} (-1)^i a_i^{1/10} \paren*{\sqrt{x^2 + a_i^2} - a_i}\]
for a sequence $0 < a_1 < \cdots < a_{r-1}$ that grows extremely quickly. Again we have a linear lower bound and an exponential upper bound; it would be interesting to narrow the gap.

\section{The upper bound in \texorpdfstring{\cref{thm:main}}{Theorem \ref{thm:main}}}\label{sec:upper}

Using the tools from the previous section, we can quickly establish the upper bound in \Cref{thm:main}.

\begin{lemma}\label{lem:main-upper}
If $d \geq 1$ and $k \geq 1$, then $\perm_{d,k}(n)=O_{d,k}(n^{2dk})$.
\end{lemma}

\begin{proof}\textcolor{red}{TOPROVE 3}\end{proof}

\Cref{lem:main-upper} settles the upper bound for all $k$ when $d \geq 2$ and for odd $k$ when $d = 1$.  The following lemma handles the remaining cases.
\begin{lemma}
If $k \geq 2$ is even, then $\perm_{1,k}(n)=O_{k}(n^{2k-2})$.
\end{lemma}
\begin{proof}\textcolor{red}{TOPROVE 4}\end{proof}

\section{The lower bound in \texorpdfstring{\cref{thm:main}}{Theorem \ref{thm:main}}: general techniques}\label{sec:lower}
\subsection{Overview of the proof strategy} \label{subsec:lowerintro}
Before diving into the technical details of the proof of~\cref{thm:main}, we will give a qualitative high-level description of our strategy.

It is reasonable to expect that a generic $n$-element set $C\subseteq \RR^d$ will have $\psi_{k}(C)$ within a constant factor of $\perm_{d,k}(n)$. The difficulty in proving the lower bound in \cref{thm:main} thus lies in finding a set $C$ that both ``behaves generically'' and has enough structure to be analyzed.  To achieve this, our construction of $C$ will contain features on many different scales.  When $d \geq 2$, this will let us use asymptotic estimates such as $\sqrt{1+R^2}\approx R$ and $\sqrt{R^2 + aR} - R \approx a/2$ for large $R$, which simplify the square roots inherent in Euclidean distances.

For a fixed $d$, our proof proceeds by inductively turning a construction with $k$ vantage points into a construction with $k+2$ vantage points.  When $d=1$, our base cases are $k=1$ and $k=2$; the result for the former is already known, and the result for the latter is an immediate consequence.  When $d \geq 2$, our base cases $k = 0$ and $k = 1$ are trivial and already known, respectively. The inductive step takes a set $C'$ from our inductive hypothesis and adds two carefully-chosen sets $C_1$ and $C_2$ ``flanking'' $C'$ such that $C'$ is located roughly halfway between $C_1$ and $C_2$. We make the scale of $C_1$ and $C_2$ much larger than the scale of $C'$, and we make the separation distances between $C'$, $C_1$, and $C_2$ even larger.

We place two vantage points $u_1$ and $u_2$ near $C_1$ and $C_2$, respectively, and place $k$ additional vantage points $v_1, \ldots, v_k$ close to $C'$.  The quantity $\Abs{c - u_1} + \Abs{c - u_2}$ will be essentially constant (in fact, exactly constant when $d=1$) on $C'$ due to the large separation distances, so the relative order of the points of $C'$ will be entirely determined by the $k$ vantage points near $C'$.  At the same time, for each $c\in C_1\cup C_2$, since the scale of $C_1$, $C_2$, $u_1$, and $u_2$ exceeds that of $C'$ and $v_1, \ldots, v_k$, the variation in the quantity $\Abs{c - u_1} + \Abs{c - u_2}$ (as $u_1$ and $u_2$ vary) will vastly exceed the variation in the quantity $\sum_i \Abs{c-v_i}$ (as the $v_i$'s vary), so the relative order of the points in $C_1 \cup C_2$ will be entirely determined by the vantage points $u_1$ and $u_2$. As a result, the relative orderings of the sets $C'$ and $C_1 \cup C_2$ can be determined independently, and the total number of orderings of $C' \cup C_1 \cup C_2$ will be at least the product of the numbers of orderings of the two component parts.

To formalize the notion of large separations, we define ``effective distance functions'' that control the relative order of the points in $C_1 \cup C_2$ in the limit where the scales of $C_1$ and $C_2$, as well as their separation distance, go to infinity. This perspective allows us to reduce the original problem to a two-``vantage-point'' subproblem involving the effective distance functions. For any given individual solution to the subproblem, we will be able to complete the inductive step by choosing sufficiently large scales and separation distances, but this limiting process can be forgotten entirely when solving the subproblem itself.  After this stage, the proofs for $d=1$ and for $d \geq 2$ diverge.

For $d=1$, our effective distance functions are piecewise linear and can be analyzed manually. The argument for $d \geq 2$ is more complicated.  After the initial reduction via large separations, our next reduction scales $C_2$ to be much larger than $C_1$; this further simplifies the effective distance functions.  We will take $C_1$ and $C_2$ to be two (scaled) copies of a $(d-1)$-dimensional configuration $C^*$ such that $\psi_1(C^*) = \Omega_d(n^{2(d-1)})$; they will be located on two hyperplanes $H_1$ and $H_2$ orthogonal to the separation axis. The projection of $u_i$ onto $H_i$ will determine the relative ordering of the points in $C_i$ for each $i$, and then the orthogonal distances from $u_1, u_2$ to $H_1, H_2$ will determine the interleaving of points from $C_1$ and points from $C_2$ in the ordering of $C_1 \cup C_2$.  Thanks to our prior simplification work, this interleaving process can be analyzed almost exactly, and we will show that generically there are $\Theta(n^4)$ possible interleavings. Combined with $\Theta_d(n^{2(d-1)})$ orderings of each of $C_1$ and $C_2$, this will give $\Theta_d(n^{4d})$ total orderings of $C_1 \cup C_2$. A schematic of the entire construction is shown in \cref{fig:constructionschematic}. 

\begin{figure}[tbp]
\begin{center}{\includegraphics[height=5.44cm]{VantagePIC1}}
\end{center}
\caption{A schematic illustration of our general recursive approach for $d \geq 2$. Dots representing vantage points are large and red.}
\label{fig:constructionschematic}
\end{figure}


\subsection{Pairs of flanking points} 
We begin by formalizing the notion of an infinite separation limit through the definition of effective distance functions. For the remainder of the proof, write $e_1 \coloneqq (1,0, 0,\ldots)$ for the first standard basis vector in $\RR^d$.

\begin{definition}
Let $k$ be a nonnegative integer, let $\hat U = (\hat u_1, \hat u_2) \in \RR^d \times \RR^d$ be an ordered pair of points, and let $\hat c\in\RR^d$ be a point.  Define
\[
\hat{D}^1_{k,\hat U}(\hat c) \coloneqq \Abs{\hat c - \hat u_1} + ((k+1)\hat c + \hat u_2) \cdot  e_1 \quad\text{and}\quad
\hat{D}^2_{k,\hat U}(\hat c) \coloneqq \Abs{\hat c - \hat u_2} + ((k+1)\hat c + \hat u_1) \cdot e_1.
\]
\end{definition}

To motivate these definitions,
we consider the multiset of vantage points \[V = \set{\hat u_1 + Re_1, -\hat u_2 - Re_1, \underbrace{0,\ldots,0}_{k\text{ points}}}\] and the candidate points $c_1 = \hat c_1 + Re_1$ and $c_2 = -\hat c_2 - Re_1$, where $R$ is a large positive real number.  With $\hat u_1, \hat u_2, \hat c_1, \hat c_2$ held constant, we have
\[\hat{D}_{k,\hat U}^1(\hat c_1) = \lim\limits_{R \to \infty} (D_{V}(c_1) - (k+1)R) \quad\text{and}\quad
\hat{D}_{k,\hat U}^2(\hat c_2) = \lim\limits_{R \to \infty} (D_{V}(c_2) - (k+1)R).\]
So, if we care about the relative sizes of quantities of the form $D_V(c_1), D_V(c_2)$ in the regime where $R$ is large, then it suffices to understand the relative sizes of the quantities $\hat{D}_{k,\hat U}^1(\hat c_1), \hat{D}_{k,\hat U}^2(\hat c_2)$.  We now define an analogue of $\psi_k(C)$ for these effective distances.  In what follows, we write $A \sqcup B$ to denote the disjoint union of the sets $A,B$ (even if $A,B$ have nonempty intersection as sets).

\begin{definition} \label{def:hatpsi}
Let $k$ be a nonnegative integer, and let $\hat C_1$ and $\hat C_2$ be sets of points in $\RR^d$. Given $\hat U \in \RR^d \times \RR^d$, let $\hat D_{k,\hat U} \colon \hat C_1 \sqcup  \hat C_2 \to \RR$ be the function that equals $\hat D^1_{k,\hat U}$ on $\hat C_1$ and equals  $\hat D^2_{k,\hat U}$ on $\hat C_2$. Let $\hat \Sigma_{\hat U}^{\hat C_1,\hat C_2}$ be the function $[\abs{\hat C_1} + \abs{\hat C_2}] \to \hat C_1 \sqcup\hat C_2$ such that $\hat D_{k,\hat V} \circ \hat \Sigma_{\hat U}^{\hat C_1,\hat C_2}$ is increasing (if such a function exists), and let $\hat \Psi_k(\hat C_1, \hat C_2)$ be the set of all such $\hat \Sigma_{\hat U}^{\hat C_1,\hat C_2}$. Finally, let $\hat \psi_k(\hat C_1,\hat C_2) = \abs{\hat \Psi_k(\hat C_1,\hat C_2)}$.
\end{definition}

The following three lemmas capture the main steps of our inductive argument.  The first is for all $d$, the second is for $d=1$, and the third is for $d \geq 2$.
 
\begin{lemma} \label{lem:lowerinduction}
Let $d \geq 1$, $k\geq 0$, and $m \geq 0$, with the additional constraint that $m=0$ if $k=0$. Then, for sets $\hat C_1, \hat C_2 \subseteq \RR^d$, we have
\[\perm_{d,k+2}(m + \abs{\hat C_1} + \abs{\hat C_2}) \geq \hat \psi_k(\hat C_1, \hat C_2) \perm_{d,k}(m).\]
\end{lemma}
\begin{lemma} \label{lem:biexistd1}
For $k,m \geq 1$, there exist sets $\hat C_1, \hat C_2 \subseteq \RR$ of size $2m$ such that ${\hat \psi_k(\hat C_1, \hat C_2) \geq m^4}$.
\end{lemma}
\begin{lemma} \label{lem:biexist}
For $d \geq 2$, $k \geq 0$, and $m \geq 1$, there exist sets $\hat C_1, \hat C_2 \subseteq \RR^d$ of size $m$ such that $\hat \psi_k(\hat C_1, \hat C_2) = \Omega_{d}(m^{4d})$.
\end{lemma}

Before proving these lemmas, let us see how they imply the desired estimates on $\perm_{d,k}(n)$ for fixed $d,k$.  We proceed by induction on $k$. For the inductive step, we note that combining \cref{lem:biexistd1,lem:biexist} yields that for all $d \geq 1$ and $k \geq 1$, there exist sets $\hat C_1, \hat C_2$ of size at most $n/3$ with $\hat \psi_k(\hat C_1, \hat C_2) = \Omega_{d}(n^{4d})$, so after applying \cref{lem:lowerinduction} we find that
\[\perm_{d,k+2}(n) \geq \psi_{d,k+2}(\floor{n/3} + \abs{\hat C_1} + \abs{\hat C_2}) \geq \Omega_{d}(n^{4d}) \perm_{d,k}(\floor{n/3}).\]
Therefore it suffices to show the base cases $k = 1,2$.

The base case $k = 1$, i.e., that $\perm_{d,1}(n) = \Omega_d(n^{2d})$ for all $d \geq 1$, follows from the result of \cite{Good1977, Zaslavsky2002} discussed in \Cref{sec:intro}. In the $k = 2$ case, we wish to prove that $\perm_{d,2}(n)$ is $\Omega(n^2)$ if $d = 1$ and $\Omega_d(n^{4d})$ otherwise. In the $d = 1$ case, this follows from the fact that $\perm_{1,2}(n) \geq \perm_{1,1}(n)$, which can be easily proven by putting the two vantage points at the same place. In the $d \geq 2$ case, we use \cref{lem:biexist} to find sets $\hat C_1$ and $\hat C_2$ of size $\floor{n/2}$ such that $\hat \psi_k(\hat C_1,\hat C_2) = \Omega_d(n^{4d})$. Then, by \cref{lem:lowerinduction} we have
\[\perm_{d,2}(n) \geq \perm_{d,2}(2\floor{n/2}) \geq \hat \psi_k(\hat C_1,\hat C_2) = \Omega_d(n^{4d}),\]
where we use the trivial fact that $\perm_{d,0}(0) = 1$.

We now dispose of \cref{lem:lowerinduction}; the proofs of \cref{lem:biexistd1,lem:biexist} will occupy the following two sections.


\begin{proof}\textcolor{red}{TOPROVE 5}\end{proof}
\section{The lower bound in \texorpdfstring{\cref{thm:main}}{Theorem \ref{thm:main}}: \texorpdfstring{$d = 1$}{d = 1}} \label{sec:lowerd1}
To prove \cref{thm:main} when $d=1$, it remains only to prove \cref{lem:biexistd1}.
\begin{proof}\textcolor{red}{TOPROVE 6}\end{proof}

\begin{figure}[tbp]
\begin{center}{\includegraphics[height=5.574cm]{VantagePIC4}}
\end{center}
\caption{A schematic illustration of the construction in \Cref{lem:biexistd1}.} 
\label{fig:section5}
\end{figure}

\section{The lower bound in \texorpdfstring{\cref{thm:main}}{Theorem \ref{thm:main}}: \texorpdfstring{$d \geq 2$}{d ≥ 2}} \label{sec:lowerdg2}
We now pick up where we left off at the end of \cref{sec:lower}.
\subsection{A further reduction}
In this section, we reduce \cref{lem:biexist} to a problem involving simpler distance functions.
\begin{definition}\label{def:check}
Let $d \geq 2$ be a positive integer. Given a quadruple \[\check V = (\check v_1, \check v_2, \check x, \check y) \in \RR^{d-1} \times \RR^{d-1} \times \RR \times \RR_{>0}\] and a point $\check c \in \RR^{d-1}$, let
\[\check D^1_{\check V}(\check c) = \sqrt{\check x^2 + \Abs{\check c - \check v_1}^2} - \check x \quad\text{and}\quad \check D^2_{\check V}(\check c) = \check y\Abs{\check c - \check v_2}^2.\]
For two sets $\check C_1,\check C_2 \subseteq \RR^{d-1}$, define $\check D_{\check V} \colon \check C_1 \sqcup \check C_2 \to \RR$ to be the function that equals $\check D_{\check V}^1$ on $\check C_1$ and equals $\check D_{\check 
V}^2$ on $\check C_2$.  Let $\check \Sigma_{\check V}^{\check C_1,\check C_2} \colon [\abs{\check C_1} + \abs{\check C_2}] \to \check C_1 \sqcup \check C_2$ be the function such that $\check D_{\check V} \circ \check \Sigma_{\check V}^{\check C_1,\check C_2}$ is increasing (if such a function exists). Let $\check \Psi(\check C_1,\check C_2)$ be the set of all possible $\check \Sigma_{\check V}^{\check C_1,\check C_2}$ as $\check V$ varies, and let $\check \psi(\check C_1,\check C_2) = \abs{\check \Psi(\check C_1,\check C_2)}$.
\end{definition}
As in \cref{def:hatpsi}, the first two functions in \cref{def:check} can be viewed as effective distance functions for a suitable infinite separation limit.  The setup of this infinite separation limit is as described in the statement of the following lemma.

\begin{lemma}\label{lem:thing}
Let $k \geq 0$ and $d \geq 2$ be integers, and let $\check C_1, \check C_2 \subseteq \RR^{d-1}$ be finite point sets.  For sufficiently large $R > 0$, we have $\hat \psi_k(\set{0} \times \check C_1, \set{0} \times R\check C_2) \geq \check \psi(\check C_1, \check C_2)$.
\end{lemma}
\begin{proof}\textcolor{red}{TOPROVE 7}\end{proof}
This lemma shows that in order to prove \cref{lem:biexist}, it remains only to find sets $\check C_1,\check C_2 \subseteq \RR^{d-1}$ each of size $m$ such that $\check\psi(\check C_1,\check C_2) = \Omega_{d}(m^{4d})$.

\subsection{The final construction}
Since the remainder of the argument is concerned with only the ``check'' variables and functions, we will dispense with diacritics except for $\check \psi$, $\check \Psi$, $\check D_V^i$, and $\check \Sigma_V^{C_1,C_2}$.


The goal of this section is to prove the following lemma.
\begin{lemma} \label{lem:psicc}
For $d \geq 2$ and $C \subseteq \RR^{d-1}$ finite, we have $\check \psi(C, C) \geq \binom{\abs{C}}{2}^2 \binom{\psi_1(C)}{2}$.
\end{lemma}
\cref{lem:biexist} can now be deduced by 
choosing $C$ to be some $m$-element subset of $\RR^{d-1}$ with $\psi_1(C) = \Theta_d(m^{2(d-1)})$; for the existence of such a set $C$, see the discussion in \Cref{sec:intro}.  For the rest of this section, fix a choice of $C \subseteq \RR^{d-1}$ ($d\geq 2$).


We start with some preliminary results that will help us understand $\check D^1_V$.  For $a > 0$, define the function $\vartheta_a \colon \RR \to \RR$ as $\vartheta_a(x) = \sqrt{x^2+a^2}-x$. For $a, b > 0$, define $\vartheta_{a,b}\colon \RR \cup \set{-\infty,+\infty} \to \RR$ by
\[\vartheta_{a,b}(x) = \begin{cases}
\vartheta_a(x)/\vartheta_b(x) & x \in \RR \\
1 & x = -\infty \\
a^2/b^2 & x = +\infty.
\end{cases}\]
We record several facts about $\vartheta_{a,b}$ that will be useful in the sequel.
\begin{lemma} \label{lem:theta}
Let $a > b > 0$.  Then $\vartheta_{a,b}$ is a continuous and strictly increasing function. Also, if $p \geq q > 0$ and $p/q \leq a^2/b^2$, then the unique $x \in \RR \cup \set{\pm\infty}$ satisfying $\vartheta_a(x)/\vartheta_b(x)= p/q$ is determined up to a sign by
\[x^2 = \frac{(a^2q^2-b^2p^2)^2}{4pq(p-q)(a^2q-b^2p)}.\]
(In particular, when $x = \pm \infty$, the numerator of the above fraction is nonzero, and the denominator is zero.)
\end{lemma}

\begin{proof}\textcolor{red}{TOPROVE 8}\end{proof}

We now turn to the expressions appearing in $\check D^2_V$.  For $v \in \RR^{d-1}$, let \[\Delta(v) = \set{\Abs{v - c_1}^2/\Abs{v - c_2}^2 : c_1, c_2 \in C, \Abs{v - c_1} > \Abs{v - c_2}}.\] 
Say a pair of points $(v_1, v_2) \in (\RR^{d-1})^2$ is \dfn{good} if the following conditions hold:
\begin{itemize}
    \item $v_1$ and $v_2$ are not elements of $C$.
    \item $\Delta(v_1)$ and $\Delta(v_2)$ are disjoint and each have size $\binom{\abs{C}}{2}$.
    \item If $x \in \RR$ and $c_1, c_2, c_3, c_4 \in C$ are such that $(c_1,c_2) \neq (c_3,c_4)$, $\Abs{v_1 - c_1} > \Abs{v_1 - c_2}$, and $\Abs{v_1 - c_3} > \Abs{v_1 - c_4}$, then $\vartheta_{\Abs{v_1 - c_1},\Abs{v_1 - c_2}}(x)$ and $\vartheta_{\Abs{v_1 - c_3},\Abs{v_1 - c_4}}(x)$ are not both in $\Delta(v_2)$. 
\end{itemize}
The following lemma says that generic pairs are good; actually verifying all of the conditions for goodness is unfortunately somewhat tedious and notation-heavy.
\begin{lemma} \label{lem:denseopen}
The set of good pairs is dense and open in $(\RR^{d-1})^2$.
\end{lemma}
\begin{proof}\textcolor{red}{TOPROVE 9}\end{proof}

For the next step, we fix $v_1, v_2 \in \RR^{d-1}$ and consider $\check \Sigma^{C,C}_{(v_1,v_2,x,y)}$ as $x$ and $y$ vary. Let us write $\Gamma(v_1,v_2)= \abs{\set{(a,b)\in \Delta(v_1) \times \Delta(v_2) : a > b}}$.
\begin{lemma} \label{lem:gdd}
If $(v_1, v_2)$ is good, then as $V$ varies over $\set{(v_1,v_2)} \times \RR \times \RR_{>0}$, there are at least $\Gamma(v_1, v_2)$ possibilities for the function $\check \Sigma^{C,C}_V$.
\end{lemma}
\begin{proof}\textcolor{red}{TOPROVE 10}\end{proof}


\begin{proof}\textcolor{red}{TOPROVE 11}\end{proof}

\section{Unlimited vantage points}\label{sec:unlimited}

In this section, we study the set $\Psi(C)=\bigcup_{k\geq 1}\Psi_k(C)$ of orderings of a set $C=\set{c_1,\ldots,c_n}\subseteq\RR^d$ of candidate points that can be obtained using an arbitrarily large (finite) multiset of vantage points. That is, $\Psi(C)$ is the set of all tuples $\Sigma_V^C$ that can be obtained by choosing a finite multiset $V$ of vantage points in $\RR^d$ that distinguishes the points of $C$. 

We first observe that the triangle inequality places a constraint on the tuples in $\Psi(C)$. Let us say a tuple $(x_1,\ldots,x_n)$ of points in $\RR^d$ is \dfn{protrusive} if for every $i\in[n-1]$, the point $x_{i+1}$ is not in the convex hull of $x_1,\ldots,x_i$ (so the convex hull of $x_1,\ldots,x_{i+1}$ ``protrudes'' out of the convex hull of $x_1,\ldots,x_i$). 

\begin{proposition}\label{prop:convexity}
Let $C=\set{c_1,\ldots,c_n}\subseteq\RR^d$. Every tuple in $\Psi(C)$ is protrusive.  
\end{proposition}

\begin{proof}\textcolor{red}{TOPROVE 12}\end{proof}

It is natural to ask if $\Psi(C)$ is exactly the set of protrusive orderings of $C$. The next theorem states that this is the case when $d=1$, but we will see later that it can fail to hold when $d=2$. 

\begin{theorem}\label{thm:protrusive_d_1}
If $C=\set{c_1,\ldots,c_n}\subseteq \RR$, then $\Psi(C)$ is the set of protrusive orderings of $C$. The number of such orderings is $2^{n-1}$. 
\end{theorem}

\begin{proof}\textcolor{red}{TOPROVE 13}\end{proof}

We now turn our attention to the Euclidean plane. We begin by showing that the analogue of \cref{thm:protrusive_d_1} is false when $d=2$.  In particular, we construct an explicit set $C \subseteq \RR^2$ of size $6$ such that $\Psi(C)$ does not contain all of the protrusive orderings of $C$.  The set $C$ will consist of the vertices of an equilateral triangle together with three points placed near the midpoints of its edges but just outside of the triangle; see \cref{fig:6points}.  In particular, this $C$ will be in convex position, so all orderings will be protrusive, but we will show that $\Psi(C)$ does not contain all orderings of $C$.

\begin{figure}[tbp]
\begin{center}{\includegraphics[height=3cm]{VantagePIC2}}
\end{center}
\caption{The set $C$ from \cref{prop:6-points}. The points $c_1,c_2,c_3$ are in blue, while $c_1',c_2',c_3'$ are in green. Note that the green points lie slightly outside of the triangle whose vertices are the blue points. }
\label{fig:6points}
\end{figure}

\begin{proposition}\label{prop:6-points}
Let $C \subseteq \RR^2$ consist of the six points
\begin{align*}
c_1&\coloneqq2e(\pi/2), &c_2&\coloneqq2e(7\pi/6),&c_3&\coloneqq2e(11\pi/6),\\
c'_1&\coloneqq-1.1e(\pi/2), &c'_2&\coloneqq-1.1e(7\pi/6), &c'_3&\coloneqq-1.1e(11\pi/6),
\end{align*}
where we have written $e(\theta)\coloneqq(\cos \theta, \sin \theta)$.
Then for any $v \in \RR$, we have
\[\Abs{v-c_1}+\Abs{v-c_2}+\Abs{v-c_3} \geq \Abs{v-c'_1}+\Abs{v-c'_2}+\Abs{v-c'_3}.\]
In particular, the ordering $(c_1, c_2, c_3, c'_1, c'_2, c'_3)$ is protrusive but is not in $\Psi(C)$.
\end{proposition}

\begin{proof}\textcolor{red}{TOPROVE 14}\end{proof}

It is still worth studying the sets $C$ with the property that $\Psi(C)$ is precisely the set of protrusive orderings of $C$. The following theorems provide two families of sets $C$ with this property. 

\begin{theorem}\label{thm:4-points}
If $C\subseteq\RR^d$ is a set of size $n\leq 4$, then $\Psi(C)$ is the set of protrusive orderings of $C$.
\end{theorem} 

\begin{proof}\textcolor{red}{TOPROVE 15}\end{proof}

In light of \cref{prop:6-points} and \cref{thm:4-points}, we are led to the following natural question. 

\begin{question}
Does there exist a set $C$ of $5$ points such that some protrusive ordering of $C$ is not in $\Psi(C)$?
\end{question}

We now formulate a sufficient condition on $C$ for $\Psi(C)$ to consist of all $\abs{C}!$ orderings of $C$.  If $C=\set{c_1, \ldots, c_n} \subseteq \RR^d$, then its \dfn{{distance matrix}} is defined to be the $n \times n$ matrix ${\mathbf M}_C$ whose $ij$-entry is $\Abs{c_i-c_j}$.  It is well known \cite{Schoenberg1937} (see also \cite{Ball1992}) that ${\mathbf M}_C$ is invertible for every choice of $C$.  Let $\mathbf{1}$ denote the all-$1$'s vector of length $n$.

\begin{lemma}\label{lem:dist-matrix}
Let $C=\set{c_1, \ldots, c_n} \subseteq \RR^d$.  If the vector $\nu\coloneqq{\mathbf M}_C^{-1}{\mathbf 1}$ has all entries strictly positive, then $\Psi(C)$ is the set of all $n!$ orderings of $C$, and every ordering is witnessed by a multiset of vantage points such that every vantage point is in $C$.
\end{lemma}

\begin{proof}\textcolor{red}{TOPROVE 16}\end{proof}

It is straightforward to check that the hypothesis of \Cref{lem:dist-matrix} holds for many particular sets $C$.  One family of examples comes from taking $C$ to be the set of vertices of a vertex-transitive polytope; in this case, the vector $\nu={\mathbf M}_C^{-1}{\mathbf 1}$ is in fact constant.

\begin{theorem}\label{thm:polytope}
If $C$ is the set of vertices of a vertex-transitive polytope $P \subseteq \RR^d$, then $\Psi(C)$ is the set of all $\abs{C}!$ orderings of $C$.
\end{theorem}
\begin{proof}\textcolor{red}{TOPROVE 17}\end{proof}

\section*{Acknowledgements}
Noga Alon was supported in part by NSF grant DMS-2154082. Colin Defant was supported by the National Science Foundation under Award No.\ 2201907 and by a Benjamin Peirce Fellowship at Harvard University. He thanks Rimma Hamaleinen for initially directing his attention to the article \cite{Carbonero2021}.  Noah Kravitz was supported in part by an NSF Graduate Research Fellowship (grant DGE-2039656).

\printbibliography
\end{document}
