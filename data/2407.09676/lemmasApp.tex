% !TEX root = mfe.tex
\section{Appendix: Useful lemmas}\label{sec:lemmasApp}
\begin{lemma}[factors of $\pi$]\label{lem:factors}
	For any nonempty circular permutation $\pi$ 
	and any prefix $y$ of $\pi$ that is not its fundamental component $x$, 
	such that $|y|>|x|$, 
	if $\pi = y^n$ then $|x|$ divides $|y|$.    
\end{lemma}
\begin{proof}
	Suppose for contradiction that such $y$ and $n$ exist, so $x^{v(\pi)} = y^n$ which means that $v(\pi)|x| = n|y|$. As $|x|$ does not divide $|y|$, from division algorithm we can write $|y| = a |x| + b$ for some remainder $0<b<|x|$.
	
	Suppose the infinite string $W$ that is just an infinite repeat of $\pi$ and hence an infinite repeat of $x$ and $y$. Let's suppose a simple function $f(i)$ that return the character at index $i$ of $W$, as $\pi = x^{v(\pi)} = y^n$ that means that $f(i) = f(i+\alpha|x|) = f(i+\beta|y|)$ for any $\alpha$ and $\beta$. Then $f(i) = f(i+ a\beta |x| + \beta b)$. For $\beta = 1$ that implies that $f(i) = f(i+b)$, which implies the  existence of a smaller repeating prefix of $\pi$ since $b<|x|$ which contradicts the fact that $x$ is the fundamental component of $\pi$.   
\end{proof}


The following lemma restricts us to deal with only specific and constant number of different folding rotational symmetries, and hence constant different symmetry corrections in total. 
\begin{lemma}\label{lem:div}
	If $S$ is $R$-fold rotational symmetric secondary structure, with a specific circular permutation $\pi$, then $R$ must be a divisor of $v(\pi)$.
\end{lemma}

\begin{proof}
	As $R = |H|$ a subgroup of $G^\pi$, and from Lagrange theorem for finite groups~\cite{nicholson2012introduction}, which says that for any finite group $G$, the order of every subgroup of $G$ divides the order of $G$. So, R divides $|G^\pi| = v(\pi)$. Also, from the fundamental theorem of cyclic groups, as we know that $H$ is a subgroup of the cyclic group $G^\pi$, then $H = \langle \rho^d \rangle$ for some $\rho^d \in G^\pi$, $H$ is generated by $\rho^d$. Hence, $H$ is the \emph{unique} subgroup of order $|H|= v(\pi)/d$. 
\end{proof}



\begin{lemma} \label{lem:nobase}
	For any connected unpseudoknotted secondary structure $S$, if there exists at least one base pair $(i,j)$ such that $\llbracket i,j \rrbracket > \frac{N}{R}$, then $S$ can not be  $R$-fold rotationally symmetric. 
\end{lemma}
\begin{proof}
	For all $R>2$, suppose for the sake of contradiction that $S$ is $R$-fold rotational symmetric secondary structure and such base pair $(i,j)$ exists, from symmetry this implies the existence of another $(R-1)$ different base pairs each with same segment length as $\llbracket i,j \rrbracket$, so the total length of the system must be higher than number of bases $N$, so at least two base pairs must intersect forming a pseudoknot, which contradicts the fact that $S$ is pseudoknot-free. 
	
	For $R=2$, suppose such base pair $(i,j)$ exists, then this can only happen if $(i,j)$ is a central base pair, a diameter in $\PolySpi$, such that $\llbracket i,j \rrbracket = \frac{N}{2}+1 >  \frac{N}{2}$. This is impossible as the symmetry implies that $i$ and $j$ are of the same type, that there exists a base which is complement to itself.  
\end{proof}



