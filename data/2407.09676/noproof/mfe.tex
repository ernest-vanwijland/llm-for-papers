\documentclass[11pt,letterpaper]{article}  % leqno
\usepackage[margin=1in]{geometry}
\usepackage{hyperref}
\usepackage{stmaryrd} % for double-square brackets [[ blah ]] in math
\usepackage[normalem]{ulem}
\usepackage{xspace,graphicx,amssymb,amsmath,amsthm,thm-restate} 
\usepackage{algorithm}  % provides the algorithm env (caption, horz lines, etc.)
\usepackage[]{algpseudocode} % this provides the algorithmic environment with commands like \EndIf , \ State ,etc.  Note that option [noend] gets rid of END lines (END IF etc.) 
\usepackage{caption} % allows for unnumbered Algorithm captions (so I can have one  algorithm spread accross multiple pages
\usepackage[capitalise]{cleveref} % must be loaded after hyperref, amsmath
\Crefname{figure}{Figure}{Figures}
\usepackage{url} 

\newtheorem{theorem}{Theorem}
\newtheorem{lemma}[theorem]{Lemma}
\theoremstyle{definition}  % makes the \newtheorem's below here use rm font (classic definition style)
\newtheorem{Definition}[theorem]{Definition}
\newtheorem{note}[theorem]{Remark} % kinda crazy
\newtheorem{remark}[theorem]{Remark}


% "Symmetry-naive MFE" and  "snMFE"
\newcommand{\snMFE}{snMFE\xspace}
\newcommand{\symnMFE}{symmetry-naive MFE\xspace}
\newcommand{\SymnMFE}{Symmetry-naive MFE\xspace}
\newcommand{\PolySpi}{\ensuremath{\mathrm{Poly}(S,\pi)}\xspace}
\newcommand{\PolySpiPrime}{\ensuremath{\mathrm{Poly}(S,\pi')}\xspace}


\newcommand{\DGnosym}{\ensuremath{\overline{\Delta G}}}
\newcommand{\algorithmicbreak}{\textbf{break}}
\newcommand{\Break}{\State \algorithmicbreak}

% Editing commands, to be commented out for submission
\setlength{\marginparwidth}{21mm}
\setlength{\marginparsep}{0.3mm}
\usepackage[textsize=tiny,
% color=lightgray
textcolor=blue,
linecolor=blue,
bordercolor=blue,
backgroundcolor=white,
]{todonotes}

% \usepackage[textsize=tiny]{todonotes} 
%\newcommand{\todoi}[1]{\todo[inline]{#1}}
% \newcommand{\dwm}[1]{\marginpar{\tiny #1}}
%\newcommand{\dwm}[1]{\todo{DW: #1}}
% \renewcommand{\dwm}[1]{} % kills \dwm
%\newcommand{\asm}[1]{\marginpar{\tiny #1}}
\newcommand{\dw}[1]{{\color{blue}#1}}
%\newcommand{\asm}[1]{\todo{AS: #1} }
\newcommand{\AS}[1]{{\color{red}#1}}
%\newcommand{\ahmedS}[2]{{\AS{\sout{#1}}} {\AS{#2}}}
%\newcommand{\dwi}[1]{\todo[inline]{\tiny DW: #1}}
\newcommand{\dws}[2]{{\dw{\sout{#1}}}{\dw{#2}}}
%\renewcommand{\dws}[2]{{}{\dw{#2}}}


\title{\Large An efficient algorithm to compute the minimum free energy\\ of interacting nucleic acid strands
}



\author{Ahmed Shalaby\thanks{Hamilton Institute and Department of Computer Science, Maynooth University, Ireland. Research supported by Science Foundation Ireland (SFI) under grant numbers 20/FFP-P/8843 and 18/ERCS/5746 and the European Union's European Research Council (ERC, Active-DNA, no 772766); European Innovation Council  (EIC, DISCO, No 101115422). Views and opinions expressed are however those of the author(s) only and do not necessarily reflect those of the European Union, European Research Council,  European Innovation Council or Science Foundation Ireland. Neither the European Union nor the granting authority can be held responsible for them.} 
\and Damien Woods\footnotemark[1]}

\date{}


\begin{document}
	\maketitle
	\pagenumbering{arabic}\pagestyle{plain}
	
	% !TEX root = mfe.tex
\begin{abstract} \baselineskip=11pt 
%	We give the first polynomial time algorithm for predicting optimal multistranded nucleic acid structures, answering an open problem from Dirks et al [SICOMP Review; 2007].
	The information-encoding molecules RNA and DNA form a combinatorially large set of secondary structures through nucleic acid base pairing.  Thermodynamic prediction algorithms predict favoured, or minimum free energy (MFE), secondary structures, and can even assign an equilibrium probability to any particular structure via the partition function---a Boltzmann-weighted sum of the free energies of the exponentially large set of secondary structures.  Prediction is NP-hard in the presence pseudoknots---base pairings that violate a restricted planarity condition.  However, unpseudoknotted structures are amenable to dynamic programming-style problem decomposition:  for a single DNA/RNA strand there are polynomial time algorithms for MFE and partition function.  For multiple strands, the problem is significantly more complicated due to extra entropic penalties.  Dirks et al [SICOMP Review; 2007] showed that for multiple ($\mathcal{O}(1)$) strands, with $N$ bases, there is a polynomial time in $N$ partition function algorithm, however their technique did not generalise to  MFE which they left open. 
	
	We give the first polynomial time ($\mathcal{O}(N^4)$) algorithm   for unpseudoknotted multiple ($\mathcal{O}(1)$) strand MFE, answering the open problem from Dirks et al.  The challenge in computing MFE lies in considering the rotational symmetry of secondary structures, a global feature not immediately amenable to dynamic programming algorithms that rely on local subproblem decomposition.  Our proof has two main technical contributions:  First, a polynomial upper bound on the number of symmetric secondary structures that need to be considered when computing the rotational symmetry penalty. Second, that bound is leveraged by a backtracking algorithm to find the true MFE in an exponential space of contenders. 
	
	Our MFE algorithm has the same asymptotic run time as Dirks et al's partition function algorithm,  suggesting a reasonably efficient handling of the global problem of rotational symmetry, although ours has higher space complexity.  Finally, our algorithm also seems reasonably tight in terms of number of strands since Codon, Hajiaghayi and Thachuk [DNA27, 2021] have shown  that unpseudoknotted MFE is NP-hard for $\mathcal{O}(N)$ strands. 
\end{abstract}
	\section{Introduction}

Given a graph $G$ and a set $T$ of its vertices called \emph{terminals}, a \emph{cut sparsifier} of $G$ with respect to $T$ is a graph $G'$ with $T\subseteq V(G')$, such that for every partition $(T_1,T_2)$ of $T$ into non-empty subsets, the size of the minimum cut separating $T_1$ from $T_2$ in $G$ and the size of the minimum cut separating $T_1$ from $T_2$ in $G'$ are within some small multiplicative factor $q\ge 1$, which is also called the \emph{quality} of the sparsifier.


Karger \cite{karger1993global,karger1999random} first considered the special case where $T=V(G)=V(G')$ and $G'$ is required to be a subgraph of $G$, and
he used sampling approaches to construct quality-$(1+\varepsilon)$ cut sparsifiers with $O(n\log n/\varepsilon^2)$ edges.
Such sparsifiers are called \emph{edge sparsifiers}, since we are only allowed to sparsify the edges (not the vertices) and the goal is to minimize $|E(G')|$.
In this paper, we allow $G'$ to have a different set of vertices, and such graphs $G'$ are known as \emph{vertex sparsifiers}.
Constructing cut-preserving vertex sparsifiers with small quality and size (measured by the number of vertices in $G'$) has been a central problem in graph compression.

Cut-preserving vertex sparsifiers were first studied in the special case where $G'$ is only allowed to contain terminals (that is, $V(G')=T$). Moitra \cite{moitra2009approximation} first showed that every  $k$-terminal graph has a cut sparsifier with quality $O(\log k/\log\log k)$.
Then Charikar, Leighton, Li and Moitra \cite{charikar2010vertex} showed that such sparsifiers can be computed efficiently.
The best quality lower bound in this case is $\Omega(\sqrt{\log k}/\log\log k)$ \cite{makarychev2010metric,charikar2010vertex}, leaving a small gap to be closed. 
%
Therefore, the natural next question, which has been a central question on cut/flow sparsifiers over the past decade, is:

\vspace{-10pt}
\[\emph{Can better quality sparsifiers be achieved by allowing a small number of Steiner vertices}? 
\]
\vspace{-10pt}


For exact cut sparsifiers ($q=1$), it has been shown that every $k$-terminal graph admits an exact cut sparsifier of size at most $2^{2^k}$ \cite{hagerup1998characterizing,khan2014mimicking}, while the strongest lower bound is $2^{\Omega(k)}$ \cite{khan2014mimicking,karpov2017exponential}, leaving an exponential gap yet to be closed.
Since then the research on cut sparsifiers was diverted into two directions.
The first direction focuses on proving better size upper bounds for special families of graphs.
For example, for planar graphs, Krauthgamer and Rika \cite{krauthgamer2013mimicking,krauthgamer2017refined} showed that $2^{O(k)}$ vertices are sufficient for an exact cut sparsifier, and then Karpov, Philipzcuk and Zych-Pawlewicz \cite{karpov2017exponential} proved a lower bound of $2^{\Omega(k)}$, showing that the single exponential upper bound is almost tight.
Another example is graphs where each terminal has degree exactly $1$. For these graphs, Chuzhoy \cite{chuzhoy2012vertex} showed the construction for $O(1)$-quality sparsifiers of size $O(k^3)$ by contracting expanders, and Kratsch and Wahlstrom \cite{kratsch2012representative} constructed quality-$1$ sparsifiers of size $O(k^3)$ via a matroid-based approach.
%
The second direction focuses on constructing cut sparsifiers with a slightly worse quality of $(1+\eps)$ for small $\eps>0$.
Andoni, Gupta, and Krauthgamer \cite{andoni2014towards} initiated the study of quality-$(1+\eps)$ flow sparsifiers \footnote{Flow sparsifiers are concretely related to cut sparsifiers. We review the previous work and discuss the connection between them in \Cref{sec: related}.} (which are stronger versions of cut sparsifiers), established a framework of constructing flow sparsifiers via contractions with a doubly exponential upper bound. However, the only family that they managed to obtain a better-than-$2^{2^k}$ upper bound for is the family of quasi-bipartite graphs, graphs where every edge is incident to some terminal. They showed an upper bound of $\poly(k/\eps)$, via a sampling-based approach.
%
For quality-$(1+\eps)$ cut sparsifiers on quasi-bipartite graphs, the bound was improved to $\tilde O(k/\eps^2)$ by Jambulapati, Lee, Liu and Sidford \cite{jambulapati2023sparsifying}, also via a sampling-based approach.

Among the above algorithmic results, the most common approach for constructing cut sparsifiers is contraction. Specifically, we compute a partition $\fset$ of the vertices of $G$ into disjoint subsets, and then contract each subset of $\fset$ into a supernode.
The resulting graph is called a \emph{contraction-based cut sparsifier}.
%
Contraction-based cut sparsifiers may only increase the value of every terminal min-cut, which partly explains why contraction is a commonly adopted approach, as to analyze its quality it suffices to show that the min-cuts are not increased by too much.
For general graphs, a lower bound of $2^{2^{\Omega(k)}}$ (almost matching the upper bound of $2^{2^k}$) was proved \cite{karpov2017exponential} for contraction-based cut sparsifiers. It is therefore natural to ask whether or not better bounds can be achieved for special graphs.
Moreover, all the previous constructions of cut sparsifiers with Steiner nodes, except for the ones using matroid-based/sampling-based approaches, are contraction-based.
Therefore, it is natural to explore if contraction can give optimal constructions for cut sparsifiers.


\subsection{Our Results}


In this paper, we make progress on the above questions. Our first results, summarized as \Cref{main: upper} and \Cref{quasi_1}, are new size upper bounds on exact or quality-$(1+\eps)$ sparsifiers for special families of graphs.

\begin{theorem}
\label{main: upper}
For every integer $k\ge 1$ and real number $\eps>0$, every planar graph with $k$ terminals admits a quality-$(1+\eps)$ cut sparsifier on $O(k\cdot\poly(\log k/\eps))$ vertices, which is also a planar graph.
\end{theorem}

For planar graphs, our near-linear size bound is in sharp contrast with the single-exponential bound $2^{\Theta(k)}$ \cite{krauthgamer2013mimicking,krauthgamer2017refined,karpov2017exponential} for exact cut sparsifiers. In other words, \Cref{main: upper} shows that, for preserving cuts in planar graphs, we can trade a small loss in quality for a significant improvement in size.

\begin{theorem}
	\label{quasi_1}
For every integer $k\ge 1$, every quasi-bipartite graph with $k$ terminals admits an exact (quality-$1$) contraction-based cut sparsifier of size $2^{O(k^2\log k)}$, and there exists a quasi-bipartite graph with $k$ terminals whose exact contraction-based cut sparsifier must contain $\Omega(2^k)$ vertices.
\end{theorem}

\Cref{quasi_1} shows that, similar to planar graphs, the family of quasi-bipartite graphs also admit a single-exponential upper bound (even achievable by contraction-based sparsifiers), better than the doubly-exponential bound for general graphs \cite{hagerup1998characterizing,khan2014mimicking,karpov2017exponential}.

Next we turn to study contraction-based sparsifiers.
Our next result shows that contraction-based sparsifiers are not optimal constructions, as they give worse bounds for quasi-bipartite graphs.

\begin{theorem}
\label{main: lower}
For every integer $k\ge 1$ and real number $\eps>0$, there exists a quasi-bipartite graph with $k$ terminals, whose quality-$(1+\eps)$ contraction-based cut sparsifier must contain $k^{\tilde\Omega(1/\eps)}$ vertices.
\end{theorem}

Compared with the previous bounds $\poly(k/\eps)$ \cite{andoni2014towards} and $\tilde O(k/\eps^2)$ \cite{jambulapati2023sparsifying} (both obtained by sampling-based and therefore not contraction-based approaches), \Cref{main: lower} shows that contraction-based sparsifiers are sometimes provably suboptimal.

We then proceed to study the size upper bound for quality-$(1+\eps)$ contraction-based cut sparsifiers of quasi-bipartite graphs. Our last result, summarized in \Cref{quasi_apx}, shows that when $\eps$ is a constant, quasi-bipartite graphs have $\poly(k)$-sized contraction-based cut sparsifiers, complementing the lower bound in \Cref{main: lower}, in contrast with the $\Omega(2^k)$ lower bound for exact sparsifiers in \Cref{quasi_1}.

\begin{theorem}
\label{quasi_apx}
For every integer $k\ge 1$ and real number $\eps>0$, every quasi-bipartite graph with $k$ terminals admits a quality-$(1+\eps)$ contraction-based cut sparsifier of size $k^{O(1/\eps^2)}\cdot f(\eps)$, where $f$ is a function that only depends on $\eps$.
\end{theorem}

We summarize our results and provide some comparison with previous work in \Cref{table}.

\setlength{\tabcolsep}{0.5em} % for the horizontal padding
{\renewcommand{\arraystretch}{1.4}% for the vertical padding
\begin{table}[h]
\centering
	\begin{tabular}{|c|c|c|c|c|}
		\hline
		Graph Type                       & Quality      & Size                          & Contraction-based? & Citation                                                                                      \\  \hline
		\multirow{3}{*}{General }          & $1$          & $2^{2^k}$                & Yes            & \cite{hagerup1998characterizing,khan2014mimicking} \\ \cline{2-5}
		& $1$          & $2^{\Omega(k)}$               & No         &\cite{karpov2017exponential,khan2014mimicking}\\ 
		\cline{2-5}
		& $1$          & $2^{2^{\Omega(k)}}$               & Yes         &\cite{karpov2017exponential}\\\hline
		\multirow{3}{*}{Planar}          & $1$          & $2^{O(k)}$               & Yes            & \cite{krauthgamer2013mimicking,krauthgamer2017refined} \\ \cline{2-5}
		& $1$          & $2^{\Omega(k)}$               & No         &\cite{karpov2017exponential}\\
		\cline{2-5} 
		& $1+\eps$ & $O(k\cdot\poly(\log k/\eps))$ & No                & \Cref{main: upper}                                                           \\ \hline
		\multirow{3}{*}{Quasi-bipartite} & $1$          & $2^{O(k^2\log k)}, \Omega(2^{k})$                  & Yes               & \Cref{quasi_1}                                                              \\ \cline{2-5} 
		& $1+\eps$ & $\tilde O (k/\eps^2)$           & No                & \cite{jambulapati2023sparsifying}                                                    \\ \cline{2-5} 
		& $1+\eps$ & $k^{\tilde\Omega(1/\eps)}, k^{O(1/\eps^2)} f(\eps)$     & Yes               & \Cref{main: lower,quasi_apx}                                                           \\ \hline
	\end{tabular}
\caption{Results on exact or quality-$(1+\eps)$ cut sparsifiers for planar and quasi-bipartite graphs}
\label{table}
\end{table}
}


\subsection{Technical Overview}
\label{sec: tech_overview}

\paragraph{Planar cut sparsifiers.}
Our construction of planar cut sparsifiers follows a similar framework as the planar distance emulator construction in \cite{chang2022near}, based on the intuition that, if we take the dual graph, then min-cuts become min-length partitioning cycles and therefore can be approximated by distance-based approaches. The algorithm consists of three steps: 
\begin{enumerate}
\item reduce general planar graphs to $O(1)$-face instances (where terminals lie on $O(1)$ faces in the planar embedding of the graph);
\item reduce $O(1)$-face instances to $1$-face instances (where all terminals lie on the boundary of a single face; such graphs are also known as \emph{Okamura-Seymour instances}); and
\item construct quality-$(1+\eps)$ planar cut sparsifiers for $1$-face instances.
\end{enumerate}
Step $1$ and Step $3$ are implemented in a similar way as \cite{chang2022near}. Step $2$ requires several new ideas. The goal is to preserve the min-cuts for all terminal partitions. As terminals may lie on $O(1)$ faces, for different terminal partitions, the dual partitioning cycle corresponding to the min-cuts may go around these $O(1)$ faces in different ways, and therefore it is not enough to just preserve pairwise shortest paths in the dual graph. Indeed, as observed by \cite{krauthgamer2017refined}, we need to preserve, for each pair of terminals, and for each pattern (how a path going around the $O(1)$ faces, e.g., on the left side of face $1$, on the right side of face $2$, etc), the shortest path following the pattern and connecting the terminal pair. We call such a sparsifier a \emph{pattern distance emulator}, which a generalization of the standard distance emulators.
%
As there are $O(1)$ faces, there are $2^{O(1)}$ different patterns (left/right for each face). We manage to construct a near-linear size pattern emulator, incurring an unharmful factor $2^{O(1)}$ loss in its size. 

We believe that the notion of pattern emulators is of independent interest, and should play a role in solving other algorithmic problems on planar graphs.








\paragraph{Upper bounds for quasi-bipartite graphs.}
For exact (quality-$1$) contraction-based cut sparsifiers, a central notion in previous constructions and lower bound proofs is the \emph{profiles} of vertices. Specifically, the profile of a vertex $v$ specifies, for each terminal partition $(S,T\setminus S)$, on which side of the cut $\mc_G(S,T\setminus S)$ lies the vertex $v$.
Vertices of the same profile can be safely contracted together, and the number of different profiles is therefore an upper bound on the size of exact cut sparsifiers. For general graphs, a doubly-exponential upper bound of $2^{2^k}$ \cite{hagerup1998characterizing} on the number of distinct profiles was established, and this was later improved to $2^{\binom{k}{k/2}}$ \cite{khan2014mimicking}. For quasi-bipartite graphs, we will utilize its structure to proved a better (single-exponential) upper bound.

By the definition of quasi-bipartite graphs, there are no edge between non-terminal vertices, so non-terminal vertices form separate stars with terminals, and therefore contribute to terminal min-cuts independently. For example, in a star centered at $v$ with $6$ edges $e_1,\ldots,e_6$ connecting to terminal $t_1,\ldots,t_6$, respectively. The profile of vertex $v$ only depends on the weights of the edges $e_1,\ldots,e_6$: for the terminal partition $(S=\set{t_1,t_2,t_3},T\setminus S=\set{t_4,t_5,t_6})$, $$\text{$v$ lies on the $S$ side of $\mc_G(S,T\setminus S)$ iff } w(e_1)+w(e_2)+w(e_3)>w(e_4)+w(e_5)+w(e_6).$$

This gives us some power to reveal properties of profiles in quasi-bipartite graphs.
For example, if we consider subsets $S\subseteq T$ being $\set{1,2},\set{3,4},\set{1,3},\set{2,4}$, then $v$ cannot lie on the $S$ side for all these terminal min-cuts. Since otherwise, letting $w=\sum_{1\le  i\le 6}w(e_i)$, we get $w(e_1)+w(e_2)>w/2$, $w(e_3)+w(e_4)>w/2$, $w(e_1)+w(e_3)\le w/2$, and $w(e_2)+w(e_4)\le w/2$, a contradiction. This means that there are some ``configurations'' in the family of terminal subsets that are simply not realizable by any vertex profiles in quasi-bipartite graphs. We implement this idea through the notion of VC-dimension, realizing ``configurations'' by ``set systems'' and ``not realizable'' by ``non-shatterable'', and obtain a bound of $k^{O(k^2)}=2^{O(k^2\log k)}$ on the number of  profiles in quasi-bipartite graphs.


For quality-$(1+\eps)$ cut sparsifiers, since we allow a small multiplicative error in accuracy, we are not restricted to contracting vertices with the same profile, but are allowed to moderately manipulate vertex profiles so as to reduce their number. We implement this idea by ``projections onto $O(1/\eps^2)$-size stars''. On the one hand, consider for example a full star centered at $v$ (that is, a star containing edges $(v,t)$ for all $t\in T$) with uniform edge weights, and a subset $S\subseteq T$ with $|S|=|T|/2-1$, so $v$ lies on the $T\setminus S$ side of $\mc_G(S,T\setminus S)$. If we uniformly at random sample $c=O(1/\eps^2)$ edges from it and obtain a ``mimicking substar'' $H_v$, then by central limit theorem, with high probability $H_v$ contains at most $c/2+O(1/\eps)$ edges to $S$ and at least $c/2-O(1/\eps)$ edges to $T\setminus S$. Therefore, even if $H_v$ fails in mimicking the behavior of the full star on the cut $(S,T\setminus S)$, in that it mistakenly selects more edges to $S$ than to $T\setminus S$ and therefore place $v$ on the $S$ side of $\mc_G(S,T\setminus S)$ rather than the $T\setminus S$ side, it only causes an error of $\frac{c/2+O(1/\eps)}{c/2-O(1/\eps)}=1+O(\eps)$ in the min-cut size, which is allowed.

On the other hand, we have shown in the exact case that the number of profiles for stars on a fixed $O(1/\eps^2)$-size terminal set is $O(1/\eps^2)^{O(1/\eps^4)}$ (replacing $k$ with $O(1/\eps^2)$ in the $k^{O(k^2)}$ bound). Since the number of $O(1/\eps^2)$-size terminal subsets is $k^{O(1/\eps^2)}$, we can bound the total number of profiles produced by $O(1/\eps^2)$-size starts by 
 $k^{O(1/\eps^2)}\cdot O(1/\eps^2)^{O(1/\eps^4)}$, obtaining a size bound of $k^{O(1/\eps^2)} f(\eps)$ for quality-$(1+\eps)$ contraction-based cut sparsifiers for quasi-bipartite graphs.



\paragraph{Lower bounds for quasi-bipartite graphs.}
As shown in \cite{chen20241+,chen2024lower}, contraction-based sparsifiers are closely related to the Steiner node version of the classic $0$-Extension problem \cite{karzanov1998minimum}. Specifically, \cite{chen20241+} showed that the best quality achievable by contraction-based flow sparsifiers is bounded by the integrality gap of the semi-metric LP relaxation. For cut sparsifiers, we observe that the best achievable quality is controlled by an even more restricted case of the $0$-Extension with Steiner Nodes problem, where the underlying graph is a boolean hypercube. We focus on this special case, construct a hard hypercube-instance which is also a quasi-bipartite graph, and prove a size lower bound for its $(1+\eps)$-approximation, leading to a same size lower bound for quality-$(1+\eps)$ contraction-based cut sparsifiers for quasi-bipartite graphs.
















\paragraph{Concurrent Work.} Independent of work, Das, Kumar, and Vaz, showed in their work \cite{das2024nearly} that quasi-bipartite graphs admit exact non-contraction-based cut sparsifiers of size $2^{k^2}$ and exact contraction-based cut sparsifiers of size $2^{k^3}$. Our \Cref{quasi_1} gives a $2^{O(k^2\log k)}$ size bound for exact contraction-based cut sparsifiers, which is slightly stronger than their $2^{k^3}$ bound, and slightly weaker than their $2^{k^2}$ bound. They also have some results on flow sparsifiers. For example, quasi-bipartite graphs admit exact flow sparsifiers of size $3^{k^3}$, and treewidth-$w$ graphs admit quality-$O(\frac{\log w}{\log\log w})$ flow sparsifiers of size $O(kw)$.




\subsection{Related Work}
\label{sec: related}

\paragraph{Edge sparsifiers.} 
After Karger's result \cite{karger1999random} on cut-preserving edge sparsifiers, there are other work using sampling-based approaches. Benzcur and Karger \cite{benczur1996approximate} sampled edges based on inverse edge-strengths, and obtained a sparsifier of $O(n\log n/\eps^2)$ edges. Fung and Harvey \cite{fung2010graph} sampled edges according to their inverse edge-connectivity, and also obtained the bound of $O(n\log^2 n/\eps^2)$. Spielman and Srivastava \cite{spielman2011graph} sampled edges based on their inverse effective-resistance to obtain a spectral sparsifier (a generalization of cut sparsifiers) with size $O(n\log n/\eps^2)$. This bound was later improved by Batson, Spielman, and Srivastava \cite{batson2012twice} to $O(n/\eps^2)$.

\paragraph{Other work on cut sparsifiers.}
There are some work on (i) constructing better cut sparsifiers for  special families of graphs, for example trees \cite{goranci2017vertex} and planar graphs with all terminals lying on the same face \cite{goranci2017improved}, and graphs with bounded treewidth \cite{andoni2014towards}; (ii) preserving terminal min-cut values up to some threshold value \cite{chalermsook2021vertex,liu2020vertex}; and
(iii) dynamic cut/flow sparsifiers and their utilization in dynamic graph algorithms \cite{durfee2019fully,chen2020fast,goranci2021expander}.

\paragraph{Flow sparsifiers.}
Flow sparsifiers are highly correlated with cut sparsifiers.
Given a graph $G$ and a set $T$ of $k$ terminals, a graph $G'$ is a \emph{flow sparsifier} of $G$ with respect to $T$ with \emph{quality $q$}, iff every $G$-feasible multicommodity flow on $T$ can be routed in $G'$, and every $G'$-feasible multicommodity flow on $T$ can be routed in $G$ if the capacities of edges in $G$ are increased by factor $q$. Flow sparsifiers are stronger than cut sparsifiers, in the sense that quality-$q$ flow sparsifiers are automatically quality-$q$ cut sparsifiers, but the other direction is not true in general.

When no Steiner nodes are allowed,
Leighton and Moitra \cite{leighton2010extensions} showed the existence of quality-$O(\log k/\log\log k)$ flow sparsifiers. 
%Charikar, Leighton, Li and Moitra then \cite{charikar2010vertex} showed that they can be computed efficiently.
On the lower bound side, the first quality lower bound in \cite{leighton2010extensions} is $\Omega(\log\log k)$, and this was later improved to $\Omega(\sqrt{\log k/\log\log k})$ by Makarychev and Makarychev \cite{makarychev2010metric}.


In the setting where Steiner nodes are allowed,
Chuzhoy \cite{chuzhoy2012vertex} showed $O(1)$-quality contraction-based flow sparsifiers with size $C^{O(\log\log C)}$ exist, where $C$ is the total capacity of all edges incident to terminals (assuming that all edges have capacity at least $1$).
On the lower bound side, Krauthgamer and Mosenzon \cite{krauthgamer2023exact} showed that there exist $6$-terminal graphs whose quality-$1$ flow sparsifiers must have an arbitrarily large size (i.e., the size bound cannot depend only on $k$ and $\eps$). Chen and Tan \cite{chen20241+} showed that there exists $6$-terminal graphs whose quality-$(1+10^{-18})$ contraction-based flow sparsifiers must have an arbitrarily large size.


\subsection{Organization}
The rest of the paper is organized as follows. We start with some preliminaries and formal definitions in \Cref{sec: prelim}.
We first provide the construction of planar cut sparsifier in \Cref{sec: planar}, proving \Cref{main: upper}. We then show the construction of contraction-based cut sparsifiers for quasi-bipartite graphs in \Cref{sec: quasi_exact} and \Cref{sec: quasi_apx}, proving \Cref{quasi_1} and \Cref{quasi_apx}. Finally, we show the lower bound for quasi-bipartite graphs, giving the proof of \Cref{main: lower} in \Cref{sec: lower}.










 
	% !TEX root = mfe.tex

\section{Definition of multi-stranded DNA systems and basic lemmas}\label{sec:mfe}

Intuitively, a single DNA strand $s$ is a sequence of nucleotide bases connected by covalent bonds which together make up the backbone of $s$, with the left end of the sequence corresponding to the $5'$ end of $s$ and the right end corresponding to the $3'$ end. When drawing $s$ we label  the $3'$ end with an arrow which also shows the strand directionality, see \cref{fig:sec struct}. Hydrogen bonds can form between Watson-Crick base pairs, namely C–G and A–T.

Formally, A DNA strand $s$ is a word over the alphabet of DNA {\em bases} $\{\mathrm{A},\mathrm{T},\mathrm{G},\mathrm{C}\}$, indexed from 1 to $|s|$, where $|s|$ denotes the length of $s$.
A base pair is a tuple $(i, j)$ such that $i<j$. 
For any $c$ strands, we will assign to each of them a unique distinct identifier in $\{1, . . . ,c\}$~\cite{dirks2007thermodynamic}. Each base is specified by a strand identifier and a position on that strand, $i_s$ denotes the base of index $i$ of strand $s$.

\subsection{Connected unpseudoknotted secondary structures and  polymer graphs}

\begin{Definition}[Secondary structure $S$]
	For any set of $c$ DNA strands, a secondary structure $S$ is a set of base pairs such that each base appears in at most one pair, i.e.~if $(i_n, j_m)\in S$ and $(k_q, l_r)\in S$ then $i_n,j_m,k_q,l_r$ are all distinct.
\end{Definition}
The {\em graph representation of a  secondary structure} $S$
is the graph $G=(V,E)$, where $V$ is the set of bases of each strand $s \in \{1, . . . ,c\}$, and $E = E_v \cup E_b$, where $E_v$ is the set of \emph{covalent backbone bonds} connecting base $i_n$ with base $(i+1)_n$ for all bases $i = 1,2, ..., |n|-1$ on all strands $n \in \{1, . . . ,c\}$, and $E_b = S$ is the set of base pairs in $S$. $E_v$ and $E_b$ are disjoint.

The set of circular permutations, $\Pi$, of $c$ strands has $(c-1)!$ distinct circular permutations~\cite{brualdi1977introductory} (e.g., for the three strands $\{A, B, C\}$, $\Pi = \{A B C, ACB\}$), 
e.g., the orderings $ABC$, $BCA$, and $CAB$ are the same on a circle. 
Next, we define a  polymer graph for each $\pi$, see also  \cref{fig:sec struct}.

\begin{Definition}[Polymer graph]
	For any secondary structure $S$, and any ordering $\pi$ of its $c$ strands, the polymer graph representation of $S$,  denoted  \PolySpi, is a graph representation of $S$, embedded in the unit disk from $\mathbb{R}^2$, where the $c$ strands are placed in succession from their $5'$ to $3'$ ends around the circumference of the circle, and the bases, $V$, are spaced evenly around the circle circumference, each element of $E_v$ is represented by an arc on the circumference between covalently-bonded bases, and each element of $E_b$ is represented by a chord between two different bases. 
\end{Definition}

\begin{Definition}[Unpseudoknotted secondary structure]
	A secondary structure $S$ is unpseudoknotted if there exists at least one circular permutation $\pi \in \Pi$ such that $\PolySpi$ is planar, otherwise $S$ is pseudoknotted. 
	An example is shown in \cref{fig:sec struct}.
\end{Definition} 


\begin{remark}
	In the rest of the paper we use $N$ to denote the total number of bases of a secondary structure $S$.
	A secondary structure $S$ is connected if the graph representation of $S$ is a connected graph. In this work, we are only interested in connected unpseudoknotted secondary structures.
\end{remark}

\begin{figure}[t]
	\centering\includegraphics[width=0.7\textwidth]{figures/sym.jpg}
	
	\caption{Three secondary structures  with their associated polymer graphs. In each case, there is a single complex with four identical (indistinguishable) strands of  strands of type $X$, but with different symmetry degree $R$. 
		(a)~Symmetry degree $R$ = 4 (rotation by $90^{\circ}$  gives the same secondary structure). 
		(b)~Symmetry degree  $R$ = 2 (rotation by $180^{\circ}$  gives the same secondary structure). 
		(c)~Symmetry degree  $R$ = 1 (asymmetric secondary structure).
	}\label{fig:sym}
\end{figure}

\subsection{Free energy of a secondary structure}
Any connected unpseudoknotted secondary structure $S$ can be decomposed into different loop types~\cite{tinoco,santa,mathews1999expanded}: namely hairpin loops, interior loops, exterior loops, stacks, bulges, and multiloops as shown in \cref{fig:sec struct}. 
As usual, let  $k_B$ be Boltzmann's constant and $T$ is the temperature in Kelvin (also a constant).\footnote{All results hold if we assume these are typical values from physics, or just 1 in appropriate units.} 
The free energy of $S$ is defined as the sum of three terms:\footnote{Throughout this paper $\log n = \log_\mathrm{e} n$.}
\begin{equation}\label{eq:DGss}
	\Delta G(S) =  \sum_{l\in S} \Delta G(l) + (c-1)\Delta G^{\textrm{assoc}} + k_\mathrm{B} T \log R.
\end{equation}  
\begin{itemize}
	\item the first is itself the sum of the (well-defined, empirically-obtained) free energies $\Delta G(l)$ of $S$'s constituent loops~\cite{dirks2007thermodynamic}, where each loop energy is defined with respect to the free energy of the unpaired reference state.
	\item $\Delta G^{\textrm{assoc}} $ is the entropic association~\cite{dirks2007thermodynamic} penalty applied for each of the $c-1$ strands added to the first strand to form a complex of $c$ strands.
	\item $R$ is the rotational symmetry of the secondary structure $S$,  illustrated in \cref{fig:sym}, and to be formally defined  in \cref{sec:sym}.  In particular, since favourable free energies are usually negative, the term  $k_\mathrm{B} T \log R \geq 0$ corresponds to the reduction in the entropic contribution of $S$, as any secondary structure with an $R$-fold rotational symmetry has a corresponding $R$-fold reduction in its distinguishable conformational space~\cite{dirks2007thermodynamic} as shown in \cref{fig:sym}.\footnote{This is perhaps counter-intuitive. In a follow-up expanded version of this paper we will give a full statistical mechanics explanation, which treats the symmetry penalty to offset the fact that non-symmetrical structures are undercounted.} Dynamic programming algorithms to date for mulit-stranded MFE ignored this term for reasons we outline in \cref{sec:sym}.
\end{itemize}



For $c$ strands, we let $\Omega$ be the set  (usually called the {\em ensemble}) of all  connected unpseudoknotted secondary structures. 
For any circular permutation $\pi \in \Pi$ of the $c$ strands, let $\Omega(\pi) \subseteq \Omega$ be the subset of $\Omega$ such that each connected unpseudoknotted secondary structure $S\in \Omega(\pi)$ is representable as a crossing-free polymer graph with circular permutation $\pi$. 
%\dw{: more precisely, $\pi$ is the ordering such that $\PolySpi$  has no crossings}.

\begin{remark}[$S$, or $\PolySpi$]\label{rm:PolySpi}
	Dirks et al.~\cite{dirks2007thermodynamic} showed, in their representation theorem (Theorem 2.1), that the sets $\Omega(\pi)$, for all $\pi \in \Pi$, form a partitioning of $\Omega$, which means that every connected unpseudoknotted secondary structure belongs to exactly one $\Omega(\pi)$ for some $\pi \in \Pi$. 
	Hence to avoid the cumbersome phrase 
	{\em $c$-strand connected unpseudoknotted secondary structure $S$ with strand ordering $\pi$ and polymer graph $\PolySpi$}
	we simply write $S$, or $\PolySpi$.
\end{remark}



Predicting the minimum free energy means finding a minimum over the ensemble $\Omega$. 
The known strategy is to deal with each partition $\Omega(\pi)$ separately,  
then finding their minimum:  
\begin{equation}\label{eq:MFE}
	\textrm{MFE} = \min_{S \in \Omega} \Delta G(S)  =\min\limits_{\pi \in \Pi} \left\{ \min_{S \in \Omega(\pi)} \Delta G(S) \right\}  
\end{equation}

\subsection{Definition of multi-stranded rotational symmetry} \label{sec:sym}

Here, we formalise rotational symmetry. 
In the previous section we assigned each one of the $c$ strands a unique identifier, dealing with them as distinct strands even if two or more have the same sequence. 
But in most experimental settings, strands with the same sequences are \emph{indistinguishable}   in the sense that they behave identically with respect to relevant measurable quantities~\cite{dirks2007thermodynamic}. 
Mathematically, we say that {\em two strands are  indistinguishable} if they have the same sequence. 
Also, two \emph{secondary structures are indistinguishable} if there exists a permutation of the implied unique strand ordering (\cref{rm:PolySpi}), 
that maps indistinguishable strands onto each other while preserving all base pairs, otherwise, the two structures are distinct~\cite{dirks2007thermodynamic}.


For any $c$ strands, not necessarily distinct, they consist of $k \leq c$ \emph{strand types}, usually denoted by uppercase English letters $X,Y,\ldots$\footnote{In contrast with some of the literature. we exclude using $\{\mathrm{A},\mathrm{T},\mathrm{G},\mathrm{C}\}$ for strand types; to avoid any confusion between strand and base types.}
A {\em multi-stranded DNA system} 
$M = \{(t_1,n_1),(t_2,n_2), ..,(t_k,n_k)\}$,  
is a \emph{multiset} of $k$ strand types $t_1,  ..., t_k$ with repetition numbers 
$n_1,  ..., n_k \in \mathbb{N}$ such that $n_1+ ...+n_k = c$.\footnote{It is known~\cite{sawada2003fast} how to efficiently reduce the circular permutation space by getting rid of  circular permutations that are redundant due to indistinguishable strands, which is important to consider when computing the  partition function but needed for MFE.} 
For such a multiset $M$
we can think of each circular permutation $\pi$  as a string over strand types such that each strand type $t_i$ appears exactly $n_i$ times (e.g., $M = \{(X,6),(Z,3)\}$ one valid $\pi$ is $\pi = XZXXZXXZX$). 



\begin{Definition}[Symmetry degree of a permutation] \label{def:sym}
	For any circular permutation $\pi$, we say $n \in \mathbb{N}$ is a symmetry degree of $\pi$ if $\pi = y^n$ for some $y$, a prefix of $\pi$. 
\end{Definition}

\noindent For example, $\{1,2,4\}$ are the symmetry degrees of $\pi = XZXZXZXZ$ since $\pi = (XZXZXZXZ)^1 = (XZXZ)^2 = (XZ)^4$.

For any circular permutation $\pi$, 
its maximum symmetry degree is denoted $v(\pi)$, and the corresponding repeating prefix $x$, such that $x^{v(\pi)} = \pi$, is the \emph{fundamental component} of $\pi$. 
It can be seen that $x$ is the smallest prefix that repeats over $\pi$. 
Indeed, $v(\pi)$ is the number of cyclic permutations that map each strand to a strand of the same type. 
Any repeating prefix of $\pi$ must be a multiple of its fundamental component, as proven in \cref{lem:factors} in \cref{sec:lemmasApp}.


\begin{remark}[Notation: $X_m^n$]
	For any circular permutation $\pi$, 
	its \emph{augmented version}  gives the full ordering information for each fundamental component. 
	For example, 
	if $\pi = XYXZ \, XYXZ$, then its fundamental component is $XYXZ$ and its augmented version is 
	$ X_1^1 Y_1^1 X_2^1 Z_1^1  \; X_1^2 Y_1^2 X_2^2 Z_1^2$, such that  $X_m^n$, means the $m$th strand of type $X$ in the $n$th fundamental component of $\pi$.   
\end{remark}

We can visualize any ordering $\pi$ by representing it as a \emph{regular $v(\pi)$-gon} with each of its $v(\pi)$ vertices   representing a fundamental component. %, evenly placed on a circle. 
Let $\rho = (1 \ 2 \ 3 \ ... \ v(\pi))$\footnote{Here we use algebraic cycle notation.  
	The order of $\rho$, denoted by $o(\rho)$, is the length of $\rho$ which is $v(\pi)$.} and consider the cyclic group\footnote{$G^\pi$ is isomorphic to $C_{v(\pi)}$, cyclic group of order $v(\pi)$.} $G^\pi$ generated by $\rho$. Intuitively, $G^\pi$ is the group of the all $v(\pi)$ rotational motions in plane of the regular $v(\pi)$-gon that give the same $v(\pi)$-gon. 
We can represent  $G^\pi$ as follows:  $G^\pi = \{\rho^0, \rho^1 ...,\rho^{v(\pi)-1}\}$, where $\rho^i$ represents rotation of the regular $v(\pi)$-gon by the angle of  
$i \times {\frac{360^{\circ}}{v(\pi)}}$, 
where  $|G^\pi| = v(\pi)$.


Now, we are ready to define the rotational symmetry of a secondary structure and a strand ordering $\pi$, intuitively, the number of rotations of its polymer graph that give the same polymer graph, as shown in \cref{fig:sym}.


\begin{Definition}[$R$-fold rotational symmetric structure]  \label{def:sym}
	A connected unpseudoknotted secondary structure $S$ and strand ordering $\pi$ 
	(and thus polymer graph $\PolySpi$) 
	is $R$-fold rotational symmetric, 
	or simply rotationally symmetric, 
	then for any base pair $(i,j)$ in the polymer graph $\PolySpi$
	the rotation of that base pair by multiples of $(360^\circ/R)$ is also in $\PolySpi$. 
	More formally:   
	$(i_{X_k^l},j_{Y_m^n}) \in \PolySpi$,   
	iff 
	$(i_{X_k^{a(l)}}, j_{Y_m^{a(n)}}) \in \PolySpi$ for all $a \in H \! \leq \! G^\pi$, where $H$ is the largest subgroup\footnote{$H \leq G$, notationally means $H$ is a subgroup of $G$. Every subgroup of cyclic group is also cyclic.} satisfying the condition, and if $|H| = R$.  
\end{Definition}


\begin{remark}
	In Def. \ref{def:sym},	we restricted $H$ to be the {\em largest} subgroup so as to be aligned with the entropic reduction penalty due to symmetry that appears in \cref{eq:DGss}, even if  from a geometric perspective any $R$-fold rotational symmetric secondary structure should be also $R'$-fold rotational symmetric if $R'$ divides $R$. 
\end{remark}


\section{A polynomial upper bound on a class of rotationally symmetric secondary structures}\label{sec:ub}


\begin{remark}\label{rm:noNicks}
	In this section, we assume a global indexing of all bases from $1$ to $N$, and use square brackets, $[i, i+1]$, to  denote the covalent bond connecting bases $i$ and $i +1$, given that both belong to the same strand. This notation also helps to differentiate covalent bonds $[i, i+1]$ from base pair $(j,k)$ (hydrogen bonds)  notation. 
\end{remark}

\begin{Definition}[$R$-symmetric backbone cut generated by a covalent bond]\label{def:cut}
	For  a connected unpseudoknotted secondary structure $S$ and strand ordering $\pi$, the {\em $R$-symmetric backbone cut generated by the covalent bond} $b=[i_{A_m^n}, (i+1)_{A_m^n}]$ is $\mathcal{C}_R^b = \{ [i_{A_m^{a(n)}}, ({i+1})_{A_{m}^{a(n)}}]: \textrm{ for all } a \in H \!\!\leq \!\! \!\ G^\pi \textrm{ such that } |H| = R \}$. We call $b$ a {\em symmetric backbone cut generator}. An example is shown in \cref{fig:sand}.
\end{Definition}


Only one covalent bond is needed to generate its corresponding $R$-symmetric backbone cut. Also, note that this definition excludes any cut through nicks by \cref{rm:noNicks}. The next lemma shows that the number of unique symmetric backbone cuts is linear in $N$.

\subsection{Linear upper bound on number of unique symmetric backbone cuts}

\begin{lemma}[Upper bound on unique symmetric backbone cuts]\label{lem:ub}
	For any connected unpseudoknotted secondary structure $S$ of $c=\mathcal{O}(1)$ strands with $N$ total bases, with a specific strand ordering $\pi$, the number of  unique symmetric backbone cuts is $\frac{N-c}{v(\pi)} \left[ \sigma(v(\pi))-v(\pi) \right] = \mathcal{O}(N)$, where $\sigma(v(\pi))$ is sum of divisors of $v(\pi)$. 
\end{lemma}

\begin{proof}\textcolor{red}{TOPROVE 0}\end{proof}


If $S$ is a connected unpseudoknotted secondary structure with ordering $\pi$, you can go from any base $i$ to $j$ in two different paths around the circumference of $\PolySpi$ (clockwise or anticlockwise). We define the \emph{length} function $l[i,j]$ to be the length of the shorter path, including both $i$ and $j$ as follows: 
\begin{equation}
	l[i,j] = \min \{|i-j|+1,N-|i-j|+1\}
\end{equation} 

Also,  $\llbracket i,j \rrbracket$  is used to denote that shorter {\em segment} of length $l[i,j]$, where the direction from base $i$ to base $j$ is the same as the system strands' direction.    

\subsection{How to slice a pizza (secondary structure)}

{\bf We} want to slice any $R$-fold rotational symmetric secondary structure $S$, {\bf like pizza}, to the centre of its $\PolySpi$, without intersecting any of its base pairs. First, we formalise (\cref{def:admissible}) a special type of backbone cut, called an \emph{admissible  $R$-symmetric backbone cut}. Then, we will prove (\cref{lem:cutexist}) its existence for $S$. 

\begin{Definition}[Admissible $R$-symmetric backbone cut] \label{def:admissible}
	For any connected unpseudoknotted secondary structure $S$ with strand ordering $\pi$, the {\em $R$-symmetric backbone cut} $\mathcal{C}_R^b$ generated by $b$ is {\em admissible}, 
	if for all covalent bonds $x \in \mathcal{C}_R^b$,  
	$x$ is not ``enclosed'' by any base pair $(i,j) \in \PolySpi$, more formally:  $x \nsubseteq \llbracket i,j \rrbracket$.   An example is shown in \cref{fig:sand}.
\end{Definition}


\begin{restatable}{lemma}{cutexist} 
	\label{lem:cutexist}
	For any $R$-fold rotationally symmetric secondary structure $S$, there exists at least one admissible $R$-symmetric backbone cut of $S$.
\end{restatable}
\begin{proof}\textcolor{red}{TOPROVE 1}\end{proof}

Note that any $R$-fold rotationally symmetric secondary structure $S$ can have more than one admissible R-symmetric cut. Before defining what we mean by a pizza slice (formally, symmetric slice in \cref{def:symmetric slice}),  the following lemma is used to  ensure such a slice is  connected.

\begin{restatable}[Pizza slicing lemma] {lemma}{connected} 
	\label{lem:connected}
	For any $R\geq 2$ and any $R$-fold rotational symmetric secondary structure $S$, let $G$ be the graph obtained from $\PolySpi$ by removing the covalent bonds of admissible $R$-symmetric backbone cut $\mathcal{C}_R^b$ generated by any covalent bond $b$, such that $G=(V(\PolySpi), E(\PolySpi)\setminus\mathcal{C}_R^b)$, then $G$ is disconnected and consists exactly of $R$ connected isomorphic components.
\end{restatable}  




\begin{proof}\textcolor{red}{TOPROVE 2}\end{proof}

\begin{Definition}[Symmetric slice]\label{def:symmetric slice}
	From the construction in \cref{lem:connected}, each of the $R$ 
	isomorphic subgraphs (components) in $ \mathcal{G}$ is called a {\em symmetric slice} of $\PolySpi$,  denoted  by $\rhd^{S}$. Also, the loop free energy of a symmetric slice is: 
	\begin{equation}
		\Delta G(\rhd^{S}) = \sum_{l\in \rhd^{S}} \Delta G(l)
	\end{equation}
\end{Definition}

For any $R$-fold symmetric secondary structure $S$, the following lemma shows the existence of a unique loop in the center of $\PolySpi$  surrounded by the outer base pairs of its symmetric slices. We call it the {\em central loop} of $S$, and denote it by $\bigcirc^S$. This central loop plays a crucial role in validating our slicing and swapping strategy  (\cref{lem:sand,lem:sand2}) and determining the exact upper bound (\cref{lem:polyub}) of symmetric secondary structures need to be backtracked.  


\begin{restatable}{lemma}{centralloop} 
	\label{lem:centralloop}
	For any $R$-fold symmetric secondary structure $S$, there exists a single loop, that we call the central loop $\bigcirc^S$, that is not contained in any of the $R$ symmetric slices. 
	If $R>2$ then $\bigcirc^S$ is a multiloop, if $R=2$ then $\bigcirc^S$ is either a multiloop, stack, or an internal loop.   
\end{restatable}
\begin{proof}\textcolor{red}{TOPROVE 3}\end{proof}


Because of the crucial importance of multiloops for our strategy, we highlight the multiloop energy model that is used in the standard dynamic programming algorithms. The free energy of a multiloop has the following linear form~\cite{dirks2007thermodynamic}:
\begin{equation} \label{eq:multi}
	\Delta G^ \textrm{multi}  = \Delta G_\textrm{init}^\textrm{multi} + b \Delta G_\textrm{bp}^\textrm{multi} + n\Delta G_\textrm{nt}^\textrm{multi}.
\end{equation}
Where, $\Delta G_\textrm{init}^\textrm{multi}$ is called the penalty for   formation of the multiloop, $\Delta G_\textrm{bp}^\textrm{multi}$ is called the penalty for each of its $b$ base pairs that border the interior of the multiloop, and $\Delta G_\textrm{nt}^\textrm{multi}$ is called the penalty for each of the $n$ free bases inside the multiloop. For any $R$-fold symmetric secondary structure~$S$, 
$b \Delta G_\textrm{bp}^\textrm{multi}$ and $n\Delta G_\textrm{nt}^\textrm{multi}$ are shared equally between the $R$ symmetric slices of $\PolySpi$, hence $R$ divides both $b$ and $n$. So, we denote  $\Delta G^ \textrm{multi}  = \Delta G_\textrm{init}^\textrm{multi} + R(\Delta G_{\rhd^{S}}^\textrm{multi})$, where $\Delta G_{\rhd^{S}}^\textrm{multi}$ is the energy contribution of each symmetric slice of $\PolySpi$ to the multiloop free energy, such that $\Delta G_{\rhd^{S}}^\textrm{multi} = \frac{b}{R} \Delta G_\textrm{bp}^\textrm{multi} + \frac{n}{R} \Delta G_\textrm{nt}^\textrm{multi}$.

\begin{note}
	For any connected unpseudoknotted secondary structure $S$ of $c$ strands, we use $\overline{\Delta G}(S)$ to denote the \snMFE of $S$, in other words $\overline{\Delta G}(S) = \sum_{l\in S} \Delta G(l)
	+  (c-1)\Delta G^{\textrm{assoc}}$. It is clear that $\overline{\Delta G}(S) \leq \Delta G(S)$, as the symmetry correction $k_\mathrm{B} T \log R \geq 0$. 
\end{note}


\begin{figure}[t]
	\centering\includegraphics[width=0.7\textwidth]{figures/sand.jpg}
	\caption{Slicing and swapping strategy for constructing new asymmetric structure by combining two symmetric structures with the same symmetric backbone cut. 
		(a)~4-fold symmetric secondary structure $S_i$, with admissible $4$-symmetric backbone cut $\mathcal{C}_R^b$.
		Black arrows: indicate the four covalent bonds forming $\mathcal{C}_R^b$ generated by the covalent bond $b$. 
		(b)~4-fold symmetric secondary structure $S_j$, sharing the same cut $\mathcal{C}_R^b$ as $S_i$.
		Black arrows: indicate the four covalent bonds forming $\mathcal{C}_R^b$. 
		(c)~Asymmetric secondary structure $S_k$ that is constructed by replacing the grey shaded `slice' from $S_i$ by its corresponding slice from $S_j$, using the proof of  \cref{lem:sand}. 
	}\label{fig:sand}
\end{figure}



Intuitively, the following lemma lets us take two symmetric pizzas with the same admissible symmetric cut for which we merely now their \snMFE (we don't know their true MFE), and swap a slice from one into the other to get a new asymmetric pizza whose true MFE lies between their {\snMFE}s. The key intuition is that we transforming symmetric secondary structures into an  asymmetric one. 

\begin{lemma}\label{lem:sand}[Free-energy sandwich theorem for two $R$-fold rotational symmetric  structures]
	For any two distinct $(R \geq 3)$-fold rotationally symmetric secondary structures, $S_i$ and $S_j$, of $c$ strands, such that $R \geq 3$ and $\DGnosym(S_i) \leq \DGnosym(S_j)$ and $S_i$ and $S_j$ have the same R-admissible backbone cut $\mathcal{C}_R^b$, then there exists at least one asymmetric secondary structure $S_k$, such that $\DGnosym(S_i) \leq \Delta G(S_k) \leq \DGnosym(S_j)$.  Furthermore, the statement holds if $R=2$ and at least one of the central loops  $\bigcirc^{S_i},  \bigcirc^{S_j}$  is a multiloop. 
\end{lemma}

\begin{proof}\textcolor{red}{TOPROVE 4}\end{proof}

\cref{lem:sand} states that, if two symmetric secondary structures, having the same admissible R-symmetric backbone cut, belong to the same energy level based on \symnMFE algorithm, ignoring symmetry entropic correction, this implies the existence of at least one asymmetric secondary structure that actually belong to the same energy level because symmetry  correction for asymmetric structures is zero. If the two secondary structures belong to two different energy levels, then there exist at least one asymmetric secondary structure that actually belong to an energy level strictly lies between the other two energy levels.


\paragraph{Intuition for the case of $R=2$ and the central loop is not a multiloop.} 
When $R=2$, and the central loop is not a multiloop, the proof of \cref{lem:sand} breaks.   
From \cref{lem:centralloop}, when $R=2$ the central loop is either a multiloop, internal loop or stack loop. 
The multiloop case has been handled already (\cref{lem:sand}), and the stack loop case can be subsumed into the  internal loop case, since stacks are considered a special type of internal loop in the standard energy model~\cite{dirks2007thermodynamic}. 
Instead of depending on having the same admissible 2-symmetric backbone cut, we depend on sharing the same central internal loop itself, this more restricted hypothesis   implies having the same admissible 2-symmetric backbone cut too, allowing us to prove \cref{lem:sand2} using a similar strategy to  \cref{lem:sand}.  


\begin{lemma}\label{lem:sand2}[Free-energy sandwich theorem for two $2$-fold rotational symmetric structures]
	For any two distinct  $2$-fold rotationally symmetric secondary structures, $S_i$ and $S_j$, of $c$ strands, such that $\DGnosym(S_i) \leq \DGnosym(S_j)$ and both have the same central internal loop $\bigcirc^{S_i} = \bigcirc^{S_j}$, then there exists at least one asymmetric secondary structure $S_k$, such that $\DGnosym(S_i) \leq \Delta G(S_k) \leq \DGnosym(S_j)$.  
\end{lemma}
\begin{proof}\textcolor{red}{TOPROVE 5}\end{proof}

We now have two sandwich theorems that we can use to construct an asymmetric structure: \cref{lem:sand,lem:sand2}. In \cref{sec:BT} we give a backtracking algorithm to search for suitable $S_i$ and $S_j$, with the goal of  applying either one of these two sandwich theorems to $S_i$ and $S_j$. 
To get an overall polynomial bound for the backtracking algorithm, we wish to bound, given $S_i$, how many secondary structures to scan before finding a suitable $S_j$.    
\cref{lem:ub} gives this upper bound on this number  when applying \cref{lem:sand}. 
Next, \cref{lem:ub2} gives this upper bound  when applying  \cref{lem:sand2}. 

Unfortunately, the bound in \cref{lem:ub2} is larger than \cref{lem:ub}, since the energy model is more complex for internal loops than multiloops~\cite{dirks2003partition}. 

\begin{lemma}[Upper bound on number of  central internal loops]\label{lem:ub2}
	For any set of $c$ strands with specific ordering $\pi$, for any set $\mathcal{T}$ of $2$-fold rotational symmetric secondary structures $(R=2)$, 
	such that each has a distinct internal central loop, the  cardinality of $\mathcal{T}$, $ |\mathcal{T}| \leq \sum_{s\in y} (\lVert \baseA \rVert_s \lVert \baseT \rVert_s+ \lVert \baseG \rVert_s \lVert \baseC \rVert_s ) \leq N^2/16$, 
	where $y$ is a fundamental component of $\pi$, such that $\pi = y^2$, and $\parallel\! \! B \!\! \parallel_s$  
	denotes the number of bases in strand $s$ of type $B$ for all $B\in\{\mathrm{A},\mathrm{T},\mathrm{G},\mathrm{C}\}$.
\end{lemma} 

\begin{proof}\textcolor{red}{TOPROVE 6}\end{proof}









\subsection{Polynomial upper bound on number of symmetric secondary structures (for future backtracking)}
\begin{lemma}\label{lem:polyub}
	
	Given an ordering $\pi$ of $c$ strands, 
	for any set $\mathcal{T}$ of distinct symmetric secondary structures such that 
	\begin{enumerate}
		\item  for any two $(R>2)$-fold symmetric secondary structures $S_i, S_j \in \mathcal{T}$, where $S_i$ and $S_j$ have different admissible R-symmetric backbone cuts (we mean all possible cuts are different), 	
		and 
		\item 
		for any two 2-fold symmetric secondary structures $S_i, S_j \in \mathcal{T}$, where $S_i$ and $S_j$ have different admissible R-symmetric backbone cuts (all possible cuts are different)
		or  different central internal loops, 
	\end{enumerate}
	then $|\mathcal{T}| \leq \mathcal{U}$, where $\mathcal{U} =  \frac{N-c}{v(\pi)} \left[ \sigma(v(\pi))-v(\pi) \right] + \frac{N^2}{16} = \mathcal{O}(N^2)$. 
\end{lemma}

\begin{proof}\textcolor{red}{TOPROVE 7}\end{proof}

The (bad) quadratic bound in \cref{lem:ub2} is not that frequent: 
In particular, that bound only appears when $R=2$ and the central loop is an internal loop for both symmetric secondary structures (since $R=2$ this implies that the repetition number for every strand type is {\em even}, which in practice, say, for random or typical systems, may not be frequent).   
In particular the following lemma gives a {\em linear} bound when the repetition number of at least one strand type is odd. 


\begin{lemma} \label{lem:even}
	For any $R$-fold rotationally symmetric  secondary structure $S$, with ordering $\pi$, such that $R$ is even, then the repetition number of each strand type must be even. Hence for any system of $c$ strands ($k$ strand types) such that the repetition number of some strand type is odd, then $\mathcal{U}$, where $\mathcal{U} =  \frac{N-c}{v(\pi)} \left[ \sigma(v(\pi))-v(\pi) \right] = \mathcal{O}(N)$. 
\end{lemma}
\begin{proof}\textcolor{red}{TOPROVE 8}\end{proof}





	% !TEX root = mfe.tex

\section{Backtracking to find the true MFE}\label{sec:BT}

In this section, we give a backtracking procedure, 
\cref{algo:2} in \cref{app:backalgo}, 
to give our main result (\cref{thm:main}). 
First,  we run our augmentation of the known \symnMFE (\snMFE) algorithm---\cref{algo:1} in \cref{sec:AlgoMFE}, which returns some matrices which are input to the backtracking algorithm, \cref{algo:2}. 
Our multistranded backtracking algorithm builds on the single-stranded backtracking algorithm of Wuchty et al.~\cite{wuchty1999complete},
which in turn follows Waterman and Byers~\cite{waterman1985dynamic}, 
 although we make several technical modifications. 
In particular, distinctions with that previous work~\cite{wuchty1999complete,waterman1985dynamic} include: 
\begin{enumerate}
	\item We generalise backtracking from single-stranded to multistranded, which has a slightly different MFE algorithm, consistent with Dirks et al and Fornace et al~\cite{dirks2007thermodynamic,fornace2020unified} (i.e.~as implemented by the NUPACK software); in particular to ensure  the connectedness of secondary structures (a non-issue for~\cite{wuchty1999complete,waterman1985dynamic}). 
	\item We make major changes to the core of the backtracking algorithm  to ensure generation of all secondary structures, 
	at each of a specified number of energy levels, in energy level order 
	(which is different from the Wuchty et al's~\cite{wuchty1999complete} approach of backtracking all sub-optimal secondary structures that lie between the \snMFE and any arbitrary upper limit above it). 
	\item We extend the refinement cases of  Wuchty et al~\cite{wuchty1999complete} to handle our auxiliary matrices in such a way that yields  a good running time. 
\end{enumerate}

\subsection{Partially and fully specified structures} 


\begin{Definition}[Partially and fully specified structure $\mathcal{S}$]
	A partially specified structure $\mathcal{S} = (\delta,\mathcal{P},E_{L_\mathcal{S}})$, where $\delta$ is a stack of disjoint segments of one or more DNA strands $\{[i,j]^t. [k,l]^{t'} \ldots\}$ where $[i,j]^t$ is the top of the stack, such that $i$ and $j$ are the end bases of the segment $[i,j]$, $t \in \{\square,b,m\}$ is the type of each segment, 
	such that $t = \square$ means the existence of a base pair between $i$ and $j$, is as yet undetermined, 
	$t =b$ means there is a base pair between $i$ and $j$, and 
	$t = m$ means that entire segment $[i,j]$ is part of a multiloop. 
	$\mathcal{P}$ is a set of base pairs formed in $\mathcal{S}$, and $E_{L_\mathcal{S}}$ is the energy of all loops that are 
	`completely formed' in $\mathcal{S}$. If $\delta = \phi$, we call $\mathcal{S}$ a fully specified structure.
\end{Definition}

A fully specified structure is a connected unpseudoknotted secondary structure. For any segment $[i,j]^t$,  label $t$ is assigned according to how a segment is generated through \emph{refinement} from another segment, formalized in \cref{sec:backhigh}, 
label $t$ is needed in switching the backtracking between the appropriate matrices of the \snMFE algorithm.
We will denote the minimum free energy attainable from segment $[i,j]^t$, by $E([i,j]^t)$, which we get directly from the appropriate matrix $M$, $M^b$, $M^m$, $M^\text{b:int}$, $M^\text{b:mul}$, or $M^\text{m:2}$, that are returned  by the \snMFE algorithm (\cref{algo:1}), based on $t$, full details are in \cref{sec:AlgoMFE}. 
The domain of the  function $E$ is extended to include the set of all partially specified structures, in addition to the set of all segments,   
$\mathcal{S} = (\delta,\mathcal{P},E_{L_\mathcal{S}})$ so that 
$E$ gives  the minimum free energy attainable from $\mathcal{S}$, 
respecting the refinement rules formalized in \cref{sec:backhigh}, as follows:
\begin{equation}\label{eq:ES}
	E(\mathcal{S}) = E_{L_{\mathcal{S}}} 
	+ \sum \limits_{[m,n]^t \in \delta} E([m,n]^t)
\end{equation}

Any partially specified structure $\mathcal{S} = (\delta,\mathcal{P},E_{L_\mathcal{S}})$ represents a set of all structures that have the base pairs $\mathcal{P}$ in common:  we can think of $\mathcal{S}$ as the root of the tree of these structures, all intermediate nodes of this tree will be partially specified structures, and its leaves will be fully specified structures, and $E(\mathcal{S})$ is the minimum free energy attainable from this tree where all its nodes, structures, are further \emph{refinements} of $\mathcal{S}$.   

\begin{Definition}[Refinement of a partially specified structure]
	A structure $\mathcal{S}' = (\delta',\mathcal{P}',E_{L_{\mathcal{S}'}})$ is called a refinement of the partially specified structure $\mathcal{S} = (\delta,\mathcal{P},E_{L_\mathcal{S}})$ if $\mathcal{P} \subseteq \mathcal{P}'$, and for each segment $[i',j']^{t'} \in \delta'$ there exist a segment $[i,j]^t \in \delta$ such that $[i',j']^{t'} \subseteq [i,j]^t$.
\end{Definition}

\subsection{Analysis of the backtracking algorithm refinement rules}\label{sec:backhigh}


The backtracking algorithm  starts with $\mathcal{S} = ([1,N]^\square, \phi,0)$, which represents the whole system of strands with a specific strand order, $\pi$, without any base pair formed, $\mathcal{P} = \phi$, hence no loops are formed too, $E_{L_\mathcal{S}} = 0$. $\mathcal{S}$ is the parent node of the tree of any possible structure. 
Now, we will outline the main refinement procedure of the generic partial structure $\mathcal{S} = ([i,j]^t.\delta, \mathcal{P}, E_{L_{\mathcal{S}}})$ that has been chosen, at the beginning of each iteration of the backtracking algorithm ({\cref{algo:2}}), 
from some array $\mathcal{R}_{z}$, the array of partially specified structures associated with each secondary structure $S_{z} \in \{S_1, \ldots, S_\mathcal{U}\}$, which is the secondary structure number $z$, of the worst case $\mathcal{U}$ secondary structures we need to scan (\cref{lem:polyub}), such that $S_z$ is completely scanned during the backtracking. 
The segment $I = [i,j]^t$, the top of the
segments stack of $\mathcal{S}$, will be popped  and  refined based on the type of  label $t$ resulting in a new refined structure $\mathcal{S}'$. Matrices $M$, $M^b$, $M^m$, and the new auxiliary matrices $M^\text{b:int}$, $M^\text{b:mul}$, and $M^\text{m:2}$, returned by  \cref{algo:1}, will be used to compute the minimum free energy $E(\mathcal{S}')$ attainable from  the refined partially specified structure~$\mathcal{S}'$.



Given that the algorithm  scans, or backtracks, all secondary structures in energy level $\mathcal{E}$, and $\mathcal{B}$ is the best candidate for the true MFE at the moment, then the \emph{acceptance criteria} of any refined partially specified structure $\mathcal{S}'$ is: 
\begin{equation}
	E(\mathcal{S}') \leq    \mathcal{B}
\end{equation}

This acceptance criteria is checked after each refinement case, and if it is satisfied, $\mathcal{S}'$ will be added on the array of partially specified structures $\mathcal{R}_u$, for some $u \in \{1, \ldots, \mathcal{U} \}$, for further refinements in future, where $\mathcal{R}_u$ is the secondary structure currently being  scanned. 
Note that, because of the strict sequential scanning of the backtracking algorithm (\cref{remark:newS}), 
the acceptance criteria implicitly implies that $\mathcal{E} \leq E(\mathcal{S}')$. Also, 
the acceptance criteria guarantees the connectedness of at least one potential fully specified structure which is a child of $\mathcal{S}'$  
(in  \cref{algo:2}, setting $E(\mathcal{S}') = +\infty$ implies a disconnected or invalid structure). 

There are $3$ cases based on the type of $t$ of the segment $I = [i,j]^t$ that has been popped (as mentioned above) from the stack $\delta$:



\begin{enumerate}\item
	$t = \square$ (recall, $\square$ means: the existence of a base pair between $i$ and $j$ is undetermined): 
	
	In this case we backtrack in matrix $M$.  
	
	$i$ and $j$ are the end bases of $I$, and the possible refinements, based on Eq.~(1) in \cref{fig:mfe}, are: 
	
	\begin{itemize}
		\item Subcase: If the base $j$, at the $3'$ end of the segment $[i,j]$, is unpaired, that will result in the new partial structure (i.e.~excluding $j$ and moving to $j-1$):
		
		$\mathcal{S}' = ([i,j-1]^\square.\delta, \mathcal{P}, E_{L_{\mathcal{S}}})$ such that $E(\mathcal{S}') = M_{i,j-1} + E_{L_{\mathcal{S}}} 
		+ \sum \limits_{[m,n]^t \in \delta} E([m,n]^t)$.
		
		\item 	Subcase: If the base $j$ forms a base pair with base $d \in [i,j-1]$,  we need to scan all such  $d \in [i,j-1]$, 
		and for each we have the new partial structure:
		
		$\mathcal{S}' = ([i,d-1]^\square.[d,j]^b.\delta, \mathcal{P}, E_{L_{\mathcal{S}}})$ such that $E(\mathcal{S}') = M_{i,d-1} + M^b_{d,j}  +E_{L_{\mathcal{S}}} 
		+ \sum \limits_{[m,n]^t \in \delta} E([m,n]^t)$. 
		
		Note that we did not add the base pair $(d,j)$ to $\mathcal{P}$ at this step, but we shall do when refining the interval
		$[d,j]^b$ enclosed by $(d,j)$~\cite{wuchty1999complete}. 	
	\end{itemize}
	
	Then the acceptance criteria will be checked after each of the two sub-cases above. The backtracking algorithm have to  check up to $\mathcal{O}(N)$ refined structures (because of $d$ spans $\leq N$ bases in subcase 2), and hence save up to  $\mathcal{O}(N)$ refined structures to $\mathcal{R}_u$ in the worst case.   
	
	\item
	Case $t = b$ (recall: a base pair is formed between the end bases $i$ and $j$ of $[i,j]^b$ segment):
	
	In this case we backtrack in matrix $M^b$. 	
	Now, assume the segment $[i,j]^b$ is popped from the stack $\delta$, based on Eq.~(2) in \cref{fig:mfe}, there are four subcases:
	
	\begin{itemize}
		
		\item \textbf{Hairpin loop} formation: If  $(i,j)$ is closing a hairpin loop (\cref{fig:mfe}(b)), this will result in the new partial structure:
		
		$\mathcal{S}' = (\delta, \mathcal{P} \cup \{(i,j)\}, E_{L_{\mathcal{S}}} + \Delta G_{i,j}^\text{hairpin} )$ such that $E(\mathcal{S}') = \Delta G_{i,j}^\text{hairpin} + E_{L_{\mathcal{S}}} 
		+ \sum \limits_{[m,n]^t \in \delta} E([m,n]^t)$. 
		
		
		\item \textbf{Interior loop} formation: We need to scan all possible base pairs $(d,e)$ that bind to form an interior loop along with $(i,j)$ (\cref{fig:mfe}(b)). 
		Scanning all pairs in a straightforward way would lead to checking up to $\mathcal{O}(N^2)$ refined structures, 
		which would end up with a poor final worse-case time complexity of the backtracking algorithm. 
		Instead, we scan these base pairs in a different way, that keeps the number of refined structures that we need to check at each iteration to $\mathcal{O}(N)$. We achieve this by introducing a new auxiliary matrix, called $M^\text{b:int}$, in the \snMFE algorithm,  \cref{algo:1}. 		
		
		Modifying the generic form of segment $I$ is essential, in this case, the new segment generic form will be $I = [i,j]^b_{\text{int}:k}$, such that $k \in [i+1,j-1]$, which means that any base $d \in[k,j-1]$ is unpaired, hence not included in the second base pair formation to complete the interior loop with $(i,j)$.
		When a new segment $[i,j]^b$ is just generated as a refinement from another segment, 
		$[i,j]^b$ will be interpreted inside this case as $ [i,j]^b_{\text{int}:j} $, which means any base $d \in [i+1,j-1]$ can be part of the second base pair closing the current interior loop. 
		Given  $I = [i,j]^b_{\text{int}:k}$, there are two cases:
		
		1) If the base $k-1$ is also unpaired, this will
		result in the new partial structure: 
		
		$\mathcal{S}' = ([i,j]^b_{\text{int}:k-1}.\delta, \mathcal{P}, E_{L_{\mathcal{S}}})$ such that $E(\mathcal{S}') = M_{i,j,k-2}^\text{b:int} + E_{L_{\mathcal{S}}} 
		+ \sum \limits_{[m,n]^t \in \delta} E([m,n]^t)$.
		
		2) If the base $k-1$ is paired with another base $d \in [i+1,k-2]$ closing the interior loop, this will result in the new partial structure: 
		
		$\mathcal{S}' = ([d,k-1]^b.\delta, \mathcal{P} \cup \{(i,j)\}, E_{L_{\mathcal{S}}} + \Delta G_{i,d,k-1,j}^\text{interior})$ such that $E(\mathcal{S}') = M^b_{d,k-1} + \Delta G_{i,d,k-1,j}^\text{interior}   +E_{L_{\mathcal{S}}} 
		+ \sum \limits_{[m,n]^t \in \delta} E([m,n]^t)$. 
		
		\item \textbf{Multiloop} formation: In this case we also need to scan all possible pairs $(d,e)$ that will used to form a multi-loop to the $3'$ end (\cref{fig:mfe}(b)), so we will follow the same strategy as the case of interior loop formation, by introducing another new auxiliary matrix, called $M^\text{b:mul}$, in the symmetry agnostic MFE algorithm,  \cref{algo:1}. 
		The generic form of $I$ in this case will be updated to $I = [i,j]^b_{\text{mul}:k}$ such that $k \in [i+1,j-1]$, which means that any base $d \in[k,j-1]$ is unpaired, hence not included in the forming any base pair inside this multiloop along with $(i,j)$.
		When a new segment $[i,j]^b$ is just generated as a refinement from another segment, 
		$[i,j]^b$ will be interpreted inside this case as $ [i,j]^b_{\text{mul}:j} $, which means any base $d \in [i+1,j-1]$ can be part of base pair formation inside this multiloop. 
		Given  $I = [i,j]^b_{\text{mul}:k}$, there are two cases:
		
		1) If $k-1$ is also unpaired, this will
		result in the new partial structure: 
		
		
		$\mathcal{S}' = ([i,j]^b_{\text{mul}:k-1}.\delta, \mathcal{P}, E_{L_{\mathcal{S}}})$ such that $E(\mathcal{S}') = M_{i,j,k-2}^\text{b:mul} + E_{L_{\mathcal{S}}} + \sum \limits_{[m,n]^t \in \delta} E([m,n]^t)$. 
		
		2) If the base $k-1$ is paired with another base $d \in [i+1,k-2]$, this will result in the new partial structure: 
		
		$\mathcal{S}' = ([i+1,d-1]^m.[d,k-1]^b.\delta, \mathcal{P} \cup \{(i,j)\}, \Delta G_\text{init}^\text{multi} + 2\Delta G_\text{bp}^\text{multi} + (j-k) \Delta G_\text{nt}^\text{multi} + E_{L_{\mathcal{S}}})$ such that $E(\mathcal{S}') =  M^m_{i+1,d-1} + M^b_{d,k-1} + \Delta G_\text{init}^\text{multi} + 2\Delta G_\text{bp}^\text{multi} + (j-k)\Delta G_\text{nt}^\text{multi} +E_{L_{\mathcal{S}}} 
		+ \sum \limits_{[m,n]^t \in \delta} E([m,n]^t)$. 
		
		\item \textbf{Exterior loop} formation: All bases $z \in [i,j]$ such that $[z,z+1]$ is a nick \cref{fig:sec struct}, transition between two strands, are scanned (\cref{fig:mfe}(b)), leading to the new partial structure: 
		
		$\mathcal{S}' = ([i+1,z]^\square.[z+1,j-1]^\square.\delta, \mathcal{P} \cup \{(i,j)\}, E_{L_{\mathcal{S}}})$ such that $E(\mathcal{S}') = M_{i+1,z} +  M_{z+1,j-1} +  E_{L_{\mathcal{S}}} + \sum \limits_{[m,n]^t \in \delta} E([m,n]^t)$.   
		
		
		
	\end{itemize}
	
	Then the acceptance criteria will be checked after each sub-case. Now, with the aid of the introduced new auxiliary matrices $M^\text{b:int}$ and $M^\text{b:mul}$, the backtracking algorithm checks up to $\mathcal{O}(N)$ refined structures, and hence saves up to  $\mathcal{O}(N)$ refined structures to $\mathcal{R}_u$ in the worst case. Without the  new auxiliary matrices (in this case of $t=b$), the backtracking algorithm will check up to $\mathcal{O}(N^2)$ refined structures, and saves up to  $\mathcal{O}(N^2)$ refined structures to $\mathcal{R}_u$.
	
	
	\item
	Case $t = m$ 	(recall: the segment $[i,j]^m$ is a part of a multiloop): 
	
	We backtrack in matrix $M^m$. 
	Now, assume the segment $[i,j]^m$ is popped from the stack, based on Eq.~(3) in \cref{fig:mfe},  there are two subcases:
	
	\begin{itemize}
		\item  If there exists exactly one additional base pair $(d,e)$ defining the multiloop (\cref{fig:mfe}(c)), then we will scan for all possible pairs $(d,e)$ that could be used. Following the same strategy of scanning introduced before (i.e.~in case 2. $t=b$, interior loop or multiloop) to reduce time in interior and multiloop formation cases when $t = b$, there are two cases: 
		
		1) If $j$ is unpaired, this will
		result in the new partial structure: 
		
		$\mathcal{S}' = ([i,j-1]^m.\delta, \mathcal{P}, E_{L_{\mathcal{S}}} + \Delta G_\text{nt}^\text{multi})$ such that $E(\mathcal{S}') = M_{i,j-1}^m + \Delta G_\text{nt}^\text{multi} +E_{L_{\mathcal{S}}} + \sum \limits_{[m,n]^t \in \delta} E([m,n]^t)$. 	 	
		
		2) If the base $j$ is paired with another base $d \in [i,j-1]$ defining the multiloop, this will result in the new partial structure: 
		
		$\mathcal{S}' = ([d,j]^b.\delta, \mathcal{P}, E_{L_{\mathcal{S}}} + \Delta G_\text{bp}^\text{multi} + (d-i) \Delta G_\text{nt}^\text{multi})$ such that $E(\mathcal{S}') = M^b_{d,j} + \Delta G_\text{bp}^\text{multi}+ (d-i) \Delta G_\text{nt}^\text{multi} +E_{L_{\mathcal{S}}} 
		+ \sum \limits_{[m,n]^t \in \delta} E([m,n]^t)$. 
		
		
		\item  If there exist more than one  base pair defining the multiloop, all possible pairs $(d,e)$ are scanned (\cref{fig:mfe}(c)). Following the same strategy of scanning, and using one of the new auxiliary matrices, called $M^\text{m:2}$, in the \snMFE algorithm, \cref{algo:1}. 
		
		
		The generic form of $I$ in this case will be updated to $I = [i,j]^m_{\text{mul}:k}$ 	such that $k \in [i,j]$, which means that any base $d \in[k,j]$ is unpaired, hence not included in the forming of any base pair inside this multiloop.
		When a new segment $[i,j]^m$ is just generated as a refinement from another segment, 
		$[i,j]^m$ will be interpreted inside this case as $ [i,j]^m_{\text{mul}:j-1} $, which means any base $d \in [i,j]$ can be part of base pair formation inside this multiloop. 
		Given  $I = [i,j]^m_{\text{mul}:k}$, there are two cases:
		
		
		
		
		1) If the base $k-1$ is also unpaired, this will
		result in the new partial structure: 
		
		$\mathcal{S}' = ([i,j]^m_{\text{mul}:k-1}.\delta, \mathcal{P}, E_{L_{\mathcal{S}}})$ such that $E(\mathcal{S}') = M_{i,j,k-2}^\text{m:2} + E_{L_{\mathcal{S}}} + \sum \limits_{[m,n]^t \in \delta} E([m,n]^t)$. 
		
		2) If the base $k-1$ is paired with another base $d \in [i,k-2]$,  this will result in the new partial structure: 
		
		$\mathcal{S}' = ([i,d-1]^m.[d,k-1]^b.\delta, \mathcal{P},  \Delta G_\text{bp}^\text{multi} + (j-k + 1) \Delta G_\text{nt}^\text{multi} + E_{L_{\mathcal{S}}})$ such that $E(\mathcal{S}') =   M^m_{i,d-1} +  M^b_{d,k-1} + \Delta G_\text{bp}^\text{multi} + (j-k + 1)\Delta G_\text{nt}^\text{multi} +E_{L_{\mathcal{S}}} 
		+ \sum \limits_{[m,n]^t \in \delta} E([m,n]^t)$. 
		
		
	\end{itemize}
	
	Then the acceptance criteria will be checked after each sub-case. Now, with the aid of the introduced new auxiliary matrices $M^\text{mul:2}$, the backtracking algorithm   checks up to $\mathcal{O}(N)$ refined structures, hence saves up to  $\mathcal{O}(N)$ refined structures to $\mathcal{R}_u$, in the worst case.
	
\end{enumerate}

\begin{remark}
	For any $\mathcal{S}$ and  $\mathcal{S}'$ such that $\mathcal{S}'$ is a refinement of $\mathcal{S}$ based on refinement rules described above, $E(\mathcal{S}) \leq E(\mathcal{S}')$. Also, note that the form of the new refined structure $\mathcal{S}'$ in each case of refinement cases is different, and hence leads to a different fully specified structure which guarantees that each secondary structure $S_u$ encountered during the backtracking is scanned exactly once. 
\end{remark} 

\begin{remark}\label{remark:spaceE}
	For all the cases above where  $\mathcal{S}'$ is refinement of $\mathcal{S}$,  the stack $\delta$ and the base pairs set $\mathcal{P}$ are parts (common) of each refined structure  $\mathcal{S}'$,  hence it is enough to save them once, which  takes $\mathcal{O}(N)$ space, and for each refined structure $\mathcal{S}'$, we need to save only the additional base pairs (one base pair in the worst case, hence $\mathcal{O}(1)$ space), or the new segments (two segments in the worst case, hence $\mathcal{O}(1)$ space) that are pushed on the top of $\delta$ based on the refinement case. In total saving the all $\mathcal{O}(N)$ refined structures that are generated from all cases requires $\mathcal{O}(N)$ space.   
	
	
\end{remark}  

\begin{remark}\label{remark:timeE}
	For all the cases above where  $\mathcal{S}'$ is refinement of $\mathcal{S}$,  $\mathcal{H} = E_{L_{\mathcal{S}}} + \sum_{[m,n]^t \in \delta} E([m,n]^t)$ 
	is repeatedly used to compute $E(\mathcal{S}')$ again and again, hence it is enough to compute it once, which takes only $\mathcal{O}(N)$ time. 
	In total, if $\mathcal{H}$ is pre-computed once in this way, computing $E(\mathcal{S}')$ takes $\mathcal{O}(1)$ time. 
\end{remark} 


After scanning the secondary structure $S_u$ completely, the partially specified structure $\mathcal{S}' \in \mathcal{R}_u$, such that $E(\mathcal{S}') =  \min\limits_{\mathcal{S} \in \mathcal{R}_u}\{ E(\mathcal{S})\}$ will be computed and saved. And a new partially specified structure $\mathcal{S}$ is chosen as follows:

\begin{equation} \label{eq:newS}
	\mathcal{S} = \min\limits_{z \in \left\{1,\ldots,u\right\}}\left\{\min\limits_{\mathcal{S}' \in \mathcal{R}_z}\left\{ E(\mathcal{S}')\right\}\right\} 
\end{equation}

Where $\{S_1,\ldots,S_u\}$ is the set of distinct secondary structures that are completely scanned until the moment, where $u<\mathcal{U}$. Then the minimum ($\min_{\mathcal{S}' \in \mathcal{R}_z}\left\{ E(\mathcal{S}')\right\}$) of the array $\mathcal{R}_{z'}$ where $\mathcal{S}$ is choose from, will be computed again and saved for future iterations. 

\begin{remark}\label{remark:newS}
	\cref{eq:newS} guarantees sequential scanning of the backtracking algorithm through energy levels without skipping any potential structure, due to free energy minimality. Also, note that for all $z \in \{1,\ldots,u\}$, the  $\min_{\mathcal{S}' \in \mathcal{R}_z}\left\{ E(\mathcal{S}')\right\}$ is already computed and saved, as in each iteration we choose only the minimum over the set of all minimum energies of each array, so we lose only one the minimum of some array $\mathcal{R}_{z'}$, so $\mathcal{R}_{z'}$ is the only one we need to compute its minimum again in $\mathcal{O}(N^2)$ time.        
\end{remark} 

The same refinement process starts again with that new selected partially specified structure $\mathcal{S}$. This backtracking procedure continues in this way until one of the three following conditions occurs first: 
\begin{enumerate}
	\item  The algorithm scans an {\em asymmetric} secondary structure $S_u$, then the true MFE $= \Delta G (S_u)$, as a direct consequence of $\Delta G (S_u) \leq \mathcal{B}$, where we recall that $\mathcal{B}$ was the best candidate value for true MFE, or 
	\item  The algorithm exceeds the upper bound $\mathcal{U}$ of the number of symmetric secondary structures to be scanned, then the true MFE $= \mathcal{E}$, the energy of the last scanned energy level which is also the \snMFE of that last completely scanned symmetric secondary structure $S_u$, as a direct consequence of \cref{lem:polyub,lem:sand,lem:sand2} (meaning we have two symmetric structures of  \snMFE equal to $\mathcal{E}$ with the same admissible cut, hence we can make a new pizza: an asymmetric secondary structure of true MFE $\mathcal{E}$), or
	\item  The algorithm starts scanning a new energy level $\mathcal{E}' > \mathcal{B}$, then the true MFE $= \mathcal{B}$, as $\mathcal{B}$ is the best candidate for the true MFE that we got from a previously scanned symmetric secondary structure.   
\end{enumerate}
Whichever of the three cases occurs, the true MFE is returned (and a secondary structure with that true MFE is constructed).

\subsection{Time and space complexity analysis of the backtracking algorithm}\label{sec:backtime}

The backtracking algorithm needs to scan up to $\mathcal{U} = \mathcal{O}(N^2)$ secondary structures in the worst case ({\cref{lem:polyub}}), so the total time complexity of the backtracking algorithms is $\mathcal{O}(\mathcal{U} \mathcal{W})$, where $\mathcal{W}$ is the time complexity of scanning only one secondary structure and setting up the scene for the next iteration by choosing the new partially specified structure required to scan the next secondary structure. 

\paragraph{Analysis for scanning only one secondary structure in the backtracking algorithm.}
To scan (construct) one secondary structure $S_u$, we need in the worst case  $N = \mathcal{O}(N)$ refinement steps, as each step either ignores a base or forms a base pair. We showed in our analysis, in \cref{sec:backhigh}, and based on $t \in \{\square,b,m\}$, that each step checks up to $\mathcal{O}(N)$ refined structures and saves up to $\mathcal{O}(N)$ refined structures to $\mathcal{R}_u$, the array of refined structures associated with $S_u$. In total, by \cref{remark:timeE}, scanning one secondary structure takes $\mathcal{O}(N^2)$, as $\mathcal{R}_u$ contains $\mathcal{O}(N^2)$ structure, hence computing the minimum of $\mathcal{R}_u$ takes $\mathcal{O}(N^2)$ time. 

The last step is to see what is the time complexity of choosing the new partially specified structure $\mathcal{S}$ for the next iteration based on \cref{eq:newS}, from \cref{remark:newS} this step takes $\mathcal{O}(N^2)$ time, as the minimum of all the $\mathcal{U} = \mathcal{O}(N^2)$ arrays is already computed and stored. 

So, in total scanning one secondary structure and setting up the scene for the next iteration by choosing the new partially specified structure $\mathcal{S}$ takes $\mathcal{O}(N^2)$ time.

\cref{remark:spaceE} shows that each array $\mathcal{R}_u$ requires $\mathcal{O}(N^2)$ space, hence in total the backtracking algorithm requires $\mathcal{O}(N^4)$ space to store the all $\mathcal{R}_u$ such that $u \in \{1, \ldots, \mathcal{U}\}$. This analysis leads to the following result.

\begin{lemma}\label{lem:BTtime}
	The running time  of the backtracking algorithm,   \cref{algo:2} and \cref{sec:backhigh},
	for a set of $c = \mathcal{O}(1)$ DNA or RNA strands of total length  $N$ bases, is $\mathcal{O}(N^4(c-1)!)$, and it requires $\mathcal{O}(N^4)$ space.  
\end{lemma}

\begin{remark}\label{remark:newT}
	We should note that, changing the strategy, used in \cref{eq:newS}, for choosing the new partially specified structure $\mathcal{S}$ can help in reducing the space complexity of the backtracking algorithm from $\mathcal{O}(N^4)$ to $\mathcal{O}(N^3)$ (the same space complexity as \snMFE \cref{algo:1}) with the trade off increasing the time complexity to be $\mathcal{O}(N^4 \log N(c-1)!)$ instead of $\mathcal{O}(N^4 (c-1)!)$. As we know that we need to scan only $\mathcal{U} = \mathcal{O}(N^2)$ secondary structures, so we do not need to store all elements of arrays $\mathcal{R}_z$ where $z \in \{1,\ldots,u\}$. 
	Only the minimum $\mathcal{U}$ candidates should be stored. By sorting $\mathcal{R}_u$, the array of partially specified structures that we obtain after constructing the secondary structure $S_u$ (the secondary structure number $u$, of the worst case $\mathcal{U}$). $\mathcal{R}_u$ can be sorted in $\mathcal{O}(N^2 \log N)$ time then merged in $\mathcal{O}(N^2)$ time with the already sorted array, that we obtain cumulatively through time from the previous iterations.        
\end{remark} 

As we noted before in \cref{lem:even}, that the (bad) quadratic upper bound $\mathcal{U}$ in \cref{lem:ub2} is very restricted and rare, and 
for systems of $c$ strands ($k$ strand types) where the repetition number of some strand type is odd, \cref{lem:even} shows that the upper bound $\mathcal{U}$ is linear. Hence, the time complexity of the backtracking algorithm is $\mathcal{O}(N^3(c-1)!)$ and it requires $\mathcal{O}(N^3)$ space.




\section{Time and space analysis of MFE algorithm}\label{sec:analysis}

\main*


\begin{proof}\textcolor{red}{TOPROVE 0}\end{proof}

 
	
	\subsection*{Acknowledgements} 
	We thank Constantine Evans for his helpful comments on the statistical mechanics origin of the MFE rotational symmetry penalty, and Dave Doty for comments on the manuscript. Ahmed Shalaby would like to thank Dvořák for composing his masterpiece, Symphony No.~9 ``From the New World''. 
	
	\bibliographystyle{plainurl}   
	\bibliography{bib}
	
	\newpage
	\appendix
	% !TEX root = mfe.tex
\section{Appendix: Useful lemmas}\label{sec:lemmasApp}
\begin{lemma}[factors of $\pi$]\label{lem:factors}
	For any nonempty circular permutation $\pi$ 
	and any prefix $y$ of $\pi$ that is not its fundamental component $x$, 
	such that $|y|>|x|$, 
	if $\pi = y^n$ then $|x|$ divides $|y|$.    
\end{lemma}
\begin{proof}\textcolor{red}{TOPROVE 0}\end{proof}


The following lemma restricts us to deal with only specific and constant number of different folding rotational symmetries, and hence constant different symmetry corrections in total. 
\begin{lemma}\label{lem:div}
	If $S$ is $R$-fold rotational symmetric secondary structure, with a specific circular permutation $\pi$, then $R$ must be a divisor of $v(\pi)$.
\end{lemma}

\begin{proof}\textcolor{red}{TOPROVE 1}\end{proof}



\begin{lemma} \label{lem:nobase}
	For any connected unpseudoknotted secondary structure $S$, if there exists at least one base pair $(i,j)$ such that $\llbracket i,j \rrbracket > \frac{N}{R}$, then $S$ can not be  $R$-fold rotationally symmetric. 
\end{lemma}
\begin{proof}\textcolor{red}{TOPROVE 2}\end{proof}



 \newpage
	% !TEX root = mfe.tex


\section{Appendix: \SymnMFE  (\snMFE) algorithm}\label{sec:AlgoMFE}

\cref{algo:1}, shown in \cref{fig:mfe}, computes \snMFE  for a constant number, $c=\mathcal{O}(1)$, of interacting nucleic acid strands. 
We should note that: \cref{algo:1} is a straightforward conversion of the partition function algorithm from Dirks et al. \cite{dirks2007thermodynamic}. 
\cref{algo:1} ignores rotational symmetry, 
if the predicted \snMFE structure from this algorithm happens to be asymmetric, 
then the output of \cref{algo:1} is the true MFE as there is no symmetry correction penalty for asymmetric secondary structures.
However, if the \snMFE structure is an $R$-fold symmetric secondary structure, then its free energy must be corrected by by $+k_\mathrm{B} T \log R$, a positive value, then it is not guaranteed that the \snMFE will be the true MFE without scanning all secondary structures in the window of $k_\mathrm{B} T \log R$ above \snMFE, and applying any needed symmetry corrections to the free energy of each secondary structure that lies in that window. Wuchty et al. \cite{wuchty1999complete} showed that this window of secondary structures could scale exponentially with $N$, which shows why this strategy fails. But in this work, we proved that only a polynomial number of these structures are enough for predicting the true MFE. 

We introduced new three-dimensional matrices $M^\text{b:int}, M^\text{b:mul}$,  $M^\text{m:2}$ to help in reducing the time complexity of the backtracking algorithm. For any segment $[i,j]^b$, if $i+2 \leq k \leq j-1$, $M_{i,j,k}^\text{b:int}$ will contain the minimum free energy attainable from the segment $[i,j]^b$ such that all bases $d$, such that $k<d<j$ are unpaired, and there exist exactly one base pair $(m,n)$, such that $i+1 \leq m < n \leq k$, $(i,j)$ and $(m,n)$ are forming together an internal loop. The same for $M_{i,j,k}^\text{b:mul}$, except that there exist more than one base pair $(m,n)$, such that $i+1 \leq m < n \leq k$, forming together a multiloop with $(i,j)$.  


For any segment $[i,j]^m$, if $i+1 \leq k \leq j$, $M_{i,j,k}^\text{m:2}$ contains the minimum free energy attainable from the segment $[i,j]^m$ such that all bases $d$, such that $k<d \leq j$ are unpaired, and there exist more than one base pair $(m,n)$, such that $i \leq m < n \leq k$, forming together a multiloop with $(i,j)$.    





\begin{figure}[H]
	\centering\includegraphics[width=1\textwidth]{figures/dirkis.jpg}
	\caption{\snMFE dynamic program recursion diagrams (left) and recursion equations (right). A solid straight line indicates a base pair and
		a dashed line demarcates a region without implying that the connected bases are paired. Shaded regions correspond to loop free energies that are	explicitly incorporated at the current level of recursion. See \cite{dirks2003partition, fornace2020unified} for  full details.  
	}\label{fig:mfe}
\end{figure}



\begin{algorithm}[H] 
	%\newgeometry{top=0.2in,bottom=0.1in,left=1in,right=1in}
	\caption{\small\SymnMFE (\snMFE) algorithm pseudocode  that takes as input: $c=\mathcal{O}(1)$ strands with total number of bases (length) $N$ and strand ordering $\pi$. 
		Runs in  $\mathcal{O}(N^4)$ time and $\mathcal{O}(N^3)$ space with
		recursive calls illustrated in \cref{fig:mfe}. Nicks between strands are denoted by half indices (e.g.~$x+ \frac{1}{2}$). 
		The function $\eta[i+ \frac{1}{2}, j+\frac{1}{2}]$ returns the number of nicks in the interval $[i+ \frac{1}{2}, j+\frac{1}{2}]$. 
		The shorthand $\eta[i+ \frac{1}{2}]$ is equivalent to $\eta[i+ \frac{1}{2}, i+\frac{1}{2}]$ and by convention, $\eta[i+ \frac{1}{2}, i-\frac{1}{2}] =0$.
	} \label{algo:1}
	\begin{algorithmic}[1]
		\footnotesize
		\State Initialize $M, M^b, M^m, M^\text{b:int}, M^\text{b:mul}$,  $M^\text{m:2}$  by setting all values to $+\infty$, except $M_{i,i-1} = 0$ for all $i=1,\ldots,N$
		
		\For{$l \gets 1 \ldots N$}
		\For{$i \gets 1 \ldots N-l+1$}
		\State $j = i+l-1$
		
		\Comment{$M^b$ recursion equations} 
		\If{$\eta[i+\frac{1}{2}, j-\frac{1}{2}] ==0$}
		$M_{i,j}^b =\Delta G_{i,j}^\text{hairpin}$
		\Comment{hairpin loop requires no nicks} 		
		
		
		
		\EndIf 
		
		\State $\text{min}_\text{int}^b =  \text{min}_\text{mul}^b = \text{min}_\text{mul}^m = +\infty$
		
		
		
		\For{$e \gets i+2 \ldots j-1$}
		\Comment{loop over all possible $3'$-most pairs $(d,e)$} 
		
		\For{$d \gets i+1 \ldots e-1$} 
		
		\If{$\eta[i+\frac{1}{2}, d-\frac{1}{2}] ==0$ and $\eta[e+\frac{1}{2}, j-\frac{1}{2}] ==0$}
		$M_{i,j}^b = \min \{ M_{i,j}^b, M_{d,e}^b + \Delta G_{i,d,e,j}^\text{interior} \}$
		
		\If{$(M_{d,e}^b + \Delta G_{i,d,e,j}^\text{interior}) < \text{min}_\text{int}^b $}
		$\text{min}_\text{int}^b = M_{d,e}^b + \Delta G_{i,d,e,j}^\text{interior} $
		
		\EndIf
		
		
		\EndIf 
		
		\If{$\eta[e+\frac{1}{2}, j-\frac{1}{2}] ==0$ and $\eta[i+\frac{1}{2}] ==0$ and $\eta[d-\frac{1}{2}] ==0$}
		\Comment{multiloop: no nicks} 
		
		\State  $M_{i,j}^b = \min \{ M_{i,j}^b, M^b_{d,e} + M^m_{i+1,d-1} + \Delta G_\text{init}^\text{multi} + 2\Delta G_\text{bp}^\text{multi} + (j-e-1)\Delta G_\text{nt}^\text{multi}\}$
		
		\If{$(  M^m_{i+1,d-1} +  M^b_{d,e} + \Delta G_\text{init}^\text{multi} + 2\Delta G_\text{bp}^\text{multi} + (j-e-1)\Delta G_\text{nt}^\text{multi}) < \text{min}_\text{mul}^b$}
		
		\State $\text{min}_\text{mul}^b =   M^m_{i+1,d-1} + M^b_{d,e} + \Delta G_\text{init}^\text{multi} + 2\Delta G_\text{bp}^\text{multi} + (j-e-1)\Delta G_\text{nt}^\text{multi} $
		
		\EndIf
		
		\EndIf
		\EndFor
		
		\State $M_{i,j,e}^\text{b:int} = \text{min}_\text{int}^b$; \ \  
		$M_{i,j,e}^\text{b:mul} = \text{min}_\text{mul}^b$ \Comment{for the new auxiliary matrices} 
		
		
		\EndFor
		
		
		
		\For{$x \in \{i, \ldots,j+1\}$ s.t. $\eta[x+\frac{1}{2}] = 1$} 
		\Comment{loop over all nicks $\in [i+\frac{1}{2}, j-\frac{1}{2}]$} 
		
		\If{($\eta[i+\frac{1}{2}] == 0$ and $\eta[j-\frac{1}{2}] == 0$) or ($i==j-1$) or  ($x==i$ and $\eta[j-\frac{1}{2}] == 0$) or
			\par \hskip\algorithmicindent  
			\hspace{6 mm} ($x==j-1$ and $\eta[i+\frac{1}{2}] == 0$)}
		
		\State  $M_{i,j}^b = \min \{ M_{i,j}^b, M_{i+1,x} + M_{x+1,j-1} \}$\Comment{exterior loops} 
		
		\EndIf
		\EndFor
		
		
		\\	
		\Comment{$M, M^m$ recursion equations} 
		
		\If{$\eta[i+\frac{1}{2}, j-\frac{1}{2}] == 0$} 
		$M_{i,j} = 0$\Comment{empty substructure} 
		
		\EndIf
		
		\For{$e \gets i+1 \ldots j$} \Comment{loop over all possible $3'$-most pairs $(d,e)$}
		
		\For{$d \gets i \ldots e-1$}
		
		\If{$\eta[e+\frac{1}{2}, j-\frac{1}{2}] == 0$} 
		
		\If{$\eta[d-\frac{1}{2}] == 0$ or $d==i$}
		$M_{i,j} = \min \{M_{i,j}, M_{i,d-1} + M_{d,e} \}$
		\EndIf
		
		\If{$\eta[i+\frac{1}{2}, d-\frac{1}{2}] == 0$}
		
		\State $M_{i,j}^m = \min \{M_{i,j}^m, M^b_{d,e} + \Delta G_\text{bp}^\text{multi} + (d-i + j-e)\Delta G_\text{nt}^\text{multi}\}$
		\Comment{single base pair}
		
		\EndIf
		\If{$\eta[d-\frac{1}{2}] == 0$}
		\State $M_{i,j}^m = \min \{M_{i,j}^m, M^b_{d,e} + M^m_{i,d-1} + \Delta G_\text{bp}^\text{multi} + (j-e)\Delta G_\text{nt}^\text{multi}\}$
		\Comment{more than one base pair}
		
		\If{$( M^m_{i,d-1} +  M^b_{d,e} +  \Delta G_\text{bp}^\text{multi} + (j-e)\Delta G_\text{nt}^\text{multi}) < \text{min}_\text{mul}^m $}
		
		\State $\text{min}_\text{mul}^m  =   M^m_{i,d-1} + M^b_{d,e} + \Delta G_\text{bp}^\text{multi} + (j-e)\Delta G_\text{nt}^\text{multi}$
		
		\EndIf
		
		\EndIf
		
		\EndIf
		
		\EndFor
		\State $M_{i,j,e}^\text{m:2} = \text{min}_\text{mul}^m $
		\EndFor
		
		\EndFor
		\EndFor \Comment{next line returns the \snMFE for ordering $\pi$, and several matrices for future backtracking} 
		\State \Return $M_{1,N} + (c-1) \Delta G^\text{assoc}$; 
		and matrices: $M, M^b, M^m, M^\text{b:int}, M^\text{b:mul}$,  $M^\text{m:2}$
		
		
		
	\end{algorithmic}
\end{algorithm}







 
	% !TEX root = mfe.tex

\section{Appendix: Backtracking algorithm to find the true MFE }\label{app:backalgo}

In this backtracking algorithm, we use  $[i,j]_{q:k}^t $ to denote the generic form of segment that has been popped from the stack $\delta$, where $t \in \{\square,b,m\}$, and $q \in \{\text{mul}, \text{int},  \text{null}\}$, where null means (nothing), and $k \in [i,j]$. For the full details, see \cref{sec:backhigh}. 


\begin{algorithm}[t] 
	\caption{\small Backtracking pseudocode that takes as input: $c=\mathcal{O}(1)$ strands with total number of bases (length) $N$ and strand ordering $\pi$. 
		Runs in  $\mathcal{O}(N^4)$ time and $\mathcal{O}(N^4)$ space, and assumes there are $k \leq c$ strand types given as a \emph{multiset}, 
		each with an associated repetition number $n_1, ..., n_k \in \mathbb{N}$, such that $n_1+ ...+n_k = c$, with total length $N$. 
		$[i,j]^t \Leftarrow \delta$ denotes  popping an element from stack $\delta$, 
		which is a segment, and assigning it to the generic segment $[i,j]^t$.  
		And $\mathcal{S} \Rightarrow \mathcal{R}$ denotes pushing structure $\mathcal{S}$ onto stack $\mathcal{R}$, and  $E(\mathcal{S})$ is defined in \cref{eq:ES}, and all refinement cases are analyzed in \cref{sec:backhigh}.
	} \label{algo:2}
	\begin{algorithmic}[1]
		\footnotesize	
		\If{($n_1, n_2, \ldots, n_k$) are all even}\Comment{use \cref{lem:even} to set symmetric structure upperbound $\mathcal{U}$}
		\State $\mathcal{U} =  \frac{N-c}{v(\pi)} \left[ \sigma(v(\pi))-v(\pi) \right] + \frac{N^2}{16}$  \Comment{in $\mathcal{O}(1)$ time}
		
		\Else
		
		\State $\mathcal{U} =  \frac{N-c}{v(\pi)} \left[ \sigma(v(\pi))-v(\pi) \right]$
		
		\EndIf
		
		\State $\mathcal{E} =$ \snMFE \Comment{\snMFE is returned by \cref{algo:1}}
		
		\State $\mathcal{B} = \mathcal{E}+ k_\mathrm{B} T \log v(\pi)$ \Comment{where $v(\pi)$ is the highest symmetry degree for the $c$-strands ordering $\pi$}
		
		
		
		\State $\mathcal{S} = ([1,N]^\square, \phi,0)$ \Comment{the initial system (the parent of any possible structure)}
		
		\State $(\delta, \mathcal{P}, E_{L_{\mathcal{S}}}) = \mathcal{S}$
		
		\State $u = 1$ \Comment{$u$ is a symmetric secondary structures counter}
		
		\While{($u \leq \mathcal{U}$)}
		\State  $y= \mathrm{False}$; \ \ $w= \mathrm{False}$  \Comment{$y$ and $w$ are indicator variables}
		
		\State $[i,j]_{q:k}^t \Leftarrow \delta$
		
		\State $\mathcal{H} = E_{L_{\mathcal{S}}} + \sum \limits_{[m,n]^t \in \delta} E([m,n]^t)$\Comment{in $\mathcal{O}(N)$, \cref{remark:timeE}}\label{line:H}
		
		\State $\mathcal{S} = (\delta, \mathcal{P}, E_{L_{\mathcal{S}}})$ \Comment{if all cases were not satisfied (just pop  $[i,j]_{q:k}^t$)}
		
		
		\If{$(t == \square)$} \Comment{backtrack in matrix $M$}
		
		\State $\mathcal{S}' = ([i,j-1]^\square.\delta, \mathcal{P}, E_{L_{\mathcal{S}}})$\Comment{base $j$ is not paired}
		
		\State $E(\mathcal{S}')  = M_{i,j-1}  + \mathcal{H}$ 
		
		
		\If{($E(\mathcal{S}') \leq \mathcal{B}$)}	
		
		\If{($y== \mathrm{False}$ and $E(\mathcal{S}') == \mathcal{E}$)}
		
		\State $\mathcal{S} = \mathcal{S}'$; \ \  $y = \mathrm{True}$
		
		\Else
		
		\State $\mathcal{S}' \Rightarrow \mathcal{R}_u$		
		\EndIf
		
		
		
		\EndIf
		
		\For{$d \in [i,j-1]$}\Comment{base $j$ is paired with some base $d$}
		
		
		\State $\mathcal{S}' = ([i,d-1]^\square.[d,j]^b.\delta, \mathcal{P}, E_{L_{\mathcal{S}}})$
		
		\State $E(\mathcal{S}')  = M_{i,d-1} + M^b_{d,j}   + \mathcal{H} $
		
		\If{($ M_{i,d-1} + M^b_{d,j}   + \mathcal{H} \leq \mathcal{B}$)}
		
		\If{($y== F$ and $E(\mathcal{S}') == \mathcal{E}$)}
		
		\State $\mathcal{S} = \mathcal{S}'$; \ \  $y = T$
		
		\Else
		
		\State $\mathcal{S}' \Rightarrow \mathcal{R}_u$		
		
		\EndIf
		\EndIf
		\EndFor
		
		\ElsIf{$(t == b)$}\Comment{backtrack in matrix $M^b$}
		
		\If{($q == \text{null}$)}\Comment{hairpin loop formation}
		
		\State$\mathcal{S}' = (\delta, \mathcal{P} \cup \{(i,j)\}, E_{L_{\mathcal{S}}} + \Delta G_{i,j}^\text{hairpin} )$ 
		
		\State $E(\mathcal{S}')  = \Delta G_{i,j}^\text{hairpin} + \mathcal{H} $
		
		
		
		
		\If{($E(\mathcal{S}') \leq \mathcal{B}$)}	
		
		\If{($y== F$ and $E(\mathcal{S}') == \mathcal{E}$)}
		
		\State $\mathcal{S} = \mathcal{S}'$; \ \  $y = T$
		
		\Else
		
		\State $\mathcal{S}' \Rightarrow \mathcal{R}_u$		
		\EndIf
		
		
		\EndIf
		\EndIf
		
		
		
		\algstore{backtracking1}		
		
	\end{algorithmic}
\end{algorithm}

\begin{algorithm}
	\caption*{Part 2 of backtracking algorithm}
	\begin{algorithmic}[1]
		\algrestore{backtracking1}
		\footnotesize
		
		\If{($q == \text{null}$)} \Comment{interpreting the segment to be in a valid form for internal loop case \cref{sec:backhigh}.}
		
		\State $q = \text{int}$; \ \ $k = j$, \ \ $w= \mathrm{True}$
		\EndIf	
		
		\If{($q == \text{int}$)}
		
		\State$\mathcal{S}' = ([i,j]^b_{\text{int}:k-1}.\delta, \mathcal{P}, E_{L_{\mathcal{S}}})$ \Comment{internal loop formation: base $k-1$ is unpaired}
		
		\State $E(\mathcal{S}')  = M_{i,j,k-2}^\text{b:int} + \mathcal{H}$
		
		\If{($E(\mathcal{S}') \leq \mathcal{B}$)}	
		
		\If{($y== F$ and $E(\mathcal{S}') == \mathcal{E}$)}
		
		\State $\mathcal{S} = \mathcal{S}'$; \ \  $y = T$
		
		\Else \ \  $\mathcal{S}' \Rightarrow \mathcal{R}_u$		
		
		\EndIf
		\EndIf
		\EndIf	
		
		
		\If{($w == \mathrm{True}$)}
		\State $q = \text{null}$; \ \ $k = \text{null}$, \ \ $w= \mathrm{False}$ 
		\EndIf	
		
		
		\If{($q == \text{null}$)}\Comment{internal loop formation: base $k-1$ is paired}
		
		\For{$d \in [i+1,k-2]$}
		
		
		\State  $\mathcal{S}' = ([d,k-1]^b.\delta, \mathcal{P} \cup \{(i,j)\}, E_{L_{\mathcal{S}}} + \Delta G_{i,d,k-1,j}^\text{interior})$
		
		\State $E(\mathcal{S}')  = M^b_{d,k-1} + \Delta G_{i,d,k-1,j}^\text{interior} + \mathcal{H}$
		
		
		\If{($E(\mathcal{S}') \leq \mathcal{B}$)}	
		
		\If{($y== F$ and $E(\mathcal{S}') == \mathcal{E}$)}
		
		\State $\mathcal{S} = \mathcal{S}'$; \ \  $y = T$
		
		\Else  \ \  $\mathcal{S}' \Rightarrow \mathcal{R}_u$		
		
		
		\EndIf
		
		
		
		\EndIf
		\EndFor
		\EndIf
		
		\If{($q == \text{null}$)}\Comment{interpreting the segment to be in a valid form for multiloop case \cref{sec:backhigh}}
		
		\State $q = \text{mul}$; \ \ $k = j$, \ \ $w= \mathrm{True}$
		\EndIf	
		
		\If{($q == \text{mul}$)}
		
		\State $\mathcal{S}' = ([i,j]^b_{\text{mul}:k-1}.\delta, \mathcal{P}, E_{L_{\mathcal{S}}})$ \Comment{multiloop formation: base $k-1$ is unpaired}
		
		\State $E(\mathcal{S}')  = M_{i,j,k-2}^\text{b:mul} + \mathcal{H}$
		
		
		\If{($E(\mathcal{S}') \leq \mathcal{B}$)}	
		
		\If{($y== F$ and $E(\mathcal{S}') == \mathcal{E}$)}
		
		\State $\mathcal{S} = \mathcal{S}'$; \ \  $y = T$
		
		\Else
		
		\State $\mathcal{S}' \Rightarrow \mathcal{R}_u$		
		\EndIf
		
		\EndIf
		
		\EndIf
		\If{($w == \mathrm{True}$)}
		\State $q = \text{null}$; \ \ $k = \text{null}$, \ \ $w= \mathrm{False}$ 
		\EndIf	
		
		\If{($q == \text{null}$)}\Comment{multiloop formation: base $k-1$ is paired}
		
		\For{$d \in [i+1,k-2]$}
		
		
		
		\State  $\mathcal{S}' = ([i+1,d-1]^m.[d,k-1]^b.\delta, \mathcal{P} \cup \{(i,j)\}, \Delta G_\text{init}^\text{multi} + 2\Delta G_\text{bp}^\text{multi} + (j-k) \Delta G_\text{nt}^\text{multi} + E_{L_{\mathcal{S}}})$
		
		\State $E(\mathcal{S}')  = M^m_{i+1,d-1} + M^b_{d,k-1} + \Delta G_\text{init}^\text{multi} + 2\Delta G_\text{bp}^\text{multi} + (j-k)\Delta G_\text{nt}^\text{multi} + \mathcal{H}$
		
		
		\If{($ E(\mathcal{S}') \leq \mathcal{B}$)}	
		
		\If{($y== F$ and $E(\mathcal{S}') == \mathcal{E}$)}
		
		\State $\mathcal{S} = \mathcal{S}'$; \ \  $y = T$
		
		\Else
		
		\State $\mathcal{S}' \Rightarrow \mathcal{R}_u$		
		\EndIf
		
		
		\EndIf
		\EndFor
		\EndIf
		
		
		
		
		\algstore{backtracking2}
		
	\end{algorithmic}
\end{algorithm}


\begin{algorithm}
	\caption*{Part 3 of backtracking algorithm}
	\begin{algorithmic}[1]
		\algrestore{backtracking2}
		\footnotesize
		
		\If{($q == \text{null}$)}\Comment{exterior loop formation}
		
		\For{$z \in [i,j]$ s.t. $\eta[z+\frac{1}{2}] ==1$}
		
		
		\State  $\mathcal{S}' = ([i+1,z]^\square.[z+1,j-1]^\square.\delta, \mathcal{P} \cup \{(i,j)\}, E_{L_{\mathcal{S}}})$
		
		\State $E(\mathcal{S}')  = M_{i+1,z} +  M_{z+1,j-1} + \mathcal{H}$
		
		
		
		\If{($E(\mathcal{S}') \leq \mathcal{B}$)}	
		
		\If{($y== F$ and $E(\mathcal{S}') == \mathcal{E}$)}
		
		\State $\mathcal{S} = \mathcal{S}'$; \ \  $y = T$
		
		\Else
		
		\State $\mathcal{S}' \Rightarrow \mathcal{R}_u$		
		\EndIf
		\EndIf
		\EndFor
		
		\EndIf
		
		
		
		\ElsIf{$(t == m)$}\Comment{backtrack in matrix $M^m$}
		
		\If{($q == \text{null}$)}\Comment{multiloop case 1: base $j$ is unpaired}
		
		\State $\mathcal{S}' = ([i,j-1]^m.\delta, \mathcal{P}, E_{L_{\mathcal{S}}} + \Delta G_\text{nt}^\text{multi})$ 
		
		\State $E(\mathcal{S}')  = M_{i,j-1}^m + \Delta G_\text{nt}^\text{multi} + \mathcal{H}$
		
		
		
		\If{($E(\mathcal{S}')  \leq \mathcal{B}$)}	
		
		\If{($y== F$ and $E(\mathcal{S}') == \mathcal{E}$)}
		
		\State $\mathcal{S} = \mathcal{S}'$; \ \  $y = T$
		
		\Else
		
		\State $\mathcal{S}' \Rightarrow \mathcal{R}_u$		
		\EndIf
		
		\EndIf
		
		\EndIf
		\If{($q == \text{null}$)}\Comment{base $j$ is paired}
		
		\For{$d \in [i,j-1]$}
		
		
		\State $\mathcal{S}' = ([d,j]^b.\delta, \mathcal{P}, E_{L_{\mathcal{S}}} + \Delta G_\text{bp}^\text{multi} + (d-i) \Delta G_\text{nt}^\text{multi})$ 
		
		\State $E(\mathcal{S}') = M^b_{d,j} + \Delta G_\text{bp}^\text{multi}+ (d-i) \Delta G_\text{nt}^\text{multi} + \mathcal{H}$
		
		
		\If{($E(\mathcal{S}')  \leq \mathcal{B}$)}	
		
		\If{($y== F$ and $E(\mathcal{S}') == \mathcal{E}$)}
		
		\State $\mathcal{S} = \mathcal{S}'$; \ \  $y = T$
		
		\Else
		
		\State $\mathcal{S}' \Rightarrow \mathcal{R}_u$		
		\EndIf
		
		
		
		\EndIf
		
		
		\EndFor
		\EndIf
		
		\If{($q == \text{null}$)}\Comment{interpreting the segment to be in a valid form for multiloop case \cref{sec:backhigh}}
		
		\State $q = \text{mul}$; \ \ $k = j-1$, \ \ $w= \mathrm{True}$
		\EndIf	
		
		\If{($q == \text{mul}$)}\Comment{multiloop case 2: base $k$ is unpaired}
		
		\State  $\mathcal{S}' = ([i,j]^m_{\text{mul}:k-1}.\delta, \mathcal{P}, E_{L_{\mathcal{S}}})$ 
		
		\State $E(\mathcal{S}') = M_{i,j,k-2}^\text{m:2} + \mathcal{H}$
		
		\If{($E(\mathcal{S}') \leq \mathcal{B}$)}	
		
		\If{($y== F$ and $E(\mathcal{S}') == \mathcal{E}$)}
		
		\State $\mathcal{S} = \mathcal{S}'$; \ \  $y = T$
		
		\Else
		
		\State $\mathcal{S}' \Rightarrow \mathcal{R}_u$		
		\EndIf
		
		\EndIf
		
		\EndIf
		
		\If{($w == \mathrm{True}$)}
		\State $q = \text{null}$; \ \ $k = \text{null}$, \ \ $w= \mathrm{False}$ 
		\EndIf	
		
		
		
		\algstore{backtracking3}
		
	\end{algorithmic}
\end{algorithm}




\begin{algorithm}
	\caption*{Part 4 of backtracking algorithm}
	\begin{algorithmic}[1]
		\algrestore{backtracking3}
		\footnotesize
		
		\If{($q == \text{null}$)}\Comment{base $k-1$ is paired}
		
		\For{$d \in [i,k-2]$}
		
		\State $\mathcal{S}' = ([i,d-1]^m.[d,k-1]^b.\delta, \mathcal{P},  \Delta G_\text{bp}^\text{multi} + (j-k+1) \Delta G_\text{nt}^\text{multi} + E_{L_{\mathcal{S}}})$ 
		
		
		\State $E(\mathcal{S}') = M^m_{i,d-1} +  M^b_{d,k-1} + \Delta G_\text{bp}^\text{multi} + (j-k + 1)\Delta G_\text{nt}^\text{multi} + \mathcal{H} $
		
		\If{($ E(\mathcal{S}') \leq \mathcal{B}$)}	
		
		\If{($y== F$ and $E(\mathcal{S}') == \mathcal{E}$)}
		
		\State $\mathcal{S} = \mathcal{S}'$; \ \  $y = T$
		
		\Else
		
		\State $\mathcal{S}' \Rightarrow \mathcal{R}_u$		
		\EndIf
		
		
		
		\EndIf
		\EndFor
		\EndIf
		\EndIf
		
		\State $(\delta, \mathcal{P}, E_{L_{\mathcal{S}}}) = \mathcal{S}$
		
		\If{$(\delta == \phi)$}\Comment{ scanning $\mathcal{S}$ is done, $\mathcal{S}$ is a fully specified structure}
		
		\State Output secondary structure $\mathcal{S}$ using its base pairs set $\mathcal{P}$
		
		\If{($\mathcal{S}$ is asymmetric)}
		\State  MFE $= \Delta G(\mathcal{S})$  \Comment{$\Delta G(\mathcal{S})$ is defined in \cref{eq:DGss}, which also equals $\mathcal{E}$ at the moment} 
		\Break  \Comment{break out of top-level while loop}
		\Else
		\State Apply rot.~symmetry correction to free energy of $\mathcal{S}$ (i.e.~$\mathcal{E}' := \mathcal{E} + k_\mathrm{B} T \log R$);  
		if $\mathcal{E}'<\mathcal{B}$ then $\mathcal{B} := \mathcal{E}'$ 
		
		
		
		\State  $\mathcal{S} = \min\limits_{z \in \left\{1,\ldots,u\right\}}\left\{\min\limits_{\mathcal{S}' \in \mathcal{R}_z}\left\{ E(\mathcal{S}')\right\}\right\}$  \Comment{\cref{eq:newS}, to ensure the sequential scanning over energy levels} 
		
		\State	$u = u+1$   \Comment{increment symmetric secondary structures counter} 
		
		
		
		\State $(\delta, \mathcal{P}, E_{L_{\mathcal{S}}}) = \mathcal{S}$
		
		\If{($E(\mathcal{S}) > \mathcal{B}$)}\Comment{compare with the updated $\mathcal{B}$}
		
		
		\State    MFE $=  \mathcal{B}$ 
		\Break  \Comment{break out of top-level while loop}
		
		\Else
		\State $\mathcal{E} = E(\mathcal{S})$
		
		\State    MFE $=  \mathcal{E}$ \Comment{in case the upper bound $\mathcal{U}$ is exceeded}
		
		\EndIf		
		
		\EndIf	
		\EndIf
		
		\EndWhile
		
		
		
		
		\State \Return MFE   \Comment{return the true MFE} 
		
		
	\end{algorithmic}
\end{algorithm}


	
\end{document}  