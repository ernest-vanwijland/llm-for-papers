% !TEX root = mfe.tex
\begin{abstract} \baselineskip=11pt 
%	We give the first polynomial time algorithm for predicting optimal multistranded nucleic acid structures, answering an open problem from Dirks et al [SICOMP Review; 2007].
	The information-encoding molecules RNA and DNA form a combinatorially large set of secondary structures through nucleic acid base pairing.  Thermodynamic prediction algorithms predict favoured, or minimum free energy (MFE), secondary structures, and can even assign an equilibrium probability to any particular structure via the partition function---a Boltzmann-weighted sum of the free energies of the exponentially large set of secondary structures.  Prediction is NP-hard in the presence pseudoknots---base pairings that violate a restricted planarity condition.  However, unpseudoknotted structures are amenable to dynamic programming-style problem decomposition:  for a single DNA/RNA strand there are polynomial time algorithms for MFE and partition function.  For multiple strands, the problem is significantly more complicated due to extra entropic penalties.  Dirks et al [SICOMP Review; 2007] showed that for multiple ($\mathcal{O}(1)$) strands, with $N$ bases, there is a polynomial time in $N$ partition function algorithm, however their technique did not generalise to  MFE which they left open. 
	
	We give the first polynomial time ($\mathcal{O}(N^4)$) algorithm   for unpseudoknotted multiple ($\mathcal{O}(1)$) strand MFE, answering the open problem from Dirks et al.  The challenge in computing MFE lies in considering the rotational symmetry of secondary structures, a global feature not immediately amenable to dynamic programming algorithms that rely on local subproblem decomposition.  Our proof has two main technical contributions:  First, a polynomial upper bound on the number of symmetric secondary structures that need to be considered when computing the rotational symmetry penalty. Second, that bound is leveraged by a backtracking algorithm to find the true MFE in an exponential space of contenders. 
	
	Our MFE algorithm has the same asymptotic run time as Dirks et al's partition function algorithm,  suggesting a reasonably efficient handling of the global problem of rotational symmetry, although ours has higher space complexity.  Finally, our algorithm also seems reasonably tight in terms of number of strands since Codon, Hajiaghayi and Thachuk [DNA27, 2021] have shown  that unpseudoknotted MFE is NP-hard for $\mathcal{O}(N)$ strands. 
\end{abstract}