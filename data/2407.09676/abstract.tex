% !TEX root = mfe.tex
\begin{abstract} \baselineskip=11pt 
%We answer an open problem from Dirks et al [SICOMP Review; 2007] on predicting optimal nucleic acid structures.  
Information-encoding molecules such as RNA and DNA form a combinatorially large set of secondary structures through nucleic acid base pairing. 
Thermodynamic prediction algorithms % embedded in tools like NUPACK and ViennaRNA used by biologists, bioengineers and molecular programmers to predict structure and interactions of DNA and RNA strands. 
%These algorithms can predict 
%  are implemented by popular tools, such as 
% to predict 
predict favoured, or minimum free energy (MFE), secondary structures, 
and can even assign an equilibrium probability to any particular structure via the partition function---a Boltzman-weighted sum of the free energies of the exponentially large set of secondary structures. % As such, they also underlie more complex kinetic simulation tools. 
Prediction is NP-hard in the presence pseudoknots---base pairings that violate a restricted planarity condition. 
However, unpseudoknotted structures   
% Algorithmically, {\em unpseudoknotted} secondary structures 
are amenable to dynamic programming-style problem decomposition:   
%  decomposing problem structure in a way amenable to dynamic programming, 
% yielding beautiful connections between biology's information carrier and efficient algorithms.  
% In particular, for a single DNA/RNA strand, there are 
% For example,  
for a single DNA/RNA strand there are polynomial time algorithms 
for MFE 
% (MFE: what is the optimal unpseudoknotted structure?) 
and partition function. 
For multiple strands, the problem is significantly more complicated due to extra entropic penalties. 
% (a Boltzman-weighted sum of all structures' free energies used to assign a probabilty to any unpseudoknotted secondary structure). 
Dirks et al [SICOMP Review; 2007] showed that for multiple ($O(1)$) strands, with $N$ bases, there is a polynomial time in $N$ partition function algorithm, however their technique did not generalise to  MFE which they left open. 
% \dw{OR: In a ground-breaking result, Dirks et al introduced a clean formalism to handle multiple strands, and proceeded to give a polynomial time in $N$ partition function algorithm  for $O(1)$ strands, with $N$ bases. However, their technique did not generalise to  MFE which they left  open. }
%\dw{Dirks et al show that this entropic penalty for symmetry is .... handled ... by their PF algo.}\dwm{not sure if I want to get into this}  
%\dw{The special case of a single strand type has been answered by XXX et al, [pub venue, year] but the general problem remained open.}\dwm{delete? it's good to who folks have tried and failed}

We give a $O(N^4)$ time algorithm for unpseudoknotted multiple ($O(1)$) strand MFE, answering the open problem from Dirks et al. 
% which if ignored, as previous algorithms do, leads to an overcounting of symmetric secondary structures, or equivalently, an undercounding of asymmetric structures, which in turn results in assignment of an incorrect MFE value. 
The challenge in computing MFE lies in considering the rotational symmetry of secondary structures, a global feature not immediately amenable to dynamic programming algorithms that rely on local subproblem decomposition. 
Our proof has two main technical contributions: 
First, a polynomial upper bound on the number of symmetric secondary structures that need to be considered when computing the rotational symmetry penalty. 
% known \SymnMFE algorithms' output. 
Second, that bound is then leveraged by a backtracking algorithm to find the true MFE in an exponential space of contenders. 
%  builds on those previous (inaccurate) \SymnMFE algorithms    
% backtrack before having the correct optimal MFE value, this technique maybe helpful for similar optimization problems, that are well suited for dynamic programming paradigm, but at the same time, depends crucially on some global properties of the full problem instances themselves. 

Our MFE algorithm has the same asymptotic run time as Dirks et al's partition function algorithm,   %(which can safely ignore symmetry corrections for the case of partition function), 
suggesting a reasonably efficient handling of the global problem of rotational symmetry, although ours has higher space complexity. 
Finally, our algorithm also seems reasonably tight in terms of number of strands since
% {it scales to polylog($N$) strands, yet} 
Codon, Hajiaghayi and Thachuk [DNA27, 2021] have shown  that unpseudoknotted MFE is NP-hard for  $O(N)$ strands. 

%The cited algorithms form the basis of popular software packages such as NUPACK and ViennaRNA, .... ours should too. 
% , there are beuatiful links between ideas in dynamic programming and thermodynamic  a polynomial time algorithm [Dirks et al;   
%
%We give an algorithm that runs in time polynomial in $N$ for the problem of computing the minimum free energy of $O(1)$ DNA or RNA strands of total length $N$ bases. 
%\dw{accurate nucleic acid prediction crucial in DNA nanostructure engineering, DNA computing, prediction of RNA / DNA interactions in biology,... we answer open problem from Dirks et al; 
%	TODO: why is this problem, or it's solution, theoretically interesting? [can we give some context to the literature?]
%	It is of practical use: nupack/viennaRNA are examples of packages in wide use; they are currently reporting the wrong value for MFE ... we correct that;}
\end{abstract}