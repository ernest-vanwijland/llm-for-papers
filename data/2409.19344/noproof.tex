\documentclass[11pt,a4paper]{article}
\usepackage{amsfonts,amsgen,amstext,amsbsy,amsopn,amsfonts,amssymb,amscd}
\usepackage[leqno]{amsmath}
\usepackage[amsmath,amsthm,thmmarks]{ntheorem}
\usepackage{epsf,epsfig}
\usepackage{float}
\usepackage{dsfont}
\usepackage{ebezier,eepic}
\usepackage{color}
\usepackage{tikz}
\usepackage{multirow}
\usepackage{mathrsfs}
\usepackage{graphicx}
\usepackage{subfigure}
\usepackage{cases}
\usepackage{epstopdf}
\setlength{\textwidth}{150mm} \setlength{\oddsidemargin}{7mm}
\setlength{\evensidemargin}{7mm} \setlength{\topmargin}{-5mm}
\setlength{\textheight}{245mm} \topmargin -18mm

\newtheorem{thm}{Theorem}[section]
\newtheorem{theoremA}{Theorem}
\newtheorem{theoremB}{Theorem}
\newtheorem{prop}[thm]{Proposition}
\newtheorem{prob}[thm]{Problem}
\newtheorem{lem}[thm]{Lemma}
\newtheorem{example}[thm]{Example}
\newtheorem{false statement}{False statement}
\newtheorem{cor}[thm]{Corollary}
\newtheorem{fact}[thm]{Fact}
\newtheorem{assumption}{Assumption}

\theoremstyle{definition}
\newtheorem{defn}[thm]{Definition}
\newtheorem{claim}[thm]{Claim}
\newtheorem{remark}[thm]{Remark}
\newtheorem{conj}[thm]{Conjecture}
\newtheorem{corollary}[thm]{Corollary}
\newtheorem{problem}{Problem}
\newtheorem{case}{Case}
\newtheorem{subcase}{Case}[case]
\newtheorem{subsubcase}{Case}[subcase]
\newtheorem{lemmaA}{Lemma}
\newtheorem{factA}[lemmaA]{Fact}

\renewcommand{\thetheoremA}{\Alph{theoremA}}
\renewcommand{\thelemmaA}{\Alph{lemmaA}}
\renewcommand{\thefactA}{\Alph{factA}}
\newcommand{\Mod}[1]{\mathrm{mod}\ #1}
\def\theequation{\thesection.\arabic{equation}}
\makeatletter \@addtoreset{equation}{section}

\baselineskip 15pt
\renewcommand{\baselinestretch}{1.1}
\newcommand{\de}{{\rm def}}
\newcommand{\ex}{{\rm ex}}
\def\hh{\mathcal{H}}
\def\hu{\mathcal{U}}
\def\hm{\mathcal{M}}
\def\hl{\mathcal{L}}
\def\hht{\mathcal{T}}
\def\he{\mathcal{E}}
\def\hf{\mathcal{F}}
\def\hg{\mathcal{G}}
\def\hk{\mathcal{K}}
\def\ha{\mathcal{A}}
\def\hb{\mathcal{B}}
\def\hs{\mathcal{S}}
\def\hr{\mathcal{R}}
\def\hc{\mathcal{C}}
\def\hd{\mathcal{D}}
\def\hi{\mathcal{I}}
\def\hj{\mathcal{J}}
\def\hp{\mathcal{P}}
\def\ex{\mathbb{E}}

\begin{document}
\title{\bf\Large On $r$-wise $t$-intersecting uniform families}
\date{}
\author{Peter Frankl$^1$, Jian Wang$^2$\\[10pt]
$^{1}$R\'{e}nyi Institute, Budapest, Hungary\\[6pt]
$^{2}$Department of Mathematics\\
Taiyuan University of Technology\\
Taiyuan 030024, P. R. China\\[6pt]
E-mail:  $^1$frankl.peter@renyi.hu, $^2$wangjian01@tyut.edu.cn
}
\maketitle

\begin{abstract}
We consider families, $\hf$ of $k$-subsets of an $n$-set. For integers $r\geq 2$, $t\geq 1$, $\hf$ is called $r$-wise $t$-intersecting if any $r$ of its members have at least $t$ elements in common. The most natural construction of such a family is the full $t$-star, consisting of all $k$-sets containing a fixed $t$-set. In the case $r=2$ the Exact Erd\H{o}s-Ko-Rado Theorem shows that the full $t$-star is largest if $n\geq (t+1)(k-t+1)$. In the present paper, we prove that for $n\geq (2.5t)^{1/(r-1)}(k-t)+k$, the full $t$-star is largest in case of $r\geq 3$. Examples show that the exponent $\frac{1}{r-1}$ is best possible. This represents a considerable improvement on a recent result of Balogh and Linz.
\end{abstract}

\section{Introduction}

Let $[n]=\{1,\ldots,n\}$ be the standard $n$-element set. Let $2^{[n]}$ denote the power set of $[n]$ and let $\binom{[n]}{k}$ denote the collection of all $k$-subsets of $[n]$. A subset $\hf\subset \binom{[n]}{k}$ is called a {\it $k$-uniform family}.

The  central notion of this paper is that of $r$-wise $t$-intersecting.

\begin{defn}
For positive integers $r,t$, $r\geq 2$, a family $\hf\subset 2^{[n]}$ is called $r$-wise $t$-intersecting if $|F_1\cap F_2\cap \ldots \cap F_r|\geq t$ for all $F_1,F_2,\ldots,F_r\in \hf$.
\end{defn}

Let us define
\begin{align*}
&m(n,r,t)= \max\left\{|\hf|\colon \hf\subset 2^{[n]} \mbox{ is $r$-wise $t$-intersecting}\right\},\\[3pt]
&m(n,k,r,t)= \max\left\{|\hf|\colon \hf\subset \binom{[n]}{k} \mbox{ is $r$-wise $t$-intersecting}\right\}.
\end{align*}

Let us define the so-called Frankl families (cf. \cite{F77PHD})
\begin{align*}
&\ha_i(n,r,t) =\{A\subset [n]\colon A\cap [t+r i]\geq t+(r-1)i\}, \ 0\leq i\leq \frac{k-t}{r},\\[3pt]
&\ha_i(n,k,r,t) =\ha_i(n,t) \cap \binom{[n]}{k}.
\end{align*}


Since $\ha_i(n,r,t)$ consists of the sets $A$ satisfying $|[t+ri]\setminus A|\leq i$, that is, sets that leave out at most $i$ elements out of the first $t+ri$, $|A_1\cap \ldots\cap A_r\cap [t+ri]|\geq t+ri-ri\geq t$ for all $A_1,\ldots,A_r\in \ha_i(n,r,t)$.

\begin{conj}[\cite{F77PHD}]
\begin{align}
&m(n,r,t) =\max_i|\ha_i(n,r,t)|;\label{ineq-1.1}\\[3pt]
&m(n,k,r,t) =\max_i |\ha_i(n,k,r,t)|.\label{ineq-1.2}
\end{align}
\end{conj}

Let us note that for $r=2$ the statement \eqref{ineq-1.1} is a consequence of the classical Katona Theorem \cite{K64}.

\begin{thm}[The Katona Theorem \cite{K64}]
\[
m(n,2,t) =|\ha_{\lfloor \frac{n-t}{2} \rfloor}(n,2,t)|.
\]
\end{thm}

The case $r=2$ of \eqref{ineq-1.2} was a longstanding conjecture. It was proved in \cite{FFu} for a wide range and it was completely established by the celebrated Complete Intersection Theorem of Ahlswede and Khachatrain \cite{AK}.

A family $\hf\subset \binom{[n]}{k}$ is called a {\it $t$-star} if there exists $T\subset [n]$ with $|T|=t$ such that $T\subset F$ for all $F\in \hf$. The family $\{F\in \binom{[n]}{k}\colon T\subset F\}$ with some $T\in \binom{[n]}{t}$ is called a {\it full $t$-star}.

Let us recall a part of it that was proved earlier.

\begin{thm}[Exact Erd\H{o}s-Ko-Rado Theorem \cite{ekr}, \cite{F78}, \cite{W84}]\label{thm-ekr}
Let $\hf\subset \binom{[n]}{k}$ be a 2-wise $t$-intersecting family. Then for $n\geq (t+1)(k-t+1)$,
\[
|\hf| \leq \binom{n-t}{k-t}.
\]
Moreover, for $n>(t+1)(k-t+1)$ equality holds if and only if $\hf$ is the full $t$-star.
\end{thm}

Theorem \ref{thm-ekr} motivates the following question that is the central problem of the present paper: determine or estimate $n_0(k,r,t)$, the minimal integer $n_0$ such that for all $n\geq n_0$ and all $r$-wise $t$-intersecting families $\hf\subset \binom{[n]}{k}$, $|\hf|\leq |\ha_0(n,k,r,t)|= \binom{n-t}{k-t}$. Theorem \ref{thm-ekr} shows $n_0(k,2,t)=(t+1)(k-t+1)$.


Since the value $\binom{n-t}{k-t}$ is independent of $r$, it should  be clear that $n_0(k,r,t)$ is a monotone decreasing function of $r$. Thus $n_0(k,r,t)\leq n_0(k,2,t)=(t+1)(k-t+1)$.  For $t=1$ the exact value of $m(n,k,r,t)$ and  thereby $n_0(k,r,t)$ is known (cf. \cite{F76}):
\begin{align}\label{ineq-1.6}
m(n,k,r,1) = \left\{ \begin{array}{ll}
                 \binom{n-1}{k-1}, & \mbox{ if } n\geq \frac{r}{r-1}k\\[5pt]
                  \binom{n}{k}, & \mbox{ if } n< \frac{r}{r-1}k.
                \end{array}\right.
\end{align}

Recently, Balogh and Linz \cite{BL} showed that
\[
n_0(k,r,t)< (t+r-1)(k-t-r+ 3).
\]

The main result of the present paper is

\begin{thm}\label{thm:main-1}
For $r= 3,4$,
\begin{align}\label{ineq-1.3}
n_0(k,r,t) \leq    \left(2.5 t\right)^{\frac{1}{r-1}}(k-t)+k.
\end{align}
For $r\geq 5$,
\begin{align}\label{ineq-1.3}
n_0(k,r,t) \leq  \left(2t\right)^{\frac{1}{r-1}}(k-t)+k.
\end{align}
\end{thm}




Let us show that \eqref{ineq-1.3} is essentially  best possible for $t\geq  2^r-r$ and $r$ sufficiently large. Precisely,  for $t\geq 2^r-r$ we have
\[
\left(\frac{t+r}{2}\right)^{\frac{1}{r-1}}(k-t)<n_0(k,r,t) \leq  \left(2t\right)^{\frac{1}{r-1}}(k-t)+k.
\]
Let us prove the lower bound by showing that $|\ha_1(n,k,r,t)|>\binom{n-t}{k-t}$ for $n=\left(\frac{t+r}{2}\right)^{\frac{1}{r-1}}(k-t-r+2)+t+r-2$.
Note that
\[
|\ha_1(n,k,r,t)| = \binom{n-t-r}{k-t-r}+(t+r) \binom{n-t-r}{k-t-r+1}=  \binom{n-t-r}{k-t-r}\left(1+\frac{(t+r)(n-k)}{k-t-r+1}\right)
\]
and
\begin{align*}
\frac{|\ha_1(n,k,r,t)|}{\binom{n-t}{k-t}} &= \frac{(k-t)(k-t-1)\ldots(k-t-r+1)}{(n-t)(n-t-1)\ldots(n-t-r+1)}\left(1+\frac{(t+r)(n-k)}{k-t-r+1}\right)\\[3pt]
&= \frac{(k-t)(k-t-1)\ldots(k-t-r+2)}{(n-t)(n-t-1)\ldots(n-t-r+2)} \frac{(t+r)n-(k+1)(t+r-1)}{n-t-r+1}\\[3pt]
&> \left(\frac{k-t-r+2}{n-t-r+2}\right)^{r-1} \frac{(t+r)n-(k+1)(t+r-1)}{n-t-r+1}.
\end{align*}
If $t\geq 2^r-r$ then $n= \left(\frac{t+r}{2}\right)^{\frac{1}{r-1}}(k-t-r+2)+t+r-2\geq 2k-t-r+2$.
Let us assume $k\geq t+r$  (this is no real restriction, cf. Proposition \ref{prop-1.7} below). It follows that
\begin{align*}
\frac{(t+r)n-(k+1)(t+r-1)}{n-t-r+1} \geq (t+r) \frac{n-k-1+\frac{k+1}{t+r}}{n-t-r+1}> \frac{(t+r)(n-k)}{n-t-r+1}>\frac{t+r}{2}.
\end{align*}
Thus,
\begin{align*}
\frac{|\ha_1(n,k,r,t)|}{\binom{n-t}{k-t}} > \left(\frac{k-t-r+2}{n-t-r+2}\right)^{r-1} \frac{t+r}{2}= 1.
\end{align*}
Therefore for $t\geq 2^r-r$ we obtain that
\begin{align*}
n_0(k,r,t)&> \left(\frac{t+r}{2}\right)^{\frac{1}{r-1}}(k-t-r+2)+t+r-2 \\[3pt]
&> \left(\frac{t+r}{2}\right)^{\frac{1}{r-1}}(k-t)+ \left(\frac{t+r}{2}\right)^{\frac{1}{r-1}}\left(2\left(\frac{t+r}{2}\right)^{\frac{r-2}{r-1}}-r\right)\\[3pt]
&> \left(\frac{t+r}{2}\right)^{\frac{1}{r-1}}(k-t)+ \left(\frac{t+r}{2}\right)^{\frac{1}{r-1}}\left(2^{r-1}-r\right)\\[3pt]
&> \left(\frac{t+r}{2}\right)^{\frac{1}{r-1}}(k-t).
\end{align*}

Our next result determines $m(n,k,3,2)$ for $n> 2k\geq 4$.


\begin{thm}\label{thm:main-2}
For $n> 2k\geq 4$,
\begin{align}\label{ineq-thm-2}
m(n,k,3,2) =\binom{n-2}{k-2}.
\end{align}
Moreover, in case of equality $\hf$ is the full 2-star.
\end{thm}

 Let us note that Balogh and Linz \cite{BL} proved this for $n\geq 4(k-2)$ and in the much older paper \cite{FT}
 the weaker result $m(n,k,3,2) =(1+o(1)){n-2 \choose k-2}$ was established for $k<0.501 n$.

 Let us give two more numerical examples.

\begin{prop}\label{prop-main1}
For $n\geq 2k$,
\[
m(n,k,4,3) =\binom{n-3}{k-3}\mbox{ and } m(n,k,4,4) =\binom{n-4}{k-4}.
\]
\end{prop}


The next result establishes the analogue of \eqref{ineq-thm-2} for a wide range of the pair $(r,t)$.

\begin{thm}\label{thm:main-3}
Let $n\geq \max\left\{2k, \frac{t(t-1)}{2\log 2} +2t-1\right\}$ and $t\leq 2^{r-2}\log 2-2$.  Then
\begin{align}\label{ineq-5.2}
m(n,k,r,t) =\binom{n-t}{k-t}.
\end{align}
Moreover, in case of equality $\hf$ is the full $t$-star.
\end{thm}




Let us show that for $k\leq t+r-2$ the only $r$-wise $t$-intersecting family is the $t$-star.

\begin{prop}\label{prop-1.7}
Suppose that $\hg$ is an $r$-wise $t$-intersecting $k$-graph that is not a $t$-star ($|\cap \hg| <t$). Then $k\geq t+r$ or $k=t+r-1$ and $\hg\subset \binom{Y}{k}$ for some $(k+1)$-element set $Y$.
\end{prop}

\begin{proof}\textcolor{red}{TOPROVE 0}\end{proof}

Based on Proposition \ref{prop-1.7} in the sequel we always assume that $n\geq k\geq t+r$.

As to the corresponding problem for the non-uniform case, Erd\H{o}s-Ko-Rado \cite{ekr} proved $m(n,2,1)=2^{n-1}$. Then the first author \cite{F77Bulletin} established $m(n,3,2) = 2^{n-2}$. After several partial results  the proof of the following result was concluded in \cite{F19}:
\begin{align}\label{ineq-1.4}
 m(n,r,t) = 2^{n-t} \mbox{ if and only if } t\leq 2^r-r-1.
\end{align}


We call  a family $\hf\subset \binom{[n]}{k}$ {\it non-trivial} if $\cap \{F\colon F\in \hf\} =\emptyset$. Define
\begin{align*}
&m^*(n,r,t)= \max\left\{|\hf|\colon \hf\subset 2^{[n]} \mbox{ is non-trivial $r$-wise $t$-intersecting}\right\},\\[3pt]
&m^*(n,k,r,t)= \max\left\{|\hf|\colon \hf\subset \binom{[n]}{k} \mbox{ is non-trivial $r$-wise $t$-intersecting}\right\}.
\end{align*}

\begin{thm}[Brace-Daykin-Frankl Theorem (cf. \cite{BD} for $t=1$ and \cite{F91} for $t\geq 2$)]\label{thm-bd}
For $t+r\leq n$ and  $t<2^r-r-1$,
\begin{align}\label{ineq-1.5}
m^*(n,r,t)= |\ha_1(n,r,t)| =(t+r+1) 2^{n-t-r}.
\end{align}
\end{thm}

Let us recall some notations and useful results.
For $i\in [n]$, define
\[
\hf(i) =\left\{F\setminus \{i\}\colon i\in F\in \hf\right\},\ \hf(\bar{i}) = \left\{F\colon i\notin F\in \hf\right\}.
\]
For $P\subset Q\subset [n]$, define
\[
\hf(Q)= \left\{F\setminus Q\colon Q\subset F\right\},\ \hf(P,Q)= \left\{F\setminus Q\colon  F\cap Q=P\right\}.
\]
Let $X$ be a finite set. 
For any $\hf\subset \binom{X}{k}$ and $1\leq b< k$, define the {\it $b$th shadow}  $\partial^{(b)} \hf$ as
\[
\partial^{(b)} \hf =\left\{E\in \binom{X}{k-b}\colon \mbox{there exists }F\in \hf \mbox{ such that }E\subset F\right\}.
\]
If $b=1$ then we simply write $\partial \hf$ and call it {\it the shadow} of $\hf$.
Define the {\it up shadow} $\partial^+ \hf$  as
\[
\partial^+ \hf =\left\{G\in \binom{X}{k+1}\colon \mbox{ there exists } F\in \hf \mbox{ such that  }F\subset G\right\}.
\]

Sperner  \cite{Sperner} proved the following result.
 
\begin{thm}[\cite{Sperner}]\label{thm-sperner}
 For $\hf\subset \binom{[n]}{k}$,
\begin{align}\label{ineq-sperner}
\frac{|\partial^+ \hf|}{\binom{n}{k+1}} \geq \frac{|\hf|}{\binom{n}{k}}.
\end{align}
\end{thm}

For $\ha,\hb\subset \binom{[n]}{k}$, we say that $\ha,\hb$ are {\it cross-intersecting} if $A\cap B\neq \emptyset$ for all $A\in \ha$ and $B\in \hb$. 

\begin{thm}[\cite{Hilton}]
Let $\ha,\hb\subset \binom{[n]}{k}$ be cross-intersecting. Then for $n\geq 2k$,
\begin{align}\label{ineq-1.7}
|\ha|+|\hb| \leq \binom{n}{k}.
\end{align}
\end{thm} 

We need the following version of the Kruskal-Katona Theorem.

\begin{thm}[\cite{Kruskal,Katona}]\label{thm-kk}
Let $n,k,m$ be positive integers with $k\leq m\leq n$ and let $\hf \subset \binom{[n]}{k}$  and. If $|\hf|>\binom{m}{k}$  then
\[
|\partial \hf|>\binom{m}{k-1}.
\]
\end{thm}

We also need an inequality concerning the $b$th shadow of an $r$-wise $t$-intersecting family. 

\begin{thm}[\cite{F91-2}]\label{thm-F91}
Let $\hf\subset \binom{[n]}{k}$ be an $r$-wise $t$-intersecting family. Then for $0<b\leq t$ we have
\begin{align}\label{ineq-key4}
|\partial^{(b)} \hf| \geq |\hf| \min_{0\leq i\leq \frac{k-t}{r-1}} \frac{\binom{ri+t}{i+b}}{\binom{ri+t}{i}}.
\end{align}
\end{thm}

\section{Shifting and lattice paths}


In \cite{ekr}, Erd\H{o}s, Ko and Rado introduced a very powerful tool in extremal set theory, called shifting.
For $\hf\subset \binom{[n]}{k}$ and $1\leq i<j\leq n$, define the shifting operator
$$S_{ij}(\hf)=\left\{S_{ij}(F)\colon F\in\hf\right\},$$
where
$$S_{ij}(F)=\left\{
                \begin{array}{ll}
                 F':= (F\setminus\{j\})\cup\{i\}, & \mbox{ if } j\in F, i\notin F \text{ and } F'\notin \hf; \\[5pt]
                  F, & \hbox{ otherwise.}
                \end{array}
              \right.
$$

It is well known (cf. \cite{F87}) that the shifting operator preserves the size of $\hf$ and the $r$-wise $t$-intersecting property. Thus one can apply the shifting operator to  $\hf$ when considering $m(n,k,r,t)$.

A family $\hf\subset \binom{[n]}{k}$ is called {\it shifted} if $S_{ij}(\hf)=\hf$ for all $1\leq i<j\leq n$. It is easy to show (cf. \cite{F87}) that every family can be transformed into a shifted family by applying the shifting operator repeatedly. Thus we can always assume that the family $\hf$ is shifted when detemining $m(n,k,r,t)$.

Let us define the shifting partial order.
Let $A=\{a_1,a_2,\ldots,a_k\}$ and $B=\{b_1,b_2,\ldots,b_k\}$ be two distinct $k$-sets with $a_1<a_2<\ldots<a_k$ and $b_1<b_2<\ldots<b_k$. We say that $A$ {\it precedes} $B$ in shifting partial order, denoted by $A\prec B$ if $a_i\leq b_i$ for $i=1,2,\ldots,k$.

Let us recall two properties of shifted families:

\begin{lem}[cf. \cite{F87}]
If $\hf\subset \binom{[n]}{k}$ is a shifted family, then $A\prec B$ and $B\in \hf$ always imply $A\in \hf$.
\end{lem}


\begin{lem}[\cite{F87}]\label{lem-2.4}
Let $\hf\subset  \binom{[n]}{k}$ be a shifted family. Then $\hf$ is $r$-wise $t$-intersecting if and only if
for every $F_1,\ldots,F_r\in \hf$ there exists $s$ such that
\begin{align}\label{ineq-2.5}
\sum_{1\leq i\leq r} |F_i\cap [s]| \geq (r-1) s+t.
\end{align}
\end{lem}

Note that $\sum\limits_{1\leq i\leq r} |F_i\cap [s]|\leq rs$ implies $s\geq t$ if such an $s$ exists.  For completeness let us include the proof.

\begin{proof}\textcolor{red}{TOPROVE 1}\end{proof}

Let $\hf\subset  \binom{[n]}{k}$ be a shifted $r$-wise $t$-intersecting family.  For any $F_1,\ldots,F_r\in \hf$, define $s(F_1,\ldots,F_r)$ to be the minimum $s$ such that
\begin{align*}
\sum_{1\leq i\leq r} |F_i\cap [s]| \geq (r-1) s+t.
\end{align*}
Set $s:=s(F_1,\ldots,F_r)$. Then we must have
\begin{align*}
\sum_{1\leq i\leq r} |F_i\cap [s]| = (r-1) s+t.
\end{align*}
Indeed, if $\sum\limits_{1\leq i\leq r} |F_i\cap [s]| \geq  (r-1) s+t+1$ then
\[
\sum_{1\leq i\leq r} |F_i\cap [s-1]| \geq  (r-1) s+t+1 -r\geq (r-1)(s-1)+t,
\]
contradicting the minimality of $s$. Set $F_1=F_2=\ldots =F_r=F$ for $F\in \hf$, we obtain $r |F\cap [s]| = (r-1) s+t$.
It follows that $\frac{s-t}{r}=:i$ is an integer.  Then $s=t+ri$ and
\[
\frac{(r-1) s+t}{r}=t+\frac{(r-1)(s-t)}{r} =t+(r-1)i.
\]
Thus $|F\cap [t+ri]|\geq t+(r-1)i$ holds and  we get the following corollary.

\begin{cor} [\cite{F87}]\label{cor-2.2}
Let $\hf\subset \binom{[n]}{k}$ be a shifted $r$-wise $t$-intersecting family. Then for every $F\in \hf$, there exists $i\geq 0$ so that $|F\cap [t+ri]|\geq t+(r-1)i$.
\end{cor}

In \cite{F78} a bijection between subsets and certain lattice paths was established. For $F\in \binom{[n]}{k}$, define  $P(F)$ to be the lattice path in the two-dimensional integer grid $\mathbb{Z}^2$ starting at origin as follows. In the $i$th step for $i=1,2,\ldots,n$,  from the current point $(x,y)$ the path $P(F)$ goes to $(x,y+1)$ if $i\in F$ and goes to $(x+1,y)$ if $i\notin F$.  Since $|F|=k$, there are exactly $k$ vertical steps. Thus the end point of $P(F)$ is $(n-k,k)$.

Let $\hf\subset \binom{[n]}{k}$ be a shifted $r$-wise $t$-intersecting family. By Corollary \ref{cor-2.2} we infer that $P(F)$ hits $y=(r-1)x+t$ for every $F\in \hf$. For $F\in \hf$, define $i(F)$ to be the minimum integer $i$ such that $|F\cap [t+ri]|=t+(r-1)i$. Define
\[
\hf_i= \left\{F\in \hf\colon  i(F)=i \right\}, i=0,1,2,\ldots, \left\lfloor \frac{k-t}{r-1}\right\rfloor.
\]
By Corollary \ref{cor-2.2}, $\hf_0,\hf_1,\ldots,\hf_{\lfloor \frac{k-t}{r-1}\rfloor}$ form  a partition of $\hf$.

The next lemma gives a universal bound ont the size of an $r$-wise $t$-intersecting family for $n\geq 2k-t$.

\begin{lem}
Let  $\hf\subset \binom{[n]}{k}$ be an $r$-wise $t$-intersecting family with $r\geq 3$ and $n\geq 2k-t$. Then
\begin{align}
|\hf|   \leq \sum_{0\leq i\leq t} \binom{t}{i}\binom{n-t}{k-t-(r-1)i}.\label{ineq-key0}
\end{align}
Moreover,
\begin{align}
&\sum_{i\geq 1} |\hf_i|   \leq \sum_{1\leq i\leq t} \binom{t}{i}\binom{n-t}{k-t-(r-1)i}.\label{ineq-key1}
\end{align}
\end{lem}


 \begin{figure}[t]
  \centering
  \includegraphics[width=0.45\textwidth]{test-1.pdf}
  \caption{The lattice path $P$ goes through $(i,t-i)$ and hits the line $y=(r-1)x+t$.}\label{latticepath}
\end{figure}

\begin{proof}\textcolor{red}{TOPROVE 2}\end{proof}

\begin{fact}\label{fact-3.1}
Suppose $\hf\subset 2^{[n]}$ is $r$-wise $t$-intersecting but $\hf$ is not a $t$-star. Then for $2\leq s<r$, $\hf$ is $s$-wise $(t+r-s)$-intersecting.
\end{fact}
\begin{proof}\textcolor{red}{TOPROVE 3}\end{proof}


\begin{cor}
Let $\hf\subset \binom{[n]}{k}$ be an $r$-wise $t$-intersecting family with $r\geq 3$. If $\hf$ is not a $t$-star, then
\begin{align}\label{ineq-key2}
|\hf|   \leq \sum_{0\leq i\leq t} \binom{t}{i}\binom{n-t}{k-t-(r-1)i} - \binom{n-t-1}{k-t}.
\end{align}
\end{cor}

\begin{proof}\textcolor{red}{TOPROVE 4}\end{proof}

\section{Proof of Theorem \ref{thm:main-1}}



\begin{proof}\textcolor{red}{TOPROVE 5}\end{proof}





\section{The probability of hitting the line, uniform vs non-uniform}


We need the following version of the Chernoff bound for the binomial distribution.

\begin{thm}[\cite{janson2011random}]
Let $X\in Bi(n,p)$ and $\lambda=np$. Then
\begin{align}\label{chernoff-small}
Pr(X< \lambda-a) \leq e^{-\frac{a^2}{2\lambda}}.
\end{align}
\end{thm}

We call $P(n)$ a {\it $p$-random walk of length $n$} if it starts at origin and goes up a unit with probability $p$ and goes right a unit with probability $1-p$ at each step.  Let $f(n,r,t,p)$ be the probability that a $p$-random walk $P(n)$ hits the line $y=(r-1)x+t$.  Set $f(r,t,p) =\lim\limits_{n\rightarrow \infty} f(n,r,t,p)$. That is, $f(r,t,p)$ is the probability that an infinite $p$-random walk hits the line $y=(r-1)x+t$.



\begin{lem}[\cite{F87},\cite{F91}]\label{lem-key1}
\begin{itemize}
  \item[(i)] $f(n,r,t,p)\leq f(n+1,r,t,p)$.
  \item[(ii)] $f(n+1,r,t,p) = p f(n,r,t-1,p) + (1-p) f(n,r,t+r-1,p)$.
  \item[(iii)]
  \[
      f(r,t,p)=\gamma^t,
  \]
where $\gamma$ is the unique root of  $x = p + (1-p) x^{r}$ in the open interval $(0,1)$.
  \item[(iv)] Let $\alpha_r$ be the unique root of  $x = \frac{1}{2}+ \frac{1}{2}x^{r}$. Then
  \[
      \alpha_3=\frac{\sqrt{5}-1}{2},\ \frac{1}{2}<\alpha_r< \frac{1}{2}+\frac{1}{2^r} \mbox{ for } r\geq 4.
  \]
  Moreover,
  \begin{align}\label{ineq-frankl}
  \frac{1}{2^r-r}<\alpha_r^r\leq \frac{1}{2^r-r-1} \mbox{ for } r\geq 3.
  \end{align}
\end{itemize}
\end{lem}


Let us define another type of random walk. We call $Q(n,i)$ a {\it uniform random walk} if it is chosen uniformly from all lattice paths  from $(0,0)$ to $(n-i,i)$. Let $g(n,i,r,t)$ be  the probability that a uniform random walk $Q(n,i)$ hits the line $y=(r-1)x+t$.

\begin{prop}\label{prop-key}
\begin{itemize}
  \item[(i)] $g(n,i,r,t)\leq g(n,i+1,r,t)$.
  \item[(ii)] $g(n+1,k,r,t)\leq g(n,k,r,t)$.
  \item[(iii)] For $r\geq 3$ and $t\geq 2$, $g(2k,k,r,t)\leq g(2k+2,k+1,r,t)$.
  \item[(iv)] $\lim\limits_{k\rightarrow \infty} g(2k,k,r,t)\leq f(r,t,\frac{1}{2})$.
\end{itemize}
\end{prop}

\begin{proof}\textcolor{red}{TOPROVE 6}\end{proof}




\begin{prop}\label{prop-key3}
For $n\geq 2k$,
\begin{align}\label{ineq-key3}
m(n,k,r,t)\leq \alpha_r^t \binom{n}{k},
\end{align}
where $\alpha_r$ is the unique root of  $x = \frac{1}{2}+ \frac{1}{2}x^{r}$ in the interval $(0,1)$.
\end{prop}
\begin{proof}\textcolor{red}{TOPROVE 7}\end{proof}

\section{Proof of Theorem \ref{thm:main-2}}

Let us prove a useful corollary of Theorem \ref{thm-F91}.

\begin{cor}\label{cor-4.1}
Let $\hf\subset \binom{[n]}{k}$ be a $3$-wise $t$-intersecting family. If $t\geq 4$ then $|\partial^{(2)} \hf|>4|\hf|$. If $t\geq 7$ then $|\partial^{(4)} \hf|>16|\hf|$.
\end{cor}

\begin{proof}\textcolor{red}{TOPROVE 8}\end{proof}
\begin{fact}\label{fact-4.7}
For $n\geq \frac{\sqrt{4t+9}-1}{2}k$, $|\ha_1(n,k,3,t)|<\binom{n-t}{k-t}$. For $n=  \left(\frac{\sqrt{4t+9}-1}{2}-\epsilon\right)k$ with some  $0<\epsilon<\frac{1}{10}$ and $k\geq \frac{t^2+2t}{2\epsilon}$, $|\ha_1(n,k,3,t)|>\binom{n-t}{k-t}$.
\end{fact}
\begin{proof}\textcolor{red}{TOPROVE 9}\end{proof}


\begin{proof}\textcolor{red}{TOPROVE 10}\end{proof}


\section{Proof of Proposition \ref{prop-main1} and Theorem \ref{thm:main-3}}




Let us prove a useful inequality.

\begin{lem}
For $n> \frac{rk-t}{r-1}$,
\begin{align}\label{ineq-2.1}
& m(n,k,r,t) \leq m(n-1,k,r,t) +m(n-1,k-1,r,t).
\end{align}
\end{lem}

\begin{proof}\textcolor{red}{TOPROVE 11}\end{proof}


\begin{lem}\label{lem-6.2}
Suppose that $m(n,k,r,t)=\binom{n-t}{k-t}$ then
\[
m(n,k-1,r,t) =\binom{n-t}{k-1-t}.
\]
\end{lem}

\begin{proof}\textcolor{red}{TOPROVE 12}\end{proof}

Let $\hf\subset \binom{[n]}{k}$ be an $r$-wise $t$-intersecting family.
We say that $\hf$ is {\it saturated} if any addition of an extra $k$-set to $\hf$ would destroy the $r$-wise $t$-intersecting property. We say $\hf_1,\hf_2,\ldots,\hf_r\subset \binom{[n]}{k}$ are {cross $t$-intersecting} if $|F_1\cap F_2\cap\ldots \cap F_r|\geq t$ for all $F_1\in \hf_1$, $F_2\in \hf_2$, $\ldots$, $F_r\in \hf_r$.

\begin{lem}\label{lem-6.4}
Let $\hf\subset \binom{[n]}{k}$ be a shifted and saturated $r$-wise $t$-intersecting family. Let $\hg_i=\hf([t+1]\setminus \{i\},[t+1])$, $i=1,2,3,\ldots,t$. If $\hf$ is not a $t$-star, then $\hg_i=\hg_j$ for all $1\leq i<j\leq t$.
\end{lem}

\begin{proof}\textcolor{red}{TOPROVE 13}\end{proof}

\begin{lem}\label{lem-6.5}
For $k\geq 3$,
\[
m(2k,k,4,3)=\binom{n-3}{k-3}.
\]
\end{lem}

\begin{proof}\textcolor{red}{TOPROVE 14}\end{proof}

\begin{lem}\label{lem-6.6}
For $k\geq 4$,
\[
m(2k,k,4,4)=\binom{n-4}{k-4}.
\]
\end{lem}

\begin{proof}\textcolor{red}{TOPROVE 15}\end{proof}

\begin{proof}\textcolor{red}{TOPROVE 16}\end{proof}


\begin{lem}\label{lem-6.3}
 If $k\geq  \frac{t(t-1)}{4\log 2} +t-1$ and  $t\leq 2^{r-2}\log 2 -2$,then
\begin{align}\label{ineq-5.3}
m(2k,k,r,t) =\binom{n-t}{k-t}.
\end{align}
Moreover, in case of equality $\hf$ is the full $t$-star.
\end{lem}
\begin{proof}\textcolor{red}{TOPROVE 17}\end{proof}



\begin{proof}\textcolor{red}{TOPROVE 18}\end{proof}

\section{Concluding remarks}


The area of research concerning $r$-wise $t$-intersecting non-uniform families is quite large and there are several results we could not even mention. The case of uniform families, that is, adding a new parameter $k$, increases this variety. In the present paper we stayed mostly in the range $k\leq \frac{1}{2}n$. However, it is completely legitimate to consider the range $k\sim cn$ for any fixed $c<1$ as long as $c\leq \frac{r-1}{r}$.

If one wants to extend the results to such a range it seems to be essential to answer the following question.

\begin{prob}
Let $c<\frac{r-1}{r}$ and denote by $p(n,k,r,t)$ the probability that a random lattice path from $(0,0)$ to $(n-k,k)$ hits the line $y=(r-1)x+t$. Let $\alpha$ be the unique root of $c-x+(1-c)x^r=0$ in $(0,1)$. Does the inequality
\begin{align}
p(n,k,r,t) <\alpha^t \mbox{ holds always if }k\leq cn?
\end{align}
\end{prob}

It seems to be rather difficult to determine the exact value of $n_0(k,r,t)$. Based on Fact \ref{fact-4.7}, let us make the following:

\begin{conj}
For $n\geq  \frac{\sqrt{4t+9}-1}{2}k$,
\begin{align*}
m(n,k,3,t) =\binom{n-t}{k-t}.
\end{align*}
\end{conj}

Another  important problem would be to determine $m^*(n,k,r,1)$, the uniform version of the Brace-Daykin Theorem (the case $t=1$ of Theorem \ref{thm-bd}). In the case $r=2$ the solution is given by the Hilton-Milner Theorem \cite{HM}.

Let us recall the Hilton-Milner-Frankl Theorem. Define
\begin{align*}
&\hb(n,k,r,t) =\left\{B\in \binom{[n]}{k}\colon [t+r-2]\subset B,\  B\cap [t+r-1,k+1]\neq \emptyset\right\}\\[2pt]
&\qquad\qquad\qquad\qquad\cup \left\{[k+1]\setminus \{j\}\colon 1\leq j\leq t+r-2\right\}.
\end{align*}


\begin{thm}[Hilton-Milner-Frankl Theorem \cite{HM,F78-2,AK0}]
For $n\geq (k-t+1)(t+1)$,
\begin{align}\label{ineq-hmfrankl}
m^*(n,k,2,t)=\max\left\{|\ha_1(n,k,2,t)|,|\hb(n,k,2,t)|\right\}.
\end{align}
\end{thm}

Note that both families $\ha_1(n,k,2,t)$ and $\hb(n,k,2,t)$ are $r$-wise $(t+2-r)$-intersecting, in particular, $(t+1)$-wise 1-intersecting. Thus in the range $(k-t+1)(t+1)<n$, i.e.,
$k<\frac{n}{t+1}+t-1$,
\[
m^*(n,k,r,t+2-r) =m^*(n,k,2,t).
\]
However the case $k\sim cn$ with $\frac{1}{t+1}<c<\frac{r-1}{r}$
 appears to be much harder. In \cite{FW} the following was proved.

\begin{thm}[\cite{FW}]
Let $0<\varepsilon<\frac{1}{10}$. For $n\geq \frac{4}{\varepsilon^2}+7$ and  $\left(\frac{1}{2}+\varepsilon \right)n\leq k\leq \frac{3n}{5}-3$,
\[
m^*(n,k,3,1) = |\ha_1(n,k,3,1)|.
\]
\end{thm}





\begin{thebibliography}{10}

\bibitem{AK0}
R. Ahlswede, L.H. Khachatrian, The complete non-trivial intersection theorem for systems of finite sets,  J. Comb. Theory, Ser. A,  76 (1996), 121--138.

\bibitem{AK}
R. Ahlswede, L.H. Khachatrian,  The complete intersection theorem for systems of finite sets, European J. Combin. 18 (1997), 125--136.


\bibitem{BL}
J. Balogh, W. Linz, Short proofs of two results about intersecting set systems, Combinatorial Theory 4 (1)(2024), \#4.

\bibitem{BD}
A. Brace, D.E. Daykin, A finite set covering theorem,  Bull. Aust. Math. Soc.,  5 (1971), 197--202.

\bibitem{ekr} P. Erd\H{o}s, C. Ko, R. Rado, Intersection theorems for systems of finite sets, Quart. J. Math. Oxford Ser. 12 (1961), 313--320.

\bibitem{F76}
    P. Frankl, On Sperner families satisfying an additional condition, J. Combinatorial Theory Ser. A 20 (1976), 1--11.

\bibitem{F77PHD}
P. Frankl, Extremal set systems, Ph.D. Thesis, Hungarian Academy of Science, 1977, in Hungarian.

\bibitem{F77Bulletin}
P. Frankl, On families of finite sets no two of which intersect in a singleton, Bull. Austral. Math. Soc. 17(1) (1977), 125--134.

 \bibitem{F78} P. Frankl, The Erd\H{o}s-Ko-Rado theorem is true for $n = ckt$,  Coll. Math. Soc. J. Bolyai 18 (1978), 365--375.

 \bibitem{F78-2}
P. Frankl, On intersecting families of finite sets, J. Combin. Theory, Ser. A 24 (1978), 146--161.

\bibitem{F87} P. Frankl, The shifting technique in extremal set theory, Surveys in Combinatorics  123 (1987), 81--110.

\bibitem{F91}
P. Frankl, Multiply intersecting families,  J. Combin. Theory Ser. B, 53 (1991), 195--234.

\bibitem{F91-2}
P. Frankl, Shadows and shifting, Graphs Combin. 7 (1991), 23--29.
\bibitem{F19}
P. Frankl,
Some exact results for multiply intersecting families, J. Combin. Theory  Ser. B 136(2019), 222--248.

\bibitem{FFu}
    P. Frankl, Z, F\"{u}redi, Beyond the Erd\H{o}s-Ko-Rado theorem, J. Combin. Theory Ser. A 56 (1991),  182--194.
\bibitem{FT}
  P. Frankl, N. Tokushige, Weighted 3-wise 2-intersecting families,  J. Combin. Theory Ser. A 100 (2002),  94--115.

\bibitem{FW}
 P. Frankl, J. Wang, Non-trivial $r$-wise intersecting families,  Acta Mathematica Hungarica  169 (2023),  510--523.

\bibitem{Hilton}
A.J.W. Hilton, An intersection theorem for a collection of families of subsets of  a finite set, J. London Math. Soc. 2 (1977), 369--376.

\bibitem{HM}
 A.J.W. Hilton, E.C. Milner, Some intersection theorems for systems of finite sets, Q. J. Math. 18  (1967), 369--384.

\bibitem{janson2011random}
S. Janson, T. Luczak, A. Rucinski, Random graphs, John Wiley \& Sons, 2011.

\bibitem{K64}
G.O.H. Katona, Intersection theorems for systems of finite sets, Acta Math. Acad. Sci. Hung. 15 (1964), 329--337.

\bibitem{Katona}
G.O.H. Katona, A theorem on finite sets, in: Theory of Graphs, Proc. Colloq. Tihany, 1966, Akad.
Kiad\'{o}, Budapest, 1968; Classic Papers in Combinatorics, 1987, pp. 381--401.

\bibitem{Kruskal}
J.B. Kruskal, The number of simplices in a complex, in: Math. Optimization Techniques, California Press, Berkeley, 1963, pp. 251--278.

\bibitem{Sperner}
E. Sperner, Ein Satz \"{u}ber Untermengen einer endlichen Menge, Math. Zeitschrift 27 (1928), 544--548.


\bibitem{W84}
R. M. Wilson, The exact bound in the Erd\H{o}s-Ko-Rado theorem, Combinatorica 4 (1984), 247--257.

\end{thebibliography}

\end{document}
