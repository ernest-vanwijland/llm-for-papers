\documentclass[a4paper,11pt]{article}
\pagestyle{plain}
\usepackage{braket,hyperref}
\usepackage{amssymb,mathtools,amsthm,amsmath,bm}
\usepackage{tikz}

\usepackage{xargs}
\usepackage{gensymb}

\usepackage{enumitem}
\usepackage{verbatim}


\usepackage{kbordermatrix}


\renewcommand{\kbldelim}{(}\renewcommand{\kbrdelim}{)}\usepackage{thm-restate}


\theoremstyle{plain}
\newtheorem{theorem}{\bf Theorem}[section]
\newtheorem{claim}[theorem]{\bf Claim}
\newtheorem{remark}[theorem]{\bf Remark}
\newtheorem{problem}[theorem]{\bf Problem}
\newtheorem{conjecture}[theorem]{\bf Conjecture}
\newtheorem{proposition}[theorem]{\bf Proposition}
\newtheorem{corollary}[theorem]{\bf Corollary}
\newtheorem{lemma}[theorem]{\bf Lemma}
\theoremstyle{definition}
\newtheorem{definition}[theorem]{\bf Definition}

\newcommand{\bigzero}{\mbox{\normalfont\Large\bfseries 0}}
\newcommand{\rvline}{\hspace*{-\arraycolsep}\vline\hspace*{-\arraycolsep}}




\newenvironment{example}[1][Example.]{\begin{trivlist}
		\item[\hskip \labelsep {\bfseries #1}]}{\end{trivlist}}





\newcommand{\Rea}{{\mathbb R}}
\newcommand{\QQ}{{\mathbb Q}}
\newcommand{\ZZ}{{\mathbb Z}}
\newcommand{\FF}{{\mathbb F}}

\DeclareMathOperator{\lk}{lk}
\DeclareMathOperator{\cost}{cost}
\DeclareMathOperator{\st}{st}
\DeclareMathOperator{\lap}{\Delta}
\DeclareMathOperator{\id}{\text{Id}}
\newcommand{\cobound}[2]{d_{#1}#2}
\newcommand{\bound}[2]{d_{#1}^*#2}

\DeclareMathOperator{\rank}{\text{rank}}

\DeclareMathOperator{\supp}{\text{supp}}



\newcommand{\cupdot}{\mathbin{\mathaccent\cdot\cup}}

\newcommand{\lfrac}[2]{\left\lfloor\frac{#1}{#2}\right\rfloor}

\newcommand{\ufrac}[2]{\left\lceil\frac{#1}{#2}\right\rceil}

\newcommand{\bv}{\mathbf v}
\newcommand{\p}{{\bf p}}
\newcommand{\px}{p^*}
\usepackage{authblk}

	\title{Rigidity expander graphs}


\author[2]{Alan Lew\thanks{\href{mailto:alanlew@andrew.cmu.edu}{alanlew@andrew.cmu.edu}. Alan Lew was partially supported by the Israel Science Foundation grant ISF-2480/20.}}
\author[1]{Eran Nevo\thanks{\href{mailto:nevo@math.huji.ac.il}{nevo@math.huji.ac.il}. Eran Nevo was partially supported by the Israel Science Foundation grant ISF-2480/20.}}
\author[1]{Yuval Peled\thanks{\href{mailto:yuval.peled@mail.huji.ac.il}{yuval.peled@mail.huji.ac.il}.}}
\author[1]{Orit E. Raz\thanks{\href{mailto:oritraz@mail.huji.ac.il}{oritraz@mail.huji.ac.il}.}}

\affil[1]{Einstein Institute of Mathematics,
 Hebrew University, Jerusalem~91904, Israel}
 \affil[2]{Dept. Math. Sciences, Carnegie Mellon University, Pittsburgh, PA 15213, USA}
	
	\date{}
\setcounter{Maxaffil}{0}
\renewcommand\Affilfont{\itshape\small}
\begin{document}



	\maketitle

\begin{abstract}
Jord\'an and Tanigawa recently introduced the $d$-dimensional algebraic connectivity $a_d(G)$ of a graph $G$. This is a quantitative measure of the $d$-dimensional rigidity of $G$ which generalizes the well-studied notion of spectral expansion of graphs. We present a new lower bound for $a_d(G)$ defined in terms of the spectral expansion of certain subgraphs of $G$ associated with a partition of its vertices into $d$ parts. In particular, we obtain a new sufficient condition for the rigidity of a graph $G$. 
As a first application, we prove the existence of an infinite family of $k$-regular $d$-rigidity-expander graphs for every $d\ge 2$ and $k\ge 2d+1$. Conjecturally, no such family of $2d$-regular graphs exists. Second, we show that $a_d(K_n)\geq \frac{1}{2}\lfrac{n}{d}$, which we conjecture to be essentially tight. In addition, we study the extremal values $a_d(G)$ attained if $G$ is a minimally $d$-rigid graph.
\end{abstract}

\section{Introduction}

Graph expansion is one of the most influential concepts in modern graph theory, with numerous applications in discrete mathematics and computer science (see ~\cite{hoory2006expander,lubotzky2012expander}). Intuitively speaking, an expander is a ``highly-connected" graph, and a standard way to quantitatively measure the connectivity, or expansion, of a graph uses the spectral gap in its Laplacian matrix. A main theme in the study of expander graphs deals with the construction of sparse expanders. In particular, bounded-degree regular expander graphs have been studied extensively in various areas of mathematics ~\cite{gabber1981explicit,LPS,zigzag,friedman_alon,MSS}. This paper studies a generalization of spectral graph expansion that was recently introduced by Jord\'an and Tanigawa via the theory of graph rigidity~\cite{jordan2022rigidity}.

A $d$-dimensional framework is a pair $(G,p)$ consisting of a graph $G=(V,E)$ and a map $p:V\to \Rea^d$. The framework is called $d$-rigid if every continuous motion of the vertices starting from $p$ that preserves the distance between every two adjacent vertices in $G$, also preserves the distance between \emph{every pair} of vertices; see e.g.~\cite{Connelly:RigiditySurvey, graver1993book} for background on framework rigidity. 

Asimow and Roth showed in \cite{AR1} that if the map $p$ is generic (e.g. if the $d|V|$ coordinates of $p$ are algebraically independent over the rationals), then the framework rigidity of $(G,p)$ does not depend on the map $p$. Moreover, they showed that for a generic $p$, rigidity coincides with the following stronger linear-algebraic notion of infinitesimal rigidity. 

For every  $u,v\in V$ we define $d_{uv}\in \Rea^d$ by
\[
    d_{u v}=\begin{cases}
           \frac{p(u)-p(v)}{\|p(u)-p(v)\|} & \text{ if } p(u)\neq p(v),\\
            0 & \text{ otherwise,}
    \end{cases}
\]
and $\bv_{u,v}:= (1_u-1_v)\otimes d_{u v}\in \Rea^{d|V|}$,
where $\{1_u\}_{u\in V}$ is the standard basis of $\Rea^{|V|}$ and $\otimes$ denotes the Kronecker product.
 Equivalently,
\[
   \bv_{u,v} ^T= \kbordermatrix{
     & &  &  & u & & & & v & & & \\
     & 0 & \ldots & 0 & d_{u v}^T & 0 & \ldots & 0 & d_{v u}^T & 0 & \ldots & 0}.
\]
 The (normalized) \emph{rigidity matrix} $R(G,p)\in \Rea^{d|V|\times |E|}$ is the matrix whose columns are the vectors $\bv_{u,v}$ for all $\{u,v\}\in E$.
We always assume that the image $p(V)$ does not lie on any affine hyperplane in $\Rea^d$. In such a case, it is possible to show (see \cite{AR1}) that $\rank(R(G,p))\leq d|V|-\binom{d+1}{2}$. The framework $(G,p)$ is called \emph{infinitesimally rigid} if this bound is attained, that is, if $\rank(R(G,p))= d|V|-\binom{d+1}{2}$.

A graph $G$ is called \emph{rigid in $\Rea^d$}, or \emph{$d$-rigid}, if it is
infinitesimally rigid with respect to some map $p$ (or, equivalently, if it is infinitesimally rigid for all generic maps~\cite{AR1}).

For $d=1$ and an injective map $p:V\to \Rea^d$, the rigidity matrix $R(G,p)$ is equal to the incidence matrix of $G$, hence both notions of rigidity coincide with graph connectivity. One can extend this analogy and define a higher dimensional version of the graph's Laplacian matrix, that is called the \emph{stiffness matrix} of $(G,p)$, and is defined by
\[
    L(G,p)=R(G,p) R(G,p)^T \in \Rea^{d|V|\times d|V|}.
\] 
We denote by $\lambda_i(A)$ the $i$-th smallest eigenvalue of a symmetric matrix $A$.
Since $\rank(L(G,p))=\rank(R(G,p))\le d|V|-\binom{d+1}{2}$, the kernel of $L(G,p)$ is of dimension at least $\binom{d+1}{2}$. Therefore, $\lambda_{\binom{d+1}{2}+1}(L(G,p))\neq 0$ if and only if $(G,p)$ is infinitesimally rigid. 

In \cite{jordan2022rigidity}, Jord\'an and Tanigawa defined the \emph{$d$-dimensional algebraic connectivity} of $G$, $a_d(G)$, as
\[
a_d(G)=\sup\left\{ \lambda_{\binom{d+1}{2}+1}(L(G,p)) \middle| \, p: V\to \Rea^d\right\}.
\]
For $d=1$, $L(G,p)$ coincides with the graph Laplacian matrix $L(G)$, and $a_1(G)=a(G)$ is the usual algebraic connectivity, or Laplacian spectral gap, of $G$, introduced by Fiedler in \cite{fiedler1973algebraic}. For every $d\ge 1$, $a_d(G) \ge 0$ since $L(G,p)$ is positive semi-definite, and $a_d(G)>0$ if and only if $G$ is $d$-rigid.

The following notion of \emph{rigidity expander graphs} extends the classical notion of (spectral) expander graphs, corresponding to the $d=1$ case:

\begin{definition}\label{def:expander}
Let $d\geq 1$. A family of graphs $\{G_i\}_{i\in\mathbb{N}}$ of increasing size is called a \emph{family of $d$-rigidity expander graphs} if there exists $\epsilon>0$ such that $a_d(G_i)\geq \epsilon$ for all $i\in\mathbb{N}$.
\end{definition}

It is well known that, for every $k\geq 3$, there exist families of $k$-regular ($1$-dimensional) expander graphs (see e.g. \cite{hoory2006expander}).
Our main result is an extension of this fact to general $d$:

\begin{theorem}\label{thm:rigidity_expanders}
Let $d\geq 1$ and $k\ge 2d+1$. Then, there exists an infinite family of $k$-regular $d$-rigidity expander graphs.
\end{theorem}

It was conjectured by Jord\'an and Tanigawa that families of $2d$-regular $d$-rigidity expanders do not exist
(see \cite[Conj. 2]{jordan2022rigidity} for the statement in the $d=2$ case, and see \cite[Conj. 6.2]{lew2022d} for the general case), and clearly families of $k$-regular $d$-rigidity expanders do not exist for $k<2d$ since, for $n$ large enough, such graphs have less than $dn-\binom{d+1}{2}$ edges, and are therefore not even $d$-rigid. Therefore, assuming this conjecture, our result is sharp.


Our main tool for the proof of Theorem \ref{thm:rigidity_expanders} is a new lower bound on $a_d(G)$, given in terms of the ($1$-dimensional) algebraic connectivity of certain subgraphs of $G$ associated with a partition of its vertex set into $d$ parts. For convenience, we let $a(G)=\infty$ if $G$ consists of a single vertex.

Let $G=(V,E)$ be a graph, and let $A,B\subset V$ be two disjoint sets. We denote by $G[A]$ the subgraph of $G$ induced on $A$, and by $G(A,B)$ the subgraph of $G$ with vertex set $A\cup B$ and edge set $E(A,B)=\{e\in E:\, |e\cap A|=|e\cap B|=1\}$. In addition, we say that a partition $V=A_1\cup\cdots\cup A_d$ is \emph{non-trivial} if $A_i\neq \emptyset$ for all $i=1,\ldots,d$. 

\begin{theorem}\label{thm:lower_bound_general_d}
Let $d\ge 2$. For every graph $G=(V,E)$ and a non-trivial partition $V=A_1\cup \cdots\cup A_d$ there holds
\[
a_d(G)\geq \min\left(\bigg\{a(G[A_i]) \bigg\}_{1\leq i\leq d} \bigcup\left\{\frac{1}{2}a(G(A_i,A_j))\right\}_{1\leq i<j\leq d}\right).
\]
In particular, if $G[A_i]$ is connected for all $i\in[d]$ and $G(A_i,A_j)$ is connected for all $1\leq i< j\leq d$, then $G$ is $d$-rigid.
\end{theorem}

\begin{remark}\label{rem:2d}
In the $d=2$ dimensional case, the statement in Theorem \ref{thm:lower_bound_general_d} can be slightly improved (by removing the constant $1/2$) to
\[a_2(G)\geq \min\{a(G[A_1]), a(G[A_2]),a(G(A_1,A_2))\},\] for every non-trivial partition $A_1,A_2$  of $V$.
\end{remark}

In the case $d=2$, we can think of Theorem \ref{thm:lower_bound_general_d} as a quantitative version of (a special case of) a theorem of Crapo \cite[Theorem 7]{crapo1990generic}. For $d\geq 3$, Theorem \ref{thm:lower_bound_general_d} seems to give, in addition to a lower bound on $a_d(G)$, a new sufficient condition for $d$-rigidity, which we believe to be of independent interest (this sufficient condition could also be derived from \cite[Theorem 5.5]{lindemann2022combinatorial}).



To derive Theorem \ref{thm:rigidity_expanders} from Theorem \ref{thm:lower_bound_general_d} we consider a balanced partition of the vertex set, and construct each of the $\binom{d+1}{2}$ subgraphs induced by the partition in separate. In the main case $k=2d+1$, our ``building blocks" are ($1$-dimensional) expander graphs with maximum degree $3$ and a large proportion of vertices of degree $2$. Such graphs are constructed by subdividing edges in  classical constructions of $3$-regular expander graphs. In Theorem \ref{thm:subdivided_spectral_gap} below, we hedge the effect of edge subdivision on the algebraic connectivity of the graph.

For another application of Theorem \ref{thm:lower_bound_general_d}, we derive a slight improvement of the previously known lower bound for $a_d(K_n)$ from \cite[Theorem 1.5]{lew2022d}.

\begin{corollary}\label{cor:kn_lower_bound}
Let $d\geq 3$ and $n\geq d+1$. Then
\[
    a_d(K_n)\geq \frac{1}{2}\left\lfloor\frac{n}{d}\right\rfloor.
\]
\end{corollary}


In addition, we establish the following upper bound on $a_d(G)$, generalizing the case $d=2$ proved by Jord\'an and Tanigawa in \cite[Theorem 4.2]{jordan2022rigidity}.

\begin{theorem}\label{thm:upper_bound}
Let $d\geq 2$, and let $G$ be a graph. Then,
\[
    a_d(G)\leq a(G).
\]
\end{theorem}
Theorem~\ref{thm:upper_bound} was proved recently and independently in~\cite{presenza2022upper}. Our proof is different, using the probabilistic method, and we believe it to be of independent interest.

Finally, we study how small and how large can $a_d(G)$ be provided that $G$ is a minimally $d$-rigid graph. A graph $G$ is called \emph{minimally $d$-rigid} if it is $d$-rigid, but $G\setminus e$ is not $d$-rigid for every edge $e\in E$. For $d=1$, these are exactly spanning trees. This question is related to the aforementioned conjecture that no $2d$-regular $d$-rigidity expanders exist (see Conjecture~\ref{conj:percent-ev}).

Among the minimally $d$-rigid graphs, $a_d(G)$ is maximized by a $d$-analog of the star graph. For every $d\geq 1$ and $n\geq d+1$, let $S_{n,d}$ be the graph consisting of a clique of size $d$, and $n-d$ additional vertices, each adjacent to all of the vertices of the clique, and not adjacent to any other vertex. It is easy to check that $S_{n,d}$ is minimally $d$-rigid.
\begin{theorem}\label{thm:upper_bound_minimally_rigid}
For every $d\geq 1$ and $T\neq K_2, K_3$ a minimally $d$-rigid graph there holds
\[
    a_d(T)\leq 1,
\]
and equality holds if $T=S_{n,d}$.
\end{theorem}
This extends a result of Fiedler (see \cite[4.1]{fiedler1973algebraic}, more explicitly stated by Merris in \cite[Cor. 2]{merris1987characteristic}) corresponding to the case $d=1$. Note that 
for $T=K_2$ (which is a minimally $1$-rigid graph), we have $a_1(K_2)=2$, and for $T=K_3$ (which is a minimally $2$-rigid graph), we have $a_2(K_3)=\frac{3}{2}$ (see \cite[Theorem 4.4]{jordan2022rigidity}).

Considering the other extreme, of minimizers of $a_d$ among all $n$-vertex $d$-rigid graphs,  it was shown by Fiedler in \cite{fiedler1973algebraic} that $a(G)\geq a(P_n)=2(1-\cos(\pi/n))$ for every connected graph $G$, where $P_n$ is the $n$-vertex path 
(see \cite{grone1990laplacian} for an explicit statement). We conjecture that a similar situation holds in higher dimensions: in Subsection~\ref{subsec:GeneralizedPathGraphs} we define \emph{generalized path graphs} $P_{n,d}$, which are certain $n$-vertex minimally $d$-rigid graphs, and provide in Proposition~\ref{prop:generalized_path_graph} bounds on their $d$-dimensional algebraic connectivity implying that 
$a_d(P_{n,d})=\Theta_d(1/n^2)$. We conjecture that these graphs are extremal:
\begin{conjecture}\label{conj:minimal_spectral_gap_obtained_by_paths}
Let $G$ be a $d$-rigid graph on $n$ vertices. Then,
\[
    a_d(G)\geq a_d(P_{n,d}).
\]
\end{conjecture}


The paper is organized as follows: In Section \ref{sec:prelims} we present some results about stiffness matrices that are used later. In particular, we recall the definition of the \emph{lower stiffness matrix} $L^{-}(G,p)$ introduced in \cite{lew2022d}. In Section \ref{sec:lower_bound} we prove Theorem \ref{thm:lower_bound_general_d}. In Section \ref{sec:subdivisions} we study the effects of edge subdivisions on the spectral gap of a graph. Section \ref{sec:rigidity_expanders} contains the proof of our main result, Theorem \ref{thm:rigidity_expanders}, showing the existence of $k$-regular $d$-rigidity expanders for $k\geq 2d+1$. In Section \ref{sec:upper_bound} we give a proof of the upper bound $a_d(G)\leq a(G)$ (Theorem \ref{thm:upper_bound}). In Section \ref{sec:minimally_rigid} we study the $d$-dimensional algebraic connectivity of minimally $d$-rigid graphs. We conclude in Section \ref{sec:concluding} with several open problems and directions for further research.


\section{Preliminaries}\label{sec:prelims}

Occasionally, it is simpler to work with the \emph{lower stiffness matrix} of the framework $(G,p)$, defined by
\[
    L^{-}(G,p)=R(G,p)^T R(G,p) \in \Rea^{|E|\times |E|}.
\] 
By standard linear algebra, we have that $\rank(L(G,p))=\rank(L^{-}(G,p))=\rank(R(G,p))$ and that the non-zero eigenvalues  of $L(G,p)$, with multiplicities, coincide with those of $L^{-}(G,p)$. In particular, assuming that $|E|\geq d|V|-\binom{d+1}{2}$, we have
\begin{equation}\label{eqn:lam_k}
    \lambda_{k}(L(G,p))= \lambda_{|E|-d|V|+k}(L^{-}(G,p)),
\end{equation}

for every $k\geq \binom{d+1}{2}+1$. In addition, the entries of $L^{-}(G,p)$ are given explicitly by the following lemma.
\begin{lemma}[{\cite[Lemma 2.1]{lew2022d}}]
\label{lemma:down_laplacian}
Let $(G,p)$ be a $d$-dimensional framework. Then, for every $e,e'\in E(G)$,
\[
    L^{-}(G,p)_{e,e'}= \begin{cases}
                    2 & \text{ if } e=e'=\{u,v\} \text{ and } p(u)\neq p(v),\\
  d_{uv}\cdot d_{uw} & \text{ if } e=\{u,v\},~e'=\{u,w\} \\
                    0 & \text{ otherwise,}
                    \end{cases} 
\]
\end{lemma}
where $d_{uv}\cdot d_{uw}$ denotes the dot product. In the case that $e=\{u,v\}$ and $e'=\{u,w\}$, we denote by
$\theta(e,e')$ the angle between $d_{uv}$ and $d_{uw}$. Hence,  
    $L^{-}(G,p)_{e,e'}=\cos(\theta(e,e'))$ (by convention, $\cos(\theta(e,e'))=0$ if $d_{uv}=0$ or $d_{uw}=0$).



\section{A lower bound on $a_d(G)$}\label{sec:lower_bound}


We turn to the proof of Theorem \ref{thm:lower_bound_general_d}, starting with the following very simple lemma about the eigenvalues of a block diagonal matrix.

For convenience, given a ``$0\times 0$" matrix $M$, we define $\lambda_1(M)=\infty$. 

\begin{lemma}\label{lemma:block_diagonal}
Let $M\in\Rea^{n\times n}$ be a block diagonal matrix, with blocks $M_1,\ldots,M_k$, where $M_i\in \Rea^{n_i\times n_i}$ is symmetric for every $1\leq i\leq k$. Then, for every $1\leq m\leq n$ and $r_1,\ldots, r_k$ satisfying $m=1-k+\sum_{i=1}^k r_i$ there holds
\[
\lambda_m(M)\geq \min \{ \lambda_{r_i}(M_i) : \, 1\leq i\leq k\}.
\]
\end{lemma}
\begin{proof}\textcolor{red}{TOPROVE 0}\end{proof}
\begin{remark}
    Note that, under the convention $\lambda_1(M_i)=\infty$ for $M_i\in \Rea^{0\times 0}$, Lemma \ref{lemma:block_diagonal} holds 
     also if we allow values $n_i=0$ and $r_i=1$ for one or more $i\in[k]$.
\end{remark}



\begin{proof}\textcolor{red}{TOPROVE 1}\end{proof}

To derive the stronger bound in the case $d=2$ mentioned in Remark \ref{rem:2d}, we note that in this case $L^-$ itself is a block diagonal matrix with $3$ blocks which are the $1$-dimensional lower stiffness matrices $L^-(G[A_1],q_1)$, $L^-(G[A_2],q_2)$ and $L^-(G(A_1,A_2),q_{12})$ (that is, there is no need for the ``correction" term $Q$). Therefore, by the same reasoning we applied to $M$ in the general case, we find that
\[
a_2(G)\ge \lambda_{m}(L^-) \ge \min(\{a_2(G[A_1]),a_2(G[A_2]),a_2(G(A_1,A_2))\}),
\]
where $m=|E|-2|V|+\binom{2+1}{2}+1$.

\begin{remark}
    The criterion for $d$-rigidity given by Theorem \ref{thm:lower_bound_general_d} is minimal in terms of the edge count. Namely, the assumption that all the $\binom {d+1}2$ graphs in the partition are connected implies that there are at least $|A_i|-1$ edges in $G[A_i]$ for $i\in [d]$, and at least $|A_i|+|A_j|-1$ edges in $G(A_i,A_j)$, for  $1\le i<j\le d.$ In total, there needs be at least
    \[
    \sum_{i=1}^{d}(|A_i|-1) + \sum_{1\le i<j\le d}(|A_i|+|A_j|-1) = d|V|-\binom{d+1}{2}
    \]
    edges in $G$ --- precisely the number of edges in a minimally $d$-rigid graph.
\end{remark}



As a consequence of Theorem \ref{thm:lower_bound_general_d}, we obtain a simple proof of Corollary \ref{cor:kn_lower_bound}, giving a lower bound on the $d$-dimensional algebraic connectivity of $K_n$.

\begin{proof}\textcolor{red}{TOPROVE 2}\end{proof}

\begin{remark}
For $d=2$, it was shown by Jord\'an and Tanigawa in \cite{jordan2022rigidity}, relying on a result by Zhu (\cite{zhu2013quantitative}), that $a_2(K_n)=n/2$. Dividing the vertex set into two parts of sizes $\lfloor n/2 \rfloor$ and $\lceil n/2 \rceil$ respectively, we obtain a bound of $a_2(K_n)\geq \lfrac{n}{2}$. This gives a simple proof of the sharp lower bound in the case that $n$ is even.
\end{remark}

We conjecture that the lower bound we obtained in Corollary \ref{cor:kn_lower_bound} is almost tight:

\begin{conjecture}\label{conj:complete}
Let $d\geq 3$ and $n\geq d+1$. Then,
\[
a_d(K_n)=\begin{cases}
1 & \text{ if }  d+1\leq n\leq 2d,\\
\frac{n}{2d} & \text{ if } n\geq 2d.
\end{cases}
\]
\end{conjecture}
Note that this is a strong version of Conjecture 6.1 in \cite{lew2022d}.






\section{Expansion under edge subdivisions}\label{sec:subdivisions}
The goal of this section is to prove the following theorem regarding the effect of edge subdivision on the algebraic connectivity of a graph. Let $G=(V,E)$ be a graph without isolated vertices. Given an edge $e$ in  $G$, replacing $e$ with an induced path containing $m\ge 0$ new internal vertices is called a subdivision of $e$ with $m$ vertices. 

\begin{theorem}\label{thm:subdivided_spectral_gap}
Let $G$ be a connected graph with minimum degree at least $2$ and maximum degree $\Delta$, and let $G'$ be obtained from $G$ by a subdivision of each edge of $G$ with at most $m$ vertices. Then,
\[
a(G')\geq \frac{\min\left\{\frac{1}{\Delta}a(G),4\right\}}{2 (m+1)^2}.
\]
\end{theorem}


 Let $D(G)$ be the diagonal matrix with $D(G)_{i,i}=\deg_G(i)$, and $\mathcal{L}(G)=D(G)^{-\frac{1}{2}} L(G) D(G)^{-\frac{1}{2}}$ be the normalized Laplacian of $G$.  
 The effect of edge subdivision on the normalized Laplacian was studied by Xie, Zhang and Comellas in ~\cite{xie2016normalized}. Denote by $s(G)$ the subdivision of $G$, that is, the graph obtained from $G$ subdividing each edge of $G$ with $1$ vertex, thus subdividing each edge into two edges. Furthermore, let $s^k(G)$ be the $k$-th iterated subdivision. That is, $s^k(G)=s(s^{k-1}(G))$ (where  $s^{0}(G)=G$).


\begin{lemma}[{\cite[Lemma 3.1]{xie2016normalized}}]\label{lemma:normalized_subdivision}
If $\lambda\neq 1$ is an eigenvalue of $\mathcal{L}(s(G))$ then $2\lambda(2-\lambda)$ is an eigenvalue of $\mathcal{L}(G)$.
\end{lemma}

In order to relate the spectral gap of the normalized Laplacian to the one of the unnormalized Laplacian, we will use the following result due to Higham and Cheng ~\cite{higham1998modifying}.



\begin{lemma}[{\cite[Theorem 3.2]{higham1998modifying}}]\label{lemma:inertia}
Let $A\in \Rea^{n\times n}$ be a symmetric matrix, and let $X\in \Rea^{n\times m}$, for some $m\leq n$. Then, for every $1\leq i\leq m$,
\[
    \lambda_{i}(X^{T} A X) = \theta_i \mu_i,
\]
where
\[
\lambda_i(A)\leq \mu_i\leq \lambda_{i+n-m}(A)
\]
and
\[
\lambda_1(X^T X) \leq \theta_i\leq \lambda_m(X^T X).
\]
\end{lemma}

\begin{lemma}\label{lemma:normalized_vs_unnormalized}
Let $G$ be a graph on $n$ vertices, with minimum degree $\delta>0$ and maximum degree $\Delta$. Then, for all $2\leq i\leq n$,
\[
  \delta\leq  \frac{\lambda_i(L(G))}{\lambda_i(\mathcal{L(G)})}\leq \Delta.
\]
\end{lemma}
\begin{proof}\textcolor{red}{TOPROVE 3}\end{proof}


\begin{proposition}\label{prop:iterated_subdivision}
Let $G$ be a connected graph with minimum degree at least $2$ and maximum degree $\Delta$. Then,
\[
a(s^k(G))\geq \frac{\min\left\{\frac{2}{\Delta}a(G),8\right\}}{4^k}.
\]
\end{proposition}
\begin{proof}\textcolor{red}{TOPROVE 4}\end{proof}

The next lemma establishes that algebraic connectivity is monotone with respect to edge subdivision.

\begin{lemma}\label{lemma:subdivision_second_direction}
Suppose that $G'=(V',E')$ is obtained from $G=(V,E)$ by a subdivision of an edge $e=uv$ of $G$ with $1$ new vertex $w$. Then,
$
    a(G')\leq a(G).
$
\end{lemma}
\begin{proof}\textcolor{red}{TOPROVE 5}\end{proof}


\begin{proof}\textcolor{red}{TOPROVE 6}\end{proof}

Next, we apply Theorem~\ref{thm:subdivided_spectral_gap} to show the existence of ($1$-dimensional) expander graphs with a desired degree sequence that are used subsequently as building blocks in the construction of $d$-dimensional rigidity expanders in the proof of Theorem~\ref{thm:rigidity_expanders}.  
\begin{corollary}\label{cor:expanders_with_many_degree_2_vertices}
For every $d\geq 1$ there exists $c=c(d)>0$ and an infinite family of  $2dn$-vertex bipartite graphs $(H_n)_{n=3}^{\infty}$ such that $a(H_n)\geq c$ and each part consists of $n$ vertices of degree $3$ and $(d-1)n$ vertices of degree $2$.    
\end{corollary}

\begin{proof}\textcolor{red}{TOPROVE 7}\end{proof}


\section{Existence of rigidity expander graphs}\label{sec:rigidity_expanders}
We turn to combine the results in the previous sections to establish the existence of a family of $k$-regular $d$-rigidity expanders for every $k\ge 2d+1$.
\begin{proof}\textcolor{red}{TOPROVE 8}\end{proof}


\section{An upper bound on $a_d(G)$}\label{sec:upper_bound}

In this section we give a proof of Theorem \ref{thm:upper_bound}, stating that for every graph $G$, $a_d(G)\leq a(G)$. In Section \ref{sec:concluding} below we discuss whether this theorem is tight.

We will need the following simple result about stiffness matrices.
\begin{lemma}[Jord\'an-Tanigawa {\cite[3.2]{jordan2022rigidity}}]
\label{lemma:quadratic_form}
Let $G=(V,E)$ be a graph, and let $p:V\to \Rea^d$ and $x\in \Rea^{d|V|}$. Then
\[
x^T L(G,p) x = \sum_{\{u,v\}\in E} \left\langle x(u)-x(v), d_{uv}\right\rangle^2,
\]
where $x(u)\in \Rea^d$ consists of the $d$ coordinates of $x$ corresponding to the vertex $u$. 
\end{lemma}

We will also need the following lemma from \cite{lew2022d}, that states that when computing $a_d(G)$, it is enough to consider maps $p:V\to \Rea^d$ that are embeddings (that is, injective).

\begin{lemma}[{\cite[Lemma 2.4]{lew2022d}}]\label{lemma:a_d_equivalent}
Let $G=(V,E)$ be a graph, and $d\geq 1$. Then
\[
a_d(G)=\sup\left\{ \lambda_{\binom{d+1}{2}+1}(L(G,p)) \middle| \, p: V\to \Rea^d, \, \text{ $p$ is injective} \right\}.
\]
\end{lemma}

\begin{proof}\textcolor{red}{TOPROVE 9}\end{proof}



\section{Minimally rigid graphs}\label{sec:minimally_rigid}

In this section we study the extremal values of the $d$-dimensional algebraic connectivity of minimally $d$-rigid graphs.   In Proposition \ref{prop:upper_bound_minimmaly_rigid} we prove the upper bound $a_d(T)\leq 1$ for minimally $d$-rigid graphs, and in 
Proposition \ref{prop:lower_bound_star_graph} we show that the upper bound is attained for ``generalized star" graphs. This establishes the proof of Theorem \ref{thm:upper_bound_minimally_rigid}. The section is concluded with a discussion about generalized path graphs and their algebraic connectivity.

\begin{proposition}\label{prop:upper_bound_minimmaly_rigid}
Let $d\geq 1$, and let $T\neq K_2, K_3$ be a minimally $d$-rigid graph. Then,
\[
    a_d(T)\leq 1.
\]
\end{proposition}

For the proof of Proposition \ref{prop:upper_bound_minimmaly_rigid} we will need the following results.

\begin{lemma}[Jord\'an-Tanigawa {\cite[Lemma 4.5]{jordan2022rigidity}}]\label{lemma:vertex_removal}
Let $d\geq 1$. Let $G=(V,E)$ and $v\in V$. Then,
\[
    a_d(G\setminus v)\geq a_d(G)-1.
\]
\end{lemma}

\begin{theorem}[{\cite[Theorem 1.2]{lew2022d}}]\label{thm:simplex}
Let $d\geq 3$. Then
\[
    a_d(K_{d+1})=1.
\]
\end{theorem}




\begin{proof}\textcolor{red}{TOPROVE 10}\end{proof}

Let $d\geq 1$ and $n\geq d+1$. Let $S_{n,d}$ be the graph on vertex set $[n]$ with edge set
\[
    E(S_{n,d})=\left\{ \{i,j\}:\, i\in[d],\,  j\in[n]\setminus\{i\}\right\}.
\]
It is easy to check that $S_{n,d}$ is minimally $d$-rigid. 

We consider the following mapping of the vertices of $S_{n,d}$ to $\Rea^d$:
Let $e_1,\ldots,e_d\in \Rea^d$ be the standard basis vectors. We define $\px:[n]\to \Rea^d$ by
\[
    \px(i)=\begin{cases}
        e_i & \text{ if } 1\leq i\leq d,\\
        0 & \text{ if } d<i\leq n.
    \end{cases}
\]

\begin{proposition}\label{claim:star_spectrum}
The spectrum of $L(S_{n,d},\px)$ is
\[
\left\{ 0^{\left(\binom{d+1}{2}\right)}, 1^{\left(dn-\binom{d+1}{2}-d\right)}, (n-d/2)^{(d-1)}, n^{(1)} \right\}
\]
(where the superscript $(m)$ indicates multiplicity $m$ of the corresponding eigenvalue).
In particular, $\lambda_{{\binom{d+1}{2}+1}}(L(S_{n,d},\px))= 1$.
\end{proposition}
\begin{proof}\textcolor{red}{TOPROVE 11}\end{proof}


\begin{proposition}\label{prop:lower_bound_star_graph}
Let $d\geq 1$ and $n\geq d+1$. Then, unless $d=2$ and $n=3$, we have
\[
    a_d(S_{n,d})=1.
\]
\end{proposition}
\begin{proof}\textcolor{red}{TOPROVE 12}\end{proof}

It would be interesting to determine whether the graphs $S_{n,d}$ are the only extremal cases in Theorem \ref{thm:upper_bound_minimally_rigid} (for $d=1$ this is a result of Merris, \cite[Cor. 2]{merris1987characteristic}).



\subsection{Generalized path graphs}\label{subsec:GeneralizedPathGraphs}

Let $n\geq d+1$. Let $P_{n,d}$ be the graph on vertex set $[n]$ with edges \[\left\{ \{i,j\} :\,1\leq i<j\leq n,\,  j-i\leq d\right\}.\] Note that, for $d=1$, $P_n=P_{n,1}$ is just the path with $n$ vertices.
It is not hard to check that, for $n\geq d+1$, $P_{n,d}$ is minimally rigid in $\Rea^d$.

As mentioned in the introduction, Fiedler~\cite{fiedler1973algebraic} showed that $a_1(G)\geq a_1(P_n)=2(1-\cos(\pi/n))$ for every connected graph $G$.
For $d>1$, we do not know the exact value of $a_d(P_{n,d})$, but the following result gives us its order of magnitude:

\begin{proposition}\label{prop:generalized_path_graph}
Let $d\geq 2$ and $n\geq d+1$. Then
\[
    1-\cos\left(\frac{\pi}{2} \ufrac{n}{d}^{-1}\right)
    \leq 
    a_d(P_{n,d})
    \leq
    2d-2\sum_{k=1}^d \cos(2k\pi/n).
\]
For $d=2$ we have a slightly better lower bound, $a_2(P_{n,2})\geq 2(1-\cos(\pi/n))$.
\end{proposition}
For large $n$, we have $1-\cos\left(\frac{\pi}{2}\ufrac{n}{d}^{-1}\right) \approx 1-\cos\left(\frac{d\pi}{2n}\right)\approx \frac{d^2 \pi^2}{8n^2}$.
Moreover, $2d-2\sum_{k=1}^d \cos(2k\pi/n)\leq \frac{2\pi^2 d(d+1)(2d+1)}{3 n^2}$.
Therefore, $a_d(P_{n,d})=\Theta_d(1/n^2)$.

\begin{proof}\textcolor{red}{TOPROVE 13}\end{proof}






\section{Concluding remarks}\label{sec:concluding}
Many fascinating open problems suggest themselves. In this paper, we showed that families of $k$-regular $d$-rigidity expanders exist for $k>2d$, and it is natural to seek for the best possible construction.
\begin{problem}
Let $d \ge 2$ and $k>2d$ be integers. What is 
$$
c_d(k) := \sup_{(G_n)_{n\in\mathbb N}}\liminf_n a_d(G_n),
$$
where $(G_n)_{n\in\mathbb N}$ runs over families of $k$-regular graphs of increasing size?
\end{problem}
The $1$-dimensional case of this problem is perhaps the most important question in the theory of expander graphs. The Alon-Boppana Theorem asserts an upper bound of $c_1(k)\le k-2\sqrt{k-1}$, and constructions attaining this bound are known as (one-sided) Ramanujan graphs~\cite{LPS,MSS}. 
For $d\ge 2$, the proof of Theorem \ref{thm:rigidity_expanders} gives a lower bound for $c_d(k)$ which applies to all $k\ge 2d+1$, and whose rate of decay is in the order of $1/d^2$ as $d\to\infty$. If $k\ge td$ for some $t\geq 3$, one can easily adapt our methods and attain a lower bound for $c_d(k) \ge c_1(t)/2$ that is independent of $d$. That is, by using $t$-regular bipartite Ramanujan graphs in Section~\ref{sec:rigidity_expanders} instead of the subdivided graphs from Corollary \ref{cor:expanders_with_many_degree_2_vertices}. The question whether $c_d(2d+1)\to 0$ as $d\to\infty$ remains open.

It is known that $a(T) = O(1/n)$ if $T$ is a bounded-degree tree ~\cite{kolokolnikov2015maximizing}, and we conjecture that this phenomenon extends to higher dimensions.
\begin{conjecture}~\label{conj:percent-ev}
Fix integers $d,b\ge 1$. Then,
\[
\max_{G_n} a_d(G_n) \to 0~~~\mbox{as }n\to\infty,
\]
where $G_n$ runs over all minimally $d$-rigid $n$-vertex graphs of max-degree $b$.
\end{conjecture}

A  stronger but still plausible conjecture --- that $$\sup_{(G_n,p_n)}\lambda_{d(d+1)+1}(L(G_n,p_n))\to 0~~~\mbox{as }n\to\infty,$$ where 
$(G_n,p_n)$ runs over all $d$-frameworks of minimally $d$-rigid $n$-vertex graphs of maximum degree $b$ --- would imply that no $2d$-regular $d$-rigidity expanders exist, via interlacing of spectra under adding an edge to a graph (see \cite[Theorem 2.3]{lew2022d}).

Regarding the relations between the values $a_d(G)$, for $G$ fixed and $d$ that varies, we propose the following strengthening of Theorem~\ref{thm:upper_bound}:

\begin{conjecture}(Monotonicity)
Let $1\le d'<d$ be integers and $G$ a graph on $n$ vertices. Then,
$a_d(G) \le a_{d'}(G)$.
\end{conjecture}


In addition, the fact that we do not know if the bound in Theorem \ref{thm:upper_bound} is tight raises the following problem:
\begin{problem}
    What is $\sup_G (a_d(G) / a(G))$ over all connected graphs $G$?
\end{problem}
The best lower bound that we currently have for this problem that applies to every $d$ is $1/d$ which is given by the generalized star graph $S_{n,d}$. Indeed,  $a_d(S_{n,d})=1$ and $a(S_{n,d})=d$.
For the special case $d=2$ we obtained by computer calculations $a_2(P_{12,2})\geq 0.667\cdot a(P_{12,2})$ (where $P_{n,d}$ is the generalized path graph). It remains a possibility that Theorem \ref{thm:upper_bound} is tight, and we suspect that generalized paths might be the extremal examples. 






\section*{Acknowledgements}
Part of this research was done while A.L. was a postdoctoral researcher at the Einstein Institute of Mathematics at the Hebrew University.




\bibliographystyle{abbrv}
\bibliography{biblio}

\end{document}
