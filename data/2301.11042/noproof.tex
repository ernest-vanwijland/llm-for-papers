\documentclass{article}
\usepackage{indentfirst,overpic}
\usepackage{amsthm,afterpage,float}
\usepackage{amsfonts}
\usepackage{amssymb}
\usepackage{overpic}

\newtheorem{remark}{Remark}

\newcommand{\COLORON}{1}
\newcommand{\NOTESON}{0}
\newcommand{\Debug}{0}

\newcommand{\forb}[1]{\mathrm{Forb}(#1)}
\newcommand{\ex}[1]{\mathrm{Ex}(#1)}
\newcommand{\prl}[1]{#1^{\obslash}}
\newcommand{\uop}{\ensuremath{U}-\OP}
\newcommand{\uopE}{\ensuremath{U\mathrm{-OP_E}}}
\newcommand{\uopCE}{\ensuremath{U\mathrm{-OP_{/E}}}}
\newcommand{\pln}{\ensuremath{\mathrm{Planar}}}
\newcommand{\plE}{\ensuremath{\mathrm{Planar_E}}}
\newcommand{\plV}{\ensuremath{\mathrm{Planar_V}}}
\newcommand{\frs}{\mathcal{F}}
\newcommand{\frsV}{\frs_\mathrm{V}}
\newcommand{\frsE}{\frs_\mathrm{E}}
\newcommand{\frsCE}{\frs_\mathrm{{/E}}}
\newcommand{\rmv}[1]{\ensuremath{#1_{\mathrm{V}}}}
\newcommand{\rme}[1]{\ensuremath{#1_{\mathrm{E}}}}
\newcommand{\rmce}[1]{\ensuremath{#1_{\mathrm{/E}}}}
\newcommand{\rmece}[1]{\ensuremath{#1_{\mathrm{E/E}}}}
\newcommand{\ope}{\rme{OP}}
\newcommand{\OuPl}{\ensuremath{{\mathrm{OP}}}}
\newcommand{\SU}{\ensuremath{\Sig_\bullet}}
\newcommand{\dbar}[1]{\ensuremath{\overline{\overline{#1}}}}

\newcommand{\cof}{co-finite}
\newcommand{\Cof}{Co-finite}
\newcommand{\uncof}{UNCOF}

\newcommand{\mm}{marked minor}
\newcommand{\umm}{\ensuremath{U}-marked minor}
\newcommand{\Sig}{\ensuremath{\Sigma}}
\newcommand{\vapf}{VAP-free}
\newcommand{\Pof}{Proof of }

\newcommand{\omdot}{\omega \cdot}


\usepackage[usenames]{color} 
\usepackage{amsthm,amssymb,amsmath,bbm,enumerate,graphicx,epsf,stmaryrd,accents}
\usepackage[bookmarks, colorlinks=false, breaklinks=true]{hyperref} 



\usepackage{authblk}



\hyphenation{com-pac-ti-fi-cation}

\newcommand{\keywordname}{{\bf Keywords:}}
\newcommand{\keywords}[1]{\par\addvspace\baselineskip 	\noindent\keywordname\enspace\ignorespaces#1}
	
\renewcommand{\theenumi}{(\roman{enumi})}
\renewcommand{\labelenumi}{\theenumi}

\newcommand{\comment}[1]{}
\newcommand{\COMMENT}[1]{}

\definecolor{darkgray}{rgb}{0.3,0.3,0.3}
\newcommand{\defi}[1]{{\color{darkgray}\emph{#1}}}


\newcommand{\acknowledgement}{\section*{Acknowledgement}}
\newcommand{\acknowledgements}{\section*{Acknowledgements}}





\comment{
	\begin{lemma}\label{}	
\end{lemma}
\begin{proof}\textcolor{red}{TOPROVE 0}\end{proof}


\begin{theorem}\label{}
\end{theorem} 
\begin{proof}\textcolor{red}{TOPROVE 1}\end{proof}

}






\newtheorem{proposition}{Proposition}[section]
\newtheorem{definition}[proposition]{Definition}
\newtheorem{theorem}[proposition]{Theorem}
\newtheorem{corollary}[proposition]{Corollary}
\newtheorem*{nocorollary}{Corollary}
\newtheorem{lemma}[proposition]{Lemma}
\newtheorem{observation}[proposition]{Observation}
\newtheorem{conjecture}{{Conjecture}}[section]
\newtheorem*{noConjecture}{{Con\-jecture}}
\newtheorem{problem}[conjecture]{{Problem}}
\newtheorem*{noProblem}{{Problem}}
\newtheorem{question}[conjecture]{{Question}}
\newtheorem*{noTheorem}{Theorem}
\newtheorem{examp}[proposition]{Example}\newtheorem{claim}{Claim}
\newtheorem*{noclaim}{Claim}

\newtheorem{vermutung}{{\color{red}Vermutung}}[section]
\newtheorem*{noVermutung}{{\color{red}Vermutung}}


\newcommand{\example}[2]{\begin{examp} \label{#1} {{#2}}\end{examp}}

\newcommand{\kreis}[1]{\mathaccent"7017\relax #1}

\newcommand{\FIG}{0}

\ifnum \NOTESON = 1 \newcommand{\note}[1]{ 

\hspace*{-30pt}
	{\color{blue}  NOTE: \color{Turquoise}{\small  \tt \begin{minipage}[c]{1.1\textwidth}  #1 \end{minipage} \ignorespacesafterend }} 
	
	}
\else \newcommand{\note}[1]{} \fi

\newcommand{\afsubm}[1]{ \ifnum \Debug = 1 {\mymargin{#1}}
\fi} 

\ifnum \Debug = 1 \newcommand{\sss}{\ensuremath{\color{red} \bowtie \bowtie \bowtie\ }}
\else \newcommand{\sss}{} \fi

\ifnum \FIG = 1 \newcommand{\fig}[1]{Figure ``{#1}''}
\else \newcommand{\fig}[1]{Figure~\ref{#1}} \fi

\ifnum \FIG = 1 \newcommand{\figs}[1]{Figures ``{#1}''}
\else \newcommand{\figs}[1]{Figures~\ref{#1}} \fi

\ifnum \Debug = 1 \usepackage[notref,notcite]{showkeys}
\fi

\ifnum \COLORON = 0 \renewcommand{\color}[1]{}
\fi

\newcommand{\Fig}{Figure}


\newcommand{\showFigu}[3]{
   \begin{figure}[htbp]
   \centering
   \noindent
   \epsfxsize=0.8\hsize
   \epsfbox{#1.eps}
\caption{\small #3}
   \label{#2}
   \end{figure}
}

\newcommand{\showFig}[2]{
   \begin{figure}[htbp]
   \centering
   \noindent
   \epsfbox{#1.eps}
   \caption{\small #2}
   \label{#1}
   \end{figure}
}

\newcommand{\showFigI}[1]{ \begin{figure}[htbp]
   \centering
   \noindent
   \epsfbox{#1.eps}
\label{#1}
   \end{figure}
}





\newcommand{\D}{\ensuremath{\mathbb D}}
\newcommand{\N}{\ensuremath{\mathbb N}}
\newcommand{\R}{\ensuremath{\mathbb R}}
\newcommand{\C}{\ensuremath{\mathbb C}}
\newcommand{\Z}{\ensuremath{\mathbb Z}}
\newcommand{\Q}{\ensuremath{\mathbb Q}}
\newcommand{\BS}{\ensuremath{\mathbb S}}

\newcommand{\ca}{\ensuremath{\mathcal A}}
\newcommand{\cb}{\ensuremath{\mathcal B}}
\newcommand{\cc}{\ensuremath{\mathcal C}}
\newcommand{\cd}{\ensuremath{\mathcal D}}
\newcommand{\ce}{\ensuremath{\mathcal E}}
\newcommand{\cf}{\ensuremath{\mathcal F}}
\newcommand{\cg}{\ensuremath{\mathcal G}}
\newcommand{\ch}{\ensuremath{\mathcal H}}
\newcommand{\ci}{\ensuremath{\mathcal I}}
\newcommand{\cj}{\ensuremath{\mathcal J}}
\newcommand{\ck}{\ensuremath{\mathcal K}}
\newcommand{\cl}{\ensuremath{\mathcal L}}
\newcommand{\cm}{\ensuremath{\mathcal M}}
\newcommand{\cn}{\ensuremath{\mathcal N}}
\newcommand{\co}{\ensuremath{\mathcal O}}
\newcommand{\cp}{\ensuremath{\mathcal P}}
\newcommand{\cq}{\ensuremath{\mathcal Q}}
\newcommand{\cs}{\ensuremath{\mathcal S}}
\newcommand{\cgr}{\ensuremath{\mathcal R}}
\newcommand{\ct}{\ensuremath{\mathcal T}}
\newcommand{\cu}{\ensuremath{\mathcal U}}
\newcommand{\cv}{\ensuremath{\mathcal V}}
\newcommand{\cx}{\ensuremath{\mathcal X}}
\newcommand{\cy}{\ensuremath{\mathcal Y}}
\newcommand{\cz}{\ensuremath{\mathcal Z}}
\newcommand{\cw}{\ensuremath{\mathcal W}}

\newcommand{\oo}{\ensuremath{\omega}}
\newcommand{\OO}{\ensuremath{\Omega}}
\newcommand{\alp}{\ensuremath{\alpha}}
\newcommand{\bet}{\ensuremath{\beta}}
\newcommand{\gam}{\ensuremath{\gamma}}
\newcommand{\Gam}{\ensuremath{\Gamma}}
\newcommand{\del}{\ensuremath{\delta}}
\newcommand{\Del}{\ensuremath{\Delta}}
\newcommand{\eps}{\ensuremath{\epsilon}}
\newcommand{\kap}{\ensuremath{\kappa}}
\newcommand{\lam}{\ensuremath{\lambda}}
\newcommand{\sig}{\ensuremath{\sigma}}
\newcommand{\al}{\ensuremath{\alpha}}
\newcommand{\el}{\ensuremath{\ell}}


\newcommand{\CCC}{\ensuremath{\mathcal C}}
\newcommand{\ccc}{\ensuremath{\mathcal C}}
\newcommand{\ccg}{\ensuremath{\mathcal C(G)}}
\newcommand{\fcg}{\ensuremath{|G|}}



\newcommand{\zero}{\mathbbm 0}
\newcommand{\one}{\mathbbm 1}
\newcommand{\sm}{\backslash}
\newcommand{\restr}{\upharpoonright}
\newcommand{\sydi}{\triangle}
\newcommand{\noproof}{\qed}
\newcommand{\isom}{\cong}
\newcommand{\subgr}{\subseteq}
\newcommand{\cls}[1]{\ensuremath{\overline{#1}}}
\newcommand{\act}{\curvearrowright}



\makeatletter
\DeclareRobustCommand{\cev}[1]{\mathpalette\do@cev{#1}}
\newcommand{\do@cev}[2]{\fix@cev{#1}{+}\reflectbox{$\m@th#1\vec{\reflectbox{$\fix@cev{#1}{-}\m@th#1#2\fix@cev{#1}{+}$}}$}\fix@cev{#1}{-}}
\newcommand{\fix@cev}[2]{\ifx#1\displaystyle
    \mkern#23mu
  \else
    \ifx#1\textstyle
      \mkern#23mu
    \else
      \ifx#1\scriptstyle
        \mkern#22mu
      \else
        \mkern#22mu
      \fi
    \fi
  \fi
}

\makeatother






\newcommand{\ztwo}{\mathbb Z_2}
\newcommand{\nin}{\ensuremath{{n\in\N}}}
\newcommand{\iin}{\ensuremath{{i\in\N}}}
\newcommand{\unin}{\ensuremath{[0,1]}}
\newcommand{\xy}{$x$-$y$}

\newcommand{\limf}[1]{\ensuremath{\lim \inf(#1)}}


\newcommand{\sgl}[1]{\ensuremath{\{#1\}}}
\newcommand{\pth}[2]{\ensuremath{#1}\text{--}\ensuremath{#2}~path}
\newcommand{\edge}[2]{\ensuremath{#1}\text{--}\ensuremath{#2}~edge}
\newcommand{\topth}[2]{topological \ensuremath{#1}\text{--}\ensuremath{#2}~path}
\newcommand{\topths}[2]{topological 
	\ensuremath{#1}\text{--}\ensuremath{#2}~paths}
\newcommand{\pths}[2]{\ensuremath{#1}\text{--}\ensuremath{#2}~paths}
\newcommand{\arc}[2]{\ensuremath{#1}\text{--}\ensuremath{#2}~arc}
\newcommand{\arcs}[2]{\ensuremath{#1}\text{--}\ensuremath{#2}~arcs}
\newcommand{\seq}[1]{\ensuremath{(#1_n)_{n\in\N}}} 
\newcommand{\seqi}[1]{\ensuremath{#1_0, #1_1, #1_2, \ldots}}
\newcommand{\sseq}[2]{\ensuremath{(#1_i)_{i\in #2}}} \newcommand{\susq}[2]{\ensuremath{(#1_{#2_i})_{\iin}}} \newcommand{\oseq}[2]{\ensuremath{(#1_\alp)_{\alp< #2}}} \newcommand{\fml}[1]{\ensuremath{\{#1_i\}_{i\in I}}} 
\newcommand{\ffml}[2]{\ensuremath{\{#1_i\}_{i\in #2}}} \newcommand{\ofml}[2]{\ensuremath{\{#1_\alp\}_{\alp< #2}}} \newcommand{\flo}[2]{\ensuremath{#1}\text{--}\ensuremath{#2}~flow} \newcommand{\flos}[2]{\ensuremath{#1}\text{--}\ensuremath{#2}~flows} 

\newcommand{\g}{\ensuremath{G\ }}
\newcommand{\G}{\ensuremath{G}}
\newcommand{\Gn}[1]{\ensuremath{G[#1]_n}}
\newcommand{\Gi}[1]{\ensuremath{G[#1]_i}}

\newcommand{\s}{s}

\newcommand{\half}{\frac{1}{2}}

\newcommand{\ceil}[1]{\ensuremath{\left\lceil #1 \right\rceil}}
\newcommand{\floor}[1]{\ensuremath{\left\lfloor #1 \right\rfloor}}




\newcommand{\ltp}{\ensuremath{|G|_\ell}}
\newcommand{\ltop}{\ensuremath{\ell-TOP}}
\newcommand{\blg}{\ensuremath{\partial^\ell G}}
\newcommand{\lER}{\ensuremath{\ell: E(G) \to \R_{>0}}}
\newcommand{\finl}{\ensuremath{\sum_{e \in E(G)} \ell(e) < \infty}}
\newcommand{\infl}{\ensuremath{\sum_{e \in E(G)} \ell(e) = \infty}}




\newcommand{\OP}{outerplanar}
\newcommand{\Ktt}{\ensuremath{K_{3,3}}}

\newcommand{\wqo}{well-quasi-ordered}


\newcommand{\knl}{Kirchhoff's node law}
\newcommand{\kcl}{Kirchhoff's cycle law}
\newcommand{\cutr}{non-elusive} \newcommand{\Cutr}{Non-elusive} 

\newcommand{\are}{\vec{e}}
\newcommand{\arE}{\vec{E}}

\newcommand{\DP}{Dirichlet problem}
\newcommand{\BM}{Brownian Motion}
\newcommand{\hf}{harmonic function}



\newcommand{\Ex}{\mathbb E}
\renewcommand{\Pr}{\mathbb{P}}



\newcommand{\Cg}{Cayley graph}
\newcommand{\Cc}{Cayley complex}
\newcommand{\pres}[2]{\ensuremath{\left<#1 \mid #2 \right>}}
\newcommand{\vt}{vertex-transitive}
\newcommand{\vtg}{vertex-transitive graph}



\newcommand{\Lr}[1]{Lemma~\ref{#1}}
\newcommand{\Lrs}[1]{Lemmas~\ref{#1}}
\newcommand{\Tr}[1]{Theorem~\ref{#1}}
\newcommand{\Trs}[1]{Theorems~\ref{#1}}
\newcommand{\Sr}[1]{Section~\ref{#1}}
\newcommand{\Srs}[1]{Sections~\ref{#1}}
\newcommand{\Prr}[1]{Pro\-position~\ref{#1}}
\newcommand{\Prb}[1]{Problem~\ref{#1}}
\newcommand{\Cr}[1]{Corollary~\ref{#1}}
\newcommand{\Cnr}[1]{Con\-jecture~\ref{#1}}
\newcommand{\Or}[1]{Observation~\ref{#1}}
\newcommand{\Er}[1]{Example~\ref{#1}}
\newcommand{\Dr}[1]{De\-fi\-nition~\ref{#1}}
\newcommand{\Qr}[1]{Question~\ref{#1}}




\newcommand{\lf}{locally finite}
\newcommand{\lfg}{locally finite graph}
\newcommand{\fg}{finite graph}
\newcommand{\ig}{infinite graph}
\newcommand{\bos}{basic open set}
\newcommand{\nlf}{non-locally-finite}
\newcommand{\nlfg}{non-locally-finite graph}
\newcommand{\fhcy}{Hamilton cycle}
\newcommand{\hcy}{Hamilton circle}
\newcommand{\et}{Euler tour}
\newcommand{\ets}{Euler tours}
\newcommand{\scl}{star-comb lemma}
\renewcommand{\iff}{if and only if}
\newcommand{\fe}{for every}
\newcommand{\Fe}{For every}
\newcommand{\fea}{for each}
\newcommand{\Fea}{For each}
\newcommand{\st}{such that}
\newcommand{\sot}{so that}
\newcommand{\ti}{there is}
\newcommand{\ta}{there are}
\newcommand{\tho}{there holds}
\newcommand{\wed}{well-defined}
\newcommand{\seth}{small enough that}
\newcommand{\pwd}{pairwise disjoint}

\newcommand{\obda}{without loss of generality}
\newcommand{\btdo}{by the definition of}
\newcommand{\Btdo}{By the definition of}
\newcommand{\btco}{by the construction of}
\newcommand{\Btco}{By the construction of}
\newcommand{\wrt}{with respect to}
\newcommand{\fim}{finitely many}
\newcommand{\inm}{infinitely many}
\newcommand{\istc}{is straightforward to check}
\newcommand{\ises}{is easy to see}
\newcommand{\leth}{large enough that}
\newcommand{\inhy}{induction hypothesis}
\newcommand{\tfae}{the following are equivalent}


\newcommand{\sico}{simply connected}
\newcommand{\tcs}{topological cycle space}
\newcommand{\tocir}{topological circle}
\newcommand{\topa}{topological path}
\newcommand{\tet}{topological Euler tour}

\newcommand{\locon}{locally connected}
\newcommand{\pacon}{path connected}
\newcommand{\pl}{piecewise linear}
\newcommand{\ple}{piecewise linear embedding}
\newcommand{\lfl}{locally flat}

\newcommand{\HM}{Hausdorff measure}
\newcommand{\CtS}{Cantor 3-sphere}

\newcommand{\FC}{Freudenthal compactification}
\newcommand{\HMT}{Hahn-Mazurkiewicz theorem}
\newcommand{\vKT}{van Kampen's theorem}

\newcommand{\pd}{properly discontinuous}

\newcommand{\Rw}{Random walk}
\newcommand{\rw}{random walk}
\newcommand{\RW}{Random Walk}




\newcommand{\labequ}[2]{ \begin{equation} \label{#1} #2 \end{equation} } 
\newcommand{\labsplitequ}[2]{ \begin{equation} \begin{split}
 \label{#1} #2 \end{split} \end{equation} } 
\newcommand{\labtequ}[2]{\begin{equation} \label{#1} 	\begin{minipage}[c]{0.9\textwidth}  #2 \end{minipage} \ignorespacesafterend \end{equation} } 

\newcommand{\labtequc}[2]{ \begin{equation} \label{#1} 	\text{   #2 } \end{equation} } 

\newcommand{\labtequstar}[1]{ \begin{equation*}  	\begin{minipage}[c]{0.9\textwidth}  #1 \end{minipage} \ignorespacesafterend \end{equation*} }



\newcommand{\mymargin}[1]{\ifnum \Debug = 1
  \marginpar{\begin{minipage}{\marginparwidth}\small \begin{flushleft}{\color{blue}#1}\end{flushleft}\end{minipage}}\fi
}

\newcommand{\extras}[1]{\ifnum \Debug = 1
\section{Extras} #1
 \fi
}

\newcommand{\mySection}[2]{}


\newcommand{\RST}{Robertson--Seymour theorem}
\newcommand{\RSTc}{Robertson--Seymour theorem \cite{GMXX}}

\newcommand{\Erd}{Erd\H{o}s}
\newcommand{\DK}{Diestel and K\"uhn}
\newcommand{\KH}{Kirchhoff}
\newcommand{\RD}{Diestel}

\newcommand{\CDB}{\cite[Section~8.5]{diestelBook05}}
\newcommand{\DB}{\cite{diestelBook05}}


\newcommand{\LemArcC}{\cite[p.~208]{ElemTop}}
\newcommand{\LemArc}[1]{
	\begin{lemma}[\LemArcC]
	\label{#1}
	The image of a topological path with endpoints $x,y$ in a Hausdorff space $X$ contains an arc in $X$ between $x$ and $y$.
	\end{lemma}
}

\newcommand{\LemKonig}[1]{
	\begin{lemma}[K\"onig's Infinity Lemma \cite{InfLemma}]
  	\label{#1}
	  Let $V_0,V_1,\dotsc$ be an infinite sequence of disjoint non-empty finite sets, and let $G$ be a graph on their union. Assume that every vertex $v$ in a set $V_n$ with $n\ge 1$ has a neighbour in $V_{n-1}$. Then $G$ contains a ray $v_0v_1\dotsm$ with $v_n\in V_n$ for all $n$.
	\end{lemma}
}

\newcommand{\LemCombStarC}{\cite[Lemma~8.2.2]{diestelBook05}} \newcommand{\LemCombStar}[1]{
	\begin{lemma}[{\LemCombStarC}]
	\label{#1}
	Let $U$ be an infinite set of vertices in a connected graph $G$. Then $G$ contains either a ray $R$ and infinitely many pairwise disjoint \pths{U}{R} or a subdivided star with infinitely many leaves in $U$. 
	\end{lemma}
}

\newcommand{\LemCycDecC}{\cite[Theorem~8.5.8]{diestelBook05}} \newcommand{\LemCycDec}[1]{
	\begin{lemma}[\LemCycDecC]
	\label{#1}
	Every element of $\ccg$ is a disjoint union of circuits.
	\end{lemma}
}

\newcommand{\ThmComp}[1]{
	\begin{theorem}
	\label{#1}
Let $K$ be an infinite set of propositional formulas, every finite subset of which is satisfiable. Then $K$ is satisfiable.
	\end{theorem}
}
\newcommand{\ThmCompC}{\cite{LMCS}} 

\newcommand{\LemCutCrit}[1]{
	\begin{lemma}
	\label{#1}
	Let $F\subseteq E(G)$. Then $F\in \mathcal C(G)$ if and only if $F$ meets every finite cut in an even number of edges. 
	\end{lemma}
}
\newcommand{\LemCutCritC}{\cite[Theorem~8.5.8]{diestelBook05}} 

\newcommand{\LemFinCut}[1]{
	\begin{lemma}
	\label{#1}
	Let $G$ be a connected locally finite graph, $\{ X, Y \}$ a partition of $V(G)$, and $F := E(X, Y )$. 
		\begin{enumerate}
		\item $F$ is finite if and only if $\overline{X} \cap \overline{Y}=\emptyset$.
		\item If $F$ is finite, there is no arc in $|G|\sm \kreis{F}$ with one endpoint in $X$ and the other in $Y$. 
		\end{enumerate}
	\end{lemma}
}
\newcommand{\LemFinCutC}{\cite[Theorem~8.5.5]{diestelBook05}} 


\newcommand{\LemInjHomC}{\cite{armstrong}} \newcommand{\LemInjHom}[1]{
	\begin{lemma}[\LemInjHomC] \label{#1}
	A continuous bijection from a compact space to a Hausdorff space is a homeomorphism.
	\end{lemma}
}

\newcommand{\ThmMacLaneC}{\cite{maclane37}}
\newcommand{\ThmMacLane}[1]{
	\begin{theorem}[\ThmMacLaneC]
	\label{#1}
A finite graph is planar if and only if its cycle space \ccg\ has a simple generating set.
	\end{theorem}
}



\newcommand{\LemHeineCaC}{\cite{}} \newcommand{\LemHeineCa}[1]{
	\begin{lemma}[Heine-Cantor Theorem] \label{#1}
	Let $M$ be a compact metric space, and let $f: M \to N$ be a continuous function, where $N$ is a metric space. Then $f$ is uniformly continuous.	\end{lemma}
}


\newcommand{\ThmHaMaC}{\cite{}} \newcommand{\ThmHaMa}[1]{
	\begin{theorem}[\HMT \cite{nadler}] \label{#1}
A metric space is a continuous image of the unit real interval if and only if it is compact, connected, and locally connected.
\end{theorem}
} 
\begin{document}
	\title{The excluded minors for embeddability into a   compact surface}
	
\author{Agelos Georgakopoulos\thanks{Supported by  EPSRC grants EP/V048821/1 and EP/V009044/1.}}
\affil{  {Mathematics Institute, University of Warwick}\\
  {\small CV4 7AL, UK}}


\date{\today}


	
\maketitle


\begin{abstract}
We determine the excluded minors characterising the class of countable graphs that embed into some compact surface.            
\end{abstract}

\medskip

{\noindent {\bf Keywords:} \small excluded minor, graphs in surfaces, outerplanar, star-comb lemma.} 

\smallskip
{\noindent \small  {\bf MSC 2020 Classification:}} 05C63, 05C83, 05C10, 05C75. 




\section{Introduction}

The main aim of this paper is to provide the excluded minors characterising the class of countable graphs that embed into a compact surface, whereby we put no restriction on the genus. We will prove
\begin{theorem}\label{main thm}
A countable graph \g embeds into a compact (orientable) surface \iff\ it does not have one of the 8 graphs of \fig{figExSig} as a minor.\footnote{Every non-trivial infinite graph has many `minor-twins'; for example, for each pair $x,y$ of vertices of infinite degree in $\Sig_i, i\geq 5$, we could add or remove the $x$--$y$~edge.}
\end{theorem}
It is an exercise to show that none of these graphs is a minor of another. Since none of these graphs embeds into a closed surface, orientable or not, our theorem remains valid if we remove the word `orientable'. 

All graphs in this paper are countable. In the \lf\ case, only the first two obstructions $\Sig_1=\omdot K_5, \Sig_2=\omdot \Ktt$ are needed (see \Cr{cor lf}).

\begin{figure}[H]
\begin{center}
\begin{overpic}[width=.9\linewidth]{figExSig.pdf} 
\put(-4,45){$\Sig_1$}
\put(37,45){$\Sig_2$}
\put(-4,31){$\Sig_3$}
\put(37,31){$\Sig_4$}
\put(73,45){$\Sig_6$}
\put(65,26){$\Sig_7$}
\put(-4,13){$\Sig_5$}
\put(55,9){$\Sig_8$}
\end{overpic}
\end{center}
\caption{The excluded minors of \Sig\ as provided by \Tr{main thm}. \newline 
    \hspace{\linewidth} $\bf \Sig_1$ (respectively $\bf \Sig_2$): the disjoint union of infinitely many copies of $K_5$ (resp.\ \Ktt);  \newline 
    \hspace{\linewidth} $\bf \Sig_3$ (resp.\ $\bf \Sig_4$):  the graph arising from $\Sig_1$ (resp.\ $\Sig_2$) by picking  one vertex from each component and identifying them; \newline 
    \hspace{\linewidth} $\bf \Sig_5$ (resp.\ $\bf \Sig_6$): the graph arising from $\Sig_1$ (resp.\ $\Sig_2$) by picking  one edge from each component and identifying them; \newline 
    \hspace{\linewidth} $\bf \Sig_7$: the graph arising from $\Sig_2$ by picking  one pair of non-adjacent vertices from each component and identifying these pairs; and ${\bf \Sig_8 }= K_{3,\omega}$.
} \label{figExSig}
\end{figure}


An analogous statement for embeddings into a fixed surface is the following theorem of Robertson \& Seymour \cite{RobSeyKur} (previously announced in \cite{BKMM,christian_embedding_2015,FulKynGen}).  \Fe\ \nin, \ti\ $g$, \st\ for every graph \g of genus at least $g$, there is some $\Sig_i$ as in  \fig{figExSig}, \st\  any subgraph of size $n$ of $\Sig_i$ is a minor of \G. I do not see a way to deduce this from \Tr{main thm} or vice-versa. An important difference between these two results is that if we restrict to the orientable case, then we need to allow the $n\times n$ projective grid as a further obstruction to embeddability into a fixed surface, but \Tr{main thm} proves that we have the same excluded minors with or without the orientability restriction when allowing arbitrarily high genus.

\medskip
One of the tools for our proof of \Tr{main thm} is the following result of independent interest, saying that a graph embeds into a compact surface \iff\ it can be decomposed into finitely many planar subgraphs with finite pairwise intersections. 
\begin{theorem}\label{thm decomp Intro}
A countable graph embeds into a compact (orientable) surface \iff\ it admits a finitary decomposition into planar pieces.
\end{theorem}
See \Sr{sec decomp} for the precise definitions. \Tr{thm decomp Intro} supports the meta-conjecture that any result proved for infinite planar graphs generalises, rather easily, to graphs embeddable into a compact surface. Examples of such results include the main results of \cite{intersection,planarPB,UKtrans,HuNaUni,kozPPP}. See \Sr{sec impl} for more.

\medskip
Part of the motivation for \Tr{main thm} comes from a well-known conjecture of Thomas \cite{ThoWel} postulating that the countable graphs are well-quasi-ordered under the minor relation. The analogous statement for finite graphs is the celebrated Graph Minor Theorem of Robertson \& Seymour \cite{GMXX}. A positive answer to Thomas's conjecture would imply that every minor-closed class $\cc$ of countable graphs is characterised by forbidding a finite list $\ex{\cc}$ of \defi{excluded minors}. This is in general hard to show even for a concrete class of graphs like the class $\Sigma$ of \Tr{main thm}; indeed, it is a-priori not even clear that $\ex{\Sigma}$ is finite. Apart from the fact that $\Sigma$ is a natural class to consider, another reason why the finiteness of $\ex{\Sigma}$ is a pressing question if one is interested in Thomas's conjecture is the important role played by classes of finite graphs embeddable in a fixed surface in the proof of the Graph Minor Theorem.

Many natural minor-closed graph classes  \cc, e.g.\ the graphs embeddable into a fixed surface, have the property that a graph is in \cc\ as soon as every finite subgraph is. This has the consequence that  $\ex{\cc}$ coincides with the list of excluded minors of the subclass of \cc\ comprising its finite elements. Apart from such classes, there are very few classes \cc\ of infinite graphs for which $\ex{\cc}$ is explicitly known. The only example I am aware of are the graphs with accumulation-free embeddings in the plane \cite{HalSom}. 



Additional motivation for \Tr{main thm} comes from a question raised by Christian, Richter \& Salazar \cite{christian_embedding_2015}, asking for a characterisation of the Peano continua that embed into a closed surface analogous to Claytor's \cite{ClaPea} characterisation of the Peano continua embeddable into $\BS^2$. The special case of graph-like continua was handled in \cite{christian_embedding_2015}, and the characterisation obtained is similar to \Tr{main thm}. But the lack of compactness does not allow using that  characterisation to deduce \Tr{main thm}. Using \Tr{main thm} and the \scl\ (\Lr{SC lem} below) it is not difficult to determine the excluded \emph{topological} minors for embeddability into a compact surface, and this could be a first step towards answering the aforementioned question of Christian et al.\ \cite{christian_embedding_2015}. 

\medskip
Our proof of \Tr{main thm} is elementary (but involved), relying only on Kuratowski's theorem, and a classical result of Youngs about cellular embeddings of finite graphs. It is carried out mostly in \Srs{sec decomp}, \ref{sec UOP} and \ref{sec proof}. On the way to \Tr{main thm} we will develop techniques that allow us to find the excluded minors of families of infinite graphs that satisfy a property up to finitely many flaws: we will characterize the graphs that become forests after deleting, or contracting, finitely many edges  (\Srs{sec AF} and~\ref{sec AFR}), as well as the graphs that are \OP\ up to deleting finitely many edges (\Sr{sec OPE}).

\medskip
The \scl\ is one of the most useful tools in infinite graph theory. In \Sr{sec SC} we obtain the following strengthening for 2-connected graphs:

\begin{theorem} \label{SC 2con intro}
Let \g be a countable, 2-connected, graph, and $U\subseteq V(G)$ infinite. Then \g contains a subdivision of an infinite ladder, or of an infinite fan, or of $K_{2,\infty}$, having infinitely many vertices in $U$.
\end{theorem}
If \g is \lf, then this results in a ray in \g containing an infinite subset of $U$. 

\medskip
The aforementioned conjecture of Thomas was studied by Robertson, Seymour \& Thomas \cite{RoSeThExcI,RoSeThExcII}, and they concluded that {\it there is not much chance of proving} it, as it would have implications about the ordering of finite graphs. It is therefore natural to try to extend the Graph Minor Theorem to an intermediate level covering all finite graphs but not necessarily all countable ones. A concrete approach for doing so is offered by \Cnr{Con good} and other questions in \Sr{final} arising from our results and methods.


\section{Preliminaries} \label{prels}



We follow the terminology of Diestel \DB. We use $V(G)$ to denote the set of vertices, and $E(G)$ the set of edges of a graph \G. For $S\subseteq V(G)$, the subgraph $G[S]$ of \g \defi{induced} by $S$ has vertex set $S$ and contains all edges of \g with both end-vertices in $S$.

The \defi{degree} $d(v)=d_G(v)$ of a vertex $v$ in a graph \G, is the number of edges of \g incident with $v$. 

A \defi{ray} is a one-way infinite path. We say that \g is \defi{\lf}, if no vertex of \g lies in infinitely many edges.

\medskip

Let $G,H$ be graphs. An $H$ \defi{minor} of $G$ is a collection of disjoint connected subgraphs $B_v, v\in V(H)$ of $G$, called \defi{branch sets}, 
and edges $E_{uv}, uv\in E(H)$ of \g such that each $E_{uv}$ has one end-vertex in $B_u$ and one in $B_v$. We write $H<G$ to express that $G$ has an $H$ minor.







Given a set $X$ of graphs, we write $\forb{X}$ for the class of graphs $H$ \st\ no element of $X$ is a minor of $H$. 

A \defi{subdivision} of a graph \g is a graph obtained by replacing some of the edges of \g by paths with the same end-vertices. 

\medskip

A \defi{surface} is a connected 2-manifold without boundary.  An \defi{embedding}  of a countable graph \g\ into a surface $S$ is a map $f: G \to S$ from the 1-complex obtained from \g when identifying each edge with the interval $[0,1]$ to  $S$  \st\ the restriction of $f$ to each finite subgraph of $G$ is an embedding in the topological sense, i.e.\ a homeomorphism onto its image. (The reason why we restrict to finite subgraphs here is that the 1-complex topology of \g is not metrizable when \g is not \lf, and so such \g cannot have an embedding into a metrizable space $S$. For example, a star with infinitely many leaves admits an embedding into $\R^2$ in our sense but it does not admit a topological embedding. Let $\gamma(G)$ denote the minimum genus of an orientable surface into which a graph \g embeds.

The following is perhaps folklore, but we sketch a proof for completeness. The locally finite case has been proved by Mohar \cite[\S 5]{mohar88}. 
\begin{lemma} \label{lem emb cof}
Let \g be a  countable graph, and $S$ an orientable  surface. Then \g admits an embedding into $S$ if each of its finite subgraphs does.
\end{lemma}

When $S$ is the sphere, Dirac \& Shuster \cite{DiracSchuster} provide a proof  by an elementary compactness argument (which they atribute to \Erd).
Our proof is a combination of this with Youngs' \Tr{Youngs} below.

\begin{proof}\textcolor{red}{TOPROVE 2}\end{proof}

We let \defi{\Sig} denote the class of countable graphs that embed into a compact orientable surface. (We will prove that every graph embeddable into a compact non-orientable surface also embeds into a compact orientable one.)

\medskip
We let \defi{$\omega$} denote the smallest infinite ordinal.
\defi{}

\subsection{The \scl} \label{sec scl}
Given a graph \G, and an infinite set $U\subseteq V(G)$, we define a \defi{$U$-star} to be a subdivision of the infinite star $K_{1,\omega}$ in \g all leaves of which lie in $U$. We define a \defi{$U$-comb} to be the union of a ray $R$ of \g with infinitely many pairwise disjoint, possibly trivial, \pths{U}{R}. We call $R$ the \defi{spine} of $C$, and the \pths{U}{R}\ its \defi{teeth}.

\begin{lemma}[Star-comb lemma {\LemCombStarC}] \label{SC lem}
Let $U$ be an infinite set of vertices in a connected graph $G$. Then $G$ contains either a $U$-star or a $U$-comb.
\end{lemma}

\section{Decomposing into planar graphs} \label{sec decomp}

The aim of this section is to prove the orientable case of \Tr{thm decomp Intro}, which will be used as a tool for the proof of \Tr{main thm}. (The non-orientable case will follow after we have proved \Tr{main thm}.)

\begin{definition} \label{def fin dec}
A \defi{decomposition} of a graph \g is a family $(G_i)_{i\in \ci}$ of subgraphs of \G, called the \defi{pieces}, \st\ $G= \bigcup_{i\in \ci} G_i$. We say that a decomposition $(G_i)_{i\in \ci}$ is \defi{finitary}, if $\ci$ is finite, and the intersection of any two distinct pieces is finite. Note that this means that at most finitely many vertices of \g lie in more than one piece.
\end{definition}

Let us collect a few lemmas for the proof of \Tr{thm decomp Intro}, and for later use. 

\begin{lemma} \label{lem S}
Let  $G \in \Sig$ and suppose $G'$ is obtained from \g by identifying two vertices $v,w\in V(G)$. Then $G' \in \Sig$.
\end{lemma}
\begin{proof}\textcolor{red}{TOPROVE 3}\end{proof}

\comment{
	To prove this we need to start with a nice enough embedding of $G$, which we now introduce.

Let $g:G \to \Gamma$ be an embedding of a countable graph \g into a compact surface. It was proved in \cite[Proposition 4.3.]{Universal} that we may assume $g$ to be \defi{generous} in the following sense. We say that $g$ is \defi{generous around vertices}, if \fe\ $v\in V(G)$ there is a topological open disc $D_v \subset \Gamma$ \st\ $D_v \cap g(V(G)) =g(v)$ and $D_v$ avoids $g(uw)$ for every edge $uw$ with $u,w\neq v$. Similarly, we say that $g$ is \defi{generous around edges}, if \fe\ $e=uv\in E(G)$ there is a topological open disc $D_e \subset \Gamma$ \st\ $D_e \cap g(G) = g(e) \sm \{g(u),g(v)\}$. We call $g$ \defi{generous} if it is generous  both around vertices and around edges. 


\begin{proof}\textcolor{red}{TOPROVE 4}\end{proof}
}

The power of \Lr{lem S} lies in our ability to apply it repeatedly. This way we obtain
\begin{corollary} \label{cor S plus}
Let \g be a countable graph admitting a finitary decomposition $G_1,\ldots, G_k$. If each $G_i$ lies in \Sig, then so does \G. \end{corollary}
\begin{proof}\textcolor{red}{TOPROVE 5}\end{proof}

Our last lemma is a classical result of Youngs about cellular embeddings. A \defi{face} of an embedding $g: G \to \Gamma$ of a graph into a surface is a component of $\Gamma \sm g(G)$.

\begin{theorem}[\cite{Youngs}] \label{Youngs}
Let $\Gamma$ be a closed orientable surface, and let $G$ be a finite connected graph that embeds into $\Gamma$ but does not embed into a closed orientable surface of smaller genus. Then for every embedding $g: G \to \Gamma$, each face of $g$ is homeomorphic to an open disc. 
\end{theorem}

We are now ready for the proof of the main result of this section, which we restate for convenience:
\begin{theorem}\label{thm decomp}
A countable graph \g embeds into a compact, orientable, surface \iff\ it admits a finitary decomposition into planar pieces.
\end{theorem}
\begin{proof}\textcolor{red}{TOPROVE 6}\end{proof}

\subsection{Implications of \Tr{thm decomp}} \label{sec impl}

We remark that \Tr{thm decomp} allows us to extend many results obtained for planar graphs, e.g.\ those of \cite{intersection,planarPB,UKtrans}, to graphs in \Sig. Motivated by the fact that some such results (e.g.\ \cite{HuNaUni,kozPPP}) only apply to graphs with vertex-accumulation-free embeddings into the plane, we will now formulate and prove a refinement of \Tr{thm decomp} that takes accumulation points into account. This refinement is not needed for the proof of \Tr{main thm}, and the reader may skip the rest of this section.
\medskip

We let \defi{$\Sig^*$} denote the class of countable graphs \g that embed into a compact orientable surface $\Gamma$ so that there are at most finitely points of $\Gamma$ that are accumulation points of vertices of \G. We can always choose our embeddings so that no such accumulation point lies in the image of \G. We define \defi{$\pln^*$} analogously, with $\Gamma$ replaced by $\BS^2$. Finally, we  say that \g is Vertex-Accumulation-Free, or \defi{\vapf} for short, if it admits an embedding in $\R^2$ with no accumulation point of vertices. We will prove

\begin{corollary}\label{cor vapf}
A countable graph \g lies in $\Sig^*$ \iff\ it admits a finitary decomposition into \vapf\ pieces.
\end{corollary}

Our proof will be a combination of \Tr{thm decomp} with the following basic fact about \vapf\ graphs:
\begin{lemma}[{\cite[LEMMA~7.1]{thoPla}}] \label{lem vapf}
A countable graph $H$ is \vapf,  \iff\ some embedding $g: H\to \BS^2$ has the property that \fe\ cycle $C$ one of the two sides of $g(C)$ contains only finitely many vertices.
\end{lemma}


\begin{proof}\textcolor{red}{TOPROVE 7}\end{proof}


\section{Graphs that have a property up to finitely many flaws} \label{sec AF} 

This section introduces classes of graphs that have a property up to finitely many `flaws', and basic techniques for finding their excluded minors. This will suffice to prove the analogue of Theorem 1.1 for locally finite graphs.

Given a minor-closed family \cc\ of infinite graphs, one can define classes of graphs that are \defi{almost} in \cc\ in the following sense. 

\begin{definition} \label{def almost} Let \defi{\rmv{\cc}} (respectively, \defi{\rme{\cc}}) denote the class of graphs \G, \st\ by removing finitely many vertices (resp.\ edges) from \g we obtain a graph belonging to \cc. Similarly, we let  \defi{\rmce{\cc}} denote the graphs that belong to \cc\ after contracting finitely many edges. 

It is easy to see that $\rmce{(\rme{\cc})} = \rme{(\rmce{\cc})}$ \fe\ \cc, and we will simply write $\rmece{\cc}$ instead.
\end{definition}

The following examples show that neither of $\cc_{\mathrm{/E}},\cc_\mathrm{E}$ is contained in the other in general. 

Example 1: Let $M$ denote the graph consisting of a ray emanating from the centre of an infinite star $K_{1,\omega}$ (we could call $M$ the infinite mop), and let $\cc:= \forb{M}$. Then $\cc_{\mathrm{/E}}=\cc \subsetneq \cc_\mathrm{E}$, because $\cc_\mathrm{E}$ contains $M$ while $\cc$ does not. 

Example 2: It is easy to prove $\Sig= \Sig_\mathrm{E}$ similarly to \Lr{lem S}. But $\Sig \subsetneq \Sig_\mathrm{/E}$, because $K_{3,\omega} \in \Sig_\mathrm{/E}$.


\begin{definition} \label{def omdot}
We write \defi{$\omdot H$} for the disjoint union of countably infinitely many copies of a graph $H$. If $H$ is  vertex-transitive, we let \defi{$\bigvee H$} denote the graph obtained from $\omdot H$ by picking one vertex from each copy of $H$ and identifying them (by vertex-transitivity, it does not matter which vertices we pick).
\end{definition} 
 For example, $\bigvee K_3$ is a \defi{bouquet of triangles}, i.e.\ an infinite union of triangles having exactly one vertex in common.

\medskip

The next proposition provides the excluded minors of the class $\frsE$ of `almost forests' (the analogous result for $\frsCE$ is given in \Sr{sec AFR}).
Although it is not formally needed, we include it here as a gentle introduction to the techniques we will later use to prove our main result (\Tr{main thm}).


\begin{proposition} \label{prop FE}
Let $\frs$ denote the class of countable forests. Then $\frsE= \forb{\omdot K_3, \bigvee K_3, K_{2,\omega}}$.
\end{proposition}
\begin{proof}\textcolor{red}{TOPROVE 8}\end{proof}

The class $\frsV$ is easier to characterise in terms of excluded minors: we have $\frsV= \forb{\omdot K_3}$. This is a special case of the following helpful fact. Its main argument is well-known in the context of Andreae's ubiquity conjecture \cite{AndUb}.

\begin{proposition} \label{prop PV}
Let $\cp= \forb{H_1, \ldots, H_k}$ be a minor-closed class of countable graphs, where the $H_i$ are finite. Then $\cp_V= \forb{\omdot H_1, \ldots, \omdot H_k}$.
\end{proposition}
\begin{proof}\textcolor{red}{TOPROVE 9}\end{proof}

As an example application of \Prr{prop PV}, we deduce that $\plV=\forb{\omdot K_5, \omdot \Ktt}$, where we write \defi{$\mathrm{Planar}$} for the class of planar graphs. Since $\plV \subset \Sig_V$, and  $\omdot K_5, \omdot \Ktt \not\in \Sig_V$ (\cite{BHKY}; see also the proof of \Tr{main thm}) this yields

\begin{corollary} \label{cor sigV}
$\Sig_V=\plV=\forb{\omdot K_5, \omdot \Ktt}$.
\end{corollary}
An alternative proof of \Cr{cor sigV} can be obtained by using \Tr{thm decomp}: the latter implies that $\Sig \subset \plV$, by removing the intersections of the pieces of any finitary planar decomposition of $G\in \Sig$. This in turn implies $\Sig_V \subset (\plV)_V = \plV$, and so  $\Sig_V=\plV$ as the converse inclusion is trivial.


\medskip
As another corollary of \Prr{prop PV}, we obtain an easy proof of the analogue of \Tr{main thm} for \lf\ graphs:
\begin{corollary} \label{cor lf}
Let \g be a locally finite graph. Then $G\in \Sig$ unless 
$\omdot K_5 < G$ or $\omdot \Ktt<G$.
\end{corollary}
\begin{proof}\textcolor{red}{TOPROVE 10}\end{proof}


\begin{remark}
\textup{The aforementioned fact that $\frsV= \forb{\omdot K_3}$ can be thought of as the infinite version of the classical result of \Erd\ \& P\'osa \cite{ErdPosInd} saying that every finite graph has either a $k\cdot K_3$ minor or a set of at most $f(k)$ vertices the removal of which results into a forest. I do not expect there to be an easy way to deduce the one from the other, as this \Erd-P\'osa property fails for non-planar graphs in the finite case \cite{GMV}, while the infinite version holds for every finite graph by \Prr{prop PV}}.
\end{remark}

\subsection{Almost planar graphs} \label{sec AP}

The following is another consequence of \Tr{thm decomp} of independent interest. It is not needed for the proof of our other results, and the reader may skip to the next section. 
\begin{proposition} \label{prop AP}
$\rmce{\Sig} = \rmce{\pln} =\rmece{\pln}$.
\end{proposition}
\begin{proof}\textcolor{red}{TOPROVE 11}\end{proof}



\section{Outerplanar graphs and related classes} \label{sec UOP}

By \Prr{prop PV} and \Cr{cor sigV}, if a graph $G$ does not lie in $\rmv{\Sig}=\rmv{\rm{Planar}}$ then it has one of the desired minors, and so the most challenging part of our proof of \Tr{main thm} is to handle the case where \g becomes planar after removing a finite vertex set $W$. When $W$ is a single vertex, the latter is tantamount to saying that $G-W$ is planar but not \OP\ \defi{relative to} the neighbourhood of $W$. The aim of this section is to characterise graphs that are \OP\ relative to a vertex set $U$, and the analogous class for embeddings in any compact surface, in terms of forbidden minors that are \defi{marked} by $U$. The precise definitions follow.



\medskip
Let \g be a graph and $U\subset V(G)$. Define the \defi{$U$-cone} $C_U(G)$ of \g to be the graph obtained from \g by adding a new vertex $u$, the \defi{cone vertex}, and joining it to each vertex in $U$ with an edge. We say that $G$ is \defi{\uop}, if $C_U(G)$ is planar. If \g is finite, it is easy to see that \g is \uop\ \iff\ it admits an embedding into $\BS^2$ \st\ all vertices in $U$ lie on a common face-boundary. Moreover, by letting $U=V(G)$ we recover the standard notion of outerplanarity: \g is outerplanar \iff\ it is $V(G)$-outerplanar.

A \defi{marked graph} is a pair consisting of a graph \g and a subset $U$ of $V(G)$, called the \defi{marked vertices}. Given two marked graphs $(G,U), (H,U')$, an $H$ \defi{marked minor} of $G$ is defined just like an $H$ minor of $G$  (see \Sr{prels}), except that for each marked vertex $v$ of $H$, we require that the corresponding branch set $B_v$ contains at least one marked vertex of $G$. We write $(G,U) <(H,U')$ when this is possible.

Our next lemma adapts the well-known fact that the finite \OP\ graphs coincide with $\forb{K_4,K_{2,3}}$. 

\begin{lemma} \label{lem Oi}
Suppose \g is a countable planar graph, and let $U\subset V(G)$. Then \g is \uop\ \iff\ $(G,U)$ does not contain one of the marked graphs $(\Theta_i, U_i), 1\leq i \leq 4$ of \fig{figOis} as a marked minor. 
\end{lemma}
\begin{figure} 
\begin{center}
\begin{overpic}[width=.7\linewidth]{figOis.eps} 
\put(9,-3){$\Theta_1$}
\put(39,-3){$\Theta_2$}
\put(63.5,-3){$\Theta_3$}
\put(88,-3){$\Theta_4$}
\end{overpic}
\end{center}
\caption{The excluded \mm s for relative outerplanarity. The sets of marked vertices $U_i$ are shown in grey.} \label{figOis}
\end{figure}

\begin{proof}\textcolor{red}{TOPROVE 12}\end{proof}

\begin{remark}
It follows from \Lr{lem Oi} that \g is \uop\ as soon as each of its finite subgraphs is.
\end{remark}



We now introduce a generalisation of \OP ity to arbitrary surfaces that will play an important role in the proof of \Tr{main thm}: 
\begin{definition} \label{def SU}
We say that a marked graph $(G,U)$ lies in \defi{\SU}, and write $(G,U)\in \SU$, if $C_U(G)\in \Sig$. \end{definition}

\begin{lemma} \label{lem SU}
Suppose $\g\in \Sig$ is a countable graph, and $(G,U)$ is not in \SU\ for some $U\subset V(G)$. Then $(G,U)$ contains one of the following marked graphs as a marked minor: \\
the graphs
$\Phi_i, 1\leq i \leq 5$, $\Phi'_i, 2\leq i \leq 4$ of \fig{figUis}, or \\ the graphs $\omega \cdot \Theta_i, 1\leq i \leq 4$, with $\Theta_i$ as in \fig{figOis}.
\end{lemma}
\begin{figure} 
\begin{center}
\begin{overpic}[width=1\linewidth]{figUis.pdf} 
\put(5,29){$\Phi_1$}
\put(26,29){$\Phi_2$}
\put(49,29){$\Phi_3$}
\put(70,29){$\Phi_4$}
\put(93,29){$\Phi_5$}
\put(26,0){$\Phi'_2$}
\put(49,0){$\Phi'_3$}
\put(70,0){$\Phi'_4$}
\end{overpic}
\end{center}
\caption{Some of the excluded \mm s of \SU. The grey vertices represent the marked ones.} \label{figUis}
\end{figure}

\Lr{lem SU} is the technically most challenging part of the proof of \Tr{main thm}. We prepare its proof with a number of lemmas. The first one is similar to \Cr{cor S plus}.

\begin{lemma} \label{cor SU}
Let \g be a countable graph admitting a finitary decomposition $G_1,\ldots, G_k$. If each $(G_i,U\cap V(G_i))$ lies in \SU\ for some $U\subseteq V(G)$, then $G\in \SU$.
\end{lemma}
\begin{proof}\textcolor{red}{TOPROVE 13}\end{proof}


Using this, we can now extend \Lr{lem Oi} from planar graphs to graphs in \Sig:

\begin{lemma} \label{lem STh}
Let \g be a  graph in \Sig, and let $U\subset V(G)$. Then $G$ lies in  $\SU$ \iff\ it does not contain one of the marked graphs $(\Theta_i,U_i), 1\leq i \leq 4$ of \fig{figOis} as a marked minor. 
\end{lemma}
\begin{proof}\textcolor{red}{TOPROVE 14}\end{proof}

Given a marked graph $(G,U)$ with $G\in \Sig$, and a vertex $x\in V(G)$, we say that $x$ is \defi{\SU-critical}, if $(G,U)$ is not in $\SU$ but $(G-x,U-x)$ is. The most difficult part of the proof of \Lr{lem SU} lies in finding a $\Phi_5$ minor in the case where \g contains at least two \SU-critical vertices. This is achieved (and refined) by the following two lemmas. Recall the definition of a $U$-star from \Sr{sec scl}.

\begin{lemma} \label{lem SU star}
Suppose $G \in \Sig$, and $x\in V(G)$ is \SU-critical for some $U\subseteq V(G)$. Then \g contains a $U$-star with $x$ as the infinite-degree vertex.
\end{lemma}
\begin{proof}\textcolor{red}{TOPROVE 15}\end{proof}

We use \Lr{lem SU star} in order to prove 

\begin{lemma} \label{lem SU dstar}
Suppose $G \in \Sig$, and \g has two \SU-critical vertices $x,y$ for some $U\subseteq V(G)$. Then \g contains  the marked double-star $\Phi_5$ as a \mm. \end{lemma}
\begin{proof}\textcolor{red}{TOPROVE 16}\end{proof}

We can now prove the main result of this section:

\begin{proof}\textcolor{red}{TOPROVE 17}\end{proof}

\begin{remark}
\textup{The converse of \Lr{lem SU} holds too, that is, if \g has one of the \umm s as in the statement, then \g is not in \SU. Indeed, the $U$-cone of each of these graphs is not in \Sig,  because such a cone contains one of the forbidden structures of \Tr{main thm} as we will see in the proof of \Tr{main thm}.} 
\end{remark}








\section{The excluded minors of \Sig} \label{sec proof}



We can now prove our main result.
\begin{proof}\textcolor{red}{TOPROVE 18}\end{proof}



\section{Almost outerplanar graphs} \label{sec OPE}

The aim of this section is to prove the analogue of \Prr{prop FE} for \OP\ graphs. This is included as a result of independent interest, proved using some of the techniques developed above. 

\medskip
Let \defi{\OuPl} denote the class of countable \OP\ graphs.
Let $G_1$ (respectively, $G_2$) be the graph obtained from $\omdot K_{2,3}$ by choosing a vertex of degree 3 (resp.\ degree 2) from each copy of $K_{2,3}$ and identifying them (\fig{figG1G2}). \begin{figure} 
\begin{center}
\begin{overpic}[width=.6\linewidth]{figG1G2.pdf} 
\put(11,0){$G_1$}
\put(72,0){$G_2$}
\end{overpic}
\end{center}
\caption{Two of the excluded minors of \Prr{prop OP}, arising by combining infinitely many copies of $K_{2,3}$.} \label{figG1G2}
\end{figure}

\begin{proposition} \label{prop OP}
$\rme{OP}= \forb{\omdot K_4, \omdot K_{2,3}, \bigvee K_4, G_1,G_2, K_{2,\omega}}$.
\end{proposition}


\begin{proof}\textcolor{red}{TOPROVE 19}\end{proof}

 
\section{Almost forests revisited} \label{sec AFR}

The aim of this section is to prove the following result, which complements \Prr{prop FE}.

\begin{proposition} \label{prop FCE}
Let $\frs$ denote the class of countable forests. Then $\frsCE= \forb{\omdot K_3, \bigvee K_3} = \rmece{\frs}$.
\end{proposition}
Combined with \Prr{prop FE}, it follows that $\frsE \subsetneq \frsCE$. For our proof we will need the following extension of the star-comb lemma.

A \defi{2-star} is a graph obtained from the star $K_{1,\omega}$ by subdividing each edge at least once. In other words, a 2-star is obtained from the disjoint union of infinitely many paths of length at least 2 by identifying their first vertices.

We say that a vertex-set $D$ \defi{dominates} another vertex-set $U$ of a graph, if $\cls{N}(D) \supseteq U$, where  $\cls{N}(D)$ consists of $D$ and all vertices sending an edge to $D$.

\begin{lemma} \label{2star comb}
Let $G$ be a connected graph, and $U\subseteq V(G)$. Then \g contains at least one of the following:
\begin{enumerate}
	\item \label{SC i} A $U$-comb;
	\item \label{SC ii} a 2-star with leaves in $U$;
	\item \label{SC iii} a finite vertex-set dominating $U$.
\end{enumerate}
\end{lemma}
\begin{proof}\textcolor{red}{TOPROVE 20}\end{proof}



\begin{proof}\textcolor{red}{TOPROVE 21}\end{proof}

\section{A star-comb lemma for 2-connected graphs} \label{sec SC}

While trying to prove \Tr{main thm} I came up with the following strengthening of the \scl\ for 2-connected graphs. Although it is not used for any of our proofs, I decided to include as it might become useful elsewhere. The \scl\ is one of the most useful tools in infinite graph theory. Some other strengthenings were obtained in a recent series of 4 papers by B\"urger \& Kurkofka \cite{BurKurDuaI}--\cite{BurKurDuaIV}. A related result determining unavoidable induced subgraphs for infinite 2-connected graphs is obtained by Allred, Ding \& Oporowski \cite{AlDiOpUna}. 

In analogy with $U$-stars and $U$-combs as in the statement of the \scl, we introduce the following structures. A \defi{double-star} is a subdivision of $K_{2,\omega}$. A \defi{ladder} consists of two disjoint rays $R,L$ and an infinite collection of pairwise disjoint \pths{R}{L}. A \defi{fan} consists of a ray $R$, a vertex $d\not\in V(R)$, and an infinite collection of \pths{d}{R}\ having only $d$ in common. \mymargin{(\fig{})} For each of these three terms, adding the prefix \defi{$U$-} means that the structure has infinitely many of its vertices in $U$. With this terminology, \Tr{SC 2con intro} from the introduction can be formulated as follows.


\begin{theorem} \label{SC 2con}
Let \g be a 2-connected graph, and $U\subseteq V(G)$ infinite. Then \g contains a $U$-double-star, or a $U$-ladder, or a $U$-fan. 
\end{theorem}




The following follows from the statement of \Tr{SC 2con}, but we need to prove it first as a first step towards the proof of the latter.
\begin{lemma} \label{comb 2con}
Let \g be a 2-connected, \lf\ graph, and $U\subseteq V(G)$ infinite. Then \g has a ray containing an infinite subset of $U$.
\end{lemma}

In this section we assume that the reader is familiar with the basics about the end-compac\-ti\-fication of a graph, and normal spanning trees; we refer to \cite{diestelBook05} therefor.
\begin{proof}\textcolor{red}{TOPROVE 22}\end{proof}

\begin{proof}\textcolor{red}{TOPROVE 23}\end{proof}

\begin{problem} Is it possible to generalise \Tr{SC 2con} to $k$-connected graphs, obtaining a finite list of subdivisions of $k$-connected graphs as unavoidable structures?
\end{problem} 

Results of similar flavour have been obtained by Gollin \& Heuer \cite{GolHeuCha}.

\section{Final remarks} \label{final}

It would be interesting to find the excluded minors for the classes $\rmce{\Sig}$, $\rme{\mathrm{Planar}}$, $\rmce{\mathrm{Planar}}$ and $\rmece{\mathrm{Planar}}$, and this should be within reach with the above methods and a little bit more work. I suspect that $$\rmce{\Sig}= \rmce{\mathrm{Planar}} = \rmece{\mathrm{Planar}} =\forb{\omdot K_5, \omdot \Ktt, \bigvee K_5, \bigvee \Ktt}.$$ The first two equalities have been proved in \Prr{prop AP}. I also suspect that  
$\rme{\mathrm{Planar}}=\forb{ \ex{\Sig} \cup \{\prl{K_5}, \prl{\Ktt} \}}$, where $\prl{K}$ is obtained from a graph $K$ by replacing each edge $uv$ by infinitely many \pths{u}{v}\ of length 2.

\comment{
	\begin{question} \label{Q SigCE}
Is $\rmce{\Sig}=\forb{\omdot K_5, \omdot \Ktt, \bigvee K_5, \bigvee \Ktt}$?
\end{question}
}

\medskip

Let us say that a minor-closed class \cc\ of graphs is \defi{good}, if $\cc=\forb{X}$ for a finite set $X$ of (possibly infinite) graphs. A well-know conjecture of Thomas \cite{ThoWel} postulates that the countable graphs are \wqo\ under the minor relation. A positive answer would imply that all minor-closed classes of countable graphs are good, but as mentioned in the introduction, this seems out of reach at the moment.  Still, we could seek to extend the Graph Minor Theorem \cite{GMXX} by finding sufficient conditions for classes of infinite graphs to be good. The following questions suggest a possible direction, and the methods of this paper  could be helpful. For further questions in a similar vein see \cite{Universal}.


\begin{question} \label{Q good}
Suppose \cc\ is a good minor-closed class of countable graphs. Must each of $\rmv{\cc},\rme{\cc},\rmce{\cc},\rmece{\cc}$ be good?
\end{question}



We say that a class \cc\ of graphs is \defi{\cof}, if $\cc = \forb{S}$ for a set $S$ of finite graphs (which set can be chosen to be finite by the Graph Minor Theorem \cite{GMXX}). Note that a graph \g belongs to such a class \cc\ \iff\ every finite minor of \g does. \Qr{Q good} is open in general even if \cc\ is \cof, except that $\rmv{\cc}$ is covered by \Prr{prop PV} in this case. This papers provides some techniques for attacking it. In a similar spirit, one can ask whether the class of graphs admitting a finitary decomposition into graphs in $\cc$ is good whenever $\cc$ is good/\cof.

We say that a class \cc\ of graphs is \defi{\uncof} (Union of Nested \Cof\ classes), if there is a sequence \seq{C} of \cof\ classes \st\ $\cc=\bigcup_{\nin} C_n$ and $C_n \subseteq C_{n+1}$ holds \fe\ $\nin$. The classes studied in this paper ($\Sig,\frsE,\frsCE, \ope$, etc.) are easily seen to be \uncof. Our results support


\begin{conjecture} \label{Con good}
Every \uncof\ class of countable graphs is good.
\end{conjecture}

Another interesting example of an \uncof\ class \cc\ comprises the graphs \g of finite Colin de Verdi\`ere invariant $\mu(G)$, whereby for infinite \g we define  $\mu(G)$ to be the supremal $m$ \st\ every finite subgraph $H\subset G$ satisfies $\mu(H)\leq m$. Is this \cc\ good? Can we determine $\ex{\cc}$? 

Not every proper minor-closed class is \uncof. For example, $\forb{K_\omega}$ is not, because it contains the disjoint union of $K_n, \nin$, which no proper \cof\ class contains. Thus \Cnr{Con good} is weaker than Thomas' conjecture. Beware however that \Cnr{Con good} implies the Graph Minor Theorem: any minor-closed class of finite graphs is shown to be \uncof\ by letting $C_n$ be its sub-class comprising the elements with at most $n$ vertices.

\acknowledgement{I thank Nathan Bowler and Max Pitz for spotting a mistake in an earlier version of the paper. I thank the anonymous referees for proposing several substantial improvements.}

\comment{
	\begin{lemma} \label{lem}


\end{lemma}
\begin{proof}\textcolor{red}{TOPROVE 24}\end{proof}

\begin{problem} \label{}

\end{problem} 
}







\bibliographystyle{plain}
\bibliography{collective}

\extras{

We generalise 
\begin{lemma}[\cite{halin74}] \label{LemHal}
Let \g be a countable, 2-connected, \lf\ graph, and $\chi$ an end of \G. Then \g contains two disjoint rays belonging to $\chi$.
\end{lemma}

... as follows: 

\begin{lemma} \label{LemLadFan}
Let \g be a countable, 2-connected, graph, and $\chi$ an end of \G. Then \g contains a subdivision of one of the graphs of \fig{} converging to $\chi$, i.e. a ladder or an infinite fan.
\end{lemma}
\begin{proof}\textcolor{red}{TOPROVE 25}\end{proof}


\begin{figure} 
\begin{center}
\end{center}
\caption{A \defi{ladder}, and a \defi{dominated ray}.} \label{figLadStar}
\end{figure}

}

\end{document}