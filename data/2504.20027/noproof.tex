\newif\iflongversion
\longversiontrue 



\iflongversion

  \documentclass[11pt]{article}
  \title{All-Subsets Important Separators with Applications to Sample Sets, Balanced Separators and Vertex Sparsifiers in Directed Graphs}
  \author{ {Aditya Anand \thanks{University of Michigan, Ann Arbor} }\and {Euiwoong Lee \thanks{University of Michigan, Ann Arbor. Supported in part by NSF grant CCF-2236669 and Google}}
  \and {Jason Li \thanks{Carnegie Mellon University}}
  \and {Thatchaphol Saranurak\thanks{
        University of Michigan,
        \texttt{thsa@umich.edu}.
        Supported by NSF Grant CCF-2238138. Partially funded by the Ministry of Education and Science of Bulgaria's support for INSAIT, Sofia University ``St.~Kliment Ohridski'' as part of the Bulgarian National Roadmap for Research Infrastructure.
    }}}
    \date{}
  
  \usepackage{amsthm}
  \usepackage{thm-restate}
  \newtheorem{theorem}{Theorem}[section]
\newtheorem{lemma}[theorem]{Lemma}
\newtheorem{observation}[theorem]{Observation}
\newtheorem{proposition}[theorem]{Proposition}
\newtheorem{definition}[theorem]{Definition}
\newtheorem{corollary}[theorem]{Corollary}
\newtheorem{claim}[theorem]{Claim}

\usepackage[margin=1in]{geometry}
\usepackage[linktocpage=true,
pagebackref=true,colorlinks,
linkcolor=DarkRed,citecolor=ForestGreen,
bookmarks,bookmarksopen,bookmarksnumbered]
{hyperref}

\else
  \documentclass[a4paper,UKenglish,cleveref,autoref,thm-restate]{lipics-v2021}
  \title{All-Subsets Important Separators with Applications to Sample Sets, Balanced Separators and Vertex Sparsifiers in Directed Graphs}
 \titlerunning{All-Subsets Important Separators with Applications}
 \author{Aditya Anand}{University of Michigan Ann Arbor, USA}{adanand@umich.edu}{}{}
\author{Euiwoong Lee}{University of Michigan Ann Arbor, USA}{euiwoong@umich.edu}{}{}
\author{Jason Li}{Carnegie Mellon University, USA}{jmli@alumni.cmu.edu}{}{}
\author{Thatchaphol Saranurak}{University of Michigan Ann Arbor, USA}{thsa@umich.edu}{}{}
\authorrunning{A. Anand, E. Lee, J. Li and T. Saranurak}

\Copyright{Aditya Anand, Euiwoong Lee, Jason Li and Thatchaphol Saranurak}

\ccsdesc[500]{Theory of computation~Fixed parameter tractability}
\ccsdesc[500]{Theory of computation~Graph algorithms analysis}
\keywords{directed graphs, important separators, sample sets, balanced separators, vertex sparsifiers}
\category{}                                          \relatedversion{}                                    \funding{Euiwoong Lee is supported in part by NSF grant CCF-2236669 and Google.}    
\EventEditors{Keren Censor-Hillel, Fabrizio Grandoni, Joel Ouaknine, and Gabriele Puppis}
\EventNoEds{4}
\EventLongTitle{52nd International Colloquium on Automata, Languages, and Programming (ICALP 2025)}
\EventShortTitle{ICALP 2025}
\EventAcronym{ICALP}
\EventYear{2025}
\EventDate{July 8--11, 2025}
\EventLocation{Aarhus, Denmark}
\EventLogo{}
\SeriesVolume{334}
\ArticleNo{35}
\fi

\newif\iflongversion
\longversiontrue 


\newcommand{\lv}[1]{\iflongversion #1 \fi}  \newcommand{\sv}[1]{\iflongversion\else #1\fi} 



\usepackage{tikz}
\usetikzlibrary{calc,fit,positioning,arrows.meta,backgrounds}







\usepackage{graphicx} \usepackage{xcolor}
\definecolor{ForestGreen}{rgb}{0.1333,0.5451,0.1333}
\definecolor{DarkRed}{rgb}{0.65,0,0}
\definecolor{Red}{rgb}{1,0,0}
\lv{}

\usepackage{enumerate}

\usepackage{tikz}
\usetikzlibrary{shapes, arrows.meta, positioning}
\usepackage{amssymb}
\usepackage{amsmath}
\usepackage{thm-restate}

\usepackage{algpseudocode}
\usepackage{algorithm}
\usepackage{float}
\usepackage{tablefootnote}
\usepackage{longtable}

\usepackage{xspace}
\usepackage[colorinlistoftodos,textsize=tiny,textwidth=2cm,color=red!25!white,obeyFinal]{todonotes}
\usepackage{cleveref}


\ifdefined\DEBUG
    \newcommand{\thnote}[1]{\todo[color=red!25!white]{TS: #1}\xspace}
 
 \newcommand{\enote}[1]{\todo[color=green!25!white]{EL: #1}\xspace}
  \newcommand{\anote}[1]{\todo[color=blue!25!white]{AA: #1}\xspace}
\else
\newcommand{\enote}[1]{}
\newcommand{\anote}[1]{}
\newcommand{\thnote}[1]{}
\fi




\newtheorem{hypothesis}[theorem]{Hypothesis}

\newcommand{\poly}{\mathrm{poly}\xspace}
\newcommand{\polylog}{\mathrm{polylog}\xspace}
\newcommand{\vol}{\mathrm{vol}}
\newcommand{\Reach}{\mathrm{Reach}\xspace}
\newcommand{\CC}{\mathcal{C}}
\newcommand{\MM}{\mathcal{M}}
\renewcommand{\SS}{\mathcal{S}}
\newcommand{\OO}{\mathcal{O}}
\newcommand{\OTIL}{\tilde{ \mathcal{O}}}
\newcommand{\IP}{\textbf{Input: }}
\newcommand{\OP}{\textbf{Output: }}
\newcommand{\IC}{{\textsc{ImproveCut}}}
\newcommand{\SC}{{\textsc{Sparsest Cut}}}
\newcommand{\VSC}{{\textsc{Vertex Sparsest Cut}}}
\newcommand{\SSVE}{{\textsc{Small Set Vertex Expansion}}}
\newcommand{\SSE}{{\textsc{Small Set Expansion}}}
\newcommand{\DSH}{{\textsc{Densest $k$-SubHypergraph}}}
\newcommand{\DKS}{{\textsc{Densest $k$-Subgraph}}}
\newcommand{\MMGP}{{\textsc{Min-Max Graph Partitioning}}}
\newcommand{\DBS}{{\textsc{Directed Bisection}}{}}
\newcommand{\MB}{{\textsc{Minimum Bisection}}{}}
\newcommand{\BS}{{\textsc{Balanced Separator}}}
\newcommand{\OC}{{\textsc{Oneway Cuts}}}

\renewcommand{\AA}{\mathcal{A}}
\newcommand{\STE}{{\textsc{Small Set Terminal Expansion}}}
\newcommand{\DB}{{\textsc{Directed Balanced Separator}}}
\newcommand{\SMC}{{\textsc{Skew Separator}}}
\newcommand{\DFAS}{{\textsc{Directed Feedback Arc Set}}}
\newcommand{\SMCE}{{\textsc{Skew Edge Separator}}}

\newcommand{\eps}{\epsilon}
\newcommand{\Wahlstrom}{Wahlstr\"{o}m\xspace}



\newcommand{\defparproblem}[4]{
  \vspace{1mm}
\begin{center}
\noindent\fbox{

  \begin{minipage}{0.85\textwidth}
  \begin{tabular*}{\textwidth}{@{\extracolsep{\fill}}lr} \textsc{#1}  & {\bf{Parameter:}} #3 \\ \end{tabular*}
  {\bf{Input:}} #2  \\
  {\bf{Question:}} #4
  \end{minipage}
 
  }
  \end{center}
 }







\begin{document}

\maketitle
\pagenumbering{gobble}

\begin{abstract}
Given a directed graph $G$ with $n$ vertices and $m$ edges, a parameter $k$ and two disjoint subsets $S,T \subseteq V(G)$, we show that the number of \emph{all-subsets important separators}, which is the number of $A$-$B$ important vertex separators of size at most $k$ over all $A \subseteq S$ and $B \subseteq T$, is at most $\beta(|S|, |T|, k) = 4^k {|S| \choose \leq k} {|T| \choose \leq 2k}$, where ${x \choose \leq c} = \sum_{i = 1}^c {x \choose i}$, and that they can be enumerated in time $\OO(\beta(|S|,|T|,k)k^2(m+n))$. This is a generalization of the folklore result stating that the number of $A$-$B$ important separators for two fixed sets $A$ and $B$ is at most $4^k$ (first implicitly shown by Chen, Liu and Lu Algorithmica '09). From this result, we obtain the following applications: 

\begin{enumerate}
    \item We give a construction for detection sets and sample sets in directed graphs, generalizing the results of Kleinberg (Internet Mathematics' 03) and Feige and Mahdian (STOC' 06) to directed graphs. 
    \item Via our new sample sets, we give the first FPT algorithm for finding balanced separators in directed graphs parameterized by $k$, the size of the separator. Our algorithm runs in time $2^{\OO(k)} \cdot (m + n)$. 
    \item Additionally, we show a $\OO(\sqrt{\log k})$ approximation algorithm for finding balanced separators in directed graphs in polynomial time. This improves the best known approximation guarantee of $\OO(\sqrt{\log n})$ and matches the known guarantee in undirected graphs by Feige, Hajiaghayi and Lee~(SICOMP' 08).
    \item Finally, using our algorithm for listing all-subsets important separators, we give a deterministic construction of vertex cut sparsifiers in directed graphs when we are interested in preserving min-cuts of size upto $c$ between bipartitions of the terminal set. Our algorithm constructs a sparsifier of size $\OO\left({t \choose \leq 3c}2^{\OO(c)}\right)$ and runs in time $\OO\left({t \choose \leq 3c} 2^{\OO(c)}(m + n)\right)$, where $t$ is the number of terminals, and the sparsifier additionally preserves the set of important separators of size at most $c$ between bipartitions of the terminals.
\end{enumerate}



\end{abstract}
 \lv{
\newpage
\tableofcontents
\newpage
}

\pagenumbering{arabic}





\newpage
\section{Introduction}


The study of parameterized algorithms, or fixed-parameter tractability (FPT) (see~\cite{cygan2015parameterized,downey2013fundamentals}) along with the classical area of approximation algorithms, has emerged as one of the most promising ways to cope with NP-completeness. Given a decision problem $Q$ with input size $n$, together with a parameter $\ell$, a parameterized (or FPT) algorithm is an algorithm running in time $f(\ell) n^{\OO(1)}$ that decides $Q$. Over the last few years, parameterized algorithms have been studied for most well-studied NP-complete problems. A particular focus has been on graph cut optimization problems, including {\textsc{Multiway Cut}}, {\textsc{Multicut}}, {\textsc{Minimum Bisection}}, {\textsc{Feedback Vertex Set}}, {\textsc{Balanced Separator}}~\cite{marx2006parameterized,fm06,chen2008fixed,marx2011fixed,cygan2014minimum,cygan2020randomized,li2022detecting}.
One of the most important tools for graph cut problems has been the technique of \emph{important separators} (see~\cite{marx2011important} for a brief survey), which has formed a building block in parameterized algorithms for many of these and other problems~\cite{marx2006parameterized,chen2008fixed,chen2009improved, chitnis2013fixed, lokshtanov2013clustering,lokshtanov2021fpt}.

In this work, our contribution is two-fold. First, we show a new structural result on important separators. Next, we show that using this result, combined with other techniques, leads to interesting consequences. We show that one can compute \emph{sample sets} in directed graphs. This in turn allows us to obtain both an FPT algorithm and an improved approximation algorithm for finding small balanced separators in directed graphs. We also show a deterministic construction of vertex cut sparsifiers in directed graphs. All our algorithms are simple and concise modulo standard results in approximation and parameterized algorithms. 






\begin{figure}[H]


\resizebox{\textwidth}{!}{\begin{tikzpicture}[
			node distance = 1cm and 0.5cm,
			mynode/.style={draw, align=left, minimum height=2em, minimum width=3em},
			mybignode/.style={ align=left, minimum height=2em, minimum width=3em},
			myarrow/.style={-Stealth},
			mylabel/.style={font=\scriptsize\itshape}
			]
			
			\node (combinatorics) [mybignode, font=\bfseries\Large, align=center, text width=5cm] {Combinatorics};
			
			\node (counting) [mynode, right=of combinatorics, text width=9cm] {\textbf{Counting important separators}\\ \emph{known}: between fixed source/sink vertices \\ \emph{this work}: between all subsets of terminals};
			\node (sample) [mynode, right=of counting, text width=8cm] {\textbf{Detection sets and sample sets}\\ \emph{known}: undirected graphs \\ \emph{this work}: directed graphs};
			\node (mimick) [mynode, below=of counting, text width=9cm] {\textbf{Deterministic directed vertex cut sparsifiers}\\ \emph{known}: $\OO(|T|^3)$ size, $2^{\OO(|T|^2)}(m + n)$ time\\ \emph{this work}: 1. $\OO({|T| \choose 3c})$ size, $\OO({|T| \choose 3c}(m + n))$ time \\2. $\OO(|T|^3)$ size, $(\frac{|T|}{3c})^{\OO(c|T|)}(m + n)$ time};
			\node (separator) [mynode, right=of mimick, text width=8cm] {\textbf{Directed balanced separators}\\ \emph{known}: \(O(\sqrt{\log n})\)-approximation\\ \emph{this work}: first FPT algorithm also running in linear time, \(O(\sqrt{\log k})\)-approximation};
			
			\node (algorithms) [mybignode,  left=of mimick, font=\bfseries\Large, align=center, text width=5cm]{Algorithms};
			
			\draw[myarrow] (counting) -- (sample);
			\draw[myarrow] (counting) -- (mimick);
			\draw[myarrow] (sample) -- (separator);
			
			\draw[dashed] ([xshift=1cm,yshift=-0.75cm]combinatorics.south west) -- ([xshift=19cm,yshift=-0.75cm]combinatorics.south east);
			
			
		\end{tikzpicture}
		}

 
\caption{Summary of contribution of our paper}

\end{figure}


\subsection{All-Subsets Important Separators}
Important separators have proved to be a very important tool in the design of many FPT algorithms, including fundamental problems such as {\textsc{Multiway Cut}}, {\textsc{Multicut}}, {\textsc{Directed Feedback Arc Set}}.
We refer to Chapter 8 of~\cite{cygan2015parameterized} for various applications of important separators to design parameterized algorithms. Given two disjoint subsets of vertices $A, B$ and another set of vertices $X$ (which may intersect $A,B$) in a directed graph, we say that $X$ is an $A$-$B$ separator if there is no directed path from any vertex of $A$ to any vertex of $B$ in $G \setminus X$\footnote{$G \setminus X$ denotes the graph obtained from $G$ by deleting the vertex set $X$ along with its incident edges}. For such a separator $X$, define $R(X)$ to be the set of vertices reachable from $A$ in $G \setminus X$. Then an $A$-$B$ separator $X$ is called an important $A$-$B$ separator if it is (a) inclusion-wise minimal - so that for any $v \in X$, $X \setminus \{v\}$ is not an $A$-$B$ separator and (b) for any other $A$-$B$ separator $Y \subseteq V(G)$ with $|Y| \leq |X|$, we do not have $R(X) \subset R(Y)$.\footnote{Throughout the paper, we use the notation $P \subset Q$ to mean that $P$ is a proper subset of $Q$.} Informally, important separators are those for which there is no other separator which is further away from $A$ without increasing the cut-size. Marx~\cite{marx2006parameterized} first introduced the notion of important separators and showed a bound of $4^{k^2}$ on the number of important $A$-$B$ separators of size at most $k$. Later this bound was improved to $4^k$ (implicit in the FPT algorithm for \textsc{Vertex Multiway Cut} by Chen, Liu and Lu~\cite{chen2009improved})\footnote{While these results focused mostly on the undirected version, the same proof works in directed graphs.}. Over the last few years, this result has been used extensively to obtain FPT algorithms for various central parameterized problems~\cite{marx2011fixed, chitnis2013fixed,cygan2014minimum,lokshtanov2021fpt}.
In this work, we show a new structural result on important separators in directed graphs: Given a directed graph $G$ and two disjoint subsets $S,T \subseteq V(G)$, we show that the number of $A$-$B$ important separators across all $A \subseteq S$ and $B \subseteq T$ is at most $\beta(|S|, |T|, k) = 4^k {|S| \choose \leq k} {|T| \choose \leq 2k}$, where ${n \choose \leq a} = \sum_{i = 1}^{a} {n \choose i}$. Note that the trivial bound is $4^k 2^{|S| + |T|}$, which follows directly from the fact that there are at most $4^k$ important $A$-$B$ separators for any fixed $A \subseteq S, B \subseteq T$. Previously, no such result is known even for undirected graphs. 

Throughout this paper, given a graph, we will denote by $n$ the number of vertices and by $m$ the number of edges/arcs. 

\begin{restatable}[All Subsets Important Separators]{theorem}{impsep}\label{thm:importantbound}
Let $G$ be a digraph, $k$ be a positive integer and let $S \subseteq V(G)$ and $T \subseteq V(G)$ be disjoint sets of source and sink vertices. Then there are at most $ \beta(|S|, |T|, k) = 4^k {|S| \choose \leq k}{|T| \choose \leq 2k}$ $A$-$B$ important separators of size $\leq k$ across all $A \subseteq S$ and $B \subseteq T$. Further, they can be enumerated in time $\OO(\beta(|S|, |T|, k)\cdot k^2\cdot(m + n))$. 
\end{restatable}


It is not difficult to show that this result is essentially tight: we show this formally in the next lemma, whose proof is deferred to the appendix. 

\begin{restatable}{lemma}{lowerbound}
For any positive integer $k$, the following two statements hold.
\begin{enumerate}
\item There exist infinitely many positive integers $c$, such that for each $c$, there is a directed graph $G_{c,k}$ and disjoint subsets $S, T \subseteq V(G)$ of vertices with $|T| = c$ and $|S| = 1$ so that there are at least ${|T| \choose \leq k}$ $A$-$B$ important separators of size at most $k$ across all choices $A \subseteq S$, $B \subseteq T$.

\item There exist infinitely many positive integers $c$, such that for each $c$, there is a directed graph $G_{c,k}$ and disjoint subsets $S,T \subseteq V(G)$ of vertices with $|S| = c$ and $|T| = k + 1$ so that there are at least ${|S| \choose \leq k}$ $A$-$B$ important separators of size at most $k$ across all choices $A \subseteq S$, $B \subseteq T$.

\end{enumerate}
\end{restatable}

In  the next few subsections, we show that this simple structural result has various interesting consequences, allowing us to obtain many new results which were previously only known for undirected graphs. 

\subsection{Detection Sets and Sample Sets in Directed Graphs} 

Kleinberg~\cite{kleinberg2004detecting} introduced the concept of a detection set in a graph. The principal motivation for this concept was that of a network failure:  given a network, can we compute a small set of representative nodes so that for any  small set of failure nodes that cuts communication between two large subsets of nodes, there are two representatives that cannot communicate? Formally, Kleinberg defined an $(\epsilon, k)$ detection set for an undirected graph $G$ as a set of terminals $T \subseteq V(G)$ which satisfies the following property. First, define a network failure as a set of 
vertices $X$ with $|X| \leq k$ so that $G \setminus X$ can be partitioned into $A \cup B$ where $|A|, |B| \geq \epsilon n$ and there are no edges between $A$ and $B$. Then for every such (vertex) failure set $X$, the set $T$ must intersect with at least two components of $G \setminus X$. Kleinberg~\cite{kleinberg2004detecting} showed that there is a detection set of size $\OO(\frac{k^3}{\epsilon} \log \frac{1}{\epsilon})$. For the edge failure version, he showed a bound of $\OO(\frac{k}{\epsilon} \log \frac{1}{\epsilon})$ which was subsequently improved to $\OO(\frac{k}{\lambda}\frac{1}{\epsilon} \log \frac{1}{\epsilon})$ by Kleinberg et al.~\cite{kleinberg2008network}, where $\lambda$ is the size of the global minimum cut in the graph. Feige and Mahdian~\cite{fm06} showed an improved bound on (vertex) detection sets: they show a bound of $\OO(\frac{k}{\epsilon} \log \frac{1}{\epsilon})$. Further, they showed a bound of $\OO(\frac{k}{\epsilon})$ for the edge failure version, removing the dependence on $\log \frac{1}{\epsilon}$. 


Feige and Mahdian~\cite{fm06} also studied a strengthening of this notion of detection sets called \emph{sample sets} for undirected graphs: on a high level, these are a small set of terminals $T$ which represent all the small cuts of the graph in proportion to their size, up to some additive error. Concretely, given an undirected graph $G$ and a parameter $k$, they show that there exists a set of terminals $T$ with $|T| = \OO(\frac{k}{\epsilon^2} \log \frac{1}{\epsilon})$, such that for any vertex set $X$ of size $|X| \leq k$ and every connected component $C$ of $G \setminus X$, we have $||C| - \frac{n}{|T|} |C \cap T|| \leq \epsilon n$. Further, the set $T$ can be obtained by simple random sampling, and~\cite{fm06} shows that any random subset $T \subseteq V(G)$ of size $\OO(\frac{k}{\epsilon^2} \log \frac{1}{\epsilon})$ is a sample set with constant probability. 


The most important feature of these results is that the size of the detection set (or sample set) $T$ does not depend on $n$. It is then natural to ask if such results are possible for directed graphs. Given a digraph $G$, let us define a network failure as a set of vertices $X$ with $|X| \leq k$, so that $G \setminus X$ can be partitioned into two parts $A$ and $B$, each of size at least $\epsilon n$, such that there is no arc from $B$ to $A$. The analogous question for detection sets is then: is there a set of terminals $T$ with $|T| = f(k, \epsilon)$, such that for any network failure $X$ there exists a pair $t_1, t_2 \in T$ so that there is no path from $t_1$ to $t_2$ in $G \setminus X$? Similarly, the analogous question for sample set becomes: is there a set of terminals $T$ with $|T| = g(k, \epsilon)$ for some function $f$, so that for any set of vertices $X$ with $|X| \leq k$, and any \emph{strongly connected component (SCC)} $C$ of $G \setminus X$, we have $||C| - \frac{n}{|T|} |C \cap T|| \leq \epsilon n$? We answer both these questions in the affirmative, showing that one can in fact asymptotically match the same bound as in the undirected case, with $f(k, \epsilon) = \OO(\frac{k}{\epsilon} \log \frac{1}{\epsilon})$ and $g(k, \epsilon) = \OO(\frac{k}{\epsilon^2} \log \frac{1}{\epsilon})$.

\begin{restatable}{theorem}{detectionset}\label{thm:detection_set}
For any directed graph $G$, given parameters $\epsilon \in (0,1)$ and $k \in \mathbb{N}$, there is an absolute constant $c$ such that there is an $(\epsilon,k)$ detection set of size $f(k,\epsilon) = c\frac{k}{\epsilon} \log \frac{1}{\epsilon}$. Further, a random set of $f(k,\epsilon)$ vertices must be an $(\epsilon,k)$ detection set with probability at least $\frac{2}{3}$.
\end{restatable}

\begin{restatable}{theorem}{sampleset}\label{thm:sample_set}
For any directed graph $G$, given parameters $\epsilon \in (0,1)$ and $k \in \mathbb{N}$, there is an absolute constant $c$ such that there is an $(\epsilon,k)$ sample set of size $f(k,\epsilon) = c\frac{k}{\epsilon^2} \log \frac{1}{\epsilon}$. Further, a random set of $f(k,\epsilon)$ vertices must be an $(\epsilon,k)$ sample with probability at least $\frac{2}{3}$.
\end{restatable}


While we believe that this is of independent interest, we also show a similar connection to parameterized and approximation algorithms for finding balanced cuts in directed graphs along the lines of the undirected case as in~\cite{fm06}, as discussed in the following subsections. In fact, by using a slightly more nuanced analysis, our results generalize that of~\cite{fm06} even for undirected graphs. 

Finally, we note that one can easily prove that there exists an absolute constant $c$ such that a random subset of size $c\frac{k \log n}{\epsilon^2}$ is a sample set with constant probability in a directed graph. This follows from a simple application of Chernoff bounds and a union bound noting that the number of vertex sets of size at most $k$ is at most ${n \choose k}$ (${m \choose k}$ in the case of edge sets). However, for most applications this bound is too weak. For instance, our parameterized algorithm for \DB{} has an exponential dependency in the size of the sample set, and hence to show results parameterized by $k$, it is essential that the size of the sample set does not depend on $n$.


\subsection{Directed Balanced Cuts} One of the most well-studied graph partitioning problems, both from the parameterized and approximation algorithms point of view, is the problem of \MB{}. Given an undirected graph $G$ and a parameter $k$, the goal of the \MB{} problem is to obtain a partition of the graph into two equal sized parts, such that the number of cut edges is at most $k$. \MB{} was shown to be fixed-parameter tractable by Cygan et al.~\cite{cygan2014minimum} and the current best parameterized algorithm is due to~\cite{cygan2020randomized} who show an algorithm with running time $2^{\OO(k \log k)}n^{\OO(1)}$. Räcke~\cite{racke2008optimal} gave an $\OO(\log n)$ approximation for (the optimization version of) \MB. If only an approximate bisection where both sides have $\Omega(n)$ vertices is desired, then this problem is essentially the \BS{} problem which is known to have an FPT algorithm running in time $2^{\OO(k)}(m + n)$ due to Feige and Mahdian~\cite{fm06} and an $\OO(\sqrt{\log n})$ approximation algorithm in polynomial time using the seminal result of Arora, Rao and Vazirani~\cite{arora2009expander}. Feige et al.~\cite{feige2005improved} showed that in fact this guarantee can be made $\OO(\sqrt{\log k})$, and also showed that one can compute vertex separators with the same approximation ratio. In directed graphs, the \MB{} problem has been typically studied as \DBS: Given a directed graph $G$ with even number of vertices, is it possible to partition the vertex set into two equal parts $A$ and $B$ so that the number of arcs from $A$ to $B$ is at most $k$? This question was first raised by Feige and Yahalom~\cite{feige2003complexity}. When $k = 0$, they referred to this problem as \OC{}, and showed that even this problem is NP-hard. However, given a \OC{} instance which admits a solution, if one relaxes the requirement so that the algorithm can output a partition $(A',B')$ of $V(G)$ such that there are no arcs directed from $A'$ to $B'$ and $||A'| - |B'|| \leq \epsilon n$ where $\epsilon = \Omega(\frac{1}{\log n})$, the problem now becomes tractable, and they show a polynomial time algorithm for \OC. Madathil et al.~\cite{madathil2021sub} showed that the \DBS{} problem is FPT with respect to $k$, even when one requires $|A| = |B|$ exactly, on a subclass of directed graphs called semi-complete digraphs, which are the class of directed graphs where for every pair of vertices $u$ and $v$, there is an arc from $u$ to $v$ or an arc from $v$ to $u$. In terms of approximation algorithms, Agarwal et al.~\cite{agarwal2005log} showed an $\OO(\sqrt{\log n})$ approximation for \DB{} , the directed analogue of \BS{}, while Even et al.~\cite{even1999fast} showed an $\OO(\log k)$ approximation, which is the best known approximation guarantee depending only on $k$.

However, there is no prior work on the fixed-parameter tractability of \DB{} or \DBS{} for general $k$ on general directed graphs, even when we relax the requirement of finding a bisection to that of finding an \emph{approximate bisection}, that is, find a partition $(A', B')$ with $|A'|, |B'| = \Omega(n)$ so that the number of arcs from $A'$ to $B'$ is at most $k$.

Our result makes the first progress on this problem. Before we state our results, we define the \DB{} problem formally. Since all our results work for both the vertex and edge versions, we state them together as one problem. We adapt our definition from the definition of \BS{} in~\cite{fm06}, whose results we generalize. For completeness, we recall the definition of \BS{} in~\cite{fm06}.
\defparproblem{\BS}{Undirected graph $G = (V,E)$}{$k,b$}{Is there a set of vertices (edges) $F$ with $|F| \leq k$, so that in $G \setminus F$, every connected component has size at most $bn$?}

\defparproblem{\DB}{Directed graph $G = (V,E)$}{$k,b$}{Is there a set of vertices (arcs) $F$ with $|F| \leq k$, so that in $G \setminus F$, every strongly connected component has size at most $bn$?}



For the sake of clarity and comparison, we state the main result of~\cite{fm06}. Given an undirected graph $G$, we say that a set of vertices/edges $F$ is a \emph{$b$-balanced separator} if every connected component of $G \setminus F$ has size at most $bn$.

\begin{theorem}[\cite{fm06}]\label{thm:fm}
Given an instance of \BS{} with $\frac{2}{3} \leq b \leq 1$ there is a randomized algorithm, that for any $\epsilon > 0$, runs in time $2^{\OO\left(k\log \left(\frac{1}{\epsilon}\right)/{\epsilon^2}  \right)}(m + n)$ and with constant probability outputs either (a) a set of vertices (edges) $F'$ of size at most $k$ such that every connected component of $G \setminus F'$ has size at most $(b + \epsilon)n$ or (b) concludes correctly that there is no $b$-balanced separator of size at most $k$.
\end{theorem}
The following theorem, which directly generalizes the result of~\Cref{thm:fm} is our main result. Given a directed graph $G$, we say that a set of vertices/arcs $F$ is a \emph{$b$-balanced separator} if every \emph{strongly connected component} (SCC) of $G \setminus F$ has size at most $bn$.


\begin{restatable}{theorem}{balancedcut}\label{thm:main}
Given an instance of \DB{} there is a randomized algorithm, that for any $\epsilon > 0$, runs in time $2^{\OO\left(k \min\{\log\frac{1}{b}, \log k\}\log \frac{1}{\epsilon}/\epsilon^2 \right)}(m + n)$ and with constant probability outputs either (a) a set of vertices (arcs) $F'$ of size at most $k$ such that every strongly connected component of $G \setminus F'$ has size at most $(b + \epsilon)n$ or (b) concludes correctly that there is no $b$-balanced separator of size at most $k$.
\end{restatable}

We observe that our algorithm has a running time of $2^{\OO\left(k\log \left(\frac{1}{\epsilon}\right)/{\epsilon^2}  \right)}(m + n)$ for any $b = \Omega(1)$, matching the run-time of~\Cref{thm:fm} for $\frac{2}{3} \leq b \leq 1$ while also extending to any parameter $b \in (0,1)$. We also observe that~\Cref{thm:main} implies~\Cref{thm:fm}, since given any undirected graph $G$, one can create a directed graph $H$ on the same vertex set, so that for every edge $\{u,v\} \in E(G)$, we have the two arcs $(u,v), (v,u) \in E(H)$ and we can apply~\Cref{thm:main} to obtain~\Cref{thm:fm}.  






Our algorithm for~\DB{} can be used to solve (approximate)~\DBS{} as well. To see this, given a graph $G$, observe that in any bisection $(A,B)$ so that the number of arcs going from $A$ to $B$ is at most $k$, the set $F$ of these at most $k$ arcs form a $\frac{1}{2}$-balanced separator, so that in $G \setminus F$, every strongly connected component is of size at most $\frac{n}{2}$. Using~\Cref{thm:main} with $b = \frac{1}{2}$, we can find a set of arcs $F'$ with $|F'| \leq k$ that forms a $(\frac{1}{2}+\epsilon)$ balanced separator. Finally, note that the strongly connected components of the graph $G \setminus F'$ form a Directed Acyclic Graph (DAG), and each strongly connected component of $G \setminus F'$ has size at most $(\frac{1}{2} + \epsilon)n$. It follows that there is a topological ordering $\{C_1, C_2 \ldots C_{\ell}\}$ of these strongly connected components of $G \setminus F'$, such that there is no arc from $C_j$ to $C_i$ for $i, j \in [\ell]$ with $j > i$. Therefore we can pick some prefix of strongly connected components in the topological ordering of $G 
\setminus F'$ to obtain a set $A'$, so that both $|A'|, |V(G) \setminus A'| \geq (1 - 2\epsilon)\frac{n}{4}$, and in $G \setminus F'$, there are no arcs from $V(G) \setminus A'$ to $A'$. It follows that the set of arcs $F'$ forms an (approximate) directed bisection. Note that the approximation is only in the balance, not in the number of arcs cut.

Next, we show a $\OO(\sqrt{\log k})$ approximation for \DB. This improves both the $\OO(\sqrt{\log n})$ approximation  of~\cite{agarwal2005log} and the $\OO(\log k)$ approximation given by Even et al.~\cite{even1999fast} for approximating \DB{} in polynomial time. 


\begin{restatable}{theorem}{approxbalsep}
There is an $\OO(\sqrt{\log k})$ approximation to \DB{} in polynomial time. Formally, given an instance of \DB{} with $b = \Omega(1)$, suppose there is a set of vertices (arcs) $F$ with $|F| \leq k$, so that every strongly connected component of $G \setminus F$ has at most $bn$ vertices. Then there is a polynomial time randomized algorithm that with constant probability finds a set of vertices (arcs) $F'$ with $|F'| \leq \OO(k\sqrt{\log k})$ so that in $G \setminus F'$, every strongly connected component has size at most $b'n$ for some $b' < 1$ depending on $b$.
\label{thm:approxbalsep}
\end{restatable}

Note that for this theorem, we do not optimize $b'$, unlike our FPT result where we were able to show that $b' \leq b + \epsilon$ for some suitably chosen $\epsilon$. Still, we remark that is not a limitation of our framework, but inherent in~\cite{agarwal2005log} due to the use of the ARV separation theorem~\cite{arora2009expander}.

\subsection{Vertex Cut Sparsifiers}
Vertex sparsification is a fundamental problem in various settings. Broadly, given a graph and a terminal set $T$, a vertex sparsifier is a smaller graph (with size typically depending only on $|T|$) that preserves some (cut based) property of the terminals $T$. Moitra~\cite{moitra2009approximation} first introduced a version of vertex sparsification in undirected graphs. Chuzhoy~\cite{chuzhoy2012vertex} generalized this notion by allowing Steiner nodes in the sparsifier. Both these notions preserve edge cuts between bi-partitions of the terminal set.


For our setting, we focus on directed graphs and vertex cuts. Given a directed graph $G$ and a set of terminals $T$ along with an integer $c$, a $(c,T)$ vertex cut sparsifier for $G$ is another graph $G'$ with $T \subseteq V(G')$, so that for every partition $\{A, B\}$ of $T$ (so that $A \cup B = T$ and $A \cap B = \emptyset$), if the size of an $A$-$B$ vertex min-cut is at most $c$ in $G$, then the size of an $A$-$B$ vertex min-cut is the same in both in $G$ and $G'$. In other words, the goal is to find a vertex sparsifier that preserves vertex min-cuts of size at most $c$ between sets of terminals. Here we allow the deletion of terminals as well.


Kratsch and \Wahlstrom~\cite{kratsch2012representative,kratsch2020representative} first studied this notion of vertex sparsification for vertex cuts (without the parameter $c$) and showed that given $G,T$,  there is a randomized polynomial time algorithm that obtains a $(|T|, T)$ vertex sparsifier $G'$ for $G$ with $|V(G')| \leq \OO(|T|^3)$. This bound was improved to $\OO(|T|^2)$ by~\cite{he2021near} for the special case of directed acyclic graphs. Their algorithm runs in linear time for fixed $|T|$, but is still randomized as it needs to compute a representation of gammoids.   

One can then ask the question as to what is known about deterministic algorithms. Recently, Misra et al.~\cite{misra2020linear} showed that a representation of a gammoid with rank $r$ over a ground set of $m$ elements can be constructed in time $\OO({m \choose r} m^{\OO(1)})$ deterministic time. The technique of~\cite{kratsch2012representative} needs to compute a representation of a gammoid whose ground set has size $n$, the number of vertices and rank $|T|$, the number of terminals. We therefore obtain the following result.

\begin{theorem}[\cite{kratsch2012representative},~\cite{misra2020linear}]
Given a $n$-vertex graph $G$ and a terminal set $T$, there is a deterministic algorithm that runs in time $\OO({n \choose |T|} n^{\OO(1)})$ and computes a $(|T|, T)$ vertex cut sparsifier for $G$ of size at most $\OO(|T|^3)$.
\label{thm:detsparsifier}
\end{theorem}


However, this algorithm (which is an XP algorithm in the notation of parameterized complexity) can be easily improved to a (still deterministic) FPT algorithm. The reason for this is simple: there are at most $2^{|T|}$ partitions of the terminal set $T$. For each partition $A \cup B$, we can compute an $(A,B)$ directed min-vertex cut $M$. Note that this min-cut is of size at most $|T|$, since one can simply delete all the terminals to obtain a cut. This gives a set $X$ of $2^{|T|} \cdot |T|$ vertices. It can now be shown that we can apply the closure operation to the set $V(G) \setminus X$, where we simply delete all vertices of $V(G) \setminus X$, and for each vertex pair $(u,w)$ such that there are vertices $(v_1, v_2)$ with $v_1, v_2 \in V(G) \setminus X$ with $(u,v_1), (v_2, w) \in E(G)$ we add an arc $(u,w)$. This results in a sparsifier of size $\OO(|T|2^{|T|})$. One can now set $n = \OO(|T|2^{|T|})$ in~\Cref{thm:detsparsifier} to obtain a sparsifer of size $2^{\OO(|T|^2)}$ in time $2^{\OO(|T|^2)}$. The total running time is $2^{\OO(T)}(m + n) + 2^{\OO(|T|^2)}$.






A similar line of research considers edge and vertex cuts in undirected graphs. For edge cuts in undirected graphs, Chalermsook et al.~\cite{chalermsook2021vertex} showed an upper bound of $\OO(|T|c^4)$ which was later improved to $\OO(|T|c^3)$ by Liu~\cite{liu2023vertex}. Both these algorithms are randomized. Saranurak and Yingchareonthawornchai~\cite{saranurak2022deterministic} considered the vertex version in undirected graphs, and gave a deterministic algorithm to compute a sparsifier of size $\OO(|T|2^{\OO(c^2)})$ in time $\OO(m^{1 + o(1)}2^{\OO(c^2)})$. 

However, no improvement on the FPT algorithm with running time $2^{O(|T|)}(m + n) + 2^{O(|T|^2)}$ based on \Cref{thm:detsparsifier} is known for deterministic algorithms for vertex cut sparsifiers in directed graphs. We show the following result, which shows that if one is interested in preserving cuts of size at most $c$, then a better deterministic algorithm is possible.


\begin{restatable}{theorem}{mimicking}\label{thm:mimicking}
Given a directed graph $G$, a set of terminals $T$ and integer $c$, there is a deterministic algorithm that runs in time $\OO(\psi(|T|,c) \cdot (m + n))$ and computes a $(c,T)$ vertex sparsifier $G'$ for $G$ of size at most $\psi(|T|,c)$, where $\psi(|T|,c) = {|T| \choose \leq 3c} 2^{\OO(c)}$. Additionally, $G'$ satisfies $T \subseteq V(G') \subseteq V(G)$, with the property that for every partition $A \cup B$ of $T$ and every subset $X \subseteq V(G)$ with $|X| \leq c$, $X$ is an $A$-$B$ important separator in $G$ if and only if $X \subseteq V(G')$ and $X$ is an $A$-$B$ important separator in $G'$. 
\end{restatable}





While the dependence on $c$ in the exponent for the size of the sparsifier seems undesirable,~\Cref{thm:mimicking} shows that the vertex sparsifier $G'$ constructed by our algorithm is more powerful: it in-fact ``preserves'' all $A$-$B$ important separators for any partition $(A,B)$ of $T$. Additionally, if one wants only a vertex sparsifier, we note that simply running the algorithm in~\Cref{thm:detsparsifier} after applying our result in~\Cref{thm:mimicking} gives a vertex sparsifier of size $\OO(|T|^3)$, but now the running time is $\left(\frac{|T|}{c}\right )^{\OO(c|T|)}$. Thus, together, we obtain a vertex cut sparsifier of size $\OO(|T|^3)$ in time ${|T| \choose \leq 3c} 2^{\OO(c)}(m + n) + \left(\frac{|T|}{c}\right)^{\OO(c|T|)}$. 

\begin{corollary}
Given  a graph $G$ and a terminal set $T \subseteq V(G)$, there is a deterministic algorithm that runs in time ${|T| \choose \leq 3c} 2^{\OO(c)}(m + n) + \left(\frac{|T|}{c}\right)^{\OO(c|T|)}$ and obtains a $(c,T)$ vertex cut sparisifer for $G$ with $\OO(|T|^3)$ vertices.
\end{corollary}










\subsection{Techniques and Overview}

In this section we give a high-level overview of our techniques. For the sake of simplicity and clarity we sometimes omit small details in this section.

\paragraph{All-Subset Important separators:} Our result on counting all-subset important separators is obtained using the connection between important separators and \emph{closest sets}. For simplicity, in this overview, suppose that we are given a single source $s$ and a set of target vertices $T$. Our goal in this simplified setting shall be to bound the number of $(s,B)$ important separators of size exactly $k$ across all $B \subseteq T$, and we will show a weaker bound of $4^k{|T| \choose \leq {(k +1)^2}}$. Before we describe the main idea, we remark that our eventual goal will be to show that every $(s,B)$ important separator of size $k$ for some $B \subseteq T$ is an $(s,B')$ important separator for some $B' \subseteq B$ with $|B'| \leq (k +1)^2$. It is easy to see that this suffices, since there are only ${|T| \choose \leq (k + 1)^2}$ choices for $B'$, and there are at most $4^k$ such $(s,B')$ important separators of size $k$ for a fixed choice of $B'$.  





Fix $B \subseteq T$, and consider an $(s,B)$ important separator $X$ of size $k$. Consider the (vertex) min-cut $X'$ between $X$ and $B$\footnote{Throughout this paper, we allow deletion of source and sink vertices in vertex cuts}. If this min-cut $X'$ has size strictly less than $|X| = k$, $X$ cannot be important: $X'$ would allow more vertices to be reachable from $s$ while having a size strictly less than $k$. This means that $X$ by itself must be a $(X,B)$ min-cut. Using simple cut-flow duality, it must be the case that there are $|X| = k$ vertex disjoint paths from $X$ to (some) vertices in $B$.

However, this flow property can be strengthened: it is easy to see that we can in fact assume that every such important separator $X$ is the $(X,B)$ min-cut ``closest'' to $B$. (We say $X$ is closest to $B$ if $X$ is the unique $(X, B)$ min-cut.)
But it can be shown that closest sets have a stronger flow property: we can now conclude that for every vertex $v \in X$, there are $k + 1$ paths from $X$ to $B$, which are vertex disjoint except that two of these paths both start at $v$ (\Cref{lemma:closest}). 

Why does this help? Fix a $v \in X$, and fix the $k + 1$ paths from $X$ to $B$. Suppose the $k + 1$ endpoints of these paths are $B^*_v \subseteq B$. Consider any set of vertices $X'$ that is a $(s,B^*_v)$ separator of size $\leq k$ that is ``closer'' to $B^*_v$ than $X$. (Here ``closer'' means that the set of vertices reachable from $s$ in $G \setminus X'$ is a superset of that in $G \setminus X$). Then $X'$ must delete $v$! For if not, since there are $k + 1$ paths from $X$ to $B^*_v$ which are vertex disjoint except $2$ paths which begin at $v$, and $|X'| \leq k$, $X'$ cannot be a $(s,B^*_v)$ separator. 

Applying the above argument for every $v \in X$ and letting $B' := \cup_{v \in X} B^*_v$ allows us to conclude that there is no $(s, B')$ separator $X'$ that is closer to $B'$ than $X$.
Thus, in fact, among all the sets of size at most $k$ separating $s$ from $B'$, $X$ is a set which is closest to $B'$. This means $X$ must be an important $(s,B')$ separator. But $|B'| = |\cup_{v \in X} B^*_v| \leq k(k + 1) \leq (k +1)^2$, and hence we are done. Our actual result uses a slightly more subtle argument using augmenting paths to obtain such a $B'$ with $|B'| \leq 2k$.


Here we remark that Lemma 4.10 of Lokshtanov et al.~\cite{lokshtanov2025wannabe} showed a similar result: it claims that there is a set $B' \subseteq B$ with $|B'| \leq k + 1$ such that any $(s,B)$ important separator is an $(s,B')$ important separator. However, there is a gap in the proof which leads to an unavoidable factor of $2$ loss. Indeed, as we show in~\Cref{lemma:impseppreserve}, there is a graph $G$, source vertex $s \in V(G)$ and $B \subseteq V(G)$ for which $|B'| = 2k$ is necessary, and hence our result is tight. Our high-level idea is flow-based, similar to~\cite{lokshtanov2025wannabe}. But as described above, the direct application of this flow-based idea gives a bound of $k(k + 1)$, which we refine using an augmenting path based argument to obtain the tight bound of $2k$.





\paragraph{Detection sets and Sample sets for Directed Graphs:}
To obtain detection sets and sample sets for directed graphs, on a high level, we follow the approach of Feige and Mahdian~\cite{fm06}.
We focus on sample sets in this brief overview. First, we consider the family of sets that arise as a strongly connected component after deleting an arbitrary $k$ vertices in the given digraph. Our key contribution is to show that this family has VC-dimension $\OO(k)$. Using the standard results on $\epsilon$-samples, similar to that in~\cite{fm06}, it then follows that a random set of $\OO(\frac{k}{\epsilon^2} \log \frac{1}{\epsilon})$ vertices is a sample set with constant probability. Thus the question becomes: how can we bound the VC-dimension of the family of sets $\SS$ formed by strongly connected components after deleting some $k$ vertices? We note that the techniques in~\cite{fm06} to bound the VC-dimension do not extend to directed graphs, and hence a different approach is needed. We accomplish this by showing a connection to our result on all-subset important separators. 

Recall the definition of VC-dimension: given a set family $\SS$ on a universe $U$, the VC-dimension is the size of the largest set $U' \subseteq U$ shattered by $\SS$. Here $U'$ is shattered by $\SS$ if for every subset $U'' \subseteq U'$ there exists an $S \in \SS$ with $|U' \cap S| = U''$. Thus to prove a bound of $\OO(k)$ for VC-dimension for our case, it is enough to show that there exists some constant $c$ so that any set of terminals $T \subseteq V(G)$ with $|T| \geq ck$ cannot be shattered by the set family $\SS$ which consists of the strongly connected components $C$ in $G \setminus F$ across all sets $F \subseteq V(G)$ with $|F| \leq k$. Fix any $T$ of size $\geq ck$ for large enough $c$ which we choose later. For the sake of contradiction, assume that the set $T$ can be shattered. Fix a ``pattern'' $P \subseteq T$, $P \neq \emptyset$. Since $T$ can be shattered by $\SS$, there exists a set $F$ of at most $k$ vertices, so that in $G \setminus F$, there is a strongly connected component $C$ that satisfies $C \cap T = P$. Fix some terminal $t \in P$. Then it must be the case that $P$ is the exact set of terminals that can both reach $t$ and can be reached from $t$ in $G \setminus F$. Equivalently, it is the exact set of terminals that can be reached from $t$ in both $G \setminus F$ and $G^R \setminus F$, where $G^R$ is obtained by reversing every arc of $G$. This motivates us to define $\Reach(t)$ for each $t \in T$, 
the collection of subsets of terminals that can be reached from $t$ after deleting a set $F$ of size at most $k$. Formally,
\begin{align*}
\Reach(t) = \{Q \subseteq T \mid\, & \text{there exists $F \subseteq V(G)$, $|F| \leq k$ such that $Q$ is reachable from $t$} \\&\text{while $T \setminus Q$ is unreachable from $t$ in $G \setminus F$}\}
\end{align*}
If we can bound $|\Reach(t)|$ for any graph by a function $B(|T|,k)$, then applying this result twice, once on $G$ and once on $G^R$, will bound the number of such patterns $P$ with $t \in P$ by $(B(|T|,k))^2$. Then a simple union bound over all $t \in T$ will give that the number of patterns $P \subseteq T$ is at most $|T|B(|T|,k))^2$. If this is less than $2^{|T|}$ whenever $|T| \geq ck$, then we have a contradiction to the fact that $T$ can be shattered. Thus the key question is to obtain $B(|T|,k)$, a good bound on $|\Reach(t)|$.

We do this as follows. Suppose $T' \in 
\Reach(t)$ some $T' \subseteq T$. Then there exists a set $F$ of at most $k$ vertices after whose deletion the set of reachable terminals from $t$ is exactly $T'$. We then show that there is a $(t, T \setminus T')$ \emph{important separator} $X$ of size at most $k$ whose deletion gives the same reachability set $T'$ as that after deleting $F$. But now using our bound on important separators, we can show that $B(|T|,k) \leq 4^k {|T| \choose \leq 2k}$. It is then easy to see that $|T|B(|T|,k)^2 < 2^{|T|}$ whenever $|T| \geq ck$ for some large enough (absolute) constant $c$. 






\paragraph{FPT algorithm for Directed Balanced Separators:}
To obtain the algorithm for \DB{}, we follow the approach of~\cite{fm06} in the undirected case. For simplicity, in this overview, we work with vertex separators, assume that $b = \frac{1}{2}$, and we will only return a $b' = (\frac{3}{4} + 2\epsilon)$-balanced separator (as opposed to the $(b + \epsilon)$ balance guaranteed in our result). We are given that there is a set of vertices $F$ with $|F| \leq k$, so that every strongly connected component of $G \setminus F$ has at most $\frac{1}{2}n$ vertices. We first compute an $(\epsilon, k)$ sample set $T$ of size $\OO\left ( \frac{k \log \frac{1}{\epsilon}}{\epsilon^2}\right)$ using our result on sample sets,~\Cref{thm:sample_set}. Since $T$ is a sample set, each SCC of $G \setminus F$ has at most $(\frac{1}{2} + \epsilon)|T|$ terminals. We will try to find a set of vertices $F'$ such that $|F'| \leq k$ and every SCC of $G \setminus F'$ has at most $(\frac{3}{4} + \epsilon)|T|$ terminals. The property of sample sets will then again imply that $F'$ is a $(\frac{3}{4} + 2\epsilon)$-balanced separator.

Consider the toplogical ordering $C_1, C_2 \ldots C_{\ell}$ of the SCC's of $G \setminus F$, so that there is no arc from $C_j$ to $C_i$ for $i,j \in [\ell]$ with $j > i$. Consider the smallest index $i^*$ so that the union $L = \bigcup_{i = 1}^{i^*} C_i$ contains at least $(\frac{1}{4} - \epsilon)|T|$ terminals. Since no component contains more than $(\frac{1}{2} + \epsilon)|T|$ terminals, it follows that $L$ contains at most $\frac{3}{4}|T|$ terminals. Also, by definition, $V(G) \setminus L$ contains at most $(\frac{3}{4} + \epsilon)|T|$ terminals.

The algorithm proceeds as follows. First, guess $T \cap L$ and $T \cap (V(G) \setminus L)$. This can be done in time $2^{\OO\left(\frac{k \log \frac{1}{\epsilon}}{\epsilon^2}\right)}$ since $|T| = \OO\left(\frac{k\log \frac{1}{\epsilon}}{\epsilon^2}\right)$. Next, we compute a directed min-vertex cut between $T \cap (V(G) \setminus L)$ and $T \cap L$. Since $F$ is a vertex cut between these sets of size at most $k$, this min-cut $F'$ must be of size at most $k$ as well. But then the set of vertices $F'$ is a $(\frac{3}{4} + \epsilon)$-balanced separator with respect to the set $T$. Using the property of sample sets, it follows that $F'$ is a $(\frac{3}{4} + 2\epsilon)$-balanced separator in $G$.
 
 In order to make the loss in the balance factor only an additive $\OO(\epsilon)$ we need a slightly more sophisticated argument. We accomplish this using a reduction to the \SMC{} problem, where we are given pairs $(s_i, t_i), i \in [\ell]$, and the goal is to separate $s_i$ from all $t_j$, $j \leq i$, for each $i$, by deleting minimum number of vertices. This problem, which is a special case of {\textsc{Directed Multicut}}{}, was first defined by~\cite{chen2008fixed} in their FPT algorithm for \DFAS.


\paragraph{Approximation algorithm for Directed Balanced Separators:} Our $\OO(\sqrt{\log k})$ approximation algorithm proceeds similar to the FPT algorithm. The only key technical difference is that to compute a balanced separator with respect to the terminal set, we cannot guess where the terminals of the sample set are (this requires FPT time); instead we use the algorithm of~\cite{agarwal2005log} to obtain a balanced separator with respect to the sample set. While the algorithm of~\cite{agarwal2005log} has an approximation ratio of $\OO(\sqrt{\log n})$, when we need a balanced separator with respect to a terminal set $T$, this algorithm can be made to work with an approximation ratio of $\OO(\sqrt{\log |T|})$. On a high level, the reason is quite simple: the structure theorem of Arora, Rao and Vazirani \cite{arora2009expander} is a statement about vectors. As long as we apply the structure theorem to only $|T|$ vectors, we get an approximation ratio of $\OO(\sqrt{\log |T|})$. 




\paragraph{Vertex Sparsifiers:} Our result on vertex sparsifiers is similar in spirit to the result by Kratsch and \Wahlstrom~\cite{kratsch2012representative} and proceeds by identifying \emph{irrelevant vertices}. However instead of using techniques based on matroids, we use our result on all-subsets important separators. Given a terminal set $T$, we are interested in preserving cuts across partitions of the terminal set of size up to $c$. We will define an irrelevant vertex as a vertex that is (a) not in $T$ and (b) not in any $A$-$B$ important separator of size $\leq c$ for any partition $A \cup B$ of $T$. By our result on all-subsets important separators, we can bound the number of relevant (non-irrelevant) vertices by ${|T| \choose  \leq 3c} 2^{\OO(c)}$, and identify this set in time ${|T| \choose \leq 3c} 2^{\OO(c)} (m + n)$. Thus at least $|V(G)| - |T| -  {|T| \choose \leq 3c} 2^{\OO(c)}$ vertices are irrelevant. Let $I$ be the set of irrelevant vertices. We now apply the standard operation of ``closing'' the irrelevant vertices $I$, which is to delete each $v \in I$, and for every pair $(w,y)$ where $w$ is an in-neighbour of $v_1$ and $y$ is an out-neighbour of $v_2$ for some $v_1,v_2 \in I$, to add the arc $(w,y)$. This operation is the vertex equivalent of edge contraction. The key observation here is that the closure operation \emph{can be applied to the entire set $I$ at once} as opposed to applying it to one vertex of $I$ and restarting the whole algorithm. This helps us achieve a linear-time algorithm. Thus we obtain another graph $G'$ with $V(G') \subseteq V(G)$, where $|V(G')| \leq |V(G) \setminus I| \leq  {|T| \choose \leq 3c} 2^{\OO(c)}$.









 



\section{All-Subsets Important Separators}
The goal of this section is to prove our result on all-subsets important separators.
\impsep*

As described in the introduction, our proof proceeds by capturing the relationship between important separators and closest sets. We then use well-known connectivity properties of closest sets to obtain our result.


\begin{definition}[Closest set]
For sets of vertices $X,T\subseteq V(G)$, $X$ is closest to $T$ if $X$ is the unique vertex mincut between $X$ and $T$.
\end{definition}

We use the following well-known fact about closest cuts.

\begin{lemma}[Lemma 18 of~\cite{he2021near}]\label{lemma:closest}
$X$ is closest to $T$ if and only if for each vertex $v\in X$, there are $|X|+1$ paths from $X$ to $T$ that are vertex-disjoint except at $v$, which appears in exactly two of these paths.
\end{lemma}

\begin{definition}[Minimal separators]
Given $X, S, T \subseteq V(G)$ such that $S \cap T = \emptyset$, $X$ is called an $S$-$T$ (inclusion-wise) minimal separator if in $G \setminus X$, there is no path from any $s \in S$ to any $t \in T$, and further, for every $v \in X$, there is a path between some $s \in S$ and $t \in T$ in $G \setminus (X \setminus \{v\})$.
\end{definition}

\begin{definition}
Given $X, S \subseteq V(G)$, the reachability set of the source vertices $S$ after removing $X$, that is, the set of all vertices $v$ reachable from some vertex of $S$ via a directed path after removing $X$, is denoted by $R^G_S(X)$.\end{definition}



\begin{definition}
Given $X, S, T \subseteq V(G)$ with $S \cap T = \emptyset$, a minimal $S$-$T$ separator $X$ is called an $S$-$T$ important separator if for every $S$-$T$ separator $X' \subseteq V(G)$ with $|X'| \leq |X|$, $R^G_S(X') \supset R^G_S(X)$ does not hold.
\end{definition}

We will drop the subscript or the superscript in $R^G_S(X)$ when the graph or the set of source vertices is clear from the context.

Informally, the idea of important separators is similar to that of a closest set: they are separators which cannot be pushed ``closer'' to the sink, without increasing the cut-size. The next lemma captures this relationship formally.

\begin{lemma}\label{lemma:equivalence}
Given $X, S, T \subseteq V(G)$, $X$ is an important $S$-$T$ separator if and only if (a) $X$ is a  minimal $S$-$T$ separator and (b) $X$ is closest to $T$.
\end{lemma}

\begin{proof}\textcolor{red}{TOPROVE 0}\end{proof}

\begin{lemma}[Theorem 8.51 of~\cite{cygan2015parameterized}]
Given a digraph $G$ and two disjoint sets $A,B \subseteq V(G)$, there are at most $4^k$ $A$-$B$ important separators of size $\leq k$.
\end{lemma}

Equipped with these results, we are now ready to prove our main theorem in this section which bounds the number of important separators between across all subsets $A \subseteq S$ and $B \subseteq T$.

\impsep*

\begin{proof}\textcolor{red}{TOPROVE 1}\end{proof}




















\section{Detection Sets and Sample Sets in Directed Graphs}
The goal of this section is to prove our result on detection sets and sample sets in directed graphs. The following theorems are our main results in this section.
\detectionset*
\sampleset*





We start by defining a notion of $(\epsilon,k)$ nets and recalling the definitions for $(\epsilon, k)$ detection sets and  $(\epsilon,k)$ sample sets from the overview. These definitions naturally extend those in~\cite{fm06} for undirected graphs.


\begin{definition}[Net]
Given a directed graph $G$, an $(\epsilon, k)$ net is a set of terminals $T \subseteq V(G)$ satisfying the following property: for every set of vertices $F$ with $|F| \leq k$, the following two conditions are met:
\begin{enumerate}
\item For every SCC $C$ of $G \setminus F$ with $|C| \geq \epsilon n$ we have 
$$|C \cap T| \geq 1.$$
\item Let $C^*$ be the union of all except one SCC in $G \setminus F$ such that $|C^*| \geq \epsilon n$, then we have
$$|C^* \cap T| \geq 1.$$
\end{enumerate}
\end{definition}

\begin{definition}[Detection Set]\label{def:detection_set}
Given a directed graph $G$, an $(\epsilon, k)$ detection set is a set of terminals $T \subseteq V(G)$ satisfying the following property: for every set of vertices $F$ with $|F| \leq k$ such that $V(G \setminus F)$ can be partitioned into $(A,B)$ with $|A|, |B| \geq \epsilon n$ and there are no arcs from $B$ to $A$, there exists $t_1, t_2 \in T$ such that there is no $t_1$-$t_2$ path in $G \setminus F$.
\end{definition}

\begin{definition}[Sample Set]\label{def:sample_set}
Given a directed graph $G$, an $(\epsilon, k)$ sample set is a set of terminals $T \subseteq V(G)$ satisfying the following property: for every set of vertices $F$ with $|F| \leq k$, and every SCC $C$ of $G \setminus F$, we have 
$$\left |\frac{|C \cap T|}{|T|} - \frac{|C|}{n}\right | \leq \epsilon.$$

\end{definition}
We note that these definitions are almost identical to the one in~\cite{fm06}, with the only difference being that we deal with directed graphs and strongly connected components in lieu of undirected graphs and connected components. Before proving the main results of this section, we show that $(\epsilon,k)$-nets are also $(\epsilon,k)$-detection sets. The proof is similar to that in undirected graphs.

\begin{lemma}
Every $(\epsilon, k)$-net is also an $(\epsilon,k)$ detection set.

\end{lemma}

\begin{proof}\textcolor{red}{TOPROVE 2}\end{proof}



Thus we henceforth focus on obtaining $(\epsilon,k)$-nets and samples.




First, we observe that showing a VC-dimension bound directly implies $(\epsilon,k)$ samples. This is similar to the approach in~\cite{fm06}. We start with a few definitions.

\begin{definition}[Shattering a set of elements] Suppose we are given a set system $(\SS, U)$ consisting of a family of sets $\SS$, where each set in $\SS$ consists of elements from a universe $U$. We say that a subset $W \subseteq U$ is \emph{shattered} by $\SS$, if for every subset $Y \subseteq W$, there exists a set $S \in \SS$ so that $S \cap W = Y$.

\begin{definition}[VC-dimension]
Given a set system $(\SS, U)$, its VC-dimension is the size of the largest subset $W \subseteq U$ that can be shattered by $\SS$.    
\end{definition}



\end{definition}


\begin{theorem}[$\epsilon$-net theorem, see~\cite{fm06}]\label{thm:eps_net}
Let $(\SS, U)$ be a set system with VC-dimension $d$, and universe size $|U| = n$. Then for every $\epsilon > 0$, there exists an absolute constant $c$ such that a random subset $T \subseteq U$ of size $c{d}\frac{1}{\epsilon} \log \frac{1}{\epsilon}$ is  an $\epsilon$-net with probability at least $\frac{2}{3}$.
Concretely, $T$ satisfies $|S \cap T| \geq 1$ for every $S \in \SS$ satisfying $|S| \geq \epsilon n$ with probability at least $\frac{2}{3}$.
\end{theorem}

\begin{theorem}[$\epsilon$-sample theorem, see~\cite{fm06}]\label{thm:eps_sample}
Let $(\SS, U)$ be a set system with VC-dimension $d$, and universe size $|U| = n$. Then for every $\epsilon > 0$, there exists an absolute constant $c$ such that a random subset $T \subseteq U$ of size ${cd}\frac{1}{\epsilon^2} \log \frac{1}{\epsilon}$ is an $\epsilon$-sample with probability at least $\frac{2}{3}$. Concretely, $T$ satisfies $\left|\frac{|S|}{n} - \frac{|S \cap T|}{|T|}\right| \leq \epsilon$ for every $S \in \SS$ with probability at least $\frac{2}{3}$.
\end{theorem}










Let $\SS$ be the family of sets consisting of all possible sets $C$ which are either (a) a strongly connected component after deleting some vertex set $F$ of size at most $k$ or (b) the union of all but one strongly connected components after deleting some vertex set $F$ of size at most $k$. Then it suffices to prove a VC-dimension bound of $\OO(k)$ for this family of sets with the universe as the vertex set - this would then directly imply~\Cref{thm:detection_set,thm:sample_set} using~\Cref{thm:eps_net,thm:eps_sample}.
Thus we will henceforth focus on upper-bounding the VC-dimension of the set system $\SS$.


Towards this, we consider a slightly different problem of reachability from a single source after deleting $k$ vertices.

\begin{definition}[Single source reachability profile]
Given a graph $G$, a source vertex $s$ and a set of sink vertices $T'$ with $s \notin T'$, we define the single source reachability profile of $s$ as $\Reach^k(s,T') = \{ R^G_{ \{ s \} } (F) \cap T' \mid F \subseteq V(G), |F| \leq k \}$ 
In other words, $\Reach^k(s,T')$ is the collection of subsets of $T'$ reachable from $s$ after deleting some set of at most $k$ vertices from $G$.

\end{definition}


The next theorem bounds the size of $\Reach^k(s, T')$. 

\begin{theorem}\label{thm:reachability}
Given a graph $G$, source vertex $s$ and sink vertices $T'$, $|\Reach^k(s, T')| \leq 4^k {|T'| \choose \leq 2k}$.
\end{theorem}

We prove~\Cref{thm:reachability} using our result on all-subset important separators,~\Cref{thm:importantbound}. We do this as follows. Given $s$ and $T'$, suppose there exists a set $X$ of $k$ vertices so that $P \subseteq T'$ is the subset of $T'$ reachable from $s$ in $G \setminus X$. Then in the following lemma, \Cref{lemma:wlgclosest}, we show that there is in fact an $s$-$(T' \setminus P)$ \emph{important separator} $X'$ of size $\leq k$, so that the subset of $T'$ reachable from $s$ in $G \setminus X$ remains $P$. The bound on all-subset important separators will then imply the bound on the reachability profile.  

\begin{restatable}{lemma}{wlgclosest}\label{lemma:wlgclosest}
Suppose there exists a set $X$ of $\leq k$ vertices so that $P \subseteq T'$ is the subset of $T'$ reachable from $s$ in $G \setminus X$. Then there is an $s$-$(T' \setminus P)$ important separator $X'$ of size $\leq k$, so that the subset of $T'$ reachable from $s$ in $G \setminus X$ remains $P$.
\end{restatable}
\lv{
\begin{proof}\textcolor{red}{TOPROVE 3}\end{proof}
}

\begin{proof}\textcolor{red}{TOPROVE 4}\end{proof}





Next, we show why~\Cref{thm:reachability} implies a VC-dimension bound.

\begin{theorem}
Let $\SS$ be the family of sets over consisting of all possible sets $C$ which are either (a) a strongly connected component after deleting some vertex set $F$ of size at most $k$ or (b) the union of all but one strongly connected components after deleting some vertex set $F$ of size at most $k$. Then the VC-dimension of $\SS$ is $\OO(k)$.
\end{theorem}

\begin{proof}\textcolor{red}{TOPROVE 5}\end{proof}











































































  
  
  
 






































 
































































































 \lv
{
\section{FPT Algorithm for Finding Directed Balanced Separators}

The goal of this section is to prove our result on \DB.

\balancedcut*






We will crucially exploit our results on sample sets,~\Cref{thm:sample_set}. We will also need the following result, which is about computing skew separators. The \SMC{} problem is a special case of {\textsc{Directed Multicut}}.

\begin{definition}[\SMC{}~\cite{chen2008fixed}] Given a directed graph $G$ and a set of $\ell$ terminal pairs $\{ (s_i, t_i) \}_{i \in [\ell]}$ and an integer $k$, find a set of vertices $F$ of size at most $k$, so that $G \setminus F$ has no $s_i - t_j$ path for $i \geq j$, $i,j \in [\ell]$.\footnote{Again, we note that we can delete terminals as well to form the solution.}   
\end{definition}
The analagous edge version, \SMCE{} can be defined similarly.

\begin{theorem}[Extension of Theorem 8.41 of~\cite{cygan2015parameterized}]\label{thm:reduction}
\SMC{} and \SMCE{} admit FPT algorithms running in time $\OO(4^k \cdot k^3 \cdot (n + m))$.
\end{theorem}

\begin{proof}\textcolor{red}{TOPROVE 6}\end{proof}







\begin{lemma}\label{lemma:aux}
Let $T$ be an $(\epsilon,k)$ sample set. Suppose that the \DB{} instance is a YES-instance, so that there exists a set of vertices (arcs) $F$ with $|F| \leq k$ whose deletion leaves every strongly connected component with at most $bn$ vertices. 
Then one can in $\OO(2^{\OO(|T|\min\{\log \frac{1}{b}, \log |T|\})}m)$ time, find a set of vertices (arcs) $F'$ with $|F'| \leq k$ so that there is a partition $(X,Y)$ of $G \setminus F'$ such that every SCC of $G \setminus F'$ has size at most $(b + \OO(\epsilon))n$.
\end{lemma}


\begin{proof}\textcolor{red}{TOPROVE 7}\end{proof}




The proof of~\Cref{thm:main} now follows immediately from~\Cref{thm:sample_set} and~\Cref{lemma:aux}.  







\section{Approximation Algorithm for Directed Balanced Separators}


In this section, the goal is to prove the following result, which shows a $\OO(\sqrt{\log k})$ approximation for \DB{}. 
\approxbalsep








Our result is obtained by essentially following the algorithm of~\cite{agarwal2005log} together with our theorem on sample sets,~\Cref{thm:eps_sample}. On a high level, the reason we get $\OO(\sqrt{\log k})$ approximation comes down to the fact that one can find a balanced separator with respect to a sample set $T$, which will automatically be a balanced separator for the entire graph as well due to the property guaranteed by sample sets. Once we have this equivalence, in order to obtain a balanced separator with respect to the set $T$, we now need to use the ARV structure theorem~\cite{arora2009expander} only on the sample set vectors, which are at most $|T|$ in number. By~\Cref{thm:sample_set} there is such a sample set $T$ of size at most $\OO(k)$. This helps replace the $\OO(\sqrt{\log n})$ factor by $\OO(\sqrt{\log k})$. However, for the sake of completeness and clarity we give the full algorithm by mostly following the algorithm of~\cite{agarwal2005log} with a few modifications/simplifications.

We will restrict our attention to edge cuts, as again the standard reduction in~\Cref{thm:reduction} can easily reduce the vertex version to the edge version.



Notice that if a separator $F$ is $b$-balanced for some $b = \Omega(1)$, then every strongly connected component of $G \setminus F$ has size at most $bn$. This in turn means that using a prefix of the topological order of the strongly connected components in $G \setminus F$, we can obtain sets $A,B$ with $|A|, |B| \geq c'n$ for  $c' = \frac{1-b}{2}$, such that there are no arcs from $A$ to $B$ in $G \setminus F$. Thus, upto constant factors in the balance, one can equivalently think of the~\DB{} problem as that of finding a set of vertices (arcs) $F$ with $|F| \leq k$ and a partition $A,B$ of $G \setminus F$, so that $|A|, |B| \geq c'n$ and there are no arcs from $A$ to $B$ in $G \setminus F$. 

Now we consider the terminal version of the problem, to which we will reduce via sample sets later. Given a terminal set $T$, our revised goal is to solve the following problem: given a parameter $c = \Omega(1)$, find a set of edges $F$ with $|F| \leq k$ so that there is a partition of $G \setminus F$ into $A, B$ such that $|A \cap T|, |B \cap T| \geq c|T|$ and there are no arcs from $A$ to $B$ in $G \setminus F$. In what follows, we will mostly follow the algorithm of~\cite{agarwal2005log} with small modifications which we state as and when required. 




We assume that the vertices are labelled $\{1,2, \ldots n\}$, and add a special vertex $0$, which will be a non-terminal and will be the reference vertex for the 
``$A$- side'' of the cut. We start with the SDP relaxation (extended in our case to the terminal version) in~\cite{agarwal2005log} and follow their algorithm.

\begin{center}
\noindent\fbox{

  \begin{minipage}{0.95\textwidth}
  $$\min \frac{1}{8} \sum_{e=\{i,j\} \in E(G)} d(i,j)$$
  $$\|v_i\|^2 = 1 \;\;\forall i \in [n] \cup \{0\}$$
  $$\|v_i - v_j\|^2 + \|v_j - v_k\|^2 \geq \|v_i - v_k\|^2 \;\;\forall i,j,k \in [n] \cup \{0\}$$
  $$\sum_{i < j, i,j \in T} \|v_i - v_j\|^2 \geq 4c(1-c)|T|^2$$
  
  
  
  \end{minipage}
 
      }
\end{center}
Here the ``directed distance'' $d(i,j) = \|v_i - v_j\|^2 +\|v_j - v_0\|^2 - \|v_i - v_0\|^2$ is as defined in~\cite{agarwal2005log}. The cannonical solution for this SDP is obtained by setting $v_i = v_0$ for each $i \in A$ and $v_i = -v_0$ for each $i \in B$.

\begin{theorem}[ARV separation theorem~\cite{arora2009expander}, as stated in~\cite{agarwal2005log}] Given a set of $\ell_2^2$ unit vectors $v_i$, $i \in [n]$, such that $\sum_{i < j} \|v_i -v_j\|^2 \geq 4c(1-c)n^2$, there exists a polynomial time algorithm that finds disjoint sets $L,R$ with $|L|, |R| = \Omega(n)$ such that for any $i \in L, j \in R$, we have $\|v_i - v_j\|^2 \geq \Omega({1}/{\sqrt{\log n}})$.
\label{thm:structure}


\end{theorem}



Now we follow Algorithm 4 of~\cite{agarwal2005log}. Step 1 solves the SDP. In Step 2, given the vectors from the SDP, we apply the ARV separation theorem to only the vectors $v_i$, $i \in T$ to find $\Delta = \frac{1}{\sqrt{\log |T|}}$ separated sets $L$ and $R$. Concretely, we obtain disjoint sets $L, R \subseteq T$ with $|L|, |R| = \Omega(|T|)$ so that $\|v_i - v_j\|^2 \geq \Delta$ for any $i \in L$ and $j \in R$.


Next, we define $r$ so that  both 
$L^+ = \{i \in L \mid |v_0 - v_i|^2 \leq r^2\}$ and $L^{-} = \{i \in L \mid |v_0 - v_i|^2 \geq r^2\}$ have more than $\frac{|L|}{2}$ vertices. Note that such an $r$ always exists. Once we fix $r$, we define $R^+ = \{i \in R \mid |v_0 - v_i|^2 \leq r^2\}$ and $R^{-} = \{i \in R \mid |v_0 - v_i|^2 \geq r^2\}$ similarly. Finally, if $|R^+| > \frac{|R|}{2}$ we compute a directed min-cut $F$ between $R^+$ and $L^-$, else we compute a directed min-cut $F$ between $L^+$ and $R-$. 

\begin{lemma}\label{lemma:approx}
$|F| \leq \OO(\frac{1}{\Delta} SDP)$ and $F$ is a $b^*$-balanced separator with respect to $T$ for some $b^* < 1$ depending on $c$, where SDP is the optimal SDP value.
\end{lemma}

\begin{proof}\textcolor{red}{TOPROVE 8}\end{proof}



\begin{proof}\textcolor{red}{TOPROVE 9}\end{proof}


\section{Vertex Cut Sparsifiers}

In this section, we present our result on vertex cut sparsifiers. We start with a few definitions.


\begin{definition}
Given a directed graph $G$, a set of terminals $T \subseteq V(G)$, and an integer $c$, a directed graph $G'$ satisfying $T \subseteq V(G')$ is said to be a $(c,T)$ vertex cut sparsifier for $G$ if for any partition $A \cup B$ of $T$ such that the size of the minimum $A$-$B$ (vertex) cut in $G$ is at most $c$, the size of the $A$-$B$ mincut is the same in $G$ and $G'$.
\end{definition}


Our next result gives our deterministic construction of vertex sparsifiers in directed graphs.
\mimicking*


To prove this result, we use our result on important separators combined with the standard approach of ``closing'' unnecessary vertices - given a graph $G$ and a vertex $v \in V(G)$, applying the closure operation to $v$ gives the graph $G'$ with the vertex set $V(G) \setminus \{v\}$. The arc set of $G'$ is the same as that of $G$, except that it excludes all arcs incident on $v$, and for every pair of vertices $(u,w)$ such that $u$ is an in-neighbour of $v$ and $w$ is an out-neighbour of $v$,
we add the arc $(u,w)$ to $E(G')$. Before we prove this result, we have the following basic observation. Recall that given a graph $H$ and sets $S,Z \subseteq V(H)$, $R_H^S(Z)$ denotes the set of vertices reachable from some vertex of $S$ in $H \setminus Z$.

\begin{lemma}\label{lemma:closurereach}
Let $A,Y \subseteq V(G)$ and let $G'$ be the graph obtained by applying the closure operation on a vertex $v \in V(G)$. Also suppose $v \notin Y$. Then,

\begin{enumerate}
\item $R_{G'}^A(Y) \subseteq R^A_{G}(Y)$.
\item $R_G^A(Y) \setminus \{v\} \subseteq R^A_{G'}(Y)$ 


\end{enumerate}

\end{lemma}

\begin{proof}\textcolor{red}{TOPROVE 10}\end{proof}










The next two lemmas show that applying the closure operation to a single vertex which is not in any $A$-$B$ important separator of size at most $c$, across all partitions $A \cup B$ of $T$, preserves the set of important separators. This in turn will imply that we can apply the closure operation at once to all such vertices.  

\begin{lemma}\label{lemma:mimicking1}
Given a directed graph $G$, disjoint sets $A$, $B$ and a vertex $v \notin A \cup B$ which is not in any $A$-$B$ important separator of size $\leq c$, let $G'$ be the graph obtained by closing $v$. Then if $X \subseteq V(G')$, $|X| \leq c$, is an important $A$-$B$ separator in $G'$, then it is an $A$-$B$ important separator in $G$ as well.
\end{lemma}
\begin{proof}\textcolor{red}{TOPROVE 11}\end{proof}










The next lemma is analogous and proves the other direction.

\begin{lemma}\label{lemma:mimicking2}
Given a directed graph $G$, disjoint sets $A$, $B$ and a vertex $v$ which is not in any $A-B$ important separator of size $\leq c$, let $G'$ be the graph obtained by closing $v$. Then if $X \subseteq V(G)$, $|X| \leq c$ is an $A$-$B$ important separator in $G$, then it is an $A$-$B$ important separator in $G'$ as well.
\end{lemma}

\begin{proof}\textcolor{red}{TOPROVE 12}\end{proof}






\begin{proof}\textcolor{red}{TOPROVE 13}\end{proof}
}
 
\bibliographystyle{alpha}
\bibliography{references}

\appendix

\section{Appendix}

\lowerbound*
\begin{proof}\textcolor{red}{TOPROVE 14}\end{proof}

\begin{lemma}[Tight example for important separator preservation]\label{lemma:impseppreserve}
There is a graph $G$, source vertex $s \in V(G)$, integer $k$, sink vertices $B \subseteq V(G)$ with $|B| = 2k$ and an $s$-$B$ important separator $X \subseteq V(G)$ with $|X| = k$, such that $X$ is not an $s$-$B'$ important separator for any $B' \subset B$.
\end{lemma}

\begin{proof}\textcolor{red}{TOPROVE 15}\end{proof}




















































































  
  
  
 






































 













































 





\end{document}
