\documentclass[]{theclass} 
\usepackage{graphicx, xfrac, lineno, float, subcaption, tasks, comment, xcolor, booktabs, multirow}
\usepackage[T1]{fontenc}
\usepackage[normalem]{ulem}
\usepackage{datetime}
\usepackage{colortbl}
\usepackage{enumitem}
\usepackage{soul}
\DeclareMathOperator{\lcm}{lcm}



\renewcommand\arraystretch{1.2}
\newcommand{\ed}[1]{{\textcolor{blue}{\footnotesize \textbf{ed: }\bf \em {#1}}}}

\begin{document}
\begin{frontmatter}

\titledata{Three-cuts are a charm:\\ acyclicity in 3-connected cubic graphs}
{The authors were partially supported by VEGA 1/0743/21, VEGA 1/0727/22 and APVV-19-0308.}

\authordata{Franti\v{s}ek Kardo\v{s}}
{LaBRI, CNRS, University of Bordeaux, Talence, F-33405, France; \\ Department of Computer Science, Faculty of Mathematics, Physics and Informatics\\ Comenius University, Mlynsk\'{a} Dolina, 842 48 Bratislava, Slovakia}
{frantisek.kardos@u-bordeaux.fr}{}
\authordata{Edita M\'{a}\v{c}ajov\'{a}}
{Department of Computer Science, Faculty of Mathematics, Physics and Informatics\\ Comenius University, Mlynsk\'{a} Dolina, 842 48 Bratislava, Slovakia}
{macajova@dcs.fmph.uniba.sk}{}
\authordata{Jean Paul Zerafa}
{St. Edward's College, Triq San Dwardu\\ Birgu (Citt\`{a} Vittoriosa), BRG 9039, Cottonera, Malta; \\ Department of Computer Science, Faculty of Mathematics, Physics and Informatics\\ Comenius University, Mlynsk\'{a} Dolina, 842 48 Bratislava, Slovakia; \\
Faculty of Economics, Management \& Accountancy, and Faculty of Education\\ L-Universit\`{a} ta' Malta, Msida, MSD 2080, Malta}
{zerafa.jp@gmail.com}
{}


\keywords{acyclicity, circuit, factor, perfect matching, cubic graph, snark}
\msc{05C15, 05C70}

\begin{abstract}
Let $G$ be a bridgeless cubic graph. In 2023, the three authors solved a conjecture (also known as the $S_4$-Conjecture) made by Mazzuoccolo in 2013: there exist two perfect matchings of $G$ such that the complement of their union is a bipartite subgraph of $G$. They actually show that given any $1^+$-factor $F$ (a spanning subgraph of $G$ such that its vertices have degree at least 1) and an arbitrary edge $e$ of $G$, there exists a perfect matching $M$ of $G$ containing $e$ such that $G\setminus (F\cup M)$ is bipartite. This is a step closer to comprehend better the Fan--Raspaud Conjecture and eventually the Berge--Fulkerson Conjecture. The $S_4$-Conjecture, now a theorem, is also the weakest assertion in a series of three conjectures made by Mazzuoccolo in 2013, with the next stronger statement being: there exist two perfect matchings of $G$ such that the complement of their union is an acyclic subgraph of $G$. Unfortunately, this conjecture is not true: Jin, Steffen, and Mazzuoccolo later showed that there exists a counterexample admitting 2-cuts. Here we show that, despite of this, every cyclically 3-edge-connected cubic graph satisfies this second conjecture. 
\end{abstract}

\end{frontmatter}

\section{Introduction}

In 2013, Giuseppe Mazzuoccolo \cite{MazzuoccoloS4} proposed three beguiling conjectures about bridgeless cubic graphs. His first conjecture, implied by the Berge--Fulkerson Conjecture \cite{BergeFulkerson}, is the following.

\begin{conjecture}[Mazzuoccolo, 2013 \cite{MazzuoccoloS4}]
Let $G$ be a bridgeless cubic graph. Then, there exist two perfect matchings of $G$ such that the complement of their union is a bipartite graph.
\end{conjecture}

This conjecture, which is no longer open, has been solved by the three authors. More precisely they prove the following stronger statement.

\begin{theorem}[Kardo\v{s}, M\'{a}\v{c}ajov\'{a} \& Zerafa, 2023 \cite{quelling1}]\label{theorem quelling 1}
Let $G$ be a bridgeless cubic graph. Let $F$ be a $1^+$-factor of $G$ and let $e\in E(G)$. Then, there exists a perfect matching $M$ of $G$ such that $e\in M$, and $G\setminus (F\cup M)$ is bipartite.
\end{theorem}

We note that a \emph{$1^+$-factor} of $G$ is the edge set of a spanning subgraph of $G$ such that its vertices have degree 1, 2 or 3. Theorem \ref{theorem quelling 1} not only shows the existence of two perfect matchings of $G$ whose deletion leaves a bipartite subgraph of $G$, but that for every perfect matching of $G$ there exists a second one such that the deletion of the two leaves a bipartite subgraph of $G$. In particular, Theorem \ref{theorem quelling 1} also implies that for every collection of disjoint odd circuits of $G$, there exists a perfect matching  which intersects at least one edge from each odd circuit (this was posed as an open problem by Mazzuoccolo and the last author in \cite{s4gmjp}, see also \cite{zerafa thesis}).

Mazzuoccolo moved on to propose two stronger conjectures, with Conjecture \ref{conjecture acyclic+} being the strongest of all three.

\begin{conjecture}[Mazzuoccolo, 2013 \cite{MazzuoccoloS4}]\label{conjecture acyclic}
Let $G$ be a bridgeless cubic graph. Then, there exist two perfect matchings of $G$ such that the complement of their union is an acyclic graph.
\end{conjecture} 

\begin{conjecture}[Mazzuoccolo, 2013 \cite{MazzuoccoloS4}]\label{conjecture acyclic+}
Let $G$ be a bridgeless cubic graph. Then, there exist two perfect matchings of $G$ such that the complement of their union is an acyclic graph, whose components are of order 2 or 3.
\end{conjecture}

Clearly, these last two conjectures are true for 3-edge-colourable cubic graphs, and Janos H\"agglund verified the strongest of these conjectures (Conjecture \ref{conjecture acyclic+}) by computer for all non-trivial snarks (non 3-edge-colourable cubic graphs) of order at most 34 \cite{MazzuoccoloS4}. 
However, 5 years later, Jin, Steffen, and Mazzuoccolo \cite{dmgt} gave a counterexample to Conjecture \ref{conjecture acyclic}. Their counterexample contains a lot of 2-edge-cuts and the authors state that the conjecture "could hold true for 3-connected or cyclically 4-edge-connected cubic graphs". In fact, as in real life, being more connected has its own benefits, and in this paper we show the following stronger statement.

\begin{theorem}\label{theorem acyclic}
Let $G$ be a cyclically 3-edge-connected cubic graph, which is not a Klee-graph. Then, for any $e\in E(G)$ and any $1^+$-factor $F$ of $G$, there exists a perfect matching $M$ of $G$ containing $e$ such that $G\setminus (F\cup M)$ is acyclic.
\end{theorem}

We 
remark that Klee-graphs (see Definition \ref{definition klee}), which are to be discussed further in Section \ref{section kleegraphs}, are 3-edge-colourable cubic graphs and so are not a counterexample to Conjecture \ref{conjecture acyclic}. However, the stronger statement given in Theorem \ref{theorem acyclic} does not hold for this class of graphs, and this is the reason why we exclude them. 



Although Theorem \ref{theorem acyclic} is not a direct consequence of the Berge--Fulkerson Conjecture, we believe that the results presented here and in \cite{quelling1} are valuable steps towards trying to decipher long-standing conjectures such as the Fan--Raspaud Conjecture \cite{FanRaspaud}, and the Berge--Fulkerson Conjecture itself.

In fact, we will prove the following statement, which is equivalent to Theorem \ref{theorem acyclic}.
\begin{theorem}\label{theorem acyclic2}
Let $G$ be a cyclically 3-edge-connected cubic graph, which is not a Klee-graph. Then, for any $e\in E(G)$ and any collection of disjoint circuits $\mathcal{C}$, there exists a perfect matching $M$ of $G$ containing $e$ such that every circuit in $\mathcal{C}$ contains an edge from $M$.
\end{theorem}

Indeed, given a collection of disjoint circuits $\mathcal{C}$, its complement is a $1^+$-factor, say $F_{\mathcal{C}}$. A perfect matching $M$ containing $e$ such that $G \setminus (F_{\mathcal{C}} \cup M)$ is acyclic must contain an edge from every circuit in $\mathcal{C}$. On the other hand, given a $1^+$-factor $F$, its complement is a collection of disjoint paths and circuits, and so it suffices to consider the collection $\mathcal{C}_F$ of circuits disjoint from $F$. A perfect matching $M$ containing $e$ such that every circuit in $\mathcal{C}_F$ contains an edge from $M$, clearly makes 
$G\setminus (F \cup M)$ acyclic. 

\subsection{Important definitions and notation}

Graphs considered in this paper are simple, that is, they cannot contain parallel edges and loops, unless otherwise stated. 

Let $G$ be a graph and $(V_1,V_2)$ be a partition of its vertex set, that is, $V_1\cup V_2=V(G)$ and $V_1\cap V_2=\emptyset$. Then, by $E(V_1,V_2)$ we denote the set of edges having one endvertex in $V_1$ and one in $V_2$; we call such a set an \emph{edge-cut}. An edge which itself is an edge-cut of size one is a \emph{bridge}. A graph which does not contain any bridges is said to be \emph{bridgeless}.

An edge-cut $X=E(V_1,V_2)$ is called \emph{cyclic} if both graphs $G[V_1]$ and $G[V_2]$, obtained from $G$ after deleting $X$, contain a \emph{circuit} (a 2-regular connected subgraph). The \emph{cyclic edge-connectivity} of a graph $G$ is defined as the smallest size of a cyclic edge-cut in $G$ if $G$ admits one; it is defined as $|E(G)|-|V(G)|+1$, otherwise. For cubic graphs, the latter only concerns $K_4$, $K_{3,3}$, and the graph consisting of two vertices joined by three parallel edges, whose cyclic edge-connectivity is thus 3, 4, and 2, respectively. An \emph{acyclic} graph is a graph which does not contain any circuits.

Let $G$ be a bridgeless cubic graph. A \emph{$1^+$-factor} of $G$ is the edge set of a spanning subgraph of $G$ such that its vertices have degree 1, 2 or 3. In particular, a \emph{perfect matching} and a \emph{$2$-factor} of $G$ are $1^+$-factors whose vertices have exactly degree 1 and 2, respectively. 

\section{Klee-graphs}\label{section kleegraphs}

\begin{definition}[\cite{klee dmtcs}]\label{definition klee}
A graph $G$ is a Klee-graph if $G$ is the complete graph on 4 vertices $K_4$ or there exists a Klee-graph $G_0$ such that $G$ can be obtained from $G_0$ by replacing a vertex by a triangle (see Figure \ref{fig:exklee}).
\end{definition}

\begin{figure}[ht]
    \centering
    \includegraphics{klee.4}
    \includegraphics{klee.6}
    \includegraphics{klee.8}
    \includegraphics{klee.10}
    \includegraphics{klee.12}
    \caption{Examples of Klee-graphs on 4 upto 12 vertices, left to right.}
    \label{fig:exklee}
\end{figure}
For simplicity, if a graph $G$ is a Klee-graph, we shall sometimes say that $G$ is Klee. We note that there is a unique Klee-graph on 6 vertices (the graph of a 3-sided prism), and a unique Klee-graph on 8 vertices. As we will see in Section \ref{subsection klee}, these two graphs are Klee ladders, and shall be respectively denoted as $KL_6$ and $KL_8$. 

\begin{lemma}[\cite{klee dmtcs}]
The edge set of any Klee-graph can be uniquely partitioned into three pairwise
disjoint perfect matchings. In other words, any Klee-graph is 3-edge-colourable, and the colouring is unique up to a permutation of the colours.
\end{lemma}

Since Klee-graphs are 3-edge-colourable, they easily satisfy the statement of Conjecture \ref{conjecture acyclic}.

\begin{proposition}\label{prop klee}
Let $G$ be a Klee-graph. Then, $G$ admits two perfect matchings $M_1$ and $M_2$ such that $G\setminus (M_1\cup M_2)$ is acyclic.
\end{proposition}

The new graph obtained after expanding a vertex of a Hamiltonian graph (not necessarily Klee) into a triangle is still Hamiltonian, and so, since $K_4$ is Hamiltonian, all Klee-graphs are Hamiltonian. Hamiltonian cubic graphs have the following distinctive property.

\begin{proposition}\label{prop ham collection}
Let $G$ be a Hamiltonian cubic graph. Then, for any collection of disjoint circuits $\mathcal{C}$ of $G$ there exists a perfect matching $M$ of $G$ which intersects at least one edge of every circuit in $\mathcal{C}$.
\end{proposition}

\begin{proof}\textcolor{red}{TOPROVE 0}\end{proof}

\begin{corollary}\label{cor klee collection}
For any collection of disjoint circuits $\mathcal{C}$ of a Klee-graph $G$ there exists a perfect matching $M$ of $G$ which intersects at least one edge of every circuit in $\mathcal{C}$.
\end{corollary}

On the other hand, we have to exclude Klee-graphs from Theorem \ref{theorem acyclic} (and Theorem \ref{theorem acyclic2}) since for some Klee-graphs there are edges contained in a unique perfect matching, as we will see in the following subsection.

\subsection{Other results about Klee-graphs}\label{subsection klee}

\begin{lemma}[\cite{klee dmtcs}]\label{lemma 2 disjoint triangles klee}
Let $G$ be a Klee-graph on at least 6 vertices. Then, $G$ has at least two triangles and all its triangles are vertex-disjoint.
\end{lemma}

Indeed, expanding a vertex into a triangle can only destroy triangles containing the vertex to be expanded. 

We will now define a series of particular Klee-graphs, which we will call \emph{Klee ladders}. Let $KL_4$ be the complete graph on 4 vertices, and let $u_4v_4$ be an edge of $KL_4$. For any even $n\ge 4$, let $KL_{n+2}$ be the Klee-graph obtained from $KL_n$ by expanding the vertex $u_n$ into a triangle. In the resulting graph $KL_{n+2}$, we denote the vertex corresponding to $v_n$ by $v_{n+2}$, and denote the vertex of the new triangle adjacent to $v_{n+2}$ by $u_{n+2}$.

In other words, the graph $KL_{2k+2}$ consists of the Cartesian product $P_2 \square P_k$ (where $P_t$ denotes a path on $t$ vertices) with two additional vertices $u_{2k+2}$ and $v_{2k+2}$ adjacent to each other, such that $u_{2k+2}$ ($v_{2k+2}$) is adjacent to the two vertices in the first (last, respectively) copy of $P_2$ in $P_2 \square P_k$ (see Figure \ref{fig:kleeLadder}).

Klee ladders can be used to illustrate why we have to exclude Klee-graphs from our main result. For a given Klee ladder $G$ there exists an edge $e$ such that $e$ is contained in a unique perfect matching of $G$, and therefore there is no hope for a statement like Theorem \ref{theorem acyclic2} to be true.

\begin{figure}[ht]
      \centering
      \includegraphics{ladders.4}
      \caption{An example of a Klee ladder, $KL_{12}$. {There is a unique perfect matching (here depicted using dotted lines) containing the edge $e$.} The complement of this perfect matching is a Hamiltonian circuit.}
      \label{fig:kleeLadder}
\end{figure}



We will frequently use the following structural property of certain Klee-graphs.

\begin{lemma}\label{lemma 4 circuit klee}
Let $G$ be a Klee-graph on at least 8 vertices having exactly two (disjoint) triangles. Then, 
\begin{enumerate}[label=(\roman*)]
    \item exactly one edge of each triangle lies on a 4-circuit; and 
    \item if $G$ admits an edge joining the two triangles, then $G$ is a Klee ladder.
\end{enumerate}
\end{lemma}

\begin{proof}\textcolor{red}{TOPROVE 1}\end{proof}

\section{Proof of Theorem \ref{theorem acyclic2}}

\begin{proof}\textcolor{red}{TOPROVE 2}\end{proof}

Here are some consequences of Theorem \ref{theorem acyclic2}. Corollary \ref{cor 3ec collection} follows by the above result and Corollary \ref{cor klee collection}.

\begin{corollary}\label{cor 3ec collection}
Let $G$ be a cyclically 3-edge-connected cubic graph and let $\mathcal{C}$ be a collection of disjoint circuits of $G$. Then, there exists a perfect matching $M$ such that $M\cap E(C)\neq \emptyset$, for every $C\in\mathcal{C}$.
\end{corollary}

\begin{corollary}
Let $G$ be a cyclically 3-edge-connected cubic graph. For every perfect matching $M_1$ of $G$, there exists a perfect matching $M_2$ of $G$ such that $G\setminus (M_1\cup M_2)$ is acyclic.
\end{corollary}

\begin{thebibliography}{99}

\bibitem{klee dmtcs}
M. Cygan, M. Pilipczuk, R. \v{S}krekovski,
A bound on the number of perfect matchings in Klee-graphs,
\emph{Discrete Math. Theor. Comput. Sci.} \textbf{15(1)} (2013), 37--54.

\bibitem{EsperetLovaszPlummer}
L. Esperet, F. Kardo\v{s}, A.D. King, D. Kr\'{a}l$\!$' and S. Norine,
Exponentially many perfect matchings in cubic graphs,
\emph{Adv. Math.} \textbf{227} (2011), 1646--1664.

\bibitem{EKK}
L. Esperet, F. Kardoš, D. Kráľ, A superlinear bound on the number of perfect matchings in cubic bridgeless graphs, \emph{Eur. J. Comb} \textbf{33} (2012), 767--798.

\bibitem{FanRaspaud}
G. Fan and A. Raspaud,
Fulkerson's Conjecture and circuit covers,
\emph{J. Combin. Theory Ser. B} \textbf{61(1)} (1994), 133--138.

\bibitem{BergeFulkerson}
D.R. Fulkerson,
Blocking and anti-blocking pairs of polyhedra,
\emph{Math. Program.} \textbf{1(1)} (1971), 168--194.



\bibitem{dmgt}
L. Jin, E. Steffen, G. Mazzuoccolo,
Cores, joins and the Fano-flow conjectures,
\emph{Discuss. Math. Graph Theory} \textbf{38(1)} (2018), 165--175.

\bibitem{quelling1} 
F. Kardoš, E. M\'{a}\v{c}ajov\'{a} and J.P. Zerafa, Disjoint odd circuits in a bridgeless cubic graph can be quelled by a single perfect matching, \emph{J. Combin. Theory Ser. B} \textbf{160} (2023), 1--14, \url{https://doi.org/10.1016/j.jctb.2022.12.003}





\bibitem{MazzuoccoloS4}
G. Mazzuoccolo,
New conjectures on perfect matchings in cubic graphs,
\emph{Electron. Notes Discrete Math.} \textbf{40} (2013), 235--238.



\bibitem{s4gmjp}
G. Mazzuoccolo and J.P. Zerafa,
An equivalent formulation of the Fan--Raspaud Conjecture
and related problems, \emph{Ars Math. Contemp.} \textbf{18} (2020), 87--103.



\bibitem{voorhoeve}
M. Voorhoeve, A lower bound for the permanents of certain $(0,1)$-matrices, \emph{Nederl. Akad. Wetensch. Indag. Math.} \textbf{41} (1979), 83--86.

\bibitem{zerafa thesis}
J.P. Zerafa, \emph{On the consummate affairs of perfect matchings}, PhD Thesis, Universit\`a degli Studi di Modena e Reggio Emilia, Italy, 2021, \url{https://hdl.handle.net/11380/1237629}.
\end{thebibliography}

\end{document}