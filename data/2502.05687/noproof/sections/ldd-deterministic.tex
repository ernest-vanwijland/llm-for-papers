\section{Near-Optimal LDDs Deterministically} \label{sec:ldd-deterministic}
In this section, we present the deterministic algorithm for computing a near-optimal LDD, thereby proving our second main theorem:

\thmMainDet*

To this end, we utilize many of the same building blocks we have already introduced in \Cref{sec:ldd-expander}. In particular, we follow the general framework of Multiplicative Weights Update to reduce the computation of an LDD to solving the cost-minimizing task. The full proof of \Cref{thm:main-det} can be found in the end of the section. The following lemma restates the MWU method algorithmically; we omit a proof as it follows exactly the proof of \cref{lem:mwu} in \Cref{sec:ldd-expander}. 

\begin{lemma}[Algorithmic Multiplicative Weight Update]
Let $G = (V, E)$ be a directed graph and let $D \geq 1$. Suppose that there is an algorithm $\mathcal A$ that, given $G$, $D$ and a cost function $c : E \to [|V|^{10}]$, computes a set of edges $S \subseteq E$ satisfying the following properties:
\begin{itemize}
	\item For any two nodes $u, v \in V$ that are part of the same strongly connected component in $G \setminus S$, we have $d_G(u, v) \leq D$ and $d_G(v, u) \leq D$.
	\item $c(S) \leq c(E) \cdot \frac{L}{D}$.
\end{itemize}
Then there is a deterministic algorithm to compute an LDD with loss $\Order(L)$ for $G$ (i.e., we compute the full support of a uniform distribution over $\Order(D \log n)$ cut sets). It runs in time~\smash{$\widetilde\Order(m D)$} and issues $\Order(D \log n)$ oracle calls to $\mathcal A$.
\end{lemma}

For the remainder of this section we will therefore focus on the same cost minimizer setting: Given a directed graph $G = (V, E, c)$ with edge capacities (and unit lengths), the goal is select a set of cut edges $S \subseteq E$ such that $c(S) \leq c(E) \cdot \Order(\frac{1}{D} \cdot \log n \log\log n)$ and such that all strongly connected components in the remaining graph $G \setminus S$ have (weak) diameter at most $D$.

The following lemma is a consequence of the lopsided expander machinery set up before:

\begin{lemma}[Finding Sparse Cuts] \label{lem:sparse-cut-det}
Let $G = (V, E, c)$ be a directed graph, let $D \geq \log \vol(V)$ and let $v \in V$. Then there is some $\psi = \Order(\frac{1}{D} \cdot \log\log \vol(V))$ and an algorithm to determine which of the following cases applies:
\begin{enumerate}[label=(\roman*)]
	\item There is a radius $0 \leq r \leq D$ with $\psi(B^+(v, r)) \leq \psi$ and $c(B^+(v, r)) \leq 0.95 \cdot \vol(V)$.\\(In this case the algorithm runs in linear time in the number of edges incident to $B^+(v, r)$.)
	\item Or, there is a radius $0 \leq r \leq D$ such that $\psi(\overline{B^-(v, r)}) \leq \psi$ and $c(B^-(v, r)) \leq 0.95 \cdot \vol(V)$.\\(In this case the algorithm runs in linear time in the number of edges incident to $B^-(v, r)$.)
	\item Or, $c(B^+(v, D) \cap B^-(v, D)) \geq 0.9 \cdot \vol(V)$.\\(In this case the algorithm runs in time~\smash{$\Order(m)$}.)
\end{enumerate}
\end{lemma}
\begin{proof}\textcolor{red}{TOPROVE 0}\end{proof}

Having established \cref{lem:sparse-cut-det}, now consider the algorithm in \cref{alg:det}. In summary, it runs in two phases. In Phase (I) we first repeatedly select a node $v$ and attempt to cut a lopsided sparse cut around $v$ (i.e., we cut the edges in $\delta^+(B^+(v, r))$ for some radius $r$). We only execute these cuts, however, until we find a node $z$ for which $c(B^+(z, D') \cap B^-(z, D')) \geq 0.9 \cdot \vol(V)$---that is, both the radius-$D'$ out- and in-balls of $z$ make up for a big constant fraction of the entire graph. We call $z$ a \emph{center} node and move on to phase (II). In this phase we repeat the same steps as in Phase (I), but we only choose nodes $v$ that have distance at least $2D'$ (in one direction or the other) to the center $z$. The intuition is that we can never find a second node $z'$ which equally makes up for the entire graph, as then $z$ and $z'$ would have to be connected by a short path. In the remainder of this section we formally analyze \cref{alg:det}.

\begin{algorithm}[t]
	\caption{The deterministic near-optimal LDD, see \Cref{thm:main-det}.} \label{alg:det}
	\begin{enumerate}[label=\arabic*.]
		\item[(I)] Repeat the following steps: Take an arbitrary node $v \in V$ and apply \cref{lem:sparse-cut-det} with parameter \smash{$D' = \floor{\frac{D}{4}}$}. Depending on the output execute the following steps:
		\begin{enumerate}[label=(\roman*)]
			\item Cut all edges in $\delta^+(B^+(v, r))$, recurse on the induced graph $G[B^+(v, r)]$, then remove all nodes in $B^+(v, r)$ from the graph.
			\item Cut all edges $\delta^-(B^-(v, r))$, recurse on the induced graph $G[B^-(v, r)]$, then remove all nodes in $B^-(v, r)$ from the graph.
			\item Remember $z \gets v$ (called the \emph{center} node) and continue with Phase (II).
		\end{enumerate}
		\item[(II)] Compute the sets
		\begin{align*}
			X &= B^+(z, D') \cap B^-(z, D'), \\
			Y &= B^+(z, 2D') \cap B^-(z, 2D').
		\end{align*}
		Then repeat the following steps while there still exists nodes in $V \setminus Y$: Take an arbitrary node~\makebox{$u \in V \setminus Y$} and apply \cref{lem:sparse-cut-det} with parameter~$D'$. Depending on the output execute the following steps:
		\begin{enumerate}[label=(\roman*)]
			\item Cut all edges in $\delta^+(B^+(v, r))$, recurse on the induced graph $G[B^+(v, r)]$, then remove all nodes in $B^+(v, r)$ from the graph.
			\item Cut all edges $\delta^-(B^-(v, r))$, recurse on the induced graph $G[B^-(v, r)]$, then remove all nodes in $B^-(v, r)$ from the graph.
		\end{enumerate}
	\end{enumerate}
\end{algorithm}

\begin{lemma}[Total Cost of \cref{alg:det}] \label{lem:ldd-det-cost}
Let $S \subseteq E$ denote the set of edges cut by \cref{alg:det}. Then $c(S) \leq c(E) \cdot \Order(\frac{1}{D} \cdot \log \vol(V) \log\log \vol(V))$.
\end{lemma}
\begin{proof}\textcolor{red}{TOPROVE 1}\end{proof}

\begin{lemma}[Well-Definedness of \cref{alg:det}] \label{lem:ldd-det-well-defined}
While executing Phase~(II) of \cref{alg:det}, the subcase~(iii) never happens.
\end{lemma}
\begin{proof}\textcolor{red}{TOPROVE 2}\end{proof}

\begin{lemma}[Correctness of \cref{alg:det}] \label{lem:ldd-det-correctness}
Let $S \subseteq E$ denote the set of edges cut by \cref{alg:det}. Then for any two nodes $u, w$ in the same strongly connected component in $G \setminus S$, it holds that $d_G(u, w) \leq D$ and $d_G(w, u) \leq D$.
\end{lemma}
\begin{proof}\textcolor{red}{TOPROVE 3}\end{proof}

\begin{lemma}[Running Time of \cref{alg:det}] \label{lem:ldd-det-time}
\cref{alg:det} runs in time $\Order(m \log \vol(V))$.
\end{lemma}
\begin{proof}\textcolor{red}{TOPROVE 4}\end{proof}

This completes the analysis of \cref{alg:det} and puts us in the position of completing the proof of \cref{thm:main-det}.

\begin{proof}\textcolor{red}{TOPROVE 5}\end{proof}