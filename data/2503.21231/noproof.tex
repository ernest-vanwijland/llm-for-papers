\documentclass[11pt]{amsart}


\usepackage[mathscr]{eucal}
\usepackage{amsmath,amssymb,amsfonts,amsthm,enumerate}
\usepackage[colorlinks]{hyperref}


\textwidth16cm \textheight21cm \oddsidemargin-0.1cm
\evensidemargin-0.1cm
\renewcommand{\baselinestretch}{1.2}

\renewcommand{\subjclassname}{\textup{2010} Mathematics Subject
Classification}


\newtheorem{theorem}{Theorem}[section]
\newtheorem{definition}{Definition}
\newtheorem{lemma}[theorem]{Lemma}
\newtheorem{corollary}[theorem]{Corollary}
\newtheorem{proposition}[theorem]{Proposition}
\newtheorem{conjecture}[theorem]{Conjecture}



\theoremstyle{definition}
\newtheorem{claim}{}[theorem]
\renewcommand{\theclaim}{\textbf{A}\textbf{\arabic{claim}}}


\newcommand{\N}{\mathbb N}
\newcommand{\Z}{\mathbb Z}
\newcommand{\R}{\mathbb R}
\newcommand{\Q}{\mathbb Q}
\newcommand{\F}{\mathbb F}
\newcommand{\C}{\mathbb C}

\DeclareMathOperator{\ord}{ord}
\DeclareMathOperator{\ind}{ind}
\DeclareMathOperator{\lcm}{lcm}
\DeclareMathOperator{\rk}{rk}
\DeclareMathOperator{\diam}{diam}
\DeclareMathOperator{\supp}{supp}
\DeclareMathOperator{\Hom}{Hom}
\DeclareMathOperator{\Tor}{Tor}
\DeclareMathOperator{\Map}{Map}
\newcommand{\gcds}{{\gcd}^*}


\newcommand{\la}{\langle}
\newcommand{\ra}{\rangle}
\newcommand{\be}{\begin{equation}}
\newcommand{\ee}{\end{equation}}
\newcommand{\und}{\;\mbox{ and }\;}
\newcommand{\nn}{\nonumber}
\newcommand{\ber}{\begin{eqnarray}}
\newcommand{\eer}{\end{eqnarray}}
\newcommand{\Sum}[2]{\underset{#1}{\overset{#2}{\sum}}}
\newcommand{\Summ}[1]{\underset{#1}{\sum}}

\newcommand{\bs}{\boldsymbol}
\newcommand{\numbrace}[2]{\underset{#1}{\underbrace{#2}}}
\newcommand{\wtilde}{\widetilde}
\newcommand{\barr}[2]{\overline{{#1\;}^{#2}}}
\newcommand\floor[1]{{\lfloor{#1}\rfloor}}
\newcommand\ceil[1]{{\lceil{#1}\rceil}}


\newcommand{\D}{\mathsf D}
\newcommand{\A}{\mathscr A}

\newcommand{\sP}{\mathscr S}
\newcommand{\Fc}{\mathcal F}
\newcommand{\vp}{\mathsf v}
\newcommand{\h}{\mathsf h}
\newcommand{\bulletprod}[1]{\underset{#1}{\bullet}}
\newcommand{\bulletprodd}[2]{\overset{#2}{\underset{#1}{\bullet}}}
\renewcommand{\t}{\, | \,}




\numberwithin{equation}{section}







\subjclass[2010]{11B30, 11P70}








\begin{document}

\title{On the inverse  problem of the $k$-th Davenport constants for groups of rank  $2$}

\author{Qinghai Zhong}

\address{University of Graz, NAWI Graz,
	Department of Mathematics and Scientific Computing \\
	Heinrichstra{\ss}e 36\\
	8010 Graz, Austria}
\email{qinghai.zhong@uni-graz.at}
\urladdr{https://imsc.uni-graz.at/zhong/}



\thanks{This work was supported by the Austrian Science Fund FWF [grant DOI:10.55776/P36852]}

\keywords{zero-sum sequences, Davenport constant, $k$-th Davenport constant}




\begin{abstract}
 For a finite abelian group $G$ and a positive integer $k$, let $\mathsf{D}_k(G)$ denote the smallest integer $\ell$ such that each sequence over $G$ of length at least $\ell$ has $k$ disjoint nontrivial zero-sum subsequences. It is known that $\mathsf D_k(G)=n_1+kn_2-1$ if $G\cong C_{n_1}\oplus C_{n_2}$ is a rank $2$ group, where $1<n_1\t n_2$. We investigate the associated inverse problem for rank $2$ groups, that is,  characterizing the structure of  zero-sum sequences of length $\mathsf D_k(G)$ that can not be partitioned into $k+1$ nontrivial zero-sum subsequences.
\end{abstract}

\maketitle


\section{Introduction}




Let $(G, +, 0)$ be a finite abelian group. By a sequence $S$
over $G$, we mean a finite sequence of terms from $G$ which is unordered, repetition of terms allowed. We say that $S$ is a zero-sum sequence if the sum of its terms equals zero and denote by $|S|$ the length of the sequence.


Let $k$ be a positive integer. 
We denote by $\mathsf{D}_k(G)$ the smallest integer $\ell$ such that every sequence over $G$
of length at least $\ell$ has $k$ disjoint nontrivial zero-sum subsequences.  We call $\mathsf{D}_k(G)$ the $k$-th Davenport constant of $G$, while the Davenport constant $\mathsf D(G)=\mathsf D_1(G)$ is one of the most important zero-sum invariants in Combinatorial Number Theory and,  together with Erd\H os-Ginzburg-Ziv constant,  $\eta$-constant, etc., has been studied a lot (see \cite{Ho-Zh24,Na20,Ad-Gr-Su12,Gi18, Si20,Olson-rk2,Gi-Sc19a,Fa-Ga-Zh11a,Ga-Ha-Pe-Su14,Li20,Bi-Gr-He20,Qu-Li-Te22,Br-Ri18a,Ed-El-Ge-Ku-Ra07,Ca96a,Ga-Lu08a,Ha-Zh19}). 
This variant $\mathsf D_k(G)$ of the Davenport constant was introduced and investigated by F.~Halter-Koch \cite{halterkoch92}, in the context of investigations on the asymptotic behavior of certain counting functions of algebraic integers defined via factorization properties ( see the monograph \cite[Section 6.1]{Ger-book},  and the survey article \cite[Section 5]{gaogeroldingersurvey}).
In 2014, K. Cziszter and M. Domokos (\cite{Cz-Do14,Cz-Do13c}) introduced the generalized Noether Number $\beta_k(G)$ for general groups, which equals $\mathsf D_k(G)$ when $G$ is abelian (see \cite{Cz-Do-Ge16,Cz-Do-Sz17,Cz-Do14a} for more about this direction).
Knowledge of those constants is highly relevant when applying the inductive method to determine or estimate the Davenport constant of certain finite abelian groups (see \cite{delormeetal01,bhowmikschalge07,bhowmikhallschlage,Pl-Sc11}).



In 2010, M. Freeze and W. Schmid (\cite{Fr-Sc10}) showed that for each finite abelian group $G$ we have $\mathsf{D}_k(G)= \mathsf{D}_0(G)+ k\exp(G)$ for some $\mathsf{D}_0(G)\in \mathbb{N}_0$ and all sufficiently large $k$.
In fact, it is known that for groups of rank at most two, and for some other types of groups, an equality of the form $\mathsf{D}_k(G)= \mathsf{D}_0(G)+ k\exp(G)$ for some $\mathsf{D}_0(G)\in \mathbb{N}_0$ holds for all $k$. In particular, for a rank two abelian group $G=C_m\oplus C_n$, where $m\mid n$, we have $\mathsf D_k(G)=m+kn-1$ (\cite[Theorem 6.1.5]{Ger-book}). 
Yet, it fails for elementary $2$ and $3$-groups of rank at least $3$ (see \cite{delormeetal01,bhowmikschalge07}). In general, computing (even bounding) $\mathsf D_k(G)$ is quite more complicated than for $\mathsf D(G)$, in particular for (elementary) $p$-groups, while $\mathsf D(G)$ is know for $p$-groups.



In  zero-sum theory, the associated inverse problems of zero-sum invariants  study the structure of extremal sequences that do not have enough zero-sum subsequences with the prescribed properties. The inverse problems of the Davenport constant, the $\eta$-constant, and the Erd\H os-Ginzburg-Ziv constant are central topics (see \cite{Yu07a,Yu-Ze11b,Sa-Ch07a,Sa-Ch12a,Ga-Pe-Wa11a,GeF21a,Eb-Gr24a,Gr-Li22a, Gr-Li22b,Sc10b,Sc12a,Gi-Sc19b,Gr-uzi}). 
The associated inverse problem of $\mathsf D_k(G)$ is to characterize the maximal length zero-sum sequences that can not be partitioned into $k+1$ nontrivial zero-sum subsequences. In particular, the inverse problem of $\mathsf D(G)$ is to characterize the structure of minimal zero-sum subsequences of length $\mathsf D(G)$, which was accomplished for groups of rank $2$ in a series of papers \cite{Reiher-propB-thesis} \cite{Gao-Ger-propB} \cite{propB-GGG} \cite{Sc10b} \cite{Schlage-case9-propB},
   where a minimal zero-sum sequence is a zero-sum sequence that can not be partitioned into two nontrivial zero-sum subsequences. Those inverse results can be used to construct minimal  generating subsets in Invariant Theory (see \cite[Proposition 4.7]{Cz-Do-Ge16}).
   
   
    Let $\mathcal B(G)$ be the set of all zero-sum sequences over $G$. We define
   \[
   \mathcal M_k(G)=\{S\in \mathcal B(G)\colon S \text{ can not be partitioned into $k+1$ nontrivial zero-sum subsequences}\}\,.
   \]
   Then it is easy to see that $\mathsf D_k(G)=\max \{|S|\colon S\in \mathcal M_k(G)\}$.
   In this paper, we investigate the inverse problem of general Davenport constant $\mathsf D_k(G)$ for all rank $2$ groups, that is, to study the structure of sequences of $\mathcal M_k(G)$ of length $\mathsf D_k(G)$. In 2003, Gao and Geroldinger (\cite[Theorem 7.1]{Gao-Ger-propB}) studied the inverse problem of $\mathsf D_k(G)$ for $G=C_n\oplus C_n$ under some assumptions of $G$, which had been confirmed later.  We reformulate this result in the following  and  a proof  will be given in Section 3.
   
\begin{theorem}\label{main1}
	Let  $G = C_{n} \oplus C_{n}$  with $n\ge 2$, let $k\ge 1$, and let $U\in \mathcal B(G)$ with $|U|=\mathsf D_k(G)$. Then  $U \in \mathcal M_k(G)$ 
	if and only if there exists a basis $(e_1,e_2)$ of $G$ such that it has one of the following two forms.
	\begin{itemize}
		\smallskip
		\item[(I)] \[
		U = e_1^{k_1n-1}  \prod_{i\in [1,k_2n]} (x_{i}e_1+e_2), \quad \text{where}
		\]
		\begin{itemize}
			\item[(a)] $k_1,k_2\in \N$ with $k_1+k_2=k+1$,
			\item[(b)] $x_1, \ldots, x_{k_2n}  \in [0, n-1]$ and $x_1 + \ldots + x_{k_2n} \equiv 1 \mod n$.
		\end{itemize}
	\end{itemize}
	
	
	
	\begin{itemize}
			\item[(II)] 
		\[
		U=e_1^{an} e_2^{bn-1} (xe_1+e_2)^{cn-1}  (xe_1+2e_2)\,, \quad \text{where}
		\]
		\begin{itemize}
			\item[(a)] $x\in [2,n-2]$ with $\gcd(x,n)=1$,
			\item[(b)] $a,b,c\geq 1$ with $a+b+c=k+1$.
		\end{itemize}
	
	Note that in this case, we have $k\ge 2$.
	\end{itemize}
\end{theorem}
   
 For general groups, we have the following  main theorem. 
   
\begin{theorem}\label{main}
	Let  $G = C_{n_1} \oplus C_{n_2}$  with $1 < n_1 \mid n_2$ and $n_1<n_2$, let $k\ge 1$, and let $U\in \mathcal B(G)$ with $|U|=\mathsf D_k(G)$. Then  $U \in \mathcal M_k(G)$ 
	if and only if it has one of the following four forms.
	\begin{itemize}
		\smallskip
		\item[(I)] \[
		U = e_1^{\ord (e_1)-1}  \prod_{i\in [1,k\ord (e_2)]} (x_{i}e_1+e_2), \quad \text{where}
		\]
		\begin{itemize}
			\item[(a)] $(e_1, e_2)$ is a basis for $G$ with $\ord(e_1)=n_1$ and $\ord(e_2)=n_2$,
			\item[(b)] $x_1, \ldots, x_{k\ord (e_2)}  \in [0, \ord (e_1)-1]$ and $x_1 + \ldots + x_{k\ord (e_2)} \equiv 1 \mod \ord (e_1)$.
		\end{itemize}
	\end{itemize}

	\begin{itemize}
		\smallskip
		\item[(II)] \[
		U = e_1^{k\ord (e_1)-1}  \prod_{i\in [1,\ord (e_2)]} (x_{i}e_1+e_2), \quad \text{where}
		\]
		\begin{itemize}
			\item[(a)] $(e_1, e_2)$ is a basis for $G$ with $\ord(e_1)=n_2$ and $\ord(e_2)=n_1$,
			\item[(b)] $x_1, \ldots, x_{\ord (e_2)}  \in [0, \ord (e_1)-1]$ and $x_1 + \ldots + x_{\ord (e_2)} \equiv 1 \mod \ord (e_1)$.
		\end{itemize}
	\end{itemize}

	
	\begin{itemize}
		\item[(III)] \[
		U = g_1^{n_1 - 1}   \prod_{i\in [1,kn_2]} ( -x_{i} g_1 +g_2) \,, \quad \text{where}
		\]
		\begin{itemize}
			\item[(a)] $(g_1, g_2)$ is a generating set of $G$ with $\ord(g_1)>n_1$ and $\ord (g_2) = n_2$,
			\item[(b)] $x_1, \ldots, x_{kn_2} \in [0, n_1-1]$ with $x_1 + \ldots + x_{kn_2} = n_1-1$.
		\end{itemize}
	\end{itemize}

	
	
	
	\begin{itemize}
		\item[(IV)]
		\[
		U = e_1^{s n_1 - 1} \prod_{i\in [1,kn_2 -(s-1)n_1]} ( (1-x_{i}) e_1+e_2) \,, \quad \text{where}
		\]
		\begin{itemize}
			\item[(a)] $(e_1, e_2)$ is a basis of $G$ with $\ord(e_1)=n_2$ and $\ord (e_2) = n_1$,
			\item[(b)] $s \in [2, kn_2/n_1 -1]$,
			\item[(c)]$x_1, \ldots, x_{kn_2 -(s-1)n_1} \in [0, n_1-1]$ with $x_1 + \ldots + x_{kn_2 -(s-1)n_1} = n_1-1$.
		\end{itemize}
	\end{itemize}

\end{theorem}



\section{Notation and Preliminaries}\label{Sec-Prelim}



Our notations and terminology are consistent with \cite{GE} and \cite{Gr-book}. Let $\mathbb{N}$ denote the set of positive integers  and $\mathbb{N}_0=\mathbb{N}\cup\{0\}$. For real numbers $a, b\in \mathbb{R}$, we set the discrete interval $[a, b]=\{x\in \mathbb{Z}\colon a\leq x\leq b\}$.  Throughout this paper, all abelian groups will be written additively, and for $n\in \mathbb{N}$, we denote by $C_n$ a cyclic group with $n$ elements.

Let $G$ be a finite abelian group. It is well-known that $|G|=1$ or $G\cong C_{n_1}\oplus \ldots \oplus C_{n_r}$ with $1<n_1\mid \ldots \mid n_r\in \mathbb{N}$, where $r=\mathsf{r}(G)\in \mathbb{N}$ is the \emph{rank} of $G$, and $n_r={\exp}(G)$ is the \emph{exponent} of $G$. We denote by $|G|$ the \emph{order} of $G$, and $\ord(g)$ the \emph{order} of  $g\in G$. 

Let $\mathcal F(G)$ be the free abelian (multiplicatively written) monoid with basis $G$. Then sequences over $G$ could be viewed as  elements of $\mathcal F(G)$. A  sequence $S\in \mathcal F(G)$ could be written as 
$$S=g_1\cdot \ldots \cdot g_l=\prod_{g\in G}g^{\mathsf v_{g}(S)}\,,$$
where  $\mathsf v_{g}(S)\in \mathbb{N}_0$ is the multiplicity of $g$ in $S$. 
We call
\begin{itemize}
	\item $\supp(S)=\{g\in G\colon \mathsf v_g(S)>0\}\subset G$ the \emph{support} of $S$, and
	\item $\sigma(S)=\sum_{i=1}^{l}g_i=\sum_{g\in G}\mathsf v_g(S)g\in G$ the \emph{sum} of $S$. 	
\end{itemize}
Let $t\in \N$. We denote  $$\Sigma_{\le t}(S)=\left\{\sum_{i\in I}g_i\colon I\subseteq [1, l] \mbox{ with } 1\leq |I|\leq t\right\}\,.$$ 



A sequence $T\in \mathcal F(G)$ is called a subsequence of $S$ if $\mathsf v_{g}(T)\leq \mathsf v_{g}(S)$ for all $g\in G$, and denoted by $T\mid S$. If $T\mid S$, then we denote
$$T^{-1} S=\prod_{g\in G}g^{\mathsf v_g(S)-\mathsf v_g(T)}\in \mathcal F(G)\,.$$
Let $T_1, T_2\in \mathcal F(G)$.  We set
$$T_1 T_2=\prod_{g\in G}g^{\mathsf v_{g}(T_1)+\mathsf v_{g}(T_2)}\in \mathcal{F}(G)\,.$$
If $T_1,\ldots, T_t\in \mathcal F(G)$ such that $T_1\cdot\ldots\cdot T_t\t S$, where $t\ge 2$, then 
 we say $T_1,\ldots,T_t$ are disjoint subsequences of $S$.





Every map of abelian groups $\phi: G\rightarrow H$ extends to a map from the sequences over $G$ to the sequences over $H$ by setting $\phi(S)=\phi(g_1)\cdot \ldots \cdot \phi(g_l)$. If $\phi$ is a homomorphism, then $\phi(S)$ is a zero-sum sequence if and only if $\sigma(S)\in \mathsf{ker}(\phi)$.


We denote by $\mathsf E(G)$ the Gao's constant which is the smallest integer $\ell$ such that every sequence over $G$ of length $\ell$ has a zero-sum subsequence of length $|G|$ and by $\eta(G)$ the smallest integer $\ell$ such that every sequence over $G$ of length $\ell$ has a zero-sum subsequence $T$ of length $1\le |T|\le \exp(G)$. 
Let $\mathsf d(G)$ be the maximal length of a sequence over $G$ that has no zero-sum subsequence. Then it is easy to see that $\mathsf d(G)=\mathsf D(G)-1$.
The following result is well-known and we may use it without further mention.

\begin{lemma}\label{le-E}
	Let $G$ be a finite abelian group. Then $\mathsf E(G)=|G|+\mathsf d(G)\le 2|G|-1$.
 	\end{lemma}
\begin{proof}\textcolor{red}{TOPROVE 0}\end{proof}

We also need the following lemmas.
\begin{lemma}\label{le-cyc}
	Let $G$ be a finite abelian group. If $\mathsf D(G)=|G|$, then $G$ is cyclic and for every minimal zero-sum sequence $S$ over $G$ of length $|G|$,  there exists $g\in G$ with $\ord(g)=|G|$ such that $S=g^{|G|}$.
\end{lemma}
\begin{proof}\textcolor{red}{TOPROVE 1}\end{proof}

\begin{lemma}\label{le-subgroup}
	Let $G$ be a finite abelian group and let $H\subset G$ be a proper subgroup. Then $\mathsf D_k(H)<\mathsf D_k(G)$ for all $k\in \N$.
\end{lemma}
\begin{proof}\textcolor{red}{TOPROVE 2}\end{proof}





\begin{theorem}\label{th-gen}
Let  $G = C_{n_1} \oplus C_{n_2}$  with $n_1 \mid n_2$, where $n_1,n_2\in \N$, and let $k\in \N$. Then $\eta(G)=2n_1+n_2-2$ and $\mathsf D_k(G)=n_1+kn_2-1$. In particular, $\mathsf D(G)=n_1+n_2-1$.
\end{theorem}
\begin{proof}\textcolor{red}{TOPROVE 3}\end{proof}

 


\begin{theorem}  \label{inverse}
	Let  $G = C_{n} \oplus C_{mn}$  with $n\geq 2$  and $m \ge 1$.  A sequence $S$
	over $G$ of length $\mathsf D (G) = n+mn-1$ is a minimal zero-sum
	sequence if and only if it has one of the following two forms{\rm
		\,:}
	\begin{itemize}
		\medskip
		\item[(I)] \[
		S = e_1^{\ord (e_1)-1} \prod_{i=1}^{\ord (e_2)}
		(x_{i}e_1+e_2),
		\]where
		\begin{itemize}\item[(a)] $\{e_1, e_2\}$ is a basis of $G$,
			\item[(b)] $x_1, \ldots, x_{\ord (e_2)}  \in
			[0, \ord (e_1)-1]$ and $x_1 + \ldots + x_{\ord (e_2)} \equiv 1
			\mod \ord (e_1)$. \end{itemize} In this case, we say that $S$ is of type I(a) or I(b) according to whether $\ord(e_2)=n$ or $\ord(e_2)=mn>n$.
		
		\medskip
		\item[(II)] \[
		S = f_1^{sn - 1} f_2^{(m-s)n+\epsilon}\prod_{i=1}^{n-\epsilon} ( -x_{i} f_1 +
		f_2),
		\] where
		\begin{itemize}
			\item[(a)] $\{f_1, f_2\}$ is a generating set for  $G$ with $\ord (f_2) =
			mn$ and $\ord(f_1)>n$,
			\item[(b)] $\epsilon\in [1,n-1]$  and
			$s \in [1, m-1]$,
			\item[(c)] $x_1, \ldots, x_{n-\epsilon} \in [1, n-1]$ with $x_1 + \ldots + x_{n-\epsilon} = n-1$,  \item[(d)] either  $s=1$ or
			$nf_1 = nf_2$, with both holding when $m=2$, and
			\item[(e)] either $\epsilon\geq 2$  or $nf_1\neq nf_2$.\end{itemize} In this case, we say that $S$ is of type II.
	\end{itemize}
\end{theorem}

\begin{proof}\textcolor{red}{TOPROVE 4}\end{proof}






\begin{lemma}\label{le-epi}
	Let $G$ be a finite abelian group, let $H$ be a cyclic subgroup of $G$, and let $\varphi\colon G\rightarrow G/H$ be the canonical epimorphism.
	If $S\in \mathcal M_k(G)$, then $\varphi(S)\in \mathcal M_{k|H|}(G/H)$.
\end{lemma}

 \begin{proof}\textcolor{red}{TOPROVE 5}\end{proof}


 




\section{Proof of main theorems}\label{sec-k=n-1}



\begin{proposition}\label{pr-key}
	Let $G$ be a finite abelian group of rank at most $2$, let $k\in \N$, and let $S$ be a zero-sum sequence over $G$ of length $\mathsf D_k(G)$. Then $S\in \mathcal M_k(G)$ if and only if $0\not\in \Sigma_{\le \exp(G)-1}(S)$.
\end{proposition}
\begin{proof}\textcolor{red}{TOPROVE 6}\end{proof}

We first investigate the associated inverse problem  for cyclic groups.

\begin{theorem}\label{le-cyclic}
	Let $G$ be cyclic, let $k\in \N$, and let $S$ be a zero-sum sequence over $G$ of length $\mathsf D_k(G)$. Then $S\in \mathcal M_k(G)$ if and only if there exists $g\in G$ with $\ord(g)=|G|$ such that $S=g^{k|G|}$.
\end{theorem}
\begin{proof}\textcolor{red}{TOPROVE 7}\end{proof}


Next, we prove Theorem \ref{main1}
 which could be handled easily by  Proposition \ref{pr-key} and \cite[Theorem 7.1]{Gao-Ger-propB}.
\begin{proof}\textcolor{red}{TOPROVE 8}\end{proof} 





\begin{lemma}\label{le-key}  Let $G=C_n\oplus C_n$ with $n\ge 2$ and let $k\ge 2$. If  $S\in\mathcal F(G)$  is a zero-sum sequence with $|S|=(k+1)n-1$ and $0\notin \Sigma_{\leq n-1}(S)$, then  there is a basis $(e_1,e_2)$ for $G$ such that either
	\begin{itemize}
		\item[1.] $\supp(S)\subseteq \{e_1\}\cup \big(\la e_1\ra+e_2\big)$ and $\vp_{e_1}(S)\equiv -1\mod n$, or
		\item[2.] $S=e_1^{an} e_2^{bn-1} (xe_1+e_2)^{cn-1}  (xe_1+2e_2)$ for some $x\in [2,n-2]$ with $\gcd(x,n)=1$, and some  $a,b,c\geq 1$ with $k+1=a+b+c$.
	\end{itemize}
\end{lemma}


\begin{proof}\textcolor{red}{TOPROVE 9}\end{proof}




The following lemma shows two special cases of Theorem \ref{main}.

\begin{lemma}\label{sch-I}
	Let  $G = C_{n_1} \oplus C_{n_2}$  with $1 < n_1 \mid n_2$ and $n_1<n_2$, let $k\ge 2$, and let  $U \in \mathcal M_k(G)$ with  $|U|=\mathsf D_k (G)$. 
	\begin{enumerate}
		\item[1.] If there is some $e_1\in \supp(U)$ such that $\ord(e_1)=n_1$ and $\vp_{e_1}(U)\geq n_1-1$, then 
		there exists $e_2\in G$ with  $\ord(e_2)=n_2$ such that  $(e_1,e_2)$ is a basis of $G$ and
		\[U = e_1^{n_1-1}  \prod_{i\in [1,kn_2]} (x_{i}e_1+e_2)\,,
		\]
		where $x_1, \ldots, x_{kn_2}  \in [0, n_1-1]$ and $x_1 + \ldots + x_{kn_2} \equiv 1 \mod n_1$.
		
		\item[2.] If there is some $e_2\in \supp(U)$ such that  $\ord(e_2)=n_2$ and $\vp_{e_2}(U)\geq kn_2-1$, then 
		there exists $e_1\in G$ with $\ord(e_1)=n_1$  such that $(e_1,e_2)$ is a basis of $G$ and
		\[U = e_2^{kn_2-1}  \prod_{i\in [1,n_1]} (e_1+x_{i}e_2)\,,
		\]
		where $x_1, \ldots, x_{n_1}  \in [0, n_2-1]$ and $x_1 + \ldots + x_{n_1} \equiv 1 \mod n_2$.
	\end{enumerate}
\end{lemma}

\begin{proof}\textcolor{red}{TOPROVE 10}\end{proof}



\bigskip
Now we are ready to prove our main Theorem \ref{main}.

\begin{proof}\textcolor{red}{TOPROVE 11}\end{proof}


\begin{thebibliography}{99}
\bibitem{Ad-Gr-Su12}	
Sukumar Das Adhikari,  David J. Grynkiewicz, and Zhi-Wei Sun,\emph{On weighted zero-sum sequences}, Advances in Applied Mathematics \textbf{48}(2012), 506-527.	
	



\bibitem{Schlage-case9-propB} G. Bhowmik, I. Halupczok, and J.
Schlage-Puchta,
\emph{The structure of maximal zero-sum free sequences},
Acta Arith. \textbf{143} (2010), 21-50.


\bibitem{bhowmikhallschlage}
G.~Bhowmik, I.~Halopczok, and J.~Schlage-Puchta,
\emph{Inductive methods and zero-sum free sequences},
 Integers, \textbf{9}(2009), 515-536.


\bibitem{bhowmikschalge07}
G.~Bhowmik and J.~Schlage-Puchta,
\emph{Davenport's constant for groups of the form {$\mathbb Z\sb 3\oplus\mathbb
	Z\sb 3\oplus\mathbb Z\sb {3d}$}},
 In {\em Additive combinatorics}, volume~43 of {\em CRM Proc. Lecture
	Notes}, pages 307--326. Amer. Math. Soc., Providence, RI, 2007.

\bibitem{Bi-Gr-He20}
J.~Bitz, S.~Griffith, and X. He, \emph{Exponential lower bounds on the generalized Erdős–Ginzburg–Ziv constant}, Discrete Mathematics \textbf{343}(2020), 112083.

\bibitem{Br-Ri18a}
F.E.~Brochero Mart\'\i nez and S.~Ribas, \emph{{E}xtremal product-one free sequences in {D}ihedral and {D}icyclic {G}roups}, Discrete Math. \textbf{341} (2018), 570 -- 578.



\bibitem{Ca96a}
Y.~Caro, \emph{Remarks on a zero-sum theorem}, J. Comb. Theory Ser. A \textbf{76} (1996), 315 -- 322.


\bibitem{Cz-Do13c}
K.~Cziszter and M.~Domokos, \emph{On the generalized {D}avenport constant and the {N}oether number}, Central European J. Math. \textbf{11} (2013), 1605 -- 1615.


\bibitem{Cz-Do14}
K.~Cziszter and M.~Domokos,, \emph{Groups with large {N}oether bound}, Ann. Inst. Fourier (Grenoble) \textbf{64} (2014), 909 -- 944.


\bibitem{Cz-Do14a}
K.~Cziszter and M.~Domokos,, \emph{The {N}oether number for the groups with a cyclic subgroup of index two}, J. Algebra \textbf{399} (2014), 546 -- 560.


\bibitem{Cz-Do-Ge16}
K.~Cziszter, M.~Domokos, and A.~Geroldinger, \emph{The interplay of invariant theory with multiplicative ideal theory and with arithmetic combinatorics}, Multiplicative {I}deal {T}heory and {F}actorization {T}heory, Springer, 2016, pp.~43 -- 95.


\bibitem{Cz-Do-Sz17}
K.~Cziszter, M.~Domokos, and I.~Sz{\"{o}}ll{\H{o}}si, \emph{The {N}oether numbers and the {D}avenport constants of the groups of order less than $32$}, J. Algebra \textbf{510} (2018), 513 -- 541.



\bibitem{delormeetal01}
Ch.~Delorme, O.~Ordaz, and D.~Quiroz,
\emph{Some remarks on {D}avenport constant},
 Discrete Math. \textbf{237}(2001), 119-128.

\bibitem{Ed-El-Ge-Ku-Ra07}
Y.~Edel, C.~Elsholtz, A.~Geroldinger, S.~Kubertin, and L.~Rackham, \emph{Zero-sum problems in finite abelian groups and affine caps}, Quart. J. Math. \textbf{58} (2007), 159 -- 186.


\bibitem{Eb-Gr24a}
J.~Ebert and D.J. Grynkiewicz, \emph{Structure of a sequence with prescribed
	zero-sum subsequences: rank two $p$-groups}, European J. Combin. \textbf{118}
(2024), Paper No. 103888, 13pp.





\bibitem{Fa-Ga-Zh11a}
Y.~Fan, W.~Gao, and Q.~Zhong, \emph{On the {E}rd{\H{o}}s-{G}inzburg-{Z}iv constant of finite abelian groups of high rank}, J. Number Theory \textbf{131} (2011), 1864 -- 1874.


\bibitem{Fr-Sc10} M.~Freeze and W.~ Schmid, \emph{Remarks on a generalization of the Davenport constant}, Discrete mathematics \textbf{310}(2010), 3373-89.





\bibitem{Gao-Ger-propB} W. Gao and A. Geroldinger, \emph{On zero-sum sequences in $\Z/n\Z\oplus \Z/n\Z$}, Integers \textbf{3} (2003), Paper A08, 45p.

\bibitem{gaogeroldingersurvey}
W.~Gao and A.~Geroldinger,
\emph{Zero-sum problems in finite abelian groups: a survey},
Expo. Math. \textbf{24}(2006), 337-369.


\bibitem{propB-GGG}  Weidong Gao, A. Geroldinger, D. J.  Grynkiewicz,  \emph{Inverse zero-sum problems III}, Acta Arith. \textbf{141}(2010), 103–152.


\bibitem{Ga-Ha-Pe-Su14}
W.~Gao, D.~Han, J.~Peng, and F.~Sun, \emph{On zero-sum subsequences of length kexp (G)}, J. Combin. Theory Ser. A \textbf{125}(2014), 240-53.



\bibitem{Ga-Lu08a}
W.~Gao and Z.~Lu, \emph{The {E}rd{\H{o}}s-{G}inzburg-{Z}iv theorem for dihedral groups}, J. Pure Appl. Algebra \textbf{212} (2008), 311 -- 319.





\bibitem{Ga-Pe-Wa11a}
W.~Gao, J.~Peng, and G.~Wang, \emph{Behaving sequences}, J. Comb. Theory, Ser.
A \textbf{118} (2011), 613 -- 622.

\bibitem{GeF21a}
F.~Ge, \emph{Solution to the index conjecture in zero-sum theory}, J. Combin.
Theory Ser. A \textbf{180} (2021), 105410.


\bibitem{GE}
A. Geroldinger, \emph{Additive group theory and non-unique factorizations},
Combinatorial {N}umber {T}heory and {A}dditive {G}roup {T}heory
(A. Geroldinger and I. Ruzsa, eds.), Advanced Courses in Mathematics CRM
Barcelona, Birkh{\"a}user, 2009, pp. 1--86.

\bibitem{Ge-Gr-Yu15}
A.~Geroldinger, D.J. Grynkiewicz, and P.~Yuan, \emph{On products of $k$ atoms
	{II}}, Mosc. J. Comb. Number Theory \textbf{5} (2015), 73 -- 129.


\bibitem{Ger-book} A. Geroldinger and F. Halter-Koch, \emph{Non-Unique Factorizations}, Algebraic, Combinatorial and Analytic Theory, Pure
and Applied Mathematics, vol. 278, 700p, Chapman \& Hall/CRC, 2006.

\bibitem{girardNEW}
B.~Girard.
\emph{On the existence of distinct lengths zero-sum subsequences},
 Rocky Mountain J. Math.  \textbf{42}(2012), 583-596.

\bibitem{Gi18}
B.~Girard,  \emph{An asymptotically tight bound for the Davenport constant}, Journal de l’École polytechnique—Mathématiques \textbf{5} (2018), 605-611.

\bibitem{Gi-Sc19a}
B.~Girard and W.A.~Schmid, \emph{Direct zero-sum problems for certain groups of rank three}, J. Number Theory \textbf{197} (2019), 297 -- 316.


\bibitem{Gi-Sc19b}
B.~Girard and W.A.~Schmid, \emph{Inverse zero-sum problems for certain groups of rank three}, Acta Math. Hungar. \textbf{160}(2020), 229-247.


\bibitem{Gr-book} D. J. Grynkiewicz, \emph{Structural Additive Theory},
Developments in Mathematics, 30, Springer, 2013, 426 pp.

\bibitem{Gr21} D. J. Grynkiewicz, \emph{Inverse Zero-Sum Problems III: Addendum},
 \href{arxiv}{https://arxiv.org/abs/2107.10619}.



\bibitem{Gr-Li22a}
D.J. Grynkiewicz and C.~Liu, \emph{A multiplicative property for zero-sums
	{I}}, Discrete Math. \textbf{345} (2022), Paper No. 112974, 21pp.

\bibitem{Gr-Li22b}
D.J. Grynkiewicz and C.~Liu, \emph{A multiplicative property for zero-sums {II}}, Electron. J.
Combin. \textbf{29} (2022), {Paper No. 3.12, 16pp.}


\bibitem{Gr-uzi}  D. J. Grynkiewicz and U. Vishne,  \emph{The index of small length sequences}, Internat. J. Algebra Comput. \textbf{30}(2020), 977–1014.




\bibitem{halterkoch92}
F.~Halter-Koch,
\emph{A generalization of {D}avenport's constant and its arithmetical
applications},
Colloq. Math. \textbf{63}(1992), 203-210.



\bibitem{Ha-Zh19}
Dongchun Han and Hanbin Zhang, \emph{Erd{\H{o}}s-{G}inzburg-{Z}iv {T}heorem and {N}oether number for ${C}_m \ltimes_{\varphi} {C}_{mn}$}, J. Number Theory \textbf{198} (2019), 159 -- 175.

\bibitem{Ho-Zh24}
S.~Hong and K.~Zhao, \emph{On the zero-sum subsequences of modular restricted lengths}, Discrete Mathematics \textbf{347}(2024), 113967.

\bibitem{Na20}
E. Naslund, \emph{Exponential bounds for the Erdős-Ginzburg-Ziv constant}, J.  Combin. Theory Ser. A \textbf{174} (2020), 105185.

\bibitem{Li20}
C.~Liu,  \emph{On the lower bounds of Davenport constant}, J. Combin. Theory Ser. A  \textbf{171}(2020), 105162.

\bibitem{Pl-Sc11}
A.~Plagne and W.~Schmid. \emph{An application of coding theory to estimating Davenport constants}, Designs, Codes and Cryptography \textbf{61}(2011), 105-118.

\bibitem{Qu-Li-Te22}
Y.~Qu, Y.~Li, and D. Teeuwsen, \emph{On a conjecture of the small Davenport constant for finite groups}, J.  Combin. Theory Ser. A \textbf{189}(2022), 105617.

\bibitem{Reiher-propB-thesis} C. Reiher, \emph{A proof of the theorem according to which every prime number possesses property B}, PhD Thesis, Rostock,
2010.


\bibitem{Sa-Ch07a}
S.~Savchev and F.~Chen, \emph{Long zero-free sequences in finite cyclic
	groups}, Discrete Math. \textbf{307} (2007), 2671 -- 2679.

\bibitem{Sa-Ch12a}
S.~Savchev and F.~Chen, \emph{Long minimal zero-sum sequences in the group ${C}_2 \oplus
	{C}_{2k}$}, Integers \textbf{12} (2012), Paper A51, 18pp.



\bibitem{Sc10b}
W.A. Schmid, \emph{Inverse zero-sum problems {II}}, Acta Arith. \textbf{143}
(2010), 333 -- 343.

\bibitem{Sc12a}
W.A. Schmid, \emph{Restricted inverse zero-sum problems in groups of rank two}, Q.
J. Math. \textbf{63} (2012), 477 -- 487.

\bibitem{Si20}
A.~Sidorenko,  \emph{On generalized Erdős–Ginzburg–Ziv constants for Z2d}, J. Combin. Theory Ser. A \textbf{174}(2020), 105254.

\bibitem{Olson-rk2} J. E. Olson,
\emph{A combinatorial problem on finite Abelian groups II},
J. Number Theory \textbf{1}(1969), 195-199.


\bibitem{Yu07a}
P.~Yuan, \emph{On the index of minimal zero-sum sequences over finite cyclic
	groups}, J. Comb. Theory, Ser. A \textbf{114} (2007), 1545 -- 1551.

\bibitem{Yu-Ze11b}
P.~Yuan and X.~Zeng, \emph{Indexes of long zero-sum free sequences over cyclic
	groups}, European J. Combin. \textbf{32} (2011), 1213 -- 1221.

\end{thebibliography}


\end{document} 