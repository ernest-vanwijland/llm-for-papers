\documentclass[a4paper,11pt]{article}
\usepackage[ngerman, english]{babel}
\usepackage[T1]{fontenc}
\usepackage[utf8]{inputenc}
\usepackage{amsmath}
\usepackage{amssymb}
\usepackage{amsthm}
\usepackage{graphicx}
\usepackage{here}
\usepackage{color}
\usepackage{xcolor}
\usepackage{enumerate}
\usepackage{lmodern}
\usepackage{fancyvrb}
\usepackage[plainpages=false]{hyperref}
\usepackage{caption}
\usepackage{subfigure}
\usepackage{epstopdf}
\captionsetup{format=hang, labelfont=bf, textfont=small, justification=centering}

\oddsidemargin=0.in
\topmargin=-1.5cm
\textheight=23cm
\textwidth=16cm

\renewcommand{\theequation}{\thesubsection.\arabic{equation}}

\renewcommand{\baselinestretch}{1.3}
\newcommand{\numberset}{\mathbb}
\newcommand{\N}{\numberset{N}}
\newcommand{\R}{\numberset{R}}
\newcommand{\Z}{\numberset{Z}}
\newcommand{\ca}{\mathcal}


\newtheorem{defi}{Definition}[section]
\newtheorem{theo}[defi]{Theorem}
\newtheorem{prop}[defi]{Proposition}
\newtheorem{lem}[defi]{Lemma}
\newtheorem{con}[defi]{Conjecture}
\newtheorem{obs}[defi]{Observation}
\newtheorem{cor}[defi]{Corollary}
\newtheorem{prob}[defi]{Problem}



\newcounter{claimcount}
\setcounter{claimcount}{0}
\newenvironment{claim}{\refstepcounter{claimcount}\textbf{Claim \arabic{claimcount}.}}{}

\usepackage{etoolbox}
\AtBeginEnvironment{proof}{\setcounter{claimcount}{0}}



\theoremstyle{remark}

\newcommand{\ENDproof}{\hfill $\blacksquare$\medskip\par}
\newcommand{\orange}[1]{\textcolor{orange}{#1}}
\newcommand{\dm}[1]{\textcolor{orange}{\footnotesize \textbf{Davide:} $\ll$#1$\gg$}}

\newcommand{\red}[1]{\textcolor{red}{#1}}
\newcommand{\iw}[1]{\textcolor{red}{\footnotesize \textbf{Isaak:} $\ll$#1$\gg$}}

\newcommand{\blue}[1]{\textcolor{blue}{#1}}
\newcommand{\yl}[1]{\textcolor{blue}{\footnotesize \textbf{Yulai:} $\ll$#1$\gg$}}

\title{Sets of $r$-graphs that color all $r$-graphs}

\author{Yulai Ma$^1$\thanks{Supported by Sino-German (CSC-DAAD) Postdoc Scholarship Program 2021 (57575640)}, Davide Mattiolo$^2$\thanks{Supported by a Postdoctoral Fellowship of the Research Foundation Flanders (FWO), project number 1268323N}, Eckhard Steffen$^1$, Isaak H.~Wolf$^1$ \thanks{Funded by Deutsche Forschungsgemeinschaft (DFG) - 445863039} \\
\footnotesize
		$^1$ Department of Mathematics, Paderborn University, Warburger Str.\ 100, 33098 Paderborn,
		Germany.
	\\
	\footnotesize
	$^2$ Department of Computer Science, KU Leuven Kulak, 8500 Kortrijk, Belgium.
\\ \footnotesize yulai.ma@upb.de, davide.mattiolo@kuleuven.be, es@upb.de, isaak.wolf@upb.de}
	


\date{}

\begin{document}

\maketitle

\begin{abstract}
An $r$-regular graph is an $r$-graph, if every odd set of vertices is connected to its complement by at least $r$ edges. Let $G$ and $H$ be $r$-graphs. An \emph{$H$-coloring} of $G$ is a mapping 
$f\colon E(G) \to E(H)$ such that each $r$ adjacent edges of $G$ are mapped to $r$ adjacent edges of $H$. For every $r\geq 3$, let $\ca H_r$ be an inclusion-wise minimal set of connected $r$-graphs, such that for every connected $r$-graph $G$ there is an $H \in \ca H_r$ which colors $G$.

We show that $\ca H_r$ is unique and characterize $\ca H_r$ by showing that 
$G \in \ca H_r$ if and only if the only connected $r$-graph coloring $G$ is $G$ itself. 

The Petersen Coloring Conjecture states that the Petersen graph $P$ colors every bridgeless cubic graph. 
We show  that if true, this is a very exclusive situation. 
Indeed, either $\ca H_3 = \{P\}$ or $\ca H_3$ is an infinite set and if
$r \geq 4$, then $\ca H_r$ is an infinite set. Similar results hold for the restriction on simple $r$-graphs.    

By definition, $r$-graphs of class $1$ (i.e.\ those having edge-chromatic number equal to $r$) can be colored with any $r$-graph.
Hence, our study will focus on those $r$-graphs whose edge-chromatic number is bigger than $r$, also called $r$-graphs of class $2$. We determine the set of smallest $r$-graphs of class 2 and show that it is a subset of $\ca H_r$.



\end{abstract}

{\bf Keywords:} perfect matchings, regular graphs, factors, $r$-graphs, edge-coloring, class 2 graphs, Petersen Coloring Conjecture, Berge-Fulkerson Conjecture.


\section{Introduction}
All graphs considered in this paper are finite and may have parallel edges 
but no loops.
The vertex set of a graph $G$ is denoted by $V(G)$ and its edge set by $E(G)$. 
A graph is \emph{$r$-regular} if every vertex has degree $r$. An
$r$-regular graph is an \emph{$r$-graph}, if $|\partial_G(X)| \geq r$ for every $X \subseteq V(G)$ of odd cardinality, where $\partial_G(X)$ denotes the set of edges 
that have precisely one vertex in $X$. 

Let $G$ be a graph and $S$ be a set. An \emph{edge-coloring} of $G$ is a mapping $f\colon E(G)\to S$. It is a 
\emph{$k$-edge-coloring} if $|S|=k$, and it is \emph{proper} 
if $f(e) \not = f(e')$ for any two adjacent edges $e$ and $e'$. 
The smallest integer $k$ for which $G$ admits a proper $k$-edge-coloring 
is the \emph{edge-chromatic number} of $G$, which is denoted by $\chi'(G)$. A \emph{matching} is a set $M\subseteq E(G)$ such that no two edges of $M$ are adjacent. Moreover, $M$ is said to be \emph{perfect} if every vertex of $G$ is incident with an edge of $M$.


If $\chi'(G)$ equals the maximum degree of $G$, then $G$ is said to be \emph{class $1$}; otherwise $G$ is \emph{class $2$}. 
If $\chi'(G)=r$, then $r$ is
the minimum number such that $E(G)$ decomposes into $r$ matchings,
which are perfect matchings in case of $r$-regular graphs. For $r \geq 1$,
let $\ca T_r$ be the set of the smallest $r$-graphs of class 2. 
For example, the only element of $\ca T_3$ is the Petersen graph, which is
denoted by $P$ throughout this paper. 

The generalized Berge-Fulkerson Conjecture \cite{seymour1979multi} states that every $r$-graph
has $2r$ perfect matchings such that every edge is in precisely two of them. For $r=3$ the conjecture  was attributed to Berge and Fulkerson \cite{fulkerson1971blocking},
who put it into print (cf.~\cite{seymour1979multi}).
As a unifying approach to study some hard conjectures 
on cubic graphs, Jaeger \cite{jaeger1988nowhere} introduced colorings with
edges of another graph. To be precise, let $G$ and $H$ be graphs. An \emph{$H$-coloring} of $G$ is a mapping $f\colon E(G) \to E(H)$ such that
\begin{itemize}
	\item if $e_1,e_2 \in E(G)$ are adjacent, then $f(e_1) \neq f(e_2)$,
	\item for every $v \in V(G)$ there exists a vertex $u \in V(H)$ with $f(\partial_G(v))=\partial_H(u)$.
\end{itemize}

If such a mapping exists, then we write $H \prec G$ and say $H$ \emph{colors} $G$.
A set $\ca A$ of connected $r$-graphs such that for every connected $r$-graph $G$ 
there is an element $H \in \ca A$ which colors $G$ is said to be $r$-\emph{complete}.
For every $r\geq 3$, let $\ca H_r$ be an inclusion-wise minimal $r$-complete set.  

For $r=3$, Jaeger \cite{jaeger1988nowhere} conjectured that the Petersen graph 
colors every bridgeless cubic graph. If true, this conjecture would have far reaching 
consequences. For instance, it would imply that the Berge-Fulkerson Conjecture
and the 5-Cycle Double Cover Conjecture (see \cite{C.-Q._Zhang_book}) are 
also true.   
The Petersen Coloring Conjecture is a starting point for research in several directions. 
Different aspects of it are studied and partial results are proved, see for instance \cite{DeVos_etal_2007, Haglund_Steffen_2014, Jaeger_5_edge_coloring, Jin_partial_2021, Giuseppe_Vahan_2020, Riste_etal_2020, Robert_2017}. 

Analogously to the case $r=3$, if all elements of $\ca H_r$ would satisfy the generalized
Berge-Fulkerson Conjecture, then every $r$-graph would satisfy it.
Mazzuoccolo et al.~\cite{MTZ_r_graphs} asked 
whether there exists a connected $r$-graph $H$ such that $H \prec G$ for every (simple)
$r$-graph $G$, for all $r \geq 3$. 
We show that $\ca H_r$ is unique and that it is an infinite set when $r \geq 4$. Furthermore,
if $r=3$, then either $\ca H_3 = \{P\}$ (if the Petersen Coloring Conjecture is true) 
or $\ca H_3$ is an infinite set. More precisely, in Section \ref{Sec: H-coloring} we characterize $\ca H_r$ and provide constructions for infinite subsets of $\ca H_r$. Similar results are proved 
for simple $r$-graphs.  


By definition, any $r$-graph $ G $ of class $1$ can be colored with any $r$-graph $ H $. Indeed, let $ M_1,\ldots, M_r$ be $ r $ pairwise disjoint perfect matchings of $ G $ and $ v $ a vertex of $ H $ with $ \partial_{H}(v)=\{e_1,\ldots,e_r\}$.
Every edge of $ M_i $ of  $ G $ can be mapped to $ e_i $ in $ H $.
Hence, the aforementioned questions and conjectures reduce to $r$-graphs of class $2$. In Section~\ref{Sec: smallest r-graphs} we determine the
set $\ca T_r$ of the smallest $r$-graphs of class 2
and prove that $|\ca T_r| \geq p'(r-3,6)$, where $p'(r-3,6)$ is the number 
of partitions of $r-3$ into at most $6$ parts. Furthermore, we show that 
if $r \geq 4$, then $\ca T_r$ is a proper subset of $\ca H_r$.   
 
The Petersen Coloring Conjecture has also been studied in the context of
quasi-orders on the set of graphs, see \cite{DeVos_etal_2007, Robert_2017}. In Section \ref{Sec: final remarks} we briefly put our 
results in this context. We conclude the paper with some open questions. 
 

\subsection{Definitions and basic results}

Let $ G $ be a graph.
For any subset $ X $ of $ V(G) $, we  use $ G-X $ to denote the graph obtained from $ G $ by deleting all vertices of $ X $ and all incident edges. Similarly, for $ F \subseteq E(G) $, denote by $ G-F $ the graph obtained by deleting all edges of $F$ from $ G $.  In particular, we simply write 
$ G-x $ and $ G-e $ for $ G-X $ and $ G-F $, respectively, when $ X=\{x\} $ and $ F=\{e\} $.  The subgraph of $ G $ induced by the vertex set $ X $ is denoted by $ G[X] $. Moreover, the graph obtained from $ G $ by identifying all vertices of $ X $ and deleting all resulting loops is denoted  by $ G/X $;  we denote the new vertex by $w_X$.
Let $Y$ be a subset of $ V(G) $ with $ X\cap Y=\emptyset $. We use $ [X,Y]_G $ to denote the set of all edges of $ G $ with one vertex in $ X $ and the other one  in $Y$. Furthermore, if $ Y=X^c=V(G)\setminus X $ and $ [X,Y]_G $ is nonempty, 
then we call it an \emph{edge-cut} of $ G $ and denote it by $\partial_G(X)$. If  $ X $ or $ Y $ consists of one vertex, we skip the set-brackets notation.
In addition,  $|\partial_{G}(x)| $ is called the \emph{degree} of $ x \in V(G)$ and it is denoted by $d_G(x)$.
If $G$ is an $r$-graph, then $\partial_G(X)$
is \emph{tight} if $|X|$ is odd and $|\partial_G(X)|=r$. A tight edge-cut is \emph{trivial} if   $X$ or $X^c$ consists of a single vertex. Moreover, for $v \in V(G)$ we denote by $N_G(v)$ the set of neighbors of $v$.

A \emph{$ 1 $-factor}  of a graph $G $ is a spanning $ 1 $-regular subgraph of  $ G$, and its edge set is a perfect matching.
A connected $ 2 $-regular graph is called a  \emph{circuit}. A circuit of length $k$ is
called a \emph{$ k $-circuit} and it is denoted by $C_k$. 

For two graphs $ G $ and $ H $, if there are two bijections $ \theta: V(G) \to V(H)$ and $ \phi:E(G)\to E(H)  $ such that  $ e=uv\in E(G) $ if and only if $ \phi (e)=\theta(u)\theta(v)\in E(H)  $, then we say that $ G $ and $ H $ are {\em isomorphic}, denoted by $ G\cong H $, and call the pair of mappings $ (\theta,\phi) $ an  {\em isomorphism} between $ G $ and $ H $. In particular, an {\em automorphism} of a graph is an isomorphism of the graph to itself. 


Let   $ H_1, \ldots, H_t $ be a sequence of graphs such that $V( H_i)\subseteq V(H_1) $ for each $ i\in\{2,\ldots,t\} $.  Denote by $H_1+E(H_2)+\ldots+E(H_t)$ the graph obtained from $H_1$ by adding a copy of every edge of $H_i$ for every $i\in\{2,\dots,t\}$. 
Let $\ca M$ be a finite multiset of perfect matchings of the Petersen graph $P$. 
The graph $P+\sum_{M\in\ca M}M$ is denoted by $P^{\ca M}$.

\begin{lem} [\cite{Grunewald_Steffen_1999}] \label{lem: P^M class 2}
For every finite multiset $\ca M$ of perfect matchings of the Petersen graph $P$,
the graph $P^{\ca M}$ is class 2.
\end{lem}

The following observation  will  frequently   be used without reference.

\begin{obs}\label{Observation-same-parity}
Let $r\geq 3$, let $G$ be an $r$-graph and let $X \subseteq V(G)$. If $\vert X \vert$ is even, then $\vert \partial_G(X) \vert$ is even. If $\vert X \vert$ is odd, then $\vert \partial_G(X) \vert$ has the same parity as $r$.
\end{obs}

One major fact that we use in this paper is that every $r$-graph can be
decomposed into a $k$-graph which is class 1 and an $(r-k)$-regular graph, for a suitable $k \in \{1, \dots ,r\}$. 
For every $r$-graph $G$ let $\pi(G)$ be the largest integer $t$ such that $G$ has $t$ pairwise disjoint perfect matchings.
Let $r\ge3$ and $k\in \{1,\dots, r\}$ be integers. Let 
$\ca G(r,k)=\{G\colon G$ is an $r$-graph with $\pi(G)=k \}$. 
Note that $\ca G(r,r-1) = \emptyset$, since
every $r$-graph with $r-1$ pairwise disjoint perfect matchings is a class 1 graph
and thus, it has $r$ pairwise disjoint perfect matchings. If $ k \leq r-2$, then the elements
of $\ca G(r,k)$ are class 2 graphs and $\ca G(r,i) \cap \ca G(r,j) = \emptyset$, if $1 \leq i \not = j \leq r-2$. We are interested in 
 the subset of $\ca G(r,k)$ consisting of all such graphs with the smallest order.
 This set is denoted by $\ca T(r,k)$.
 By definition, $\ca T_r \subseteq \bigcup_{i=1}^{r-2} \ca T(r,i)$.

\section{Smallest $r$-graphs of class $2$} \label{Sec: smallest r-graphs}

\subsection{Determination of $\ca T_r$} 




The following theorem extends Lemma \ref{lem: P^M class 2} and characterizes the perfect matchings $M$ on $V(P)$ such that $P+M$ is a class $2$ graph.

\begin{theo}\label{theo:P+matching}
	Let $P$ be the Petersen graph and $H$ be a 1-regular graph on 
	$V(P)$ with edge set $M$. Then $P+M$ is class $2$ if and only if $M\subseteq E(P)$.
\end{theo}
\begin{proof}\textcolor{red}{TOPROVE 0}\end{proof}


\begin{figure}[htbp]
	\centering
	\scalebox{1}{\begingroup \makeatletter \providecommand\color[2][]{\errmessage{(Inkscape) Color is used for the text in Inkscape, but the package 'color.sty' is not loaded}\renewcommand\color[2][]{}}\providecommand\transparent[1]{\errmessage{(Inkscape) Transparency is used (non-zero) for the text in Inkscape, but the package 'transparent.sty' is not loaded}\renewcommand\transparent[1]{}}\providecommand\rotatebox[2]{#2}\newcommand*\fsize{\dimexpr\f@size pt\relax}\newcommand*\lineheight[1]{\fontsize{\fsize}{#1\fsize}\selectfont}\ifx\svgwidth\undefined \setlength{\unitlength}{139.29602089bp}\ifx\svgscale\undefined \relax \else \setlength{\unitlength}{\unitlength * \real{\svgscale}}\fi \else \setlength{\unitlength}{\svgwidth}\fi \global\let\svgwidth\undefined \global\let\svgscale\undefined \makeatother \begin{picture}(1,0.78588845)\lineheight{1}\setlength\tabcolsep{0pt}\put(0,0){\includegraphics[width=\unitlength,page=1]{C5_ev2.pdf}}\put(0.48985459,0.49265317){\makebox(0,0)[lt]{\lineheight{1.25}\smash{\begin{tabular}[t]{l}$e$\end{tabular}}}}\put(-0.00222725,0.46814327){\makebox(0,0)[lt]{\lineheight{1.25}\smash{\begin{tabular}[t]{l}$u_2$\end{tabular}}}}\put(0.47706804,0.76550256){\makebox(0,0)[lt]{\lineheight{1.25}\smash{\begin{tabular}[t]{l}$u_1$\end{tabular}}}}\put(0.92059197,0.47064011){\makebox(0,0)[lt]{\lineheight{1.25}\smash{\begin{tabular}[t]{l}$u_5$\end{tabular}}}}\put(0.21913928,0.00632801){\makebox(0,0)[lt]{\lineheight{1.25}\smash{\begin{tabular}[t]{l}$u_3$\end{tabular}}}}\put(0.6998391,0.00773603){\makebox(0,0)[lt]{\lineheight{1.25}\smash{\begin{tabular}[t]{l}$u_4$\end{tabular}}}}\end{picture}\endgroup  } 
	\caption{The $5$-circuit $C^1_5$ with the edge $ e $.}
	\label{fig: C5+e}
\end{figure}













\begin{theo}\label{Theorem-cong-P+M_V2}
	For all $r\ge 3$,
$\ca T_r = \ca T(r,r-2) = \{P^{\ca M}\colon \ca M $ is a multiset of $r-3$ perfect matchings of the Petersen graph $ P\}$.
\end{theo}

\begin{proof}\textcolor{red}{TOPROVE 1}\end{proof}

\subsection{Lower bounds for $|\ca T_r|$}

The following lemma is a direct consequence of the fact that the Petersen graph is 
3-arc-transitive, see e.g.~Corollary 1.8 in \cite{Babai_Handbook}.
That is, for any two paths of length 3 of $P$ there is an
automorphism of $P$ which maps one to the other. 


\begin{lem}\label{lem:autP_fixing_3_pm}
	Let $M_1,\dots,M_6$ be the six perfect matchings of the Petersen graph $P$. Moreover, let $N_1,N_2,N_3 \in \{M_1,\dots,M_6\}$ and $g\colon \{N_1,N_2,N_3\} \to \{M_1,\dots,M_6\}$ be an injective function. There is an automorphism $ (\theta,\phi) $ of $P$ such that, for all $i\in\{1,2,3\}$, $\phi(N_i)=g(N_i)$.
\end{lem}

\begin{proof}\textcolor{red}{TOPROVE 2}\end{proof}


We now consider partitions of integers, which are ways of writing an integer as a sum of positive integers, see e.g.~\cite{matouvsek2008invitation}. 
We are interested in partitions of an integer into a fixed number of parts.
We allow $0$ to be a part of a partition.
A \emph{partition} of an integer $n$ into $k$ parts is a multiset of $k$ integers
$n_1, \dots,n_k$ with $n_i \geq 0$ for $i \in \{1, \dots,k\}$ 
such that $n = \sum_{i=1}^k n_i$. Two partitions of $n$ are equal if
they yield the same multiset, i.e.\ if they differ only in the order of their elements. 
For two positive integers $k\le n$, let $p'(n,k)$ be the number of partitions of $n$ into $k$ parts. Set $p'(0,k)=1$.   

\begin{theo} \label{thm: lower bound S(r,r-2)}
If $3\leq r \leq 8$, then $|\ca T_r| = p'(r-3,6)$, and 
	if $r \geq 9$, then  $|\ca T_r| > p'(r-3,6)$.
\end{theo}

\begin{proof}\textcolor{red}{TOPROVE 3}\end{proof}


\section{Complete sets} \label{Sec: H-coloring}

In this section we give the following characterization of $\ca H_r$: $G\in \ca H_r$ if and only if the only connected $r$-graph coloring $G$ is $G$ itself. Moreover, we show that $\ca H_r$ is an infinite set when $r\ge 4$. For $r=3$ it turns out that, if the Petersen Coloring Conjecture is false, then $\ca H_3$ is an infinite set too.
We prove similar results for the restriction on simple $r$-graphs.

We start with some preliminary technical results. In particular, we introduce a lifting operation for $r$-graphs.


\subsection{Substructures and lifting}

Let $G$ be a graph and $F \subseteq E(G)$. We say that $F$ \emph{induces} a subgraph $H$
of $G$ if $E(H) = F$ and $V(H)$ contains all vertices of $G$ which are incident with
an edge of $F$. We denote such a subgraph $H$ by $G[F]$. 
A spanning subgraph $ G' $ of $ G $ is a 
$ \{K_{1,1}, C_m\colon m\geq3\} $-factor if each component of $ G' $ is isomorphic to an element of $ \{K_{1,1}, C_m\colon m\geq3\} $, where $ K_{s,t} $ is the complete bipartite graph with two partition sets of sizes $ s $ and $ t $.
Some of the following observations appear also in  \cite{MTZ_r_graphs}.

\begin{obs}
	\label{obs:coloring_basics}
	Let $H$ and $G$ be graphs and let $f$ be an $H$-coloring of $G$.
	\begin{itemize}
		\item[(i)] $\chi'(G) \leq \chi'(H)$.
		\item[(ii)] If $M_1,\dots, M_k$ are $k$ pairwise disjoint perfect matchings in $H$, then $f^{-1}(M_1),\dots, f^{-1}(M_k)$ are $k$ pairwise disjoint perfect matchings in $G$.
		\item[(iii)] If $C$ is a $2$-regular subgraph of $H$, then $f^{-1}(E(C))$ induces a $2$-regular subgraph in $G$.
		\item[(iv)] If $ H' $ is a $ \{K_{1,1}, C_m\colon m\geq3\} $-factor in $H$, then $ f^{-1} (E(H'))$ induces a $ \{K_{1,1}, C_m\colon m\geq3\} $-factor in $ G $.
	\end{itemize}
\end{obs}
\begin{proof}\textcolor{red}{TOPROVE 4}\end{proof}




Let $G$ be a graph and let $x \in V(G)$ with $ |N_G(x)|\geq2 $. A \emph{lifting} (of $G$) at $x$ is the following operation: Choose two distinct neighbors $y$ and $z$ of $x$, delete an edge $e_1$ connecting $x$ with $y$, delete an edge $e_2$ connecting $x$ with $z$ and add a new edge $e$ connecting $y$ with $z$; additionally, if $ e_1 $ and $ e_2 $ were the only two edges incident with $ x $, then delete the vertex $ x $ in the new graph. We say $e_1$ and $e_2$ are \emph{lifted to} $e$; the new graph is denoted by $G(e_1,e_2)$. 

We will make use of the following fact. Let $G$ be a graph, then 
$\vert \partial_G(X \cap Y) \vert + \vert \partial_G(X \cup Y) \vert \leq \vert \partial_G(X) \vert + \vert \partial_G(Y) \vert$ for every $X,Y \subseteq V(G)$.



\begin{lem}\label{Lem:r-graph lifting}
	Let $ r \geq 2$ be an integer and let $G$ be a connected graph of order at least $ 2 $ with a vertex $x \in V(G)$ such that
	\begin{itemize}
		\item $d_G(v)=r$ for all $v \in V(G)\setminus\{x\}$, and
		\item if $\vert V(G) \vert$ is even, then $d_G(x)\neq r$, and
		\item $\vert \partial_G(S) \vert \geq r$ for every $S\subseteq V(G)\setminus\{x\}$ of odd cardinality.
	\end{itemize}
	Then, for every labeling $\partial_G(x)=\{e_1,\ldots, e_{d_G(x)}\}$ there exists an  $i \in \mathbb{Z}_{d_G(x)}$ such that $G(e_i,e_{i+1})$ is a connected graph with $\vert \partial_{G(e_i,e_{i+1})}(S') \vert \geq r$ for every $S'\subseteq V(G(e_i,e_{i+1}))\setminus\{x\}$ of odd cardinality.
\end{lem}

\begin{proof}\textcolor{red}{TOPROVE 5}\end{proof}

The previous lemma can be used in $r$-graphs as follows.

\begin{theo}
	\label{theo:r-graph_lifting}
	Let $r\geq 2$ be an integer, let $G$ be a connected $r$-graph and let $X$ be a non-empty proper subset of $V(G)$. If $\vert X \vert$ is even, then $G/X$ can be transformed into a connected $r$-graph by  applying $\frac{1}{2}\left \vert \partial_G(X)\right|$ lifting operations at $w_X$. If $\vert X \vert$ is odd, then $G/X$ can be transformed into a connected $r$-graph by applying $\frac{1}{2}\left( \vert \partial_G(X) \vert - r \right)$ lifting operations at $w_X$.
\end{theo}

\begin{proof}\textcolor{red}{TOPROVE 6}\end{proof}

Note that the previous lifting operations can be applied so that they preserve embeddings of graphs in surfaces.

\subsection{Characterization of $\ca H_r$}


Let $ f$ be an $ H $-coloring of $ G $.
The subgraph of $H$ induced by the edge set $Im(f)$ is denoted by $H_f$. Observe that $H_f$ also colors $G$. Furthermore, if $H$ has no two vertices $u_1,u_2$ with $\partial_H(u_1)=\partial_H(u_2)$, then $f$ induces a mapping $f_V\colon V(G) \to V(H)$, where every $v \in V(G)$ is mapped to the unique vertex $u \in V(H)$ with $f(\partial_G(v))=\partial_H(u)$. Note that $f_V$ is well defined if $H$ is a connected graph with $|V(H)|>2$. A vertex of $V(H)\setminus Im(f_V)$ is called \emph{unused}.






\begin{theo}
\label{theo:coloring_graphs_in_S(r,k)_generalisation}
Let $r \geq 3$ and let $G$ be an $r$-graph of class 2 that cannot be colored by an $r$-graph of smaller order. If $H$ is a connected $r$-graph and $f$ is an $H$-coloring of $G$, then $(f_V,f)$ is an isomorphism, i.e. $H \cong G$.
\end{theo}

\begin{proof}\textcolor{red}{TOPROVE 7}\end{proof}

In \cite{Mkrtchyan_Pet_col}, Mkrtchyan proves that if a connected $3$-graph $H$ colors the Petersen graph $P$, then $H\cong P$. The following result is implied by Theorem~\ref{theo:coloring_graphs_in_S(r,k)_generalisation} together with Observation \ref{obs:coloring_basics} $(ii)$ and  gives a generalization of Mkrtchyan's result in the $r$-regular case.

\begin{cor}
\label{cor:coloring_graphs_in_S(r,k)}
Let $r \geq 3$ and let $G$ be an $r$-graph of class 2 such that $\pi(G')>\pi(G)$ for every $r$-graph $G'$ with $|V(G')|<|V(G)|$. If $H$ is a connected $r$-graph with $H \prec G$, then $H \cong G$.
\end{cor}

 By Theorem \ref{thm: lower bound S(r,r-2)}, $\ca T_r = \ca T(r,r-2) =\{P^{\ca M}\colon \ca M $ is a set of $r-3$ perfect matchings of the Petersen graph $ P\}$.
Hence, with Corollary~\ref{cor:coloring_graphs_in_S(r,k)} we obtain the following theorem.

\begin{theo}
\label{theo:coloring_P^M}


Let $r\geq3$, let $H$ be a connected $r$-graph and let $G\in \ca T(r,r-2)\cup \ca T(r,1)$. If $H \prec G$, then $H \cong G$.
\end{theo}

\begin{theo}
\label{theo:characterisation H_r}
	Let $r\ge3$ and let $G$ be a connected $r$-graph. The following statements are equivalent.
	\begin{itemize}
		\item[1)] $G\in\ca H_r$.
		\item[2)] The only connected $r$-graph coloring $G$ is $G$ itself.
		\item[3)] $G$ cannot be colored by a smaller $r$-graph.
	\end{itemize}
\end{theo}

\begin{proof}\textcolor{red}{TOPROVE 8}\end{proof}

\begin{cor}
For every $r \geq 3$, there exists only one inclusion-wise minimal $r$-complete set, i.e. $\ca H_r$ is unique.
\label{cor:H_r unique}
\end{cor}


For $r=3$, we have $\ca T(r,r-2)=\ca T(r,1) = \{P\}$. The Petersen Coloring Conjecture states that $\ca H_3=\{P\}$. This situation is very exclusive as we show in the following subsection.

\subsection{Infinite subsets of $\ca H_r$}



\begin{lem}\label{Lemma-2-cut-image}
	Let $r \geq 3$, let $G$ and $H$ be two connected $ r $-graphs and let $f$ be an $H$-coloring of $G$. For any $ 2 $-edge-cut $ F=\{e_1,e_2\}\subseteq E(G) $, either $ |f(F)|=1 $ or $ f(F) $ is a $2$-edge-cut of $ H $.
\end{lem}

\begin{proof}\textcolor{red}{TOPROVE 9}\end{proof}











Let $G,H$ be two graphs, let $f \colon E(G) \to E(H)$, $g \colon V(G) \to V(H)$ and let $G'$ be a subgraph of $G$. The restriction of $f$ to $E(G')$ is denoted by $f|_{G'}$; the restriction of $g$ to $V(G')$ is denoted by $g|_{G'}$.


\begin{lem}\label{Lemma-PM-e-cong-itselt}
	Let $G$ and $H$ be two $ r $-graphs, where $r \geq 3$, and let $f$ be an $H$-coloring of $G$. 
	Let $\ca M$ be a multiset of $r-3 $ perfect matchings of  $P$ and let $ e_0\in E(P^{\ca M}) $. Let $ G' $ be an induced subgraph of $ G $ isomorphic to $P^{\ca M}-e_0$ and $ H' $ be the subgraph of $ H $ induced by $ f(E(G')) $. 
Then, $ (f_V|_{G'} , f|_{G'} ) $ is an isomorphism between $ G' $ and $ H' $, i.e.\ $ H'\cong G' $.  
\end{lem}

\begin{proof}\textcolor{red}{TOPROVE 10}\end{proof}

Let $ G $ and $ G' $ be two disjoint $ r $-graphs  of class $ 2 $  with $ e\in E(G) $ and $ e'\in E(G') $. Denote by $ (G,e)| (G',e')$  the set of all  graphs obtained from $ G $ by replacing the edge $ e $ of $ G $ by $ (G',e')$, that is,  deleting $ e $ from $ G $ and $ e' $ from $ G' $, and then adding two edges between $ V(G) $ and $ V(G') $ such that the resulting graph is regular (see Figure~\ref{fig:replacing_edge}).


\begin{figure}[htbp]
\centering
\scalebox{1}{\begingroup \makeatletter \providecommand\color[2][]{\errmessage{(Inkscape) Color is used for the text in Inkscape, but the package 'color.sty' is not loaded}\renewcommand\color[2][]{}}\providecommand\transparent[1]{\errmessage{(Inkscape) Transparency is used (non-zero) for the text in Inkscape, but the package 'transparent.sty' is not loaded}\renewcommand\transparent[1]{}}\providecommand\rotatebox[2]{#2}\newcommand*\fsize{\dimexpr\f@size pt\relax}\newcommand*\lineheight[1]{\fontsize{\fsize}{#1\fsize}\selectfont}\ifx\svgwidth\undefined \setlength{\unitlength}{177.47439233bp}\ifx\svgscale\undefined \relax \else \setlength{\unitlength}{\unitlength * \real{\svgscale}}\fi \else \setlength{\unitlength}{\svgwidth}\fi \global\let\svgwidth\undefined \global\let\svgscale\undefined \makeatother \begin{picture}(1,0.19149986)\lineheight{1}\setlength\tabcolsep{0pt}\put(0,0){\includegraphics[width=\unitlength,page=1]{replacing_edge.pdf}}\put(0.03757375,0.04343616){\color[rgb]{0,0,0}\makebox(0,0)[lt]{\lineheight{0.46875152}\smash{\begin{tabular}[t]{l}$e$\end{tabular}}}}\put(0.68543645,0.10734421){\color[rgb]{0,0,0}\makebox(0,0)[lt]{\lineheight{0.46875152}\smash{\begin{tabular}[t]{l}$G'-e'$\end{tabular}}}}\put(0,0){\includegraphics[width=\unitlength,page=2]{replacing_edge.pdf}}\end{picture}\endgroup  } 
\caption{A replacement of the edge $e$ by $(G',e')$.}
\label{fig:replacing_edge}
\end{figure}


In fact, any graph in $ (G,e)| (G',e')$ is an $ r $-graph of class $ 2 $. Furthermore, we use $ G | (G',e')$ to denote the set of all  graphs obtained from $ G $ by replacing each edge of $ G $ by $ (G',e')$.

\begin{theo}
\label{theo:inductive construction}
	Let $\ca M$ be a multiset of $r-3 $ perfect matchings of $P$, where $r \geq 3$, and let $ e_0\in E(P^{\ca M}) $. Let $G$ be an $r$-graph such that $ G\ncong  P^{\ca M} $. If $ G \in \ca H_r $ , then $ G | (P^{\ca M},e_0) \subset \ca H_r$. 
\end{theo}
\begin{proof}\textcolor{red}{TOPROVE 11}\end{proof}




The following corollary answers the question of 
\cite{MTZ_r_graphs} whether for each $r \geq 4$, there exists a
connected $r$-graph $H$ with $H \prec G$ for every $r$-graph $G$.

\begin{cor}
	\label{cor:H_3 = P or infinite}
	Either $\ca H_3 = \{P\}$ or $\ca H_3$ is an infinite set.	
	Moreover, if $r\geq 4$, then $\ca H_r$ is an infinite set.
\end{cor}

\begin{proof}\textcolor{red}{TOPROVE 12}\end{proof}








\subsection{Simple $r$-graphs}

In \cite{MTZ_r_graphs} the authors also asked whether for every $r\geq 4$, there is a connected $r$-graph coloring all simple $r$-graph. In this section we answer this question by showing that there is no finite set of connected $r$-graphs $\ca H_r'$ such that every connected simple $r$-graph can be colored by an element of $\ca H_r'$. 


\begin{lem}[\cite{jin2017covers}]\label{lem:1-fator_avoiding_r-1_edges}
	Let $r$ be a positive integer, $G$ be an $r$-graph and $F \subseteq E(G)$. 
	If $|F|\le r-1$, then $G-F$ has a $1$-factor.
\end{lem}

Recall that, for an $r$-graph $G$ and an odd set $X\subseteq V(G)$, an edge-cut $\partial_G(X)$ is \emph{tight} if it consists of exactly $r$ edges. 

\begin{lem}\label{Proposition-tightcut-to-tightcut}
	Let $r\geq3$, let $G,H$ be connected $r$-graphs and let $f$ be an $H$-coloring of $G$. If $F\subseteq E(G)$ is a tight edge-cut in $G$, then $f(F)$ is a tight edge-cut in $H$.
\end{lem}

\begin{proof}\textcolor{red}{TOPROVE 13}\end{proof}



\begin{lem}\label{Lemma-H-noclass1-subgraph}
	Let $r \geq 3$, let $ G $ and $ H $ be two $ r $-graphs, and let $ X $ be a subset of $ V(H) $ such that $ \partial_H(X) $ is a tight cut and $ \chi'(H/X^c)= r$. If  $H\prec G $,  then $H/X \prec G $.
\end{lem}

\begin{proof}\textcolor{red}{TOPROVE 14}\end{proof}

For any graph $ G $, the number of isolated vertices of $ G $ is denoted by $ iso(G) $.
A simple graph $ H $ is \emph{regularizable} if we can obtain a regular graph from $H $ by replacing each edge of $ H $ by a nonempty set of parallel edges. We need the following lemma, which follows from two results of \cite{Berge-Vergnas-1978} and \cite{Pulleyblank-1979}. The equivalence of the first two statements is shown in \cite{Berge-Vergnas-1978}; the equivalence of the first and the third statement is shown in \cite{Pulleyblank-1979}.

\begin{lem}\label{Lemma-regular=factor}
	Let $ G $ be a simple connected graph which is not bipartite with two partition sets of the same cardinality. The following statements are equivalent:
	\begin{itemize}
		\item  $ iso(G-S) < |S|  $, for all $ S \subseteq V(G)$.
		\item   $ G $ is regularizable \cite{Berge-Vergnas-1978}.
		\item for every $ v\in V(G) $, both $ G-v $ and $ G $ have a $ \{K_{1,1}, C_m\colon m\geq3\} $-factor \cite{Pulleyblank-1979}.
	\end{itemize}
\end{lem}


\begin{lem}\label{Lemma-contract-H-class1}
	Let $r\geq 3$, let $ G$ and $ H $ be $ r $-graphs, where $H$ is connected, and let $S \subseteq V(G) $ such that $ \partial_G(S) $ is a tight cut and $ G[S] $ has no $ \{K_{1,1}, C_m:m\geq3\} $-factor. If $ G $ has an $ H $-coloring $f\colon E(G)\to E(H)$
	and $\partial_H(X)=f(\partial_G(S))$ for an $X \subseteq V(H)$, then   $H/X$ or $H/X^c$ is a bipartite graph with two partition sets of the same cardinality.
\end{lem}
\begin{proof}\textcolor{red}{TOPROVE 15}\end{proof}

	Let  $G$ be an $r$-regular graph with a vertex $v\in V(G)$. A \emph{Meredith extension} of $ G $ at $ v $ is the following operation. Delete the vertex $ v $ from $ G $ and add a copy $K$ of the complete bipartite graph $ K_{r,r-1} $. Finally add $ r $ edges between $ V(G-v)$ and $ V(K) $ such that the resulting graph is $ r $-regular.

\begin{lem}[Rizzi \cite{rizzi1999indecomposable}]\label{Lemma-keep-r-graph}
	Let $ G $ be a graph and $ X\subseteq V(G) $ with $ |X| $ odd. If $G/X$ and $G/X^c$ are both $ r $-graphs, then $ G $ is an $r$-graph.
\end{lem}

 


\begin{theo}\label{theo:reduction_simple_case}
	Let $r\geq 3$ and let $\ca H$ be a set of connected $r$-graphs such that every  $H\in \ca H$ does not contain a  non-trivial tight edge-cut $\partial_H(X)$ such that  $H/X$ or $H/X^c$ is class 1.
	If 
	every connected simple $r$-graph can be colored by an element of $\ca H$, 
then every connected $r$-graph can be colored by an element of $\ca H$.
\end{theo}

\begin{proof}\textcolor{red}{TOPROVE 16}\end{proof}




We obtain the main result of this section as a corollary.

\begin{cor}\label{cor:color-simple-r-graphs}
	Let $ r\geq3 $ and let $\ca H_r'$ be a set of connected $r$-graphs such that every connected simple $r$-graph can be colored by an element of $\ca H_r'$.
\begin{itemize}
\item[$i)$] If the Petersen Coloring Conjecture is false, then $\ca H_3'$ is an infinite set.
\item[$ii)$] If $r \geq 4$, then $\ca H_r'$ is an infinite set. 
\end{itemize}	
	 
\end{cor}



\begin{proof}\textcolor{red}{TOPROVE 17}\end{proof}



\section{Concluding remarks} \label{Sec: final remarks}

\subsection{Quasi-ordered sets}
Jaeger \cite{Jaeger1980} initiated the study of the Petersen Coloring Conjecture in terms of partial ordered sets. DeVos, Ne\v{s}et\v{r}il and Raspaud \cite{DeVos_etal_2007} studied cycle-continuous mappings and asked whether there is an infinite set $ \ca G $ of  bridgeless graphs such that every two of them are cycle-continuous incomparable, i.e.\ there is no cycle-continuous map between any two graphs in $\ca G$.
\v{S}\'amal \cite{Robert_2017} gave an affirmative answer to the above question by constructing such an infinite set $ \ca G $ of  bridgeless cubic graphs.
In fact, he also mentioned that
this result can be considered in view of a quasi-order induced by cycle-continuous mappings on the set of bridgeless cubic graphs. That is, this quasi-ordered set contains infinite antichains.

For every integer $r \geq 1$, $H$-colorings of $r$-graphs induce a quasi-order on the set of $r$-graphs.
Then, our result on $ r $-graphs can be restated  as follows: for any $ r\geq4 $, there is an infinite set $ \ca H_r $ of  $ r $-graphs such that each of them  is  incomparable to any other $ r $-graph, and such infinite set exists for $ r=3 $ if the Petersen Coloring Conjecture is false.  
In particular, the set $\ca H_r $ is an infinite antichain.
 




\subsection{Open problems}

The edge connectivity of an $r$-graph is equal to $r$ or it is an even number.  
We have shown that $\ca T(r,r-2) \cup \ca T(r,1) \subseteq \ca H_r$.
Thus, for $r \not =5$, for each possible edge-connectivity $t$ there is 
a $t$-edge-connected $r$-graph in $\ca H_r$. For $r=5$, we do not know any 
$5$-edge-connected $5$-graph with this property, see \cite{MMSW_pdpm}
for a discussion of this topic. However, we know only a finite 
number of $t$-edge-connected $r$-graphs of $\ca H_r$ if $t \geq 3$. 

\begin{prob}
	For $r,t \geq 3$, does $\ca H_r$ contain infinitely many $t$-edge-connected 
	$r$-graphs?
\end{prob}

It is also not clear whether $\ca H_r$ contains elements
of $\ca T(r,k)$ for $k \in \{2, \dots, r-3\}$. 
So far, these sets are not determined for $k \in \{1, \dots, r-3\}$. 
Indeed, we even do not know the order of their elements. 
Let 
$o(r,k)$ be the order of the graphs of $\ca T(r,k)$. 

\begin{prob}\label{o(r,k)}
	For all $r \geq 3$ and $k \in \{1, \dots, r-2 \}\colon$ Determine $o(r,k)$.  
\end{prob}

By our results, $o(r,r-2) = 10$.
By results of Rizzi \cite{rizzi1999indecomposable}, $o(r,1) \leq 2 \times 5^{r-2}$. We conjecture the following to be true.

\begin{con} \label{conj: order}
	For all $r \geq 3$ and $k \in \{2, \dots, r-2\} \colon o(r,k-1) \geq o(r,k)$.
\end{con} 

If Conjecture \ref{conj: order} would be true, then it would follow
with Corollary~\ref{cor:coloring_graphs_in_S(r,k)} that $\ca T(r,k) \subset \ca H_r$ 
for each $k \in \{1, \dots, r-2\}$. 

Similar problems arise for simple $r$-graphs. Let $o_s(r,k)$ be the smallest
order of a simple $r$-graph $G$ with $\pi(G)=k$.
Small simple $r$-graphs of class 2 can be obtained as follows. Consider a perfect matching $M$ of $P$ and the graph $G=P+(r-3)M$. Let $H$ be a simple $r$-graph of smallest order and $v \in V(H)$. Then, $H$ is class 1 and $|V(H)|=r+1$ if $r$ is odd and $|V(H)|=r+2$ if $r$ is even. Now, replace appropriately five vertices of $G$ by $H-v$ to obtain a simple $r$-graph $G'$. Since $H$ is class 1 and $\pi(G)=r-2$, we have $\pi (G')=r-2$.   Therefore, if $r$ is odd, then $o_s(r,r-2) \leq 5(r+1)$ and
if $r$ is even, then $o_s(r,r-2) \leq 5(r+2)$. Furthermore, bounds for $o_s(r,k)$ can be obtained by using Meredith extensions, since if $G'$ is a Meredith extension of an $r$-graph $G$, then $\pi(G')=\pi(G)$.








	


\bibliography{Lit_reg_graphs}{}
\addcontentsline{toc}{section}{References}
\bibliographystyle{abbrv}

\end{document}
