\documentclass[english]{article}
\usepackage{amsmath,amssymb,amsthm,mathtools}
\usepackage[shortlabels]{enumitem}
\usepackage[pdftex,colorlinks,backref=page,citecolor=blue,bookmarks=false]{hyperref}
\usepackage{tikz}
\usetikzlibrary{shapes.misc,calc,intersections,patterns,decorations.pathreplacing}
\usetikzlibrary{arrows,shapes,positioning}
\usetikzlibrary{decorations.markings}
\usepackage{cleveref}
\newcommand{\shoham}[1]{\textcolor{blue}{#1}}

\setlength{\oddsidemargin}{0in}
\setlength{\evensidemargin}{0in}
\setlength{\marginparwidth}{0in}
\setlength{\marginparsep}{0in}
\setlength{\marginparpush}{0in}
\setlength{\topmargin}{0in}
\setlength{\headsep}{8pt}
\setlength{\footskip}{.3in}
\setlength{\textheight}{8.7in}
\setlength{\textwidth}{6.5in}
\setlength{\parskip}{3pt}
\allowdisplaybreaks

\usepackage{setspace}
\setstretch{1.3}
\setlength{\parskip}{\medskipamount}
\setlength{\parindent}{0pt}

\theoremstyle{plain}
\newtheorem{theorem}{Theorem}[section]		
\newtheorem{lemma}[theorem]{Lemma}
\newtheorem{claim}[theorem]{Claim}
\newtheorem{proposition}[theorem]{Proposition}
\newtheorem{observation}[theorem]{Observation}
\newtheorem{corollary}[theorem]{Corollary}
\newtheorem{conjecture}[theorem]{Conjecture}
\newtheorem{definition}[theorem]{Definition}
\newtheorem{problem}[theorem]{Problem}
\theoremstyle{remark}
\newtheorem{question}[theorem]{Question}
\newtheorem*{remark}{Remark}



\def\CC{\mathcal{C}}
\def\C{\mathcal{C}}
\def\X{\mathcal{X}}
\def\TT{\mathcal{D}}
\def\TTT{\mathcal{T}}
\def\RR{\mathcal{R}}
\def\BB{\mathcal{B}}
\def\B{\mathcal{B}}
\def\DD{\mathscr{U}}
\def\N{\mathbb{N}}
\def\Z{\mathbb{Z}}
\def\ss{\mathbb{S}}
\def\SS{\mathscr{S}}
\def\HH{\mathcal{H}}
\def\GG{\mathcal{G}}
\def\PP{\mathcal{P}}
\def\Ex{\mathbb{E}}

\let\phi\varphi

\def \Sp {S^+}
\def \Sm {S^-}
\def \sp {s^+}
\def \smm {s^-}
\def \Ap {A^+}
\def \Am {A^-}
\def \Np {N^+}
\def \Nm {N^-}
\def \dpp {d^+}
\def \dm {d^-}
\def \Up {U^+}
\def \Um {U^-}
\def \Yp {Y^+}
\def \Ym {Y^-}
\def \yp {y^+}
\def \ym {y^-}

\def \Vgp {V_{\good}^+}
\def \Vbp {V_{\bad}^+}
\def \Vbm {V_{\bad}^-}
\def \Vgp {V_{\good}^+}
\def \Vgm {V_{\good}^-}
\def \Vo {V_{\okay}}
\def \Vg {V_{\good}}
\def \Vb {V_{\bad}}
\def \ap {a^+}
\def \am {a^-}

\def \Vgm {V_{\good}^-}

\def \Nmu{N^{\mu}_{\good}}
\def \Nnu{N^{\nu}_{\good}}
\def \Nnuo{N^{\nu}_{\okay}}

\def \Npo{N^+_{\okay}}
\def \Nmo{N^-_{\okay}}
\def \Npg{N^+_{\good}}
\def \Nmg{N^-_{\good}}

\def \Dm {\Delta^-}

\def \Yp {Y^+}
\def \Ym {Y^-}
\def \Np {N^+}
\def \Nm {N^-}

\renewcommand{\Pr}{\mathbb{P}}




\DeclareMathOperator\Deg{d}
\DeclareMathOperator\ex{ex}
\newcommand{\eps}{\ensuremath{\varepsilon}}
\let\emptyset\varnothing
\newcommand*{\abs}[1]{\lvert#1\rvert}
\newcommand{\kzero}{33}

\newcommand{\e}{\ensuremath{\varepsilon}}
\newcommand{\al}{\alpha}

\let\originalleft\left
\let\originalright\right
\renewcommand{\left}{\mathopen{}\mathclose\bgroup\originalleft}
\renewcommand{\right}{\aftergroup\egroup\originalright}

\makeatletter
\def\imod#1{\allowbreak\mkern10mu({\operator@font mod}\,\,#1)}
\makeatother

\newcommand{\urlprefix}{}

\newcommand\todo[1]{\textcolor{red}{#1}}

\newcommand{\floor}[1]{\left\lfloor #1 \right \rfloor}
\newcommand{\ceil}[1]{\left\lceil #1 \right \rceil}
\newcommand{\subs}{\subseteq}
\newcommand{\sm}{\setminus}
\newcommand{\cplt}{\rightrightarrows} \newcommand{\tA}{\tilde{A}}
\newcommand{\tB}{\tilde{B}}
\newcommand{\A}{\mathcal{A}}





\DeclareMathOperator{\gadget}{gadget}
\DeclareMathOperator{\remainder}{remainder}
\DeclareMathOperator{\bad}{bad}
\DeclareMathOperator{\good}{good}
\DeclareMathOperator{\okay}{okay}


\usepackage{framed}

\newcommand{\rem}[1]{\begin{framed} #1 \end{framed}} 

\title{Partitioning a tournament into sub-tournaments of high connectivity }
\author{Ant\'onio Gir\~ao\thanks{
		Mathematical Institute, 
		University of Oxford,
		Andrew Wiles Building, 
		Radcliffe Observatory Quarter, 
		Woodstock Road,
		Oxford, UK.
		Email: \texttt{girao}@\texttt{maths.ox.ac.uk}.
		Research supported by EPSRC grant EP/V007327/1.
	}
	\and
	Shoham Letzter\thanks{
		Department of Mathematics, 
		University College London, 
		Gower Street, London WC1E~6BT, UK. 
		Email: \texttt{s.letzter}@\texttt{ucl.ac.uk}. 
		Research supported by the Royal Society.   
	}
}
\date{}

\begin{document}

\maketitle

\begin{abstract}
	\setlength{\parskip}{\medskipamount}
    \setlength{\parindent}{0pt}
    \noindent

	We prove that there exists a constant $c > 0$ such that the vertices of every strongly $c \cdot kt$-connected tournament can be partitioned into $t$ parts, each of which induces a strongly $k$-connected tournament. This is clearly tight up to a constant factor, and it confirms a conjecture of K\"uhn, Osthus and Townsend (2016).

\end{abstract}

\section{Introduction}

	A classical result of Hajnal \cite{hajnal1983partition} and Thomassen \cite{thomassen1983graph} asserts that for every integer $k \ge 1$ there exists an integer $K$ such that the vertices of every $K$-connected graph can be partitioned into two sets inducing $k$-connected subgraphs. There is now a whole area of combinatorial problems concerned with questions of this type; namely, to understand whether for a certain (di)graph property any (di)graph which \textit{strongly} satisfies that property has a partition into many parts where each part still has the property. In this paper, we consider the analogue of Hajnal and Thomassen's results for tournaments.

	A digraph $D$ is said to be \emph{strongly connected} if for every $u, v \in V(D)$ there is a directed path from $u$ to $v$, and it is \emph{strongly $k$-connected} if $|D| \ge k+1$ and $D \setminus Z$ is strongly connected for every subset $Z \subseteq V(D)$ of size at most $k$. Recall that a \emph{tournament} is an orientation of a complete graph. 
	Thomassen asked (see \cite{reid1989three}) if for every sequence $k_1, \ldots, k_t$ of positive integers there exists $K$ such that if $T$ is a strongly $K$-connected tournament then there is a partition $\{V_1, \ldots, V_t\}$ of $V(T)$ such that $T[V_i]$ is $k_i$-connected for every $i \in [t]$. Denote the minimum such $K$ by $f_t(k_1, \ldots, k_t)$ (and put $f_t(k_1, \ldots, k_t) := \infty$ if there is no such $K$).

	It is easy to see that $f_t(k, 1, \ldots, 1) \le k + 3t - 3$. Chen, Gould and Li \cite{chen2001partitioning} proved that every strongly $t$-connected tournament on at least $8t$ vertices can be partitioned into $t$ strongly connected tournaments (this is clearly optimal, apart from the assumption on the number of vertices). The existence of $f_2(2,2)$ remained open until the work of K\"uhn, Osthus and Townsend \cite{kuhn2016proof} who proved that $f_t(k_1, \ldots, k_t)$ is finite for all positive integers $k_1, \ldots, k_t$. Specifically, they showed $f_t(k, \ldots, k) = O(k^7 t^4)$ and conjectured $f_t(k, \ldots, k) = O(kt)$ (which would be tight up to the implicit constant factor). More recently, Kang and Kim~\cite{KangKim} proved a better upper bound on  $f_t(k,\ldots ,k)$ showing that any tournament on $n$ vertices which is $O(k^4t)$-strongly connected can be partitioned into $t$ strongly connected tournaments where each part has a prescribed size provided all sizes are $\Omega(n)$. 
	Our main result proves the conjecture of K\"uhn, Osthus and Townsend.  
	
	\begin{theorem} \label{thm:main}
		There exists a constant $c > 0$\footnote{It probably suffices to take $c = 10^{100}$.} such that for every positive integers $k$ and $t$, if $T$ is a strongly $c \cdot kt$-connected tournament, then there is a partition $\{V_1, \ldots, V_{t}\}$ of $V(T)$ such that $T[V_i]$ is strongly $k$-connected for $i \in [t]$.
	\end{theorem} 

	We give an overview of the proof in \Cref{sec:overview}, state a few simple probabilistic tools in \Cref{sec:prelims}, and dive into the proof of \Cref{thm:main} in \Cref{sec:proof}. We conclude the paper in \Cref{sec:conclusion} with some open problems.

	Throughout the paper, when we say a tournament is \emph{$k$-connected} we mean that it is strongly $k$-connected, and by a \emph{path} we mean a directed path. We will omit floor and ceiling signs whenever it does not affect the argument.

\section{Overview of proof} \label{sec:overview}

	Let $c$ be a large constant, and let $G$ be a $c\cdot kt$-connected tournament.
	We start the proof by finding $\Omega(kt)$ pairwise disjoint `gadgets' $U(\alpha)$, with $\alpha \in \A$ for some index set $\A$, with special sets $\Sp(\alpha), \Sm(\alpha) \subseteq U(\alpha)$, such that the following properties hold:
	for every $u \in \Sm(\alpha)$ and $v \in \Sp(\alpha)$, there is a directed path in $U(\alpha)$ from $u$ to $v$; most vertices in $G$ have an out-neighbour in all but at most $kt$ sets $\Sm(\alpha)$; and similarly for in-neighbours in $\Sp(\alpha)$ (see \Cref{subsec:gadgets}). We note that similar gadgets are constructed in \cite{kuhn2016proof}. One new ingredient allows us to obtain the following additional property: there is a vertex $\sp(\alpha) \in \Sp(\alpha)$ such that almost every in-neighbour of $\sp(\alpha)$ is also an in-neighbour of $u$, for all but $O(1)$ vertices $u \in U(\alpha)$; and there exists $\smm(\alpha) \in \Sm(\alpha)$ with the analogous property for out-neighbours.

	To sketch the remainder of the proof, let us pretend that \emph{all} vertices have out-neighbours in all but at most $kt$ sets $\Sm(\alpha)$ and in-neighbours in all but at most $kt$ sets $\Sp(\alpha)$.
	We now proceed in four steps.

	In the first step (given in \Cref{subsec:available}) we remove some of the gadgets, deterministically and randomly, so that every vertex $u$ in a surviving gadget $U(\alpha)$ has $\Omega(kt)$ out- and in-neighbours that are either in $U(\alpha)$ or are not in a surviving gadget. Here it is crucial to have the latter property regarding $\sp(\alpha)$ and $\smm(\alpha)$, because effectively this means that we need to guarantee that $u$ satisfies the above property for $O(1)$ vertices $u$ in $U(\alpha)$, even if $U(\alpha)$ itself is large.

	In the second step (see \Cref{subsec:eligible}) we find $\Theta(t)$ disjoint groups of $\Theta(k)$ gadgets, such that every vertex $u$ in one of these gadgets $U(\alpha)$ has $\Omega(kt)$ out- and in-neighbours (either in $U(\alpha)$ or outside of these gadgets), each of which has an out-neighbour in $\Sm(\beta)$ for all but at most $t$ gadgets in $U(\alpha)$'s group, and an in-neighbour in $\Sp(\beta)$ for all but at most $t$ gadgets in the same group. To achieve this, we randomly partition the collection of gadgets from the previous step into $\Theta(t)$ parts, and then remove some of the parts and some of the gadgets.
	
	The third step (see \Cref{subsec:connected}) finds $t$ disjoint $k$-connected sets, each containing at least $10k$ gadgets. To do this, we first randomly assign each of the vertices not covered by the gadgets described in the previous paragraph into one of the groups of gadgets, and show that with positive probability, many of these augmented groups of gadgets contain a $k$-connected set.

	Finally, in \Cref{subsec:partition}, we assign each uncovered vertex $u$ to a $k$-connected set $U$ found in the previous paragraph which has at least $k$ in- and out-neighbours of $u$. 
	(The assumption that each group contains at least $10k$ gadgets helps here.)

	Recall, though, that this proof sketch assumed that every vertex has an out-neighbour in all but at most $kt$ sets $\Sm(\alpha)$ and similarly for in-neighbours. This need not be the case, however, and that complicates each of the above four steps. Let $\Vgp$ be the set of vertices that have out-neighbours in all but at most $kt$ sets $\Sm(\alpha)$, and define $\Vgm$ similarly. In the first step, instead of aiming for $\Omega(kt)$ out-neighbours not covered by gadgets, we aim for either $\Omega(kt)$ out-neighbours in $\Vgp \cap \Vgm$, or $\Omega(kt)$ out-neighbours in $\Vgp$, each of which has $\Omega(kt)$ in-neighbours in $\Vgp \cap \Vgm$, etc. We make similar adjustments in other steps.
	

\section{Notation and preliminaries} \label{sec:prelims}
	In this section we state a few probabilistic results. The following is a corollary of Hoeffding's inequality. 

	\begin{proposition} \label{prop:hoeffding}
		Let $\eta_1, \eta_2$ satisfy $\eta_1 > 4\eta_2 > 0$ and suppose that $m_1, \ldots, m_r \in [0, \eta_2 \ell]$ satisfy $m_1 + \ldots + m_r \ge \eta_1 \ell$. If $X_1, \ldots, X_r$ are independent random variables such that $X_j$ takes values $0$ and $m_j$ and $\Pr[X_j = m_j] \ge 1/2$, then
		\begin{equation} \label{eqn:hoeffding}
			\Pr[X_1 + \ldots + X_r \ge \eta_2 \ell] \ge 1 - \exp(-\eta_1/8\eta_2).
		\end{equation}
	\end{proposition}

	\begin{proof}\textcolor{red}{TOPROVE 0}\end{proof}

	The next proposition is a simple probabilistic observation that we will use many times.

	\begin{proposition} \label{prop:markov}
		Let $X_1, \ldots, X_r$ be $0,1$-random variables. Suppose that $\Pr[X_i = 1] \ge 1 - \eta^2$ for every $i \in [r]$. Then 
		\begin{equation*}
			\Pr[X_1 + \ldots + X_r \ge (1 - \eta)r] \ge 1 - \eta.
		\end{equation*}
	\end{proposition}

	\begin{proof}\textcolor{red}{TOPROVE 1}\end{proof}

	To conclude the section, we state Chernoff's bounds, which we will use extensively.

	\begin{lemma} \label{lem:chernoff}
		Let $X$ be the sum of independent random variables taking values in $\{0, 1\}$, and write $\mu := \Ex[X]$. Then the following holds for $\delta \in [0, 1]$.
		\begin{align*}
			& \Pr[X \le (1-\delta)\mu] \le \exp(-\delta^2 \mu / 2) \\
			& \Pr[X \ge (1+\delta)\mu] \le \exp(-\delta^2 \mu / 3).
		\end{align*}
	\end{lemma}

\section{The proof} \label{sec:proof}

	In this section we prove our main theorem, \Cref{thm:main}.


	\begin{proof}\textcolor{red}{TOPROVE 2}\end{proof}




\section{Conclusion} \label{sec:conclusion}

	Recall that $f_t(k_1, \ldots, k_r)$ is the minimum $K$ such that the vertices of every strongly $K$-connected tournament can be partitioned into $t$ sets, the $i^{\text{th}}$ of which induces a strongly $k_i$-connected tournament. We showed that $f_t(k, \ldots, k) = O(kt)$, which is tight up to the implicit constant factor. It would be interesting to evaluate $f_t(k_1, \ldots, k_t)$ when possibly $k_1, \ldots, k_t$ vary significantly. 

	\begin{question}
		Is it true that $f_t(k_1, \ldots, k_t) = O(k_1 + \ldots + k_t)$?
	\end{question}
    Note that it would be enough to show that for every  $k_1\geq k_2$ one has that $f_2(k_1,k_2)=k_1+O(k_2)$.
    
	It would also be very interesting, but probably very hard, to determine if the analogue of $f_t(k_1, \ldots, k_t)$ for digraphs (which are not necessarily tournaments) holds. 

	\begin{question} [Question 1.3 in \cite{kuhn2016proof}]
		Is there a function $g$ such that, for every positive integer $k$, the vertices of every strongly $g(k)$-connected digraph can be partitioned into two sets inducing strongly $k$-connected subdigraphs?
	\end{question}

	Finally, we remark that Kim, K\"uhn and Osthus \cite{kim2016bipartitions} proved that for every integer $k \ge 1$ there exists $K$ such that if $T$ is a strongly $K$-connected tournament, then there is a partition $\{V_1, V_2\}$ of $V(T)$ such that $T[V_1]$, $T[V_2]$ and $T[V_1, V_2]$ are $k$-strongly connected. Denote the minimum such $K$ by $h(k)$. Their proof shows $h(k) = O(k^6 \log k)$. It would be interesting to determine the correct order of magnitude of $h(k)$.
	
	\begin{question}
		Is $h(k) = O(k)$?
	\end{question}

\subsection*{Acknowledgements}

	We would like to thank Lantao Zou for pointing out an error in a previous version of this paper. We would also like to thank the referees for helpful comments.

	\bibliography{main}
	\bibliographystyle{amsplain}
\end{document}
