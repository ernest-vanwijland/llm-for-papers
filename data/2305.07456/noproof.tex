\documentclass[a4paper]{article}
\usepackage{amsthm,amssymb,amsmath,enumerate,graphicx,epsf}

\newcommand{\COLORON}{1}
\newcommand{\NOTESON}{0}
\newcommand{\Debug}{0}
\usepackage{bbold}
\usepackage[usenames]{color} 
\usepackage{amsthm,amssymb,amsmath,bbm,enumerate,graphicx,epsf,stmaryrd,accents}
\usepackage[bookmarks, colorlinks=false, breaklinks=true]{hyperref} 



\usepackage{authblk}



\hyphenation{com-pac-ti-fi-cation}

\newcommand{\keywordname}{{\bf Keywords:}}
\newcommand{\keywords}[1]{\par\addvspace\baselineskip 	\noindent\keywordname\enspace\ignorespaces#1}
	
\renewcommand{\theenumi}{(\roman{enumi})}
\renewcommand{\labelenumi}{\theenumi}

\newcommand{\comment}[1]{}
\newcommand{\COMMENT}[1]{}

\definecolor{darkgray}{rgb}{0.3,0.3,0.3}
\newcommand{\defi}[1]{{\color{darkgray}\emph{#1}}}


\newcommand{\acknowledgement}{\section*{Acknowledgement}}
\newcommand{\acknowledgements}{\section*{Acknowledgements}}





\comment{
	\begin{lemma}\label{}	
\end{lemma}
\begin{proof}\textcolor{red}{TOPROVE 0}\end{proof}


\begin{theorem}\label{}
\end{theorem} 
\begin{proof}\textcolor{red}{TOPROVE 1}\end{proof}

}






\newtheorem{proposition}{Proposition}[section]
\newtheorem{definition}[proposition]{Definition}
\newtheorem{theorem}[proposition]{Theorem}
\newtheorem{corollary}[proposition]{Corollary}
\newtheorem*{nocorollary}{Corollary}
\newtheorem{lemma}[proposition]{Lemma}
\newtheorem{observation}[proposition]{Observation}
\newtheorem{conjecture}{{Conjecture}}[section]
\newtheorem*{noConjecture}{{Con\-jecture}}
\newtheorem{problem}[conjecture]{{Problem}}
\newtheorem*{noProblem}{{Problem}}
\newtheorem{question}[conjecture]{{Question}}
\newtheorem*{noTheorem}{Theorem}
\newtheorem{examp}[proposition]{Example}\newtheorem{claim}{Claim}
\newtheorem*{noclaim}{Claim}

\newtheorem{vermutung}{{\color{red}Vermutung}}[section]
\newtheorem*{noVermutung}{{\color{red}Vermutung}}


\newcommand{\example}[2]{\begin{examp} \label{#1} {{#2}}\end{examp}}

\newcommand{\kreis}[1]{\mathaccent"7017\relax #1}

\newcommand{\FIG}{0}

\ifnum \NOTESON = 1 \newcommand{\note}[1]{ 

\hspace*{-30pt}
	{\color{blue}  NOTE: \color{Turquoise}{\small  \tt \begin{minipage}[c]{1.1\textwidth}  #1 \end{minipage} \ignorespacesafterend }} 
	
	}
\else \newcommand{\note}[1]{} \fi

\newcommand{\afsubm}[1]{ \ifnum \Debug = 1 {\mymargin{#1}}
\fi} 

\ifnum \Debug = 1 \newcommand{\sss}{\ensuremath{\color{red} \bowtie \bowtie \bowtie\ }}
\else \newcommand{\sss}{} \fi

\ifnum \FIG = 1 \newcommand{\fig}[1]{Figure ``{#1}''}
\else \newcommand{\fig}[1]{Figure~\ref{#1}} \fi

\ifnum \FIG = 1 \newcommand{\figs}[1]{Figures ``{#1}''}
\else \newcommand{\figs}[1]{Figures~\ref{#1}} \fi

\ifnum \Debug = 1 \usepackage[notref,notcite]{showkeys}
\fi

\ifnum \COLORON = 0 \renewcommand{\color}[1]{}
\fi

\newcommand{\Fig}{Figure}


\newcommand{\showFigu}[3]{
   \begin{figure}[htbp]
   \centering
   \noindent
   \epsfxsize=0.8\hsize
   \epsfbox{#1.eps}
\caption{\small #3}
   \label{#2}
   \end{figure}
}

\newcommand{\showFig}[2]{
   \begin{figure}[htbp]
   \centering
   \noindent
   \epsfbox{#1.eps}
   \caption{\small #2}
   \label{#1}
   \end{figure}
}

\newcommand{\showFigI}[1]{ \begin{figure}[htbp]
   \centering
   \noindent
   \epsfbox{#1.eps}
\label{#1}
   \end{figure}
}





\newcommand{\D}{\ensuremath{\mathbb D}}
\newcommand{\N}{\ensuremath{\mathbb N}}
\newcommand{\R}{\ensuremath{\mathbb R}}
\newcommand{\C}{\ensuremath{\mathbb C}}
\newcommand{\Z}{\ensuremath{\mathbb Z}}
\newcommand{\Q}{\ensuremath{\mathbb Q}}
\newcommand{\BS}{\ensuremath{\mathbb S}}

\newcommand{\ca}{\ensuremath{\mathcal A}}
\newcommand{\cb}{\ensuremath{\mathcal B}}
\newcommand{\cc}{\ensuremath{\mathcal C}}
\newcommand{\cd}{\ensuremath{\mathcal D}}
\newcommand{\ce}{\ensuremath{\mathcal E}}
\newcommand{\cf}{\ensuremath{\mathcal F}}
\newcommand{\cg}{\ensuremath{\mathcal G}}
\newcommand{\ch}{\ensuremath{\mathcal H}}
\newcommand{\ci}{\ensuremath{\mathcal I}}
\newcommand{\cj}{\ensuremath{\mathcal J}}
\newcommand{\ck}{\ensuremath{\mathcal K}}
\newcommand{\cl}{\ensuremath{\mathcal L}}
\newcommand{\cm}{\ensuremath{\mathcal M}}
\newcommand{\cn}{\ensuremath{\mathcal N}}
\newcommand{\co}{\ensuremath{\mathcal O}}
\newcommand{\cp}{\ensuremath{\mathcal P}}
\newcommand{\cq}{\ensuremath{\mathcal Q}}
\newcommand{\cs}{\ensuremath{\mathcal S}}
\newcommand{\cgr}{\ensuremath{\mathcal R}}
\newcommand{\ct}{\ensuremath{\mathcal T}}
\newcommand{\cu}{\ensuremath{\mathcal U}}
\newcommand{\cv}{\ensuremath{\mathcal V}}
\newcommand{\cx}{\ensuremath{\mathcal X}}
\newcommand{\cy}{\ensuremath{\mathcal Y}}
\newcommand{\cz}{\ensuremath{\mathcal Z}}
\newcommand{\cw}{\ensuremath{\mathcal W}}

\newcommand{\oo}{\ensuremath{\omega}}
\newcommand{\OO}{\ensuremath{\Omega}}
\newcommand{\alp}{\ensuremath{\alpha}}
\newcommand{\bet}{\ensuremath{\beta}}
\newcommand{\gam}{\ensuremath{\gamma}}
\newcommand{\Gam}{\ensuremath{\Gamma}}
\newcommand{\del}{\ensuremath{\delta}}
\newcommand{\Del}{\ensuremath{\Delta}}
\newcommand{\eps}{\ensuremath{\epsilon}}
\newcommand{\kap}{\ensuremath{\kappa}}
\newcommand{\lam}{\ensuremath{\lambda}}
\newcommand{\sig}{\ensuremath{\sigma}}
\newcommand{\al}{\ensuremath{\alpha}}
\newcommand{\el}{\ensuremath{\ell}}


\newcommand{\CCC}{\ensuremath{\mathcal C}}
\newcommand{\ccc}{\ensuremath{\mathcal C}}
\newcommand{\ccg}{\ensuremath{\mathcal C(G)}}
\newcommand{\fcg}{\ensuremath{|G|}}



\newcommand{\zero}{\mathbbm 0}
\newcommand{\one}{\mathbbm 1}
\newcommand{\sm}{\backslash}
\newcommand{\restr}{\upharpoonright}
\newcommand{\sydi}{\triangle}
\newcommand{\noproof}{\qed}
\newcommand{\isom}{\cong}
\newcommand{\subgr}{\subseteq}
\newcommand{\cls}[1]{\ensuremath{\overline{#1}}}
\newcommand{\act}{\curvearrowright}



\makeatletter
\DeclareRobustCommand{\cev}[1]{\mathpalette\do@cev{#1}}
\newcommand{\do@cev}[2]{\fix@cev{#1}{+}\reflectbox{$\m@th#1\vec{\reflectbox{$\fix@cev{#1}{-}\m@th#1#2\fix@cev{#1}{+}$}}$}\fix@cev{#1}{-}}
\newcommand{\fix@cev}[2]{\ifx#1\displaystyle
    \mkern#23mu
  \else
    \ifx#1\textstyle
      \mkern#23mu
    \else
      \ifx#1\scriptstyle
        \mkern#22mu
      \else
        \mkern#22mu
      \fi
    \fi
  \fi
}

\makeatother






\newcommand{\ztwo}{\mathbb Z_2}
\newcommand{\nin}{\ensuremath{{n\in\N}}}
\newcommand{\iin}{\ensuremath{{i\in\N}}}
\newcommand{\unin}{\ensuremath{[0,1]}}
\newcommand{\xy}{$x$-$y$}

\newcommand{\limf}[1]{\ensuremath{\lim \inf(#1)}}


\newcommand{\sgl}[1]{\ensuremath{\{#1\}}}
\newcommand{\pth}[2]{\ensuremath{#1}\text{--}\ensuremath{#2}~path}
\newcommand{\edge}[2]{\ensuremath{#1}\text{--}\ensuremath{#2}~edge}
\newcommand{\topth}[2]{topological \ensuremath{#1}\text{--}\ensuremath{#2}~path}
\newcommand{\topths}[2]{topological 
	\ensuremath{#1}\text{--}\ensuremath{#2}~paths}
\newcommand{\pths}[2]{\ensuremath{#1}\text{--}\ensuremath{#2}~paths}
\newcommand{\arc}[2]{\ensuremath{#1}\text{--}\ensuremath{#2}~arc}
\newcommand{\arcs}[2]{\ensuremath{#1}\text{--}\ensuremath{#2}~arcs}
\newcommand{\seq}[1]{\ensuremath{(#1_n)_{n\in\N}}} 
\newcommand{\seqi}[1]{\ensuremath{#1_0, #1_1, #1_2, \ldots}}
\newcommand{\sseq}[2]{\ensuremath{(#1_i)_{i\in #2}}} \newcommand{\susq}[2]{\ensuremath{(#1_{#2_i})_{\iin}}} \newcommand{\oseq}[2]{\ensuremath{(#1_\alp)_{\alp< #2}}} \newcommand{\fml}[1]{\ensuremath{\{#1_i\}_{i\in I}}} 
\newcommand{\ffml}[2]{\ensuremath{\{#1_i\}_{i\in #2}}} \newcommand{\ofml}[2]{\ensuremath{\{#1_\alp\}_{\alp< #2}}} \newcommand{\flo}[2]{\ensuremath{#1}\text{--}\ensuremath{#2}~flow} \newcommand{\flos}[2]{\ensuremath{#1}\text{--}\ensuremath{#2}~flows} 

\newcommand{\g}{\ensuremath{G\ }}
\newcommand{\G}{\ensuremath{G}}
\newcommand{\Gn}[1]{\ensuremath{G[#1]_n}}
\newcommand{\Gi}[1]{\ensuremath{G[#1]_i}}

\newcommand{\s}{s}

\newcommand{\half}{\frac{1}{2}}

\newcommand{\ceil}[1]{\ensuremath{\left\lceil #1 \right\rceil}}
\newcommand{\floor}[1]{\ensuremath{\left\lfloor #1 \right\rfloor}}




\newcommand{\ltp}{\ensuremath{|G|_\ell}}
\newcommand{\ltop}{\ensuremath{\ell-TOP}}
\newcommand{\blg}{\ensuremath{\partial^\ell G}}
\newcommand{\lER}{\ensuremath{\ell: E(G) \to \R_{>0}}}
\newcommand{\finl}{\ensuremath{\sum_{e \in E(G)} \ell(e) < \infty}}
\newcommand{\infl}{\ensuremath{\sum_{e \in E(G)} \ell(e) = \infty}}




\newcommand{\OP}{outerplanar}
\newcommand{\Ktt}{\ensuremath{K_{3,3}}}

\newcommand{\wqo}{well-quasi-ordered}


\newcommand{\knl}{Kirchhoff's node law}
\newcommand{\kcl}{Kirchhoff's cycle law}
\newcommand{\cutr}{non-elusive} \newcommand{\Cutr}{Non-elusive} 

\newcommand{\are}{\vec{e}}
\newcommand{\arE}{\vec{E}}

\newcommand{\DP}{Dirichlet problem}
\newcommand{\BM}{Brownian Motion}
\newcommand{\hf}{harmonic function}



\newcommand{\Ex}{\mathbb E}
\renewcommand{\Pr}{\mathbb{P}}



\newcommand{\Cg}{Cayley graph}
\newcommand{\Cc}{Cayley complex}
\newcommand{\pres}[2]{\ensuremath{\left<#1 \mid #2 \right>}}
\newcommand{\vt}{vertex-transitive}
\newcommand{\vtg}{vertex-transitive graph}



\newcommand{\Lr}[1]{Lemma~\ref{#1}}
\newcommand{\Lrs}[1]{Lemmas~\ref{#1}}
\newcommand{\Tr}[1]{Theorem~\ref{#1}}
\newcommand{\Trs}[1]{Theorems~\ref{#1}}
\newcommand{\Sr}[1]{Section~\ref{#1}}
\newcommand{\Srs}[1]{Sections~\ref{#1}}
\newcommand{\Prr}[1]{Pro\-position~\ref{#1}}
\newcommand{\Prb}[1]{Problem~\ref{#1}}
\newcommand{\Cr}[1]{Corollary~\ref{#1}}
\newcommand{\Cnr}[1]{Con\-jecture~\ref{#1}}
\newcommand{\Or}[1]{Observation~\ref{#1}}
\newcommand{\Er}[1]{Example~\ref{#1}}
\newcommand{\Dr}[1]{De\-fi\-nition~\ref{#1}}
\newcommand{\Qr}[1]{Question~\ref{#1}}




\newcommand{\lf}{locally finite}
\newcommand{\lfg}{locally finite graph}
\newcommand{\fg}{finite graph}
\newcommand{\ig}{infinite graph}
\newcommand{\bos}{basic open set}
\newcommand{\nlf}{non-locally-finite}
\newcommand{\nlfg}{non-locally-finite graph}
\newcommand{\fhcy}{Hamilton cycle}
\newcommand{\hcy}{Hamilton circle}
\newcommand{\et}{Euler tour}
\newcommand{\ets}{Euler tours}
\newcommand{\scl}{star-comb lemma}
\newcommand{\vapf}{VAP-free}

\newcommand{\Pof}{Proof of }
\renewcommand{\iff}{if and only if}
\newcommand{\fe}{for every}
\newcommand{\Fe}{For every}
\newcommand{\fea}{for each}
\newcommand{\Fea}{For each}
\newcommand{\st}{such that}
\newcommand{\sot}{so that}
\newcommand{\ti}{there is}
\newcommand{\ta}{there are}
\newcommand{\tho}{there holds}
\newcommand{\wed}{well-defined}
\newcommand{\seth}{small enough that}
\newcommand{\pwd}{pairwise disjoint}

\newcommand{\obda}{without loss of generality}
\newcommand{\btdo}{by the definition of}
\newcommand{\Btdo}{By the definition of}
\newcommand{\btco}{by the construction of}
\newcommand{\Btco}{By the construction of}
\newcommand{\wrt}{with respect to}
\newcommand{\fim}{finitely many}
\newcommand{\inm}{infinitely many}
\newcommand{\istc}{is straightforward to check}
\newcommand{\ises}{is easy to see}
\newcommand{\leth}{large enough that}
\newcommand{\inhy}{induction hypothesis}
\newcommand{\tfae}{the following are equivalent}


\newcommand{\sico}{simply connected}
\newcommand{\tcs}{topological cycle space}
\newcommand{\tocir}{topological circle}
\newcommand{\topa}{topological path}
\newcommand{\tet}{topological Euler tour}

\newcommand{\locon}{locally connected}
\newcommand{\pacon}{path connected}
\newcommand{\pl}{piecewise linear}
\newcommand{\ple}{piecewise linear embedding}
\newcommand{\lfl}{locally flat}

\newcommand{\HM}{Hausdorff measure}
\newcommand{\CtS}{Cantor 3-sphere}

\newcommand{\FC}{Freudenthal compactification}
\newcommand{\HMT}{Hahn-Mazurkiewicz theorem}
\newcommand{\vKT}{van Kampen's theorem}

\newcommand{\pd}{properly discontinuous}

\newcommand{\Rw}{Random walk}
\newcommand{\rw}{random walk}
\newcommand{\RW}{Random Walk}




\newcommand{\labequ}[2]{ \begin{equation} \label{#1} #2 \end{equation} } 
\newcommand{\labsplitequ}[2]{ \begin{equation} \begin{split}
 \label{#1} #2 \end{split} \end{equation} } 
\newcommand{\labtequ}[2]{\begin{equation} \label{#1} 	\begin{minipage}[c]{0.9\textwidth}  #2 \end{minipage} \ignorespacesafterend \end{equation} } 

\newcommand{\labtequc}[2]{ \begin{equation} \label{#1} 	\text{   #2 } \end{equation} } 

\newcommand{\labtequstar}[1]{ \begin{equation*}  	\begin{minipage}[c]{0.9\textwidth}  #1 \end{minipage} \ignorespacesafterend \end{equation*} }



\newcommand{\mymargin}[1]{\ifnum \Debug = 1
  \marginpar{\begin{minipage}{\marginparwidth}\small \begin{flushleft}{\color{blue}#1}\end{flushleft}\end{minipage}}\fi
}

\newcommand{\extras}[1]{\ifnum \Debug = 1
\section{Extras} #1
 \fi
}

\newcommand{\mySection}[2]{}


\newcommand{\RST}{Robertson--Seymour theorem}
\newcommand{\RSTc}{Robertson--Seymour theorem \cite{GMXX}}

\newcommand{\Erd}{Erd\H{o}s}
\newcommand{\DK}{Diestel and K\"uhn}
\newcommand{\KH}{Kirchhoff}
\newcommand{\RD}{Diestel}

\newcommand{\CDB}{\cite[Section~8.5]{diestelBook05}}
\newcommand{\DB}{\cite{diestelBook05}}


\newcommand{\LemArcC}{\cite[p.~208]{ElemTop}}
\newcommand{\LemArc}[1]{
	\begin{lemma}[\LemArcC]
	\label{#1}
	The image of a topological path with endpoints $x,y$ in a Hausdorff space $X$ contains an arc in $X$ between $x$ and $y$.
	\end{lemma}
}

\newcommand{\LemKonig}[1]{
	\begin{lemma}[K\"onig's Infinity Lemma \cite{InfLemma}]
  	\label{#1}
	  Let $V_0,V_1,\dotsc$ be an infinite sequence of disjoint non-empty finite sets, and let $G$ be a graph on their union. Assume that every vertex $v$ in a set $V_n$ with $n\ge 1$ has a neighbour in $V_{n-1}$. Then $G$ contains a ray $v_0v_1\dotsm$ with $v_n\in V_n$ for all $n$.
	\end{lemma}
}

\newcommand{\LemCombStarC}{\cite[Lemma~8.2.2]{diestelBook05}} \newcommand{\LemCombStar}[1]{
	\begin{lemma}[{\LemCombStarC}]
	\label{#1}
	Let $U$ be an infinite set of vertices in a connected graph $G$. Then $G$ contains either a ray $R$ and infinitely many pairwise disjoint \pths{U}{R} or a subdivided star with infinitely many leaves in $U$. 
	\end{lemma}
}

\newcommand{\LemCycDecC}{\cite[Theorem~8.5.8]{diestelBook05}} \newcommand{\LemCycDec}[1]{
	\begin{lemma}[\LemCycDecC]
	\label{#1}
	Every element of $\ccg$ is a disjoint union of circuits.
	\end{lemma}
}

\newcommand{\ThmComp}[1]{
	\begin{theorem}
	\label{#1}
Let $K$ be an infinite set of propositional formulas, every finite subset of which is satisfiable. Then $K$ is satisfiable.
	\end{theorem}
}
\newcommand{\ThmCompC}{\cite{LMCS}} 

\newcommand{\LemCutCrit}[1]{
	\begin{lemma}
	\label{#1}
	Let $F\subseteq E(G)$. Then $F\in \mathcal C(G)$ if and only if $F$ meets every finite cut in an even number of edges. 
	\end{lemma}
}
\newcommand{\LemCutCritC}{\cite[Theorem~8.5.8]{diestelBook05}} 

\newcommand{\LemFinCut}[1]{
	\begin{lemma}
	\label{#1}
	Let $G$ be a connected locally finite graph, $\{ X, Y \}$ a partition of $V(G)$, and $F := E(X, Y )$. 
		\begin{enumerate}
		\item $F$ is finite if and only if $\overline{X} \cap \overline{Y}=\emptyset$.
		\item If $F$ is finite, there is no arc in $|G|\sm \kreis{F}$ with one endpoint in $X$ and the other in $Y$. 
		\end{enumerate}
	\end{lemma}
}
\newcommand{\LemFinCutC}{\cite[Theorem~8.5.5]{diestelBook05}} 


\newcommand{\LemInjHomC}{\cite{armstrong}} \newcommand{\LemInjHom}[1]{
	\begin{lemma}[\LemInjHomC] \label{#1}
	A continuous bijection from a compact space to a Hausdorff space is a homeomorphism.
	\end{lemma}
}

\newcommand{\ThmMacLaneC}{\cite{maclane37}}
\newcommand{\ThmMacLane}[1]{
	\begin{theorem}[\ThmMacLaneC]
	\label{#1}
A finite graph is planar if and only if its cycle space \ccg\ has a simple generating set.
	\end{theorem}
}



\newcommand{\LemHeineCaC}{\cite{}} \newcommand{\LemHeineCa}[1]{
	\begin{lemma}[Heine-Cantor Theorem] \label{#1}
	Let $M$ be a compact metric space, and let $f: M \to N$ be a continuous function, where $N$ is a metric space. Then $f$ is uniformly continuous.	\end{lemma}
}


\newcommand{\ThmHaMaC}{\cite{}} \newcommand{\ThmHaMa}[1]{
	\begin{theorem}[\HMT \cite{nadler}] \label{#1}
A metric space is a continuous image of the unit real interval if and only if it is compact, connected, and locally connected.
\end{theorem}
} 
\newcommand{\andim}{\ensuremath{\mathrm{ANdim}}}

\newcommand{\PP}{P.~Papasoglu}
\newcommand{\forb}[1]{\mathrm{Forb}(#1)}
\newcommand{\ex}[1]{\mathrm{Ex}(#1)}
\newcommand{\bd}{\partial}

\newcommand{\rank}{\mathrm{Rank}}
\newcommand{\diam}{\mathrm{diam}}
\newcommand{\ball}{\mathrm{Ball}}

\newcommand{\mb}{meta-bridge}
\newcommand{\game}{$\gamma$--edge}
\newcommand{\asmi}{asymptotic minor}


\newtheorem{remark}{Remark}
\usepackage{mathtools,enumitem}
\usepackage[percent]{overpic}

\DeclarePairedDelimiter\abs{\lvert}{\rvert}
\DeclarePairedDelimiter\norm{\lVert}{\rVert}
\usepackage{authblk}
\usepackage[greek,english]{babel}  
\usepackage{hyperref}
\RequirePackage{latexsym} \RequirePackage{amsthm}
\RequirePackage{amsmath}

\usepackage{enumerate}
\RequirePackage{amssymb} \RequirePackage{makeidx}
\usepackage{dsfont}
\usepackage{mathrsfs}

\newtheorem{Def}{Definition}

\newcommand{\ncn}{near-connected}
\newcommand{\ncm}{near-component}
\newcommand{\ncnx}[1]{\ensuremath{#1}-near-connected}
\newcommand{\ncmx}[1]{\ensuremath{#1}-near-component}
\newcommand{\pp}[1]{ \ifnum \Debug = 1 {\marginpar{\tiny #1 --pp}} \fi}
\newcommand{\fm}[3]{\ensuremath{#1 \prec_{#2-\mathrm{fat}} #3 }}
\newcommand{\kfm}[2]{\fm{#1}{K}{#2}}
\newcommand{\kfmg}[1]{\kfm{#1}{G}}
\newcommand{\nfm}[3]{\ensuremath{#1 {\not \prec_{#2-\mathrm{fat}}} #3 }}
\newcommand{\nkfm}[2]{\nfm{#1}{K-\mathrm{fat}}{#2}}
\newcommand{\nkfmg}[1]{\nkfm{#1}{G}}
\newcommand{\com}[3]{\ensuremath{#1 \prec_{#2-\mathrm{coarse}} #3 }}
\newcommand{\kcom}[2]{\com{#1}{K}{#2}}
\newcommand{\kcomg}[1]{\kcom{#1}{G}}
\newcommand{\ncom}[3]{\ensuremath{#1 {\not \prec_{#2-\mathrm{coarse}}} #3 }}
\newcommand{\nkcom}[2]{\ncom{#1}{K-\mathrm{coarse}}{#2}}
\newcommand{\nkcomg}[1]{\nkcom{#1}{G}}
\newcommand{\asfm}[2]{\ensuremath{#1 \prec^\infty_{\mathrm{fat}} #2 }}
\newcommand{\ascom}[2]{\ensuremath{#1 \prec^\infty_{\mathrm{coarse}} #2 }}
\newcommand{\asunm}[3]{\ensuremath{#1 \prec^\infty_{#2} #3}}
\newcommand{\asm}[2]{\ensuremath{#1 \prec^\infty #2}}
\newcommand{\nasm}[2]{\ensuremath{#1 \not \prec^\infty #2}}

\newcommand{\asmin}{\mathrm{asmin}}

\newcommand{\asdim}{\mathrm{asdim}}
\newcommand{\fan}{\mathrm{Fan}}

\usepackage{authblk}

\begin{document}

\title{Graph minors and metric spaces}

\author[1]{Agelos Georgakopoulos\thanks{Supported by the European Research Council (ERC) under the European Union's Horizon 2020 research and innovation programme (grant agreement No 639046), and EPSRC grants EP/V048821/1 and EP/V009044/1.}}
\affil[1]{  {Mathematics Institute}\\ {University of Warwick}\\  {CV4 7AL, UK}}
\author[2]{Panos Papasoglu}
\affil[2]{
{Mathematical Institute,
University of Oxford,
Andrew Wiles Building, 
Radcliffe Observatory Quarter, 
Woodstock Road, 
Oxford, 
OX2 6GG, 
UK\\
papazoglou@maths.ox.ac.uk  }}


\date{\today}
\maketitle


\begin{abstract}
We present problems and results that combine graph-minors and coarse geometry. For example, we ask whether every geodesic metric space (or graph) without a fat $H$ minor is quasi-isometric to a graph with no $H$ minor, for an arbitrary finite graph $H$. We answer this affirmatively for a few small $H$. We also present a metric analogue of Menger's theorem and K\"onig's ray theorem. We conjecture metric analogues of the \Erd--P\'osa Theorem and Halin's grid theorem.
\end{abstract}

{\bf{Keywords:} } coarse geometry, quasi-isometry, asymptotic minor, Menger's theorem.\\

{\bf{MSC 2020 Classification:}} 51F30, 05C83, 05C10, 05C63, 20F69.

\maketitle


\section{Introduction}

Gromov's \cite{GroAsyInv} coarse geometry perspective had a groundbreaking impact on geometric group theory, also spreading onto nearby areas. As an example highlight, it has been proved that (any \Cg\ of) every finitely generated group  with finite asymptotic dimension embeds coarsely into Hilbert space \cite{YuNov}, and therefore it satisfies the  Novikov conjecture \cite{GroAsyInv}. 

Several recent papers study the interplay between Gromov's asymptotic dimension and graph-theoretic notions such as graph minors \cite{quasitrees,BBEGLPS,FujPapCoa,FujPapAsy,JorLanGeo,OstExp,OstRosMet}. With these in the background, we present some results and questions with a strong interplay between geometry and graph theory. We hope that this will evolve into a coherent theory that could be called \defi{`Coarse Graph Theory'} or \defi{`Graph-Theoretic Geometry'}.

We would like to view graphs `from far away' to see their large-scale geometry and its implications. More generally, the same idea can be applied to Riemannian manifolds, or more generally, arbitrary length spaces. In what follows $X$ can stand for a length space, or an infinite graph, or a family of finite graphs, endowed with their graph-distance to turn them into metric spaces; none of our statements are affected by this choice, the reader may choose their preferred setting. Our favourite such problem is


\begin{conjecture} \label{conj fat min}
Let $X$ be a graph or a length space, and let $H$ be a finite graph. Then $X$ has no $K$-fat $H$ minor for some $K\in \N$  \iff\ $X$ is $f(K)$-quasi-isometric to a graph with no $H$ minor, where $f: \N\to \N$ depends on $H$ only.\footnote{The backward direction is straightforward.}
\end{conjecture}

A quasi-isometry is a generalisation of a bi-Lipschitz map that allows for an additive error; see \Sr{sec MS}. This notion lies at the heart of Gromov's programme \cite{GroAsyInv}. A  fat minor is a coarse version of the notion of a graph minor  defined as follows.
Recall that $X$ has a (usually finite) graph $H$ as a \defi{minor}, if we can find connected \defi{branch sets} and \defi{branch paths} in $X$, \st\ after contracting each branch set to a point, and each branch path to an edge, we obtain a copy of $H$. We say that $X$ has $H$ as a \defi{$K$-fat minor}, if we can choose the above sets to be pairwise at distance $\geq K$, except that we do not require this for incident branch set--branch edge pairs ---which must be at distance 0 by definition; see also \Dr{def fat} below. This notion of fat minor also features in \cite{BBEGLPS}, and a slight variation features in \cite{CDNRV}. 


If $X$ has a $K$-fat $H$ minor for every $K$, then we say that $X$ has $H$ as an \defi{\asmi}, and write \defi{$\asm{H}{X}$}. An important feature of asymptotic minors is that they are preserved under quasi-isometries, and in particular they are stable under changing the generating set in a \Cg\ of a group (\Or{invariance}).

For example, let $L$ be the square lattice on $\Z^2$, and let $L':= L \times K_2$ be its cartesian product with an edge. Then one can easily show that $L'$ has every finite graph as a minor, but it does not have $K_5$ or $K_{3,3}$ as a 2-fat minor. Thus this notion correctly captures the intuition that $L'$ looks like the plane from a coarse point of view. As another example, the cubic lattice on $\Z^3$ has every finite graph $H$ as an \asmi; to see this, embed $H$ in $\R^3$, fatten it slightly, then zoom $\Z^3$ out while keeping your $H$ fixed. 

\medskip

We know that \Cnr{conj fat min} holds for some very small graphs $H$. If $H=K_2$ (a pair of vertices with an edge), then  \Cnr{conj fat min} becomes the obvious fact that if $X$ has finite diameter then it is quasi-isometric to a point. The case $H=K_3$ is already non trivial and it follows from Manning's characterization of quasi-trees
\cite{Manning}. We give a self contained exposition of this
which entails a shorter proof of Manning's result (\Tr{triangle-free}). Further alternative characterisations of quasi-trees have been recently obtained by Berger \& Seymour \cite{BerSeyBou}. 

Chepoi, Dragan, Newman, Rabinovich \& Vax\`es \cite{CDNRV} settled the case $H=K_{2,3}$ of \Cnr{conj fat min} using slightly different terminology.  The similar case $H=K_4^-$, i.e.\ $K_4$ with an edge removed, has been settled by Fujiwara and the second author \cite{FujPapCoa}, along with a stronger result characterising quasi-cacti.  In \Sr{sec stars} we settle the case where $H$ is a star $K_{1,m}, m\geq 1$. This is, as far as we know, a new result.

Call two graphs $H_1,H_2$ \defi{asymptotically equivalent}, if $\asm{H_1}{X}$ implies $\asm{H_2}{X}$ and vice-versa \fe\ space $X$. In \Sr{sec homeo} we show that if two finite graphs $H_1,H_2$ are homeomorphic as 1-complexes, then they are asymptotically equivalent (the converse is true as well if we allow $X$ to be disconnected). This immediately settles \Cnr{conj fat min} for paths and cycles $H$. Moreover, combined with the aforementioned result of \cite{FujPapCoa}, it implies that a graph is quasi-isometric to a cactus \iff\ it is quasi-isometric to an outerplanar graph (\Cr{cor OP}), a fact previously proved by Chepoi et al.\ \cite{CDNRV}. 

\pp{I think the converse IS true, for graphs with no cut vertices or cut edges, so maybe we should think a bit before stating the question. Also the converse IS NOT true is general eg take O and figure 8.}

In \Cnr{conj fat min} we are mostly interested in the case where $H$ is finite, but we point out some infinite cases of interest: in \Sr{sec Konig} we prove a coarse analogue of K\"onig's theorem, and in \Sr{sec Halin} we discuss coarse analogues of  Halin's grid theorem.
\medskip

We observe below (\Or{obs distortion}) that when $H$ is 2-connected in \Cnr{conj fat min}, then instead of a quasi-isometry we can equivalently ask for a bi-Lipschitz map. Bi-Lipschitz embeddings, aka.\ embeddings of bounded multiplicative distortion, into various targets, most notably tree metrics and normed spaces, are an actively researched topic, see e.g.\ \cite{CDNRV,eriksson-bique,NaorInt,Ostrovskii} and references therein. Embeddings of bounded additive distortion are also of interest \cite{CDNRV}. This raises:
\begin{question}
For which finite graphs $H$ is it true that if a graph $X$ has no $K$-fat $H$ minor for some $K\in \N$, then $X$ admits a map of bounded additive distortion onto a graph with no $H$ minor?
\end{question}
We know that this is the case when $H$ is a cycle \cite{KerTre} (see also claim \eqref{fourteen} in \Sr{quasitree}) or $H=K_{1,3}$. 

\medskip
Ostrovskii \& Rosenthal \cite{OstRosMet} proved that (infinite, locally finite) graphs with a forbidden minor have finite asymptotic dimension. Bonamy et al.\ \cite{BBEGLPS} improve this by proving that such graphs have asymptotic dimension at most 2 (as was conjectured in \cite{FujPapCoa}). They ask whether forbidding a fat minor implies bounded asymptotic dimension. Note that a positive answer to \Cnr{conj fat min} would imply this.  The following question is similar in spirit.

\begin{conjecture}[Coarse Hadwiger conjecture] \label{coa Had}
\Fe\ \nin, if $X$ has no asymptotic $K_{n+1}$ minor, then $X$ has Assouad-Nagata dimension $\andim(X)$ at most $n-1$.
\end{conjecture}

Since $\andim(X)$ can be defined via colouring (\Sr{sec asdim})---and $\andim(X)=n-1$ means that $n$ colours suffice--- \Cnr{coa Had}  can be thought of as the coarse version of the Hadwiger conjecture, perhaps the oldest well-known open problem of graph theory, which asserts that every graph with no $K_{n+1}$ minor is $n$--colourable. If we drop the word `asymptotic' in the  statement of \Cnr{coa Had}, then a better bound  $\andim(X)\leq 2$ was recently obtained by Distel \cite{DisPro} and by Liu \cite{LiuAss}.  In fact 2 might be the right upper bound in \Cnr{coa Had} as well.  A somewhat related result relaxes the Hadwiger conjecture by allowing monochromatic components of bounded size \cite{DEMW}.


Apart from its theoretic interest due to the connection with  the Hadwiger conjecture, \Cnr{coa Had} is also interesting from an algorithmic perspective. Given the theoretical and algorithmic importance of minor-closed graph classes, one hopes that the graphs we encounter in practice come from such a class. This is however rarely the case, because e.g.\ a little bit of randomness will result in arbitrarily large cliques. Thus it is natural to seek extensions of the theory of graph minors to a larger setting. This has been a mainstream trend in computer science, see e.g.\ \cite{GrKrSiDec} and references therein. \Cnr{coa Had}  pursues something similar in a different direction. The kind of practical problems we have in mind are those typically encountered in computational geometry, e.g.\ the Traveling Salesman Problem (TSP). Computer scientists have developed a notion called \defi{sparse partition schemes}, and have shown how these can be used to solve important practical problems such as the Universal Steiner Tree and the Universal TSP problem. The definition of sparse partition schemes is a slightly more quantitative version of the colourings used to define the Assouad-Nagata dimension (\Sr{sec asdim}); see \cite{BBEGLPS} and references therein for the precise relationship and further applications).
Recall that the presence of large cliques in real-world graphs is an obstruction to using algorithms coming from graph minor theory. Our coarse approach overcomes such obstructions: a clique of any size has diameter 1, and so it looks just like a point when seen from far away. 

In the hope to develop more basic tools towards \Cnr{conj fat min} and \Cnr{coa Had}, we considered the following coarse version of Menger's theorem:

\begin{conjecture}[Coarse Menger Theorem] \label{conj MM}
For every $\nin$, there is a constant $C_n$, \st\ the following holds \fe\ graph $G$ and every two subsets  $A,Z$ of its vertex set. \Fe\ $r\in \Z$, either there is a set of $n$ \pths{A}{Z}\ in \g at distance at least  $r$ from each other, 
or there is a set $S$ of less than $n$ vertices of \G, \st\ removing the ball of radius $C_n r$ around $S$ separates $A$ from $Z$ in \G.
\end{conjecture}
\pp{changed the footnote-I wonder if their gap was serious or not, or whether we should mention it-their constant now is better anyway. My initial proof had a gap too, that's why I wonder if they made the same mistake.}
In \Sr{sec Menger} we will prove the special case $n=2$ of this conjecture\footnote{An alternative proof was independently obtained by Albrechtsen et al.\ \cite{AHJKW}, who also state \Cnr{conj MM}. }  We warn the reader that \Cnr{conj MM} seems very difficult even for $n=3$. In \Sr{sec OP} we discuss related problems that might be more accessible.

\comment{
\begin{conjecture} \label{conj MM NPc}
Fix $k\in \N$, and let \g be a graph on $n$ vertices, and   $A,Z$  two subsets of its vertex set. 

there is a constant $C_n$, \st\ the following holds \fe\ graph $G$  \Fe\ $r\in \Z$, either there is a set of $n$ \pths{A}{Z}\ in \g at distance at least  $r$ from each other, 
or there is a set $S$ of less than $n$ vertices of \G, \st\ removing the ball of radius $C_n r$ around $S$ separates $A$ from $Z$ in \G.
\end{conjecture}
}

\section{Preliminaries}



\subsection{Metric spaces} \label{sec MS}

Let $(X,d)$ be a metric space. Given $x\in X$ and $R\in \R_+$, we let $\ball_x(R)$ denote the ball of radius $R$ around $x$.

For $Y,Z\subseteq X$ we define $d(Y,Z):= \inf_{y\in Y, z\in Z} d(y,z)$.


A \defi{length space} is a metric space $(X,d)$ \st\ for every $x,y\in X$ and $\epsilon >0$ there is an $x$--$y$~arc of length at most $d(x,y)+\epsilon$. Thus every geodesic metric space, in particular every graph endowed with its standard metric, is a length space.


An  \defi{$(M,A)$-quasi-isometry} between graphs $G=(V,E)$ and $H=(V',E')$ is a map\\ $f: V \to V'$ \st\ the following hold for fixed constants $M\geq 1, A\geq 0$:
\begin{enumerate}
\item \label{q i} $M^{-1} d(x,y) -A \leq d(f(x),f(y))\leq M d(x,y) +A $ \fe\ $x,y \in V$, and
\item \label{q ii}  \fe\ $z\in V'$ \ti\ $x\in V$ \st\ $d(z,f(x))\leq A$.
\end{enumerate}
Here $d(\cdot,\cdot)$ stands for the graph distance in the corresponding graph $G$ or $H$. We say that $G$ and $H$ are \defi{quasi-isometric}, if such a map $f$ exists. Quasi-isometries between arbitrary metric spaces are defined analogously. The following is well-known, see e.g.\ {\cite[Proposition~3.B.7(6)]{cornulier_metric_2016}}:

\begin{observation} \label{qi graph}
Every length space is quasi-isometric to a connected graph.
\end{observation}
\begin{proof}\textcolor{red}{TOPROVE 2}\end{proof}

When $f: X \to Y$ between metric spaces satisfies \ref{q i} 
with $A=0$, then we say that $f$ has bounded \defi{multiplicative distortion}. When $M=1$, we say that $f$ has bounded \defi{additive distortion}. We observe that in many cases where $Y$ is not fixed, but allowed to vary inside a class of graphs, then a quasi-isometry can be turned into a map of bounded multiplicative distortion: 
\begin{observation} \label{obs distortion}
Let $\ch$ be a set of 2-connected graphs, and let $\cc=\forb{\ch}$ be the class of graphs that do not have any element of \ch\ as a minor. Then a graph $X$ is quasi-isometric to an element of \cc, \iff\ $X$ admits a map of bounded multiplicative distortion into an element of \cc. 
\end{observation}
\begin{proof}\textcolor{red}{TOPROVE 3}\end{proof}

\subsection{Asymptotic dimension} \label{sec asdim}
We recall one of the standard definitions of asymptotic dimension $\asdim(X)$ of a metric space $(X,d)$  from e.g.\ \cite{JorLanGeo}. The reader can think of $X$ as being a graph endowed with its graph distance $d$. See  \cite{BelDraAsy} for a survey of this rich topic.
 \smallskip
 
Let \cu\ be a collection of subsets of $(X,d)$ ---usually a cover. For $s > 0$, we say that \cu\ is \defi{$s$-disjoint}, if
$d(U,U') := \inf\ \{d(x, x') \mid x \in U, x' \in U'\} \geq s$
whenever $U,U'\in \cu$ are distinct. More generally, we say that \cu\ is \defi{$(n+1,s)$-disjoint}, if $\cu = \bigcup_{i=1}^{n+1} \cu_i$ for subcollections $\cu_i$ each of which is $s$-disjoint. The indices $1,\ldots,n+1$ are the \defi{colours} of \cu. We define the \defi{asymptotic dimension} $\asdim(X)$ as the smallest $n$ \st\ \fe\ $s\in \R_+$ \ti\ an $(n+1,s)$-disjoint cover \cu\ of $X$ with $\sup_{U\in \cu} \diam(U)< \infty$. Here, the diameter $\diam(U)$ is measured \wrt\ the metric $d$ of $X$. We say that \cu\ is \defi{$D$-bounded} for any $D> \sup_{U\in \cu} \diam(U)$. 

The \defi{Assouad-Nagata dimension} $\andim(X)$ is defined by putting a stronger restriction on the diameters: we define $\andim(X)$  to be the smallest $n$ for which there is $c\in \R$ \st\ \fe\ $s\in \R_+$ \ti\ an $(n+1,s)$-disjoint and $cs$-bounded cover \cu\ of $X$.


\subsection{Fat minors} \label{sec FM}

\begin{Def}[Fat minors] \label{def fat}
Let $(X,d)$ be a metric space and $\Delta$ a graph. A $K$-\defi{fat} $\Delta$-\defi{minor} of $X$ is a collection of disjoint connected subspaces $B_v, v\in V(\Delta)$ of $X$ (called \defi{branch-sets}), 
and disjoint arcs $P_e,e\in E(\Delta)$ (called \defi{branch-paths}) such that:
\begin{enumerate}
\item \label{fm i} Each  $P_{uw}$ intersects $\bigcup_{v\in V(\Delta)} B_v$ exactly at its two endpoints, which lie in $B_u$ and $B_w$.

\item \label{fm ii} For any $Y,Z \in \{ B_v \mid v\in V(\Delta) \}\cup  \{P_e \mid e\in E(\Delta)\}$, we have $d(Y,Z)\geq K$, unless $Y=Z$, or $Y=P_{vw}$ and $Z=B_v$ for some $v,w\in V(\Delta)$, or vice versa.
\end{enumerate}
\end{Def}

\begin{definition}[Asymptotic minors] \label{def asym}
Let $\Delta$ be a finite graph. We say that $X$ has $\Delta$ as an \defi{asymptotic  minor}, and write $\Delta \prec^\infty X$,  if $X$ has $\Delta $ as a $K$-fat minor  for all $K>0$. \end{definition}

\begin{observation} \label{invariance}
The property of having a fixed finite graph $H$ as an \asmi\ is preserved under quasi-isometries between length spaces.
\end{observation}

In fact, \asmi s are preserved under the more general \defi{coarse embeddings} as defined in \cite{GroAsyInv,Ostrovskii}, but we will not use this fact here.

\comment{
\section{A Coarse Ramsey Theorem}

\begin{lemma} \label{ramsey}
(Coarse Ramsey Theorem)  Let $(X,d)$ be a metric space, and $U\subset X$ infinite. Then \fe\ $R\in \R_+$, there is either\\
A) an infinite $U'\subset U$ \st\ $d(u,v)>R$ \fe\ $u,v\in U'$, or \\
B) a ball of radius $R$ in $X$ containing all but finitely many elements of $U$.
\end{lemma}
\begin{proof}\textcolor{red}{TOPROVE 4}\end{proof}
 
\begin{remark}

\end{remark}
}


\section{Manning's theorem --- quasi-trees} \label{quasitree}

Manning's theorem  \cite{Manning} says that a graph is quasi-isometric to a tree \iff\ it has the following \defi{$\delta$-\defi{Bottleneck Property}} for some $\delta\in \R_+$:

\begin{Def}
We say that a metric space $X$ has the $\delta$-\defi{Bottleneck Property},  if for all $x, y \in X$  there is a `midpoint' $m = m(x,y)$ with $d(x,m) = d(y,m) = d(x,y)/2$ such that any path from $x$ to $y$ is at distance less than $\delta$ from $m$.
\end{Def}

Our next result contains Manning's theorem, and at the same time proves the case $H=K_3$ of \Cnr{conj fat min}:

\begin{theorem} \label{triangle-free}
The following are equivalent for every graph $G$
\begin{enumerate}
\item \label{tf i} $G$ satisfies the $\delta$-Bottleneck Property for some $\delta>0$;
\item \label{tf ii} $G$ has no $K$-fat $K_3$-minor for some $K>0$; and
\item \label{tf iii} $G$ is quasi-isometric to a tree.
\end{enumerate}
\end{theorem}
\noindent {\bf Remark 1:} In this formulation our statement only makes sense for graphs $G$ of infinite diameter, but our proof yields explicit relations between the constants involved that make it meaningful for finite $G$ as well. \\
{\bf Remark 2:} Kr\"on \& M\"oller \cite{KroMolQua} provide an alternative characterisation of quasi-trees (for graphs $G$ of infinite diameter). Kerr \cite{KerTre} provides another proof of Manning's theorem, with multiplicative constant 1 for the quasi-isometry. Similar results had previously been obtained by Chepoi et al.\ \cite{CDNRV}. Our proof of \Tr{triangle-free} in fact also gives Kerr's theorem (with a different proof), as we will obtain a $(1,10K)$-quasi-isometry to a tree \eqref{fourteen}.  In other words, \ref{tf i}--\ref{tf iii} of \Tr{triangle-free} are equivalent to admitting an embedding into a tree with bounded additive distortion. 

\medskip
We will use the following terminology. Let $(X,d)$ be a metric space. We say that $C\subseteq X$ is \defi{$M$-near-connected} for $M\in \R_+$, if for every $x,y\in C$, there is a sequence $x=x_0,x_1, \ldots, x_k=y$ of points of $C$ such that $d(x_i,x_{i+1})\leq M$ \fe\ $i<k$. Such a sequence $P=\{x_i\}$ will be called an $M$-\defi{near path} from $x$ to $y$. An \defi{$M$-\ncm} is a maximal subset of $X$ that is $M$-near-connected. 

Our method below bears some similarity with \cite{CDNRV}, but our use of $M$-\ncm s instead of components in a layered partition is an essential difference. 

\begin{proof}\textcolor{red}{TOPROVE 5}\end{proof}

\comment{
The \defi{$k$-fan} $\fan_k$ is the graph obtained from a path $P$ of $k$ vertices by adding a new vertex and joining it to \mymargin{new paragraph} each vertex of $P$ with an edge. By a straightforward modification of the above proof we will obtain the following fact 
\begin{corollary} \label{corol fan}
Let $G$ be a graph with no $K$-fat $\fan_k$-minor, and $M\in \N_*$. Then there is $D=D(K,k,M)$ such that each $MK$-near-component of any `sphere' $S_n, n\in \N$ (defined as above) has diameter at most $D$.
\end{corollary}
\begin{proof}\textcolor{red}{TOPROVE 6}\end{proof}
}

\section{\Cnr{conj fat min} for stars} \label{sec stars}

In this section we establish the special case of \Cnr{conj fat min} where $H$ is a \defi{star}, i.e.\ $H=K_{1,m}$:

\begin{theorem} \label{thm star}
Let $X$ be a length space, and $m\in \N_{>0}$. Then $X$ has no asymptotic $K_{1,m}$ minor \iff\ $X$ is quasi-isometric to a graph with no $K_{1,m}$ minor.
\end{theorem}
\begin{proof}\textcolor{red}{TOPROVE 7}\end{proof}

\noindent  {\bf Remark:} In the special case $m=3$ of \Tr{thm star}, the resulting graph \g becomes a path or a cycle. In this case, after subdividing each edge of \g $M$ times, the quasi-isometry from $X$ to $G$ ---obtained by mapping each vertex to its super-box--- becomes a map of bounded additive distortion, i.e.\ a $(1,K')$-quasi-isometry. This can be proved similarly to \eqref{fourteen}. We do not see how to extend this to the case $m>3$.

\section{More on \Cnr{conj fat min}} \label{sec homeo}

Despite our \Cnr{conj fat min}, we remark that the correspondence between minor-closed families of graphs and quasi-isometry classes is not 1--1. \Tr{thm subdiv} below makes this point clear, but it also provides a tool for studying \Cnr{conj fat min}. Here we think of graphs as topological spaces, by considering the corresponding  1-complex.

\begin{theorem} \label{thm subdiv}
Let $\Gamma,\Delta$ be finite graphs that are homeomorphic as topological spaces, and let $X$ be a length space. Then  \asm{\Gamma}{X} \iff\ \asm{\Delta}{X}. 
\end{theorem}

 
Let  $\Theta$ be the graph obtained from $K_4$ by removing one edge (or from $K_{2,3}$ by contracting one). Recall that outerplanar graphs can be characterised as $\mathrm{Forb}(K_4, K_{2,3})$. Combining this with  \Tr{thm subdiv}, and a result of Fujiwara \& Papasoglou \cite{FujPapCoa} stating that a graph is quasi-isometric to a cactus \iff\ it has no asymptotic $\Theta$ minor, we obtain

\begin{corollary} \label{cor OP}
The following are equivalent for a length space $X$:
\begin{enumerate}
         \item \label{OP ii} $X$ is quasi-isometric to a cactus; 
         \item \label{OP i} $X$ is quasi-isometric to an outerplanar graph;
      \item \label{OP iii} $X$ has no asymptotic $K_4$ or  $K_{2,3}$ minor, and
      \item \label{OP iv} \ti\ a subdivision $\Theta'$ of $\Theta$ \st\ $X$ has no asymptotic $\Theta'$ minor.
\end{enumerate}
\end{corollary}
Here, we say that a graph $\Delta'$ is a \defi{subdivision} of a graph $\Delta$, if $\Delta'$ is obtained by replacing some of the edges of $\Delta$ by independent paths with the same end-vertices.

\begin{proof}\textcolor{red}{TOPROVE 8}\end{proof}



Note that two graphs are homeomorphic (as topological spaces) \iff\ they are subdivisions of the same graph (which can be obtained by suppressing all vertices of degree 2). Therefore \Tr{thm subdiv} follows by applying the following lemma twice:

\begin{lemma} \label{lem subdiv}
Let $\Delta'$ be a subdivision of a finite graph $\Delta$, and let $X$ be a length space. Then \asm{\Delta}{X} \iff\ \asm{\Delta'}{X}.
\end{lemma}
\begin{proof}\textcolor{red}{TOPROVE 9}\end{proof}

The converse of \Tr{thm subdiv} is false in general: for example, we can let $\Gamma$ be $K_2$, and $\Delta$ be its complement. But if we insist that $\Gamma,\Delta$ are both connected, and allow $X$ to be disconnected, i.e.\ we allow for infinite distances between pairs of points of $X$, then the converse becomes true and not hard to prove\footnote{For this, given non-homeomorphic $\Gamma_1,\Gamma_2$, let $X_i, i=1,2$ be the disjoint union of \oo\ copies of $\Gamma_i$, with each edge of the $n$th copy subdivided $n$ times. Then  $\asm{\Gamma_i}{X_i}$, but $\nasm{\Gamma_i}{X_{3-i}}$.}. If we insist that $X$ be connected however, then the converse fails: if $\Gamma$ is a cycle, and $\Delta$ the union of two cycles with exactly one common vertex, then $\Gamma,\Delta$ are asymptotically equivalent though non-homeomorphic. 
Still, we can obtain a converse to \Tr{thm subdiv} (for connected $X$) by restricting to 2-connected graphs. We leave the details to the interested reader.





\section{Coarse K\"onig Theorem} \label{sec Konig}
K\"onig's classical theorem states that every infinite connected graph contains a countably infinite star or a ray as a subgraph \cite[Proposition 8.2.1.]{DiestelBook05}. It is a cornerstone of infinite graph theory as it facilitates compactness arguments. In this section we prove a metric analogue. 
\smallskip

Let $P_n$ denote the path (in the graph theoretic sense) of length $n$. Let $\bigvee P_n$ be the graph obtained from the disjoint union of \seq{P}\ by identifying the first vertex of each $P_n$. A \defi{ray} is a 1-way infinite path. 


\begin{theorem} \label{thm Konig}
(Coarse K\"onig Theorem) Let $X$ be a (connected) length space of infinite diameter. Then $X$ has either the ray or $\bigvee P_n$ as an asymptotic minor.
\end{theorem}

We prepare the proof of this with a couple of lemmas of independent interest. A ray $P$ in a graph \g is called \defi{divergent}, if $P\cap S$ is finite \fe\ subset $S$ of $V(G)$ with bounded diameter. 

\begin{lemma} \label{lem ray}
Let $G$ be a graph that has a divergent ray. Then $G$ has  the ray as an asymptotic minor.
\end{lemma}
\begin{proof}\textcolor{red}{TOPROVE 10}\end{proof}

By the Infinite Ramsey theorem, every infinite graph contains either an infinite clique or an infinite independent set. We will use the following corollary of this.
\begin{lemma} \label{Ramsey}
Let $G_1 \supseteq G_2 \supseteq G_3 \ldots$ be graphs with a common infinite vertex set $V$. Then either some $G_i$ contains an infinite independent set, or there are infinite subsets $S_1 \supseteq S_2 \supseteq S_3 \ldots$ of $V$ \st\ $G_i[S_i]$ is complete \fe\ $i$.
\end{lemma}
\begin{proof}\textcolor{red}{TOPROVE 11}\end{proof}

We can now prove the main result of this section.
\begin{proof}\textcolor{red}{TOPROVE 12}\end{proof}



\comment{For convenience we will assume $X$ to be a graph, which we can because every length space is quasi-isometric to a graph (\Or{qi graph}). 

Since $X$ has  infinite diameter, there is a sequence of geodesics $p_n:[0,n]\to X$ parametrised by arc length with $p_n(0)=a\in X$ a fixed point.
Call this set of geodesic paths $S$.
Fix $R>0, R\in \mathbb N$. We will show that $X$ contains either an $R$-fat $\bigvee P_n$
or an $R$-fat ray.

Consider the following property that we call $G(n)$: There is an infinite subset of $S$, say $S_n$, such that
ifor any $p_1,p_2\in S_n$ then $d(p_1(t_1),p_2(t_2))\leq R$ implies that $t_i\leq n$ for at least one $i (i=1,2)$. Note that
if $d(p_1(t_1),p_2(t_2))\leq R$ then $|t_2-t_1|\leq R$ as $p_1,p_2$ are geodesics.

Clearly if $G(n)$ holds for some $n\geq R$ then there is a subset of $S_n$
which is a $\bigvee P_n$ $R$-fat minor. Indeed it suffices to take a sequence $q_k\in S_n$
such that the length of $q_k$ is greater, say, than $2kR+n$. We set 
$$B_{v_0}=\bigcup q_k([0,n])$$
$B_{v_{ij}}=q_i(jR+n)$ for $j\geq 1, j\in \mathbb N$ to be the branch sets, where $j<i$. For $j=i$,  if $\ell=length(q_i)$ we set the branch set
$B_{v_{ii}}$ to be
$$B_{v_{ii}}=q_i([iR+n,\ell]).$$
We take $P_{vw}$  to be the obvious 
paths in the geodesics $q_k$ joining `successive' branch sets $B_v$, $B_w$. This defines an $\bigvee P_n$ $R$-fat minor. 

 
So we may assume that $G(n)$
does not hold for any $n$.

Let $S'\subseteq S$ be infinite. We say that a path $p\in S'$ is a $bad(n)$ path if there are infinitely many paths $q\in S'$
such that $d(p(t),q(s))\leq R$ for some $t\geq n, s\in \mathbb N$.

Assume that for some $n$ there are no $bad(n)$ paths in some infinite subset of $S$, say $S'$. Pick then a path $q_1$ in $S'$ and take out from $S'$ the finite set
of paths $q$ such that $d(q_1(t),q(s))\leq R$ for some $t\geq n$ to obtain a set $A_1\subseteq S$. Now repeat, take $q_2\in A_1$
and eliminate from $A_1$ all paths $q$ such that $d(q_2(t),q(s))\leq R$ for some $t\geq n$ to obtain
a new infinite set of paths $A_2$.
 And so on. This produces an infinite set of paths $\{q_1,q_2,...\}$
that satisfies $G(n)$, a contradiction.  So there are $bad(n)$ paths in $S'$. In particular there are $bad(n)$ paths in $S$ for any $n$.

Let $q_1$ be $bad(n)$ in $S$ for $n=10R$ and let $C_1$ be the set of all paths $q$ such that $d(q_1(t),q(s))\leq R$ 
for some $t\geq n$. Since $q_1$ is a finite path there is some $k_1\in \mathbb N, k_1\geq n$ and an infinite
subset of $C_1$ (which we still denote $C_1$ to keep notation simple) so that for each $q\in C_1$
there is some $s\in \mathbb N$ for which
$d(q_1(k_1),q(s))\leq R$. Since both $q_1,q$ are geodesic paths $|s-k_1|\leq R$, hence
$$d(q_1(k_1),q(k_1))\leq 2R.$$

Since $C_1$ is an infinite subset of $S$, it contains a $bad(10k_1)$ path $q_2$. We repeat
and we obtain as before $k_2\in \mathbb N, k_2\geq 10k_1$ and an infinite subset $C_2$ of $C_1$ so that
so that for each $q\in C_2$ there is some $s$ so that
$d(q_2(k_2),q(s))\leq R$. As before for all $q\in C_2$ we have
$$d(q_2(k_2),q(k_2))\leq 2R.$$

We continue inductively constructing infinite subsets $S=C_0\supset C_1\supset C_2 \ldots$
and paths $q_1,q_2,\ldots$ such that $q_i\in C_{i-1}$  with the following property:
$d(q_i(k_i),q_j(k_i))\leq 2R$ for all $j>i$ where the sequence $\{k_i\}$ satisfies $k_{i+1}\geq 10k_i$, $k_1\geq 10R$.

We use this sequence of paths to construct an $R$-fat ray $r$ as follows. For $i\geq 1$ we pick a geodesic path $r_i$ joining $q_i(k_i)$ to $q_{i+1}(k_{i})$. We define the branch set $B_{v_i}$ to be the set 

$$q_i([k_i-2R,k_i])\cup r_i\cup q_{i+1}([k_{i},k_{i}+2R])$$

For $i=0$ we set $B_{v_0}=q_1(0)$

We define $P_{e_i}$ for $i\geq 1$ to be the arc
$$q_{i+1}([k_{i}+2R,k_{i+1}-2R]).$$

For $i=0$ we define $P_{e_0}=q_1([0,k_1-2R ])$.
It is clear then that by construction the set
$$r=\bigcup B_{v_i} \bigcup P_{e_i}$$ is an $R$-fat ray.
} 

\comment{
A potential approach towards \Cnr{Cnr Konig} is to take, \fe\ $K\in \N$, a $K$-net $G_n$ in $X$, and apply the following to $G_n$:
\begin{lemma} \label{lem}
Let $G$ be a (connected) graph of infinite diameter. Then $G$ contains the ray or $\bigvee P_n$ as a subgraph.
\end{lemma}
\begin{proof}\textcolor{red}{TOPROVE 13}\end{proof}

But I cannot yet use \Lr{lem} to prove \Cnr{Cnr Konig}.
}

\section{Coarse Halin Theorems} \label{sec Halin}

For a cardinal $k$, and a graph $H$, let $k \cdot H$ denote the disjoint union of $k$ copies of $H$. Let $R$ denote the ray, i.e.\ the 1-way infinite path. Halin \cite{halin65} proved that \fe\ graph $G$, if $n\cdot R \subset G$ holds \fe\ $\nin$, then $\omega \cdot R \subset G$. Does the coarse version hold?

\begin{conjecture} \label{Cnr HRT}
(Coarse Halin Ray Theorem) Let $X$ be a length space. Suppose $n\cdot R \prec^\infty X$ holds \fe\ \nin. Then $\omega \cdot R \prec^\infty X$.
\end{conjecture}

Let $\mathcal{H}$ denote the `half' square grid on $\N \times \N$, and let $\mathcal{F}$  denote the `full' square grid on $\Z \times \Z$. Halin's grid theorem \cite{halin65} says that every 1-ended graph satisfying $n\cdot R \subset G$ \fe\ \nin\ contains a subdivision of $\mathcal{H}$.

\begin{problem}
(Coarse Halin Grid Theorem) Let $X$ be an 1-ended length space. Suppose $\omega \cdot R \prec^\infty X$. Then $\mathcal{H} \prec^\infty X$.
\end{problem}

\begin{problem}
Let $G$ be a \Cg\ of an 1-ended finitely generated group. Must $\mathcal{H} \prec^\infty G$ hold? Must $\mathcal{F} \prec^\infty G$ hold?
\end{problem}


\section{Coarse Menger Theorem} \label{sec Menger}

In this section we prove the case $n=2$ of \Cnr{conj MM}. In \Sr{sec div rays} below we present a corollary for diverging rays in infinite graphs. We restate our special case for convenience: 

\begin{theorem} \label{Menger}
Let $X$ be a graph or a geodesic metric space, and $A,Z$ two subsets of $X$. For every $K>0$, there is either
\begin{enumerate}
\item \label{S} a set $S \subset X$ with $\diam(S)\leq K$ such that $X \setminus S$ contains no path joining $A$ to $Z$, or 
\item \label{paths} two paths joining $A$ to $Z$ at distance at least $a=K/16\cdot 17=K/272$ from each other.
\end{enumerate}
Moreover, \fe\ $x\in A, z\in Z$, we can choose the paths of \ref{paths} so that both $x,z$ are among their four endpoints.

\end{theorem}




We prepare the proof with some preliminaries. 
We will assume that \ref{S} does not hold and we will show that \ref{paths} holds.


We fix throughout the proof a shortest path $\gamma :[0,\ell]\to X $ joining $A$ to $Z$.
 
\begin{definition}[Bridges] \label{bridge}
A \textit{bridge} $B$ is a union of three successive paths $c_1\cup b\cup c_2$ 
such that:

\begin{itemize}
\item for any $x\in b, y\in \gamma $, we have  $d(x,y)\geq K/8$.

\item If $b\cap A\ne \emptyset $ then $c_1$ is an endpoint of $b$ and similarly if
$b\cap Z\ne \emptyset $ then $c_2$ is an endpoint of $b$; otherwise
$c_1,c_2$ are (geodetic) paths of length $K/8$ joining the endpoints
of $b$ to $\gamma $. 

\end{itemize}

\end{definition}

We call $c_1,c_2$ the \defi{legs} of $B$, and $b$ its \defi{spine}. The two endpoints of $B$ on $\gamma$, if they exist, are denoted by $B^0,B^1$ so that $B^0< B^1$. The \defi{interval} of $B$ is the interval $[B^0,B^1]$ of $\gamma$. If $c_1$ is a point (of $A$) then the interval of $B$ is $[\gamma (0),v]$ where $v$ is the
endpoint of $c_2$, and we define similarly the interval of $B$ when $c_2$ is a point of $Z$.

We say that a bridge is \textit{maximal} if its interval is not contained properly in the interval of another bridge.	
 
\begin{proposition} \label{max}
Let $B$ be a maximal bridge with spine $b$. For any point $x\in b$, let $x'$ be a point of $\gamma$ minimizing $d(x,x')$. Then $x'\in [B^0,B^1]$.
\end{proposition}
\begin{proof}\textcolor{red}{TOPROVE 14}\end{proof}

We assume that $\gamma $ is parametrized by arc-length. If $a\in \gamma $ and $r>0$, we denote by 
$a-r$  the point at distance $r$ from $a$ on $\gamma $ which is before $a$. We define $a+r$ similarly.

\begin{definition}[Surrounding] \label{surround}
We say that a bridge $B$ $d$-\textit{surrounds} a point $x=\gamma (t)\in \gamma $ if the interval $[t_1,t_2]$ of $B$
contains $x-d$ and $x+d$. Here if $t<d$ or $t>\ell -d$ we set $x-d=\gamma (0)$ and $x+d=\gamma (\ell)$ respectively.
\end{definition}

We restrict our attention now to maximal bridges.


\begin{definition}[Order] \label{order}
Let  $B_1, B_2$ be maximal bridges. If $B_2^0 > B_1^0$ (in which case $B_2^1 > B_1^1$ holds too by maximality), we write $B_2>B_1$. 
\end{definition}


For $R\in \R$, we say that $B_2$ \defi{$R$-crosses}  $B_1$ if $B_2$ $R$-surrounds $B_1^1$. 


\begin{lemma} \label{bridge}
Assume condition \ref{S} of \Tr{Menger} fails. Then for any $x\in \gamma $ there is some bridge $B$ that $3K/8$-surrounds $x$. 
\end{lemma}
\begin{proof}\textcolor{red}{TOPROVE 15}\end{proof}

\begin{lemma} \label{disjoint}

Let $B<C$ be maximal bridges with $C^0> B^1$.
Assume that there are $x\in B, y\in C$ such that $d(x,y)\leq K/4$. Then  $d(B^1,C^0)< 3K/4$.  
\end{lemma}
\begin{proof}\textcolor{red}{TOPROVE 16}\end{proof}

We choose a smallest set $\mathcal U$ of maximal bridges such that any $x\in \gamma $ is $K/4$-surrounded by some 
bridge in $\mathcal U$, and 
\labtequ{length}{every $B\in \cu$ has an interval of length at least $3K/4$.} Such a \cu\ exists by \Lr{bridge}. We remark that $\mathcal U$ is finite because $\gamma$ is compact, and any bridge that $3K/4$-surrounds $x\in \gamma $ also $K/4$-surrounds any point in an open interval of $x$ on $\gamma$. 


The minimality of $\mathcal U$ implies that the initial points of its bridges are not too dense on $\gamma$:


\begin{proposition} \label{U i}  
Let $B,C,D$ be bridges in $\mathcal U$. Suppose that
		 $C$ $r$-crosses $B$, and $D$ $r$-crosses $C$ for some $r\leq K/4$. Then $d(B^0,D^0)\geq r$. 
\end{proposition}
\begin{proof}\textcolor{red}{TOPROVE 17}\end{proof}

A sequence $\cb= B_1, \ldots B_n$ of bridges is said to be \defi{$R$-crossing} for some $R\in \R$, if $B_1$ $R$-surrounds $\gamma(0)$, each $B_i, i\geq 2$ $R$-crosses $B_{i-1}$, and $B^1_n\in Z$. Note that $B_n$ must $R$-surround $\gamma(1)$ as $\gamma(1)> B^1_{n-1}$.
Such a sequence is \defi{perfect}, if in addition its initial points are $R$-apart, i.e.\ $d(B_i^0,B_j^0)\geq R$ holds \fe\ $1\leq i,j \leq n$.

\smallskip
\noindent {\it {\bf Remark:} A perfect $R$-crossing sequence \cb\ remains a perfect $R$-crossing sequence if we exchange the roles of $A$ and $Z$ and reverse the direction of $\gamma$ and the ordering of \cb. But we will not use this fact.}

\smallskip
It is straightforward to check that if we order the elements of \cu\ by $<$ then the resulting sequence $S$ is a $K/4$-crossing sequence. Our next lemma says that we can obtain a perfect subsequence by sacrificing half of our crossing constant.
\begin{lemma} \label{initials} 
Let $S= \{B_i\}_{1\leq i \leq n}$ be a $K/4$-crossing sequence of bridges.  Then there is a perfect $K/8$-crossing subsequence $S'$ of $S$. \end{lemma}
\begin{proof}\textcolor{red}{TOPROVE 18}\end{proof}


\begin{proposition} \label{U ii}   
Suppose that $B_1,B_2,B_3,B_4,B_5$ are bridges such that each $B_i, i\geq 2$, $K/8$-crosses $B_{i-1}$, and $B_i^0$ lies in the interval of $B_1$ for every $2\leq i \leq 5$.  Then at least one $B_i$ does not lie in $\mathcal U$.
\end{proposition}
\begin{proof}\textcolor{red}{TOPROVE 19}\end{proof}

\begin{lemma} \label{U iii} Suppose that $B_i:i=1,\ldots,n$ are bridges in $\mathcal U$ such that $B_i$ crosses $B_{i-1}$ for $i=2,\ldots,n$, and that 
		there are $x\in B_1, y\in B_n$ such that $d(x,y)<K/4$.  Then $n\leq 8$.
\end{lemma}
\begin{proof}\textcolor{red}{TOPROVE 20}\end{proof}


Our plan is to construct the two paths $\alpha_1,\alpha_2$ joining $A$ to $Z$ required by \Tr{Menger} by combining subpaths of $\gamma$ with bridges from a sequence $S'$ as in \Lr{initials}. However, we can have bridges $B,C$ in $S'$ arbitrarily close to each other (but not intersecting, as this would contradict maximality), in which case we have to avoid putting $B$ in $\alpha_1$ and $C$ in $\alpha_2$. Therefore, if $d(B,C)<a$, we will use a \pth{B}{C} $\alpha$ of length less than $a$ to join $B,C$, and produce a new path called a \defi{\mb}. We may need to iterate the process, joining two nearby \mb s into a new \mb. But the following lemma will exploit \Lr{U iii} to upper bound the complexity of \mb s. 

Formally, we can simply define a \defi{\mb} $B$ to be a path
with endpoints in $\gamma \cup A \cup Z$ whose interior is disjoint from $\gamma \cup A \cup Z$. But the \mb s we will actually work with will obey further restrictions, being constructed recursively starting from bridges as explained above. Much of the notation we introduced for bridges can be applied verbatim to \mb s, and we will do so without further comment.

A \defi{join} of a \mb\ $D$ is a maximal subpath of $D$ consisting of points that lie on no bridge of $S'$. 

Choose $a=  K/(16\cdot 17)$. 
\begin{lemma} \label{meta} 
There is a perfect, $K/8$-crossing, sequence $M$ of \mb s \st\
\begin{enumerate}
\item \label{m ii} each join $J$ of an element of $M$ has length $\ell(J) < 15a$; 
\item \label{m iii} we have $d(J, \gamma)\geq a$ for each join $J$ of an element of $M$, and
\item \label{m i} if $x,y$ are points of distinct \mb s $B,C$ of $M$, then $d(x,y)\geq a$ unless there are endpoints $B^i,C^{1-i}, i\in \{0,1\}$ with $d(B^i,C^{1-i})<K/8$. \end{enumerate}
\end{lemma}
\begin{proof}\textcolor{red}{TOPROVE 21}\end{proof}

\noindent {\bf Remark:} The last inequality is the bottleneck for maximizing the distance $a$ between the two paths $\alpha_i$ of \Tr{Menger}.

\bigskip

\comment{
\begin{proposition} \label{}  
Let $x,y$ be points of meta-bridges $X,Y$ \st\ $d(x,\gamma), d(y,\gamma)\leq r$. Then $d(x,y)\geq a$ unless there are endpoints $x',y'$ of $X,Y$ \st\ $d(x',y')\leq ...$
\end{proposition}
\begin{proof}\textcolor{red}{TOPROVE 22}\end{proof}
}

We have collected all the necessary ingredients for the proof of our \Tr{Menger}.

\begin{proof}\textcolor{red}{TOPROVE 23}\end{proof}





\subsection{Alternative construction of the paths --- reducing to the classical Menger theorem}

This section offers an alternative proof of \Tr{Menger}, which is simpler but gives a weaker lower bound $a$ on $d(\alpha_1,\alpha_2)$. The main difference is that instead of constructing $\alpha_1,\alpha_2$ explicitly, we apply Menger's theorem to an auxiliary graph, consisting of \game s and \mb s in $M$.

For this, we first adapt \ref{m i} of \Lr{meta} by lowering the constant $K/8$ to $K/20$, say, and lowering $a$ proportionately; the proof can be repeated verbatim after this change. This change allows us to deduce the following corollary of \Lr{meta}: 
\begin{corollary} \label{cor meta}
If $B,C$ are \mb s in $M$ or \game s, then $d(B,C)\geq a$ unless $B,C$ are incident or as in \ref{m i} of \Lr{meta} (i.e.\ they have $K/20$-close end-points).
\end{corollary}
\begin{proof}\textcolor{red}{TOPROVE 24}\end{proof}

\begin{proof}\textcolor{red}{TOPROVE 25}\end{proof}



\subsection{Choosing endpoints}
We remark that $\gamma(0),\gamma(1)$ are among the four endpoints of $\alpha_1, y\in \alpha _2$ in the above construction. This can be seen directly from the construction, or by noticing that $\cu$, and hence $S'$ and $M$, contains at most one bridge with an endpoint in each of $A$ and $Z$ by maximality. Recall that $\gamma$ was chosen as an arbitrary shortest \pth{A}{Z}. By modifying $X$ we can strengthen the above conclusion:

\begin{corollary} \label{endpoints}
For every $x\in A$ and $z\in Z$, we can choose the paths of \Tr{Menger} so that both $x,z$ are among their four endpoints.
\end{corollary}
\begin{proof}\textcolor{red}{TOPROVE 26}\end{proof}

\subsection{An application --- diverging rays} \label{sec div rays}

It is not hard to extend \Tr{Menger} to the situation where $Z$ is replaced by $\infty$; that is, we have

\begin{corollary} \label{cor inf Menger}
Let $X$ be a \lf\ metric graph, and $A \subset X$. For every $K>0$, there is either
\begin{enumerate}
\item \label{S} a set $S \subset X$ with $\diam(S)\leq K$ such that $X \setminus S$ contains no path of infinite length starting at $A$, or 
\item \label{paths} two paths of infinite length starting at $A$ at distance at least $a=K/16\cdot 17$ from each other.
\end{enumerate}
\end{corollary}
\begin{proof}\textcolor{red}{TOPROVE 27}\end{proof}

\noindent {\bf Remark:} the local-finiteness condition is necessary in \Cr{cor inf Menger}. \mymargin{added this remark} A non-locally-finite counterexample can be obtained as follows. Let $R=x_0x_1\ldots$ be  a ray, and for each $i>1$, join $x_0$ to $x_i$ by a path $P_i$ of length $i$, so that $P_i\cap P_j = \{x_0\}$ \fe\ $i\neq j$, to obtain a graph $X$ (in which $x_0$ is the only vertex of infinite degree, see \fig{figNLF}). Let $A$ be any infinite set of vertices of $X$. Easily, no set of finite diameter separates $A$ from infinity in $X$, i.e.\ \ref{S} fails. Moreover, if $X$ contains two infinite paths, then so does $X':=X-x_0$. But $X'$ is an 1-ended tree. Thus \ref{paths} fails as well.

\begin{figure} 
\begin{center}
\begin{overpic}[width=.45\linewidth]{figNLF2.pdf} 
\put(-2,28){$x_0$}
\put(19,28){$x_1$}
\put(60,5){$P_4$}

\end{overpic}
\end{center}
\caption{A non-locally-finite graph failing the conclusion of \Cr{cor inf Menger}.} \label{figNLF}
\end{figure}

\medskip
We say that two rays $R,L$ in a graph \g  \defi{diverge}, if
\fe\ $n\in \N$ they have tails $R' \subseteq R, L' \subseteq L$ satisfying $d(R',L')>n$.

\begin{theorem} \label{thm div rays}
Let \g be a graph with bounded degrees which has an infinite set of pairwise disjoint rays. Then \g has a pair of diverging rays. 
\end{theorem}

The idea is to apply \Cr{cor inf Menger}, letting the scale $K$ increase as we progress towards infinity. Instead of increasing $K$, we will use the equivalent but more convenient method of scaling down the edge-lengths as the distance from a fixed origin tends to infinity. This is the role of the following lemma:

\begin{lemma} \label{lem balls}
Let $G=(V,E)$ be a graph with bounded degrees which has an infinite set of pairwise disjoint rays. Then \ti\ a finite vertex-set $A\subset V$ and an assignment of edge-lengths $\ell : E \to \R_{>0}$ with the following properties: 
\begin{enumerate}
\item \label{non sep} no ball of radius $1$ in the corresponding metric $d_\ell$ separates $A$ from $\infty$, and
\item \label{to zero} $\lim_{e\in E} \ell(e) = 0$.\footnote{This means that \fe\ $\epsilon>0$ all but finitely many $e\in E$ satisfy $\ell(e)<\epsilon$.}
\end{enumerate}
\end{lemma}

Let us first see how this implies \Tr{thm div rays}.

\begin{proof}\textcolor{red}{TOPROVE 28}\end{proof}

It remains to prove \Lr{lem balls}.

\begin{proof}\textcolor{red}{TOPROVE 29}\end{proof}

\section{Further problems} \label{sec OP}
\pp{added questions and made comments in the next couple of paragraphs}
As we saw in the introduction, \Cnr{conj fat min} can be very difficult even for a simple graph $H$. The following special case is of particular interest (\Cnr{conj fat min} can be formulated with $H$ replaced by a finite set of finite graphs):
\begin{conjecture}[Coarse Kuratowski--Wagner theorem] \label{Kur}
A length space $X$ is quasi-isometric to a planar graph \iff\ it has no asymptotic $K_5$ or $K_{3,3}$ minor. 
\end{conjecture}
Here, and more generally in \Cnr{conj fat min}, we may have to allow the resulting planar graph to be edge-weighted; we do not know whether every planar graph with arbitrary real-valued edge-lengths (used to induce a length space) is quasi-isometric to a planar graph with all edge-lengths equal to 1. 

This special case is well-motivated in itself due to widespread interest in planar graph metrics, and due to the interest in groups acting on planar surfaces \cite{Kleinian}. 

Another interesting class of graphs is that of graphs of finite tree width.

\begin{conjecture}[Coarse grid theorem] \label{con grid}
A length space $X$ is quasi-isometric to a graph of finite tree width \iff\ it does not have the $n\times n$  grid as an   asymptotic minor for some \nin. 
\end{conjecture}
One can also ask whether under the same hypothesis $asdim (X)\leq 1$ (as this is weaker). 

Our \Cnr{conj fat min}  would imply that the fundamental Robertson-Seymour graph minor theory has a coarse geometric generalisation. One can of course formulate weaker versions of our conjecture in the same direction. For example, one could ask whether excluding an infinite family of asymptotic minors is equivalent to excluding a finite such family. In geometry and topology one often asks for the `nicest' metric among all metrics in a homeomorphism class. This is the classical topic of uniformisation of surfaces,
and this point of view was crucial in the recent classification of 3-manifolds after Thurston and Perelman.  \Cnr{conj fat min}  can be seen as a coarse analog to
uniformisation: we ask for a `nice' representative in a quasi-isometry class of graphs.

\medskip

Our next conjecture is a coarse version of a result of Thomassen  \cite{ThoHad} saying that an 1-ended vertex transitive graph is either planar or has an infinite clique as a minor. 

\begin{conjecture} \label{Tho}
Let $G$ be a locally finite, vertex-transitive graph. Then either $G$ is quasi-isometric to a planar graph, or it contains every finite graph as an asymptotic minor.
\end{conjecture}

Recall that a graph \g is \defi{$k$-planar}, if it can be drawn in $\R^2$ so that each of its edges crosses at most $k$ other edges.
\begin{problem} \label{k planar}
Let $G$ be a locally finite, vertex-transitive graph. Is it true that $G$ is quasi-isometric to a planar graph \iff\ it is $k$-planar for some $k\in \N$?
\end{problem}
Vertex-transitivity is essential here, for otherwise we could let \g be a sequence of subdivisions of $K_5$.

Our next problem is motivated by the following theorem of Khukhro: 
\begin{theorem}[\cite{KhuCha}] \label{Khukhro}
A finitely generated, infinite group $\Gamma$ is virtually free, if and only if for every finitely generated Cayley graph \g of $\Gamma$, there is some finite graph that is not a minor of \G.
\end{theorem}

\begin{conjecture} \label{virt planar}
A finitely generated group $\Gamma$ is virtually planar, if and only if for every finitely generated Cayley graph \g of $\Gamma$, there is some finite graph that is not an asymptotic minor of \G. 
\end{conjecture}
\defi{Virtually planar} here means that $\Gamma$ has a finite index subgroup $H$ which has a planar \Cg.  The forward implication of this is easy: by the Svarc--Milnor lemma, any finitely generated \Cg s of  $H$ and $\Gamma$
 are quasi-isometric, and so the statement follows from \Or{invariance}. (The analogous implication of \Tr{Khukhro} was proved `by hand' by Ostrovskii \& Rosenthal \cite{OstRosMet}.)

\medskip
As already mentioned, we expect \Cnr{conj MM} to be very difficult for $n\geq 3$. The following is a related algorithmic question. Let MM3 denote the algorithmic problem that takes as input a graph $G$ on $n$ vertices and two subsets  $A,Z$ of its vertex set, and asks whether there is a triple of \pths{A}{Z}\ that are pairwise at distance at least 3. 
\begin{conjecture} \label{conj MM NPc}
MM3 is NP-complete.
\end{conjecture}



A classical result of \Erd\ \& P\'osa \cite{ErdPosInd} says that every finite graph has either a $k\cdot K_3$ minor or a set of at most $f(k)$ vertices the removal of which results into a forest, where $f(k) \sim k \log k$ is universal. The following proposes a coarse analogue, motivated by both \Cnr{conj fat min} and \Cnr{conj MM}:

\begin{conjecture}[Coarse \Erd--P\'osa Theorem] \label{Cnr EP}
 There is a function $f: \N \to \N$ and a universal constant $C$, \st\ the following holds \fe\ length space  $X$.
If $n \cdot K_3$ is not a $K$-fat minor of $G$, then \ti\ a set $S$ of at most $f(n)$ points of $X$ \st\ the ball of radius $C K$ around $S$ meets all $K$-fat $K_3$ minors of $X$. (Alternatively, $X$ is a graph, and removing $B_{CK}(S)$ results into a quasi-forest.)
\end{conjecture}

\section{Latest progress}
Many of the questions of this paper have been answered between the original submission and the revision. The aim of this section is to provide an update.

Nguyen, Scott \& Seymour \cite{NgScSeCou} found  a counterexample to \Cnr{conj MM}. This was used by Davies, Hickingbotham,  Illingworth \& McCarty \cite{DHIM} to disprove \Cnr{conj fat min}. \Cnr{conj MM NPc} has been proved by Balig\'acs \& MacManus \cite{BalMcMmet}.

The case $H=K_4$ of  \Cnr{conj fat min} has been proved by Albrechtsen, Jacobs,  Knappe, \& Wollan \cite{AJKW}. Special cases of \Cnr{conj MM}, for $r=2$, and sparse graphs, have been proved by Gartland, Korhonen \& Lokshtanov \cite{GaKoLo} and by Hendrey, Norin, Steiner \& Turcotte \cite{HNST}.
The special case of \Cnr{Tho} for finitely presented Cayley graphs has been proved by MacManus \cite{McMFat}. The “only if” direction of \Prr{k planar} has been proved by Esperet \& Giocanti \cite{EsGiCoa}.




\bibliographystyle{plain}
\bibliography{../../collective}
\end{document}
