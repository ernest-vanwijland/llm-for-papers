\documentclass[11pt,a4paper]{amsart}
\usepackage[margin=1in]{geometry}
\usepackage{amsmath, amssymb, latexsym, cmll, stmaryrd, eqnarray}
\usepackage{scalerel}
\usepackage{mathtools}
\usepackage{xspace}
\usepackage{xcolor}
\usepackage{ifthen}
\usepackage{hyperref}
\usepackage[capitalise]{cleveref}
\usepackage{lineno}
\usepackage{amsaddr}
\usepackage{wrapfig}
\usepackage[normalem]{ulem}

\usepackage{xcolor}





\newcommand{\ssection}{\section}

\newcommand{\note}[1]{\marginpar{\tiny #1}}

\newtheorem{lm}{Lemma}[section]
\newtheorem{fact}[lm]{Fact}
\newtheorem{thm}[lm]{Theorem}
\newtheorem{prp}[lm]{Proposition}
\newtheorem{obs}[lm]{Observation}
\newtheorem{cor}[lm]{Corollary}
\newtheorem{rmk}[lm]{Remark}
\newtheorem{ex}[lm]{Example}
\newtheorem{df}[lm]{Definition}
\newtheorem{prob}{Problem}
\newtheorem{conj}[prob]{Conjecture}

\newcommand{\myqed}{\hfill$\Box$}


\newcounter{senumi}[section]
\newcounter{senumip}[section]
\newcounter{temp}[section]

\def\thesenumi{\thesection.\arabic{senumip}}
\def\p@senumip\thesenumip{\thesenumi}
\newenvironment{senumerate}{\begin{list}{\hspace{-2em}(\thesenumi)}{\usecounter{senumip}}
\setcounter{senumip}{\value{temp}}
    }{\setcounter{temp}{\value{senumip}}
     \end{list}}
\newcounter{penumi}[section]
\newcounter{penumip}[section]
\newcounter{ptemp}[section]

\def\thepenumi{\arabic{penumi}}
\newenvironment{penumerate}{\begin{enumerate}\setcounter{penumi}{\value{ptemp}}\setcounter{enumi}{\value{ptemp}}}{\setcounter{ptemp}{\value{enumi}}
     \end{enumerate}}
\newcounter{ppenumi}[section]
\newcounter{ppenumip}[section]
\newcounter{pptemp}[section]

\def\theppenumi{\theptemp.\arabic{ppenumi}}
\newenvironment{ppenumerate}{\begin{list}{(\theppenumi)}{\usecounter{ppenumi}\setlength{\rightmargin}{\leftmargin}}
        \setcounter{ppenumi}{\value{pptemp}}
    }{\setcounter{pptemp}{\value{ppenumi}}
     \end{list}}


\newcounter{entmp}



\newenvironment{tenumerate}{\begin{enumerate}}{\end{enumerate}}





\newcommand{\polsat}[1]{\textsc{PolSat}\left( {\m #1} \right)}
\newcommand{\polsatstar}[1]{\textsc{PolSat$\star$}\left( {\m #1} \right)}
\newcommand{\cpolsatstar}[1]{\textsc{PolSatC$\star$}\left( {\m #1} \right)}

\newcommand{\npc}{\textsf{NP}-complete\xspace}
\newcommand{\conpc}{\textsf{co-NP}-complete\xspace}
\newcommand{\ptime}{\textsf{P}\xspace}
\newcommand{\rptime}{\textsf{RP}\xspace}
\newcommand{\prptime}{\textsf{(R)P}\xspace}
\newcommand{\usp}{Uniform Solution Property}
\newcommand{\USP}{USP\xspace}



\newcommand{\m}[1]{{\uppercase {\bf{#1}}}}
\newcommand{\rel}[1]{{\uppercase {\mathbb{#1}}}}



\newcommand{\vr}[1]{{\uppercase {\mathcal {#1}}}}



\newcommand{\set}[1]{{\left\{ {#1} \right\} }}
\newcommand{\ci}{\subseteq}
\newcommand{\co}{\supseteq}

\newcommand{\card}[1]{\left| #1 \right|}
\newcommand{\cardd}[1]{\# #1}
\newcommand{\equa}[1]{\left\| #1 \right\|}
\newcommand{\tup}[1]{\langle #1 \rangle}

\newcommand{\intv}[2]{I\left[#1,#2\right]}
\newcommand{\vpair}[2]{{{#1}\choose{#2}}}

\renewcommand{\leq}{\leqslant}
\renewcommand{\geq}{\geqslant}
\renewcommand{\le}[1]{\leqslant_{#1}}
\renewcommand{\ge}[1]{\geqslant_{#1}}


\newcommand{\comp}{\leq\!\geq}
\newcommand{\scomp}{<\!>}

\newcommand{\dist}[2]{{\sf dist}\!\left( #1,#2 \right)}
\newcommand{\distt}[3]{{\sf dist}_{#3}\!\left( #1,#2 \right)}



\renewcommand{\mapsto}{\longmapsto}
\newcommand{\tomaps}{\longmapsfrom}


\newcommand{\join}{\vee}
\newcommand{\meet}{\wedge}
\newcommand{\jjoin}{\bigvee}
\newcommand{\mmeet}{\bigwedge}
\newcommand {\bc}[1]{{\overline {#1}} }




\newcommand{\con}[1]{{\sf Con\:\m{#1}}}
\newcommand{\cn}[1]{{\sf Con\:\m{#1}}}
\newcommand{\Cn}[1]{{{\sf Con\:}\m{#1}}}
\newcommand{\cg}[3]{{\rm Cg}^{{\m {#1}}}({#2},{#3})}
\newcommand{\cgt}[3]{{\rm Cg}^{{\m {#1}}}({#2} \times {#3})}

\newcommand{\Pw}[1]{{\mathcal P} ({#1})}
\newcommand{\Pwp}[1]{{\mathcal P}^+ ({#1})}

\newcommand{\pol}[1]{{\rm Pol\:\m #1}}
\newcommand{\poln}[2]{{\rm Pol}_{#1}\m #2}

\renewcommand{\d}{\po D}
\newcommand{\q}{\po Q}


\newcommand{\po}[1]{{\mathbf {#1}}}
\newcommand{\te}[1]{{\mathbf {#1}}}
\newcommand{\tn} [1]{{\bf {#1}}}
\newcommand{\typ}{{\rm typ}}
\newcommand{\typset}[1]{\typ\set{#1}}
\newcommand{\rst}[2]{ {#1} |_{{#2}} }

\newcommand{\prect}[1]{\prec_{\tn #1}}

\newcommand{\minim}[3]{M_{\m #1}\left(#2,#3\right)}


\newcounter{note}
\newcounter{claim}


\renewcommand{\o}[1]{\overline {#1}}
\newcounter{ttable}
\newcommand{\ttable}{\refstepcounter{ttable}{Table \arabic{ttable}:\ \ }}





\newcommand{\centr}[2]{\left(#2 : #1\right)}
\newcommand{\comm}[2]{\left[ #1 , #2 \right]}






\newcommand{\efdef}{{\rm ($\star$)}}




\newcommand{\map}{\longrightarrow}

\newcommand{\congruent}[1]{\stackrel{#1}{\equiv}}

\newcommand{\h}[1]{\widehat{#1}}
\newcommand{\oo}[1]{\overrightarrow{#1}}

\newcommand{\hsp}{{\sf {HSP}}}
\newcommand{\hs}{{\sf {HS}}}

\newcommand{\fj}{\varphi}
\newcommand{\epsi}{\varepsilon}






 \usepackage{todonotes}
\newcommand{\stodo}[1]{\todo[inline,color=green!40]{\small #1}}
\newcommand{\mtodo}[1]{\todo[color=yellow!40]{\tiny #1}}
\newcommand{\piq}[1]{\textcolor{green}{#1}}


\newcommand{\mute}[1]{}




\newcommand{\gProblem}[2]{\ensuremath{\operatorname{\textup{\textsc{{#2}}}}
		\ifthenelse{\equal{#1}{}}{}{\!\left( {#1} \right)}}}

\renewcommand{\polsat}[1]{\gProblem{#1}{PolSat}}
\newcommand{\poleqv}[1]{\gProblem{#1}{PolEqv}}

\newcommand{\ceqv}[1]{\gProblem{#1}{CEqv}}
\newcommand{\csat}[1]{\gProblem{#1}{CSat}}

\newcommand{\progpolsat}[1]{\gProblem{#1}{ProgSat}}
\newcommand{\progcsat}[1]{\gProblem{#1}{ProgCSat}}
\newcommand{\listcsat}[1]{\gProblem{#1}{ListCSat}}
\newcommand{\listpolsat}[1]{\gProblem{#1}{ListPolSat}}
\newcommand{\twolistpolsat}[1]{\gProblem{#1}{2-ListPolSat}}
\newcommand{\twolistcsat}[1]{\gProblem{#1}{2-ListCSat}}
\newcommand{\twolistceqv}[1]{\gProblem{#1}{2-ListCEqv}}



\newcommand{\mreduces}{\leq_m}


\newcommand{\cm}{congruence modular }
\newcommand{\cp}{congruence permutable }

\newcommand{\pupi}{PUPI }

\newcommand{\nr}[1]{{\mathsf{nr}\left(#1\right)}}
\newcommand{\sr}[1]{{\mathsf{sr}\left(#1\right)}}

\newcommand{\charr}{\mathsf{char}}
\newcommand{\charrset}[1]{\charr\set{#1}}

\newcommand{\setm}{-}

\renewcommand{\aa}{a}
\newcommand{\bb}{b}
\newcommand{\cc}{c}
\newcommand{\dd}{d}
\newcommand{\ee}{e}

\newcommand{\aai}{a_i}
\newcommand{\bbi}{b_i}
\newcommand{\cci}{c_i}
\newcommand{\ddi}{d_i}
\newcommand{\eei}{e}

\newcommand{\alphai}{\alpha_i}
\newcommand{\alphao}{\alpha_{1-i}}
\newcommand{\betai}{\beta_i}
\newcommand{\betao}{\beta_{1-i}}
\newcommand{\alphami}{\alpha^-_i}
\newcommand{\alphamo}{\alpha^-_{1-i}}
\newcommand{\gammai}{\gamma_i}
\newcommand{\gammao}{\gamma_{1-i}}
\newcommand{\gammaz}{\gamma_0}
\newcommand{\gammaj}{\gamma_1}

\newcommand{\vi}{V_i}
\newcommand{\vo}{V_{1-i}}
\newcommand{\vz}{V_0}
\newcommand{\vj}{V_1}





\newcommand{\fji}{\fj_i}
\newcommand{\fjpi}{\fj^+_i}
\newcommand{\psii}{\psi_i}


\newcommand{\alpham}{\alpha^-}


\newcommand{\z}{\mathbb{Z}}
\newcommand{\N}{\mathbb{N}}







\DeclareMathOperator*{\amper}{\scalerel*{\&}{\sum}}


\newcommand{\matr}[2]{\mbox{End}{\left(\z_{#1}^{#2}\right)}}







\newcommand{\progg}[4]{\left(#1,#2,#3,#4\right)}
\newcommand{\prog}[2]{\left(#1\right)\!\left[#2\right]}
\newcommand{\progb}[3]{\left(#1\right)\!\left[#2,#3\right]}
\newcommand{\fcirc}[1]{#1^\circ}
\newcommand{\beq}{\mbox{\tt ==}}

\renewcommand{\b}{\textsf{b}}
\newcommand{\true}{1}
\newcommand{\false}{0}
\newcommand{\bool}{\set{\false,\true}}

\newcommand{\sacik}{\po {sat}}

\newcommand{\sumpk}[2]{\Sigma_{(#1,#2)}}
\newcommand{\aexp}{a}           






\newcommand{\zero}{e}

\newcommand{\ccc}{c}    \newcommand{\s}{s}      


\newcommand{\csize}{\lambda}   


\newcommand{\ccand}{\mathsf{AND}}
\newcommand{\ccmod}{\mathsf{MOD}}
\newcommand{\ccor}{\mathsf{OR}}

\newcommand{\sdiv}[1]{\Delta\left(#1\right)}    \newcommand{\ppdiv}{\delta}
\newcommand{\pdiv}[1]{\ppdiv\left(#1\right)}    \newcommand{\ar}[1]{\mu\left({#1}\right)}       \newcommand{\maxar}[1]{\mu\left({\m #1}\right)} 



\newcommand{\cdhh}{CDH\xspace}
\newcommand{\cdhhh}{Constant Degree Hypothesis\xspace}
\newcommand{\ethh}{ETH\xspace}
\newcommand{\rethh}{rETH\xspace}
\newcommand{\prethh}{(r)ETH\xspace}


 

\begin{document}


\title{Nonuniform Deterministic Finite Automata \\over finite algebraic structures}


\author{Paweł M. Idziak}
\address{Department of Theoretical Computer
Science,\\ Jagiellonian University,\\
Kraków, Poland}
\email{pawel.idziak@uj.edu.p}

\author{Piotr Kawałek}
\address{Institute of Discrete Mathematics and Geometry,\\ TU Wien, Austria \\[10pt] Department of Theoretical Computer
Science,\\ Jagiellonian University,\\
Kraków, Poland}
\email{piotr.kawalek@tuwien.ac.at}

\thanks{Piotr Kawałek: This research was funded in whole or in part by National Science Centre, Poland \#2021/41/N/ST6/03907. For the purpose of Open Access, the author has applied a CC-BY public copyright licence to
 any Author Accepted Manuscript (AAM) version arising from this submission.
Funded by the European Union (ERC, POCOCOP, 101071674). Views
 and opinions expressed are however those of the author(s) only and do not necessarily reflect those
 of the European Union or the European Research Council Executive Agency. Neither the European
 Union nor the granting authority can be held responsible for them.}

\author{Jacek Krzaczkowski}

\address{Department of Computer Science,\\
Maria Curie-Skłodowska University,\\
Lublin, Poland}
\email{krzacz@umcs.pl}
\thanks{Jacek Krzaczkowski: This research was funded in whole or in part by National Science Centre, Poland \#2022/45/B/ST6/02229. For the purpose of Open Access, the author has applied a CC-BY public copyright licence to any Author Accepted Manuscript (AAM) version arising from this submission.}


\begin{abstract}
Nonuniform deterministic finite automata (NUDFA) over monoids were invented by Barrington in \cite{Barrington85} to study boundaries of nonuniform constant-memory computation. Later, results on these automata helped to indentify interesting classes of groups for which  equation satisfiability problem (\polsat{}) is solvable in (probabilistic) polynomial-time \cite{GoldmannR02, IdziakKKW22-icalp}. Based on these results, we present a full characterization of groups, for which the identity checking problem (called \poleqv{}) has a probabilistic polynomial-time algorithm. We also go beyond groups, and propose how to generalise the notion of NUDFA to arbitrary finite algebraic structures. We study satisfiability of these automata in this more general setting. As a consequence, we present full description of finite algebras from congruence modular varieties for which testing circuit equivalence $\ceqv{}$ can be solved by a probabilistic polynomial-time procedure. In our proofs we use two computational complexity assumptions: randomized Expotential Time Hypothesis and Constant Degree Hypothesis.
\end{abstract}



\maketitle

\newpage

\section{Introduction}
There are many interactions between mathematics and (theoretical) computer science. 
Many branches of these sciences influence each other, sometimes in quite surprising ways.
Relatively recent example of such an influence is so-called algebraic approach to Constraint Satisfaction Problem (CSP) which led to complete classification of computational complexity of CSP \cite{Bulatov-dichotomy,Zhuk-dichotomy}.
It is really impressive how in this case (universal) algebra, combinatorics, logic, computational complexity and algorithmic work together to give new results in each of these fields.


Another example of synergy between different fields of mathematics and theoretical computer science can be observed on the borderline of circuit complexity, automata theory and (universal) algebra.
The most significant example here is the role of monoids played in automata theory and formal languages.


Usually deterministic finite automaton (DFA) is determined by 
an alphabet $\Sigma$
acting over a set $Q$ of states
by a function $\delta : \Sigma \times Q \ni (\sigma,q) \mapsto \sigma\cdot q \in Q$.
This action can be extended (in an obvious way) to the action of the free monoid $\Sigma^*$. 
To decide if a word $w\in\Sigma^*$ is accepted by a particular DFA
we need to endow it with a starting state $q_0$ and a set $F\ci Q$ of accepting states.  
Then $w$ gets accepted if $w\cdot q_0\in F$.
For our purposes we prefer, first to treat the set $Q^Q$ of functions 
as the monoid with $f\cdot g = g\circ f$, 
and then to treat the action $\delta$ as a function $a : \Sigma \map Q^Q$
given by $a(\sigma)(q)=\sigma\cdot q$.
Now, the word $\sigma^1\ldots\sigma^n$ gets accepted 
if $a(\sigma^1)\cdot\ldots\cdot a(\sigma^n)\in S$, 
where $S$ consists of transitions determined by the words $w\in\Sigma^n$ 
satisfying $w\cdot q_0\in F$.

Before restating Barrington's definition of Non-uniform Deterministic Finite Automata\break (NUDFA) over monoids \cite{Barrington86} we note that $a(\sigma^1)\cdot\ldots\cdot a(\sigma^n)$
is nothing else but\break $\po t_n(a(\sigma^1),\ldots,a(\sigma^n))$, 
where $\po t_n(x_1,\ldots,x_n)=x_1\cdot\ldots\cdot x_n$. 
In NUDFA over the monoid $\m M$ to accept the word from $\Sigma^n$ 
we are going to relax the term $x_1\cdot\ldots\cdot x_n$
to an arbitrary semigroup term $\po t(x_1,\ldots,x_k)$ 
and replace one action $a:\Sigma\map Q^Q$ by a bunch of functions $a^x:\Sigma\map M$,
one for each variable of $\po t$. 
Now a $\po t$-program (with inputs from $\Sigma^n$ represented by $n$-variable word $b_1\ldots b_n$) 
consists of:
\begin{itemize}
  \item a set of $k$-instructions, one for each variable $x$ of $\po t$, 
        of the form $\iota(x)=(b^x,a^x)$, 
        where $b^x$ is one of the variables $b_i$,
  \item and a set $S\ci M$ of accepting values.
\end{itemize}
Finally, a NUDFA over $\m M$  is a sequence (possibly even nonrecursive) of programs
$(\po t_n, n, \iota_n, S_n)_{n\in\N}$ 
with $\po t_n=(x_1,\ldots,x_{k_n})$ being some terms of $\m M$, 
$S_n \ci M$ and $\iota_n$ being the instructions for the variables of $\po t_n$. 
A word $b^1\ldots b^n\in\Sigma^n$ gets accepted by such a NUDFA if\break
$\po t_n(a^{x_1}(b^{x_1}),\ldots,a^{x_{k_n}}(b^{x_{k_n}}))\in S_n$. 
Originally Barrington considered mainly the Boolean case where $\Sigma=\set{0,1}$ 
to study computational boundaries of non-uniform constant-memory computation. 
Such automata can be used to compute the Boolean functions of the form $\set{0,1}^n\map\set{0,1}$. 
They also give an interesting algebraic insight into the internal structure of the class $NC^1$ \cite{BarringtonST90}. Note that in the Boolean case $\Sigma = \{0,1\}$ the function $a^x$ is given by the pair of values $a^x(0), a^x(1)$. In such the case,  for simplicity we write $\iota(x) = (\b^x,a^x(0),a^x(1))$.

A natural question that arises here is 
whether the language accepted by a particular NUDFA over $\m M$ is nonempty.
This problem reduces to:
\begin{itemize}
  \item[]$\progpolsat{\m M}$: \quad
        Decide if a given program over $\m M$ accepts at least one word.
\end{itemize}
The problem $\progpolsat{}$ proved itself to be extremely useful in studying groups (and monoids)
for which determining if an equation has a solution (\polsat{}) is in $\ptime$.
In fact the proof of Goldmann and Russell in \cite{GoldmannR02}
that each nilpotent group $\m G$ has tractable $\polsat{\m G}$
modifies a polynomial time algorithm for $\progpolsat{\m G}$.
A study of connections between \polsat{} and \progpolsat{} for finite monoids is given in \cite{BarringtonMMTT00}.
Recently a complete classification of finite groups $\m G$ with $\progpolsat{\m G} \in \rptime$, together with its consequences for \polsat{} has been provided by Idziak, Kawałek and Krzaczkowski in \cite{IdziakKKW22-icalp}.


Now the results of \cite{IdziakKKW22-icalp} allows us to complete the long term extensive investigations \cite{BurrisL04, HorvathS11, FoldvariH19, Weiss20, IdziakKKW22TOCS}
on equivalence problem \poleqv{} of polynomials (i.e. terms with some variables already evaluated) over finite groups.

The lower bound in our characterization relies
on the randomized version of the Exponential Time Hypothesis (\rethh),
while the upper bounds explore the so-called Constant Degree Hypothesis (\cdhh).
This hypothesis, introduced in \cite{BarringtonST90}, can be rephrased to state
that for a fixed integer $d$, a prime $p$ and an integer $m$ which is not just a power of $p$,
any 3-level circuits of the form $\ccand_d\circ\ccmod_m\circ\ccmod_p$
require exponential size to compute $\ccand_n$ of arbitrary large arity $n$ (in which $\ccand_d$ gates are in the input layer, $\ccmod_m$ gates are in the middle layer, and there is one $\ccmod_p$ gate in the output layer). 
The best known lower bound, due to Chattopadhyay et al. \cite{chat-lowerbounds}, 
for the size of $\ccand_d\circ\ccmod_m\circ\ccmod_p$ computing $\ccand_n$
is only superlinear.
Much earlier \cdhh has been considered in many different contexts.
Already in \cite{BarringtonST90} the case $d=1$, i.e. no $\ccand_d$ layer, has been confirmed. 
Also restrictions put either on the number of $\ccand_d$ used locally, 
or on the local structure of $\ccand_d\circ\ccmod_m$ fragments, 
allowed Grolmusz and Tardos \cite{GrolmuszT00, Grolmusz01} to confirm \cdhh. 
Very recently Kawałek and Wei\ss{}  \cite{KW-symmetric-gates} confirmed \cdhh for symmetric circuits. 

In this paper, under these two complexity hypothesis (i.e. \rethh and \cdhh), the characterization of finite groups with tractable \progpolsat{} from \cite{IdziakKKW22-icalp} is applied  
to get unexpectedly different characterization for tractable \poleqv{}. 

\begin{thm}
\label{eqv-groups}
Let $\m G$ be a finite group.
Assuming \rethh and \cdhh
the problem \poleqv{\m G} is in \rptime if and only if $\m G$ is solvable and has a  nilpotent normal subgroup $\m H$ with the quotient $\m G/\m H$ being also nilpotent.
\end{thm}

A surprising part of these investigations is that such characterization can not only be done,
but that it can be stated, in terms of algebraic structure of the groups.
In fact this reveals another connection between (universal) algebra
and circuit complexity theory. 

The goal of this paper is to leave the group realm and generalize \cref{eqv-groups} 
to a much broader setting of algebras. As we will see shortly this generalization is two-fold.
First we generalize the concept of NUDFas to cover automata working over arbitrary finite algebraic structure $\m A$.
This requires to define a program over $\m A$ and can be done by simply replacing the monoid $\m M$ by the algebra $\m A$ and assume that this time $\po t$ is a term of the algebra $\m A$.

\begin{wrapfigure}{l}{0.39\textwidth}
\begin{center}
\includegraphics[width=0.93\linewidth]{Rysunek4.pdf}
{\small Compressing the size of $\po t_n$.}
\\{\small \copyright Idziak, Krzaczkowski \cite{IdziakK22}}
\label{fig:comm-circuit}
\end{center}

\end{wrapfigure}

Second, in general algebraic context it is not clear which operations are to be chosen to be the basic ones.
And this choice may be extremely important from computational point of view.
Recall after \cite{IdziakK22}, that adding the binary commutator operation 
$[x,y]=x^{-1}y^{-1}xy$ to the language of a group may exponentially shorten the size of the input.
Indeed the term $\po t_n(x_1,\ldots,x_n) =[\ldots [[x_1,x_2],x_3] \ldots, x_n]$ has linear size if the commutator operation is allowed, while after writing this term in the pure group language (of multiplication and the inverse) we see that the size $\card{\po t_n}$ of $\po t_n$ is 
$2\card{\po t_{n-1}}+2$, as 
$\po t_n(x_1,\ldots,x_n)=
\po t_{n-1}(x_1,\ldots,x_{n-1})^{-1}\cdot x_n^{-1}\cdot\po t_{n-1}(x_1,\ldots,x_{n-1})\cdot x_n$, 
so that $\card{\po t_n}$ is exponential on $n$.
Actually for the alternating group $\m A_4$ it is shown in \cite{HorvathS12} that 
$\polsat{\m A_4}$ is in $\ptime$, while after endowing $\m A_4$ with the commutator operation 
(and therefore shortening the size of inputs) the problem becomes \npc.
A solution to this phenomena has been proposed by Idziak and Krzaczkowski in \cite{IdziakK22} by presenting a term by the algebraic circuit that computes it. 
For example $\po t_n(x_1,\ldots,x_n)$ can be computed by the circuit of size $6n-5$ as shown by Figure \ref{fig:comm-circuit}.

\medskip\noindent
Representing polynomials of an algebra with algebraic circuits leads to a modified version of \progpolsat{} which we explore in this paper.
\begin{itemize}
  \item[]$\progcsat{\m A}$: \quad
        Decide if a given program $(\po t, n, \iota, S)$  over $\m A$ accepts at least one word, where $\po t$ is
        given by a circuit over $\m A$. 
\end{itemize}

Measuring the size of the input, i.e. the expression of the form $\po p=\po q$ by the lengths of the polynomials $\po p$ and $\po q$ or by the sizes of their algebraic circuits used to compute $\po p$ and $\po q$ give rise to either \polsat{} or \csat{} in the satisfiability setting 
or to \poleqv{} or \ceqv{} in the equivalence setting.
Note here that \cite{IdziakK22} argues that the complexity of $\csat{\m A}$ and $\ceqv{\m A}$ is independent of which term operations of the algebra $\m A$ are chosen to be the basic ones.
This independence gives a hope for a characterization of algebras $\m A$ with tractable 
$\csat{\m A}$ or $\ceqv{\m A}$ in terms of algebraic structure of $\m A$. 
Actually already a series of papers 
\cite{IdziakKK20,Weiss20, IdziakKK22STACS,IdziakKKW22-icalp,IdziakK22,Kompatscher21} 
enforces a bunch of such necessary algebraic conditions for an algebra to have \csat{} or \ceqv{} tractable.
Not surprisingly solvability and nilpotency are among these conditions.

To define these two notions of solvability and nilpotency outside the group realm we need a notion of a commutator. However we need to work with a commutator $\comm{\alpha}{\beta}$ of two congruences $\alpha,\beta$ (that in the group setting correspond to normal subgroups) instead of a commutator of elements of an algebra. 
For more details on the definition and the properties of commutator, we refer to the book \cite{fm} and Section \ref{sec-notions-1}.
Here we only note that this concepts of commutator of congruences works smoothly only in some restricted setting of the so called congruence modular varieties, i.e. equationally definable classes of algebras with modular congruence lattices.
Fortunately this setting includes groups, rings, quasigroups, loops, Boolean algebras, Heyting algebras, lattices and almost all algebras related to logic.
In groups, rings or Boolean/Heyting algebras the congruences are determined by normal subgroups, ideals or filters respectively.
Obviously this new concept of commutator of normal subgroups coincides with the old one. 
The commutator of two ideals $I,J$ of a commutative ring 
is simply their algebraic product $I\cdot J$, 
while the commutator of filters in a Boolean/Heyting algebra is their intersection. 
Now we can say that a congruence $\alpha$ is abelian, nilpotent or solvable if  $\comm{\alpha}{\alpha}=0_{\m A}$, 
$\comm{\ldots\comm{\comm{\alpha}{\alpha}}{\alpha}}{\ldots\alpha}=0_{\m A}$
or
$\comm{\comm{\comm{\alpha}{\alpha}}{\comm{\alpha}{\alpha}}}
{\ldots\comm{\comm{\alpha}{\alpha}}{\comm{\alpha}{\alpha}}}=0_{\m A}$ respectively (for some number of nested commutators). 
Here, $0_{\m A}$ is the identity relation/congrueence of $\m A$.
The algebra $\m A$ itself is said to be abelian, nilpotent or solvable if the total congruence $1_{\m A}$ collapsing everything is abelian, nilpotent or solvable, respectively.

Note that, for nilpotent groups, boolean programs (of NUDFAs) compute $\ccand$ functions only of bounded arity, 
i.e. for each nilpotent group $\m G$ there is a constant $k$ such that $\ccand_k$ is computable by no program over $\m G$ \cite{BarringtonST90}. 
This nonexpressibility phenomena does not transfer to nilpotent algebras in general congruence modular context. The most natural example here is the algebra
$(\z_6; +,\%2)$, i.e. the group $(\z_6; +)$ endowed with the unary operation $\%2$ computing the parity.
In this algebra all the circuits of the form $\ccmod_2\circ\ccmod_3$ can be modelled so that $\ccand_n$ can be expressed for all $n$ (however by exponential size of the circuits).
This action of the prime $2$ acting over prime $3$ cannot occur in nilpotent groups.
Indeed, due to the Sylow theorem, each finite nilpotent group is a product of $p$-groups.
This decomposition prevents interaction between different primes, as they occur on different stalks/coordinates. And the lack of such interactions is crucial in bounding the arity of expressible $\ccand_n$'s. 
Also in our considerations the finite nilpotent algebras that decompose into a product of algebras of prime power order occurs naturally. 
They are known as supernilpotent  algebras  \cite{bulatov-kom, aichmud-2010}.
We will return to this concept of supernilpotent algebras and its relativization to supernilpotent congruences in Section \ref{sec-notions-1}. 

Now we are ready to state the other two main results of the paper.

\begin{thm}
\label{thm:cm-progcsat}
Let $\m A$ be a finite algebra from a congruence modular variety. 
Assuming \rethh and \cdhh
the problem $\progcsat{\m A}$ is in \rptime if and only if $\m A$ is nilpotent 
and has a supernilpotent congruence $\sigma$ with supernilpotent quotient $\m A/\sigma$ 
and such that all cosets of $\sigma$ have sizes that are powers of the same prime number $p$.
\end{thm}


\begin{thm}
\label{thm:cm-ceqv}
Let $\m A$ be a finite algebra from a congruence modular variety. 
Assuming \rethh and \cdhh
the problem $\ceqv{\m A}$ is in \rptime if and only if $\m A$ is nilpotent 
and has a supernilpotent congruence $\sigma$ with supernilpotent quotient $\m A/\sigma$. 
\end{thm}

Results from \cite{IdziakKKW22-icalp} that we use in the proof of \cref{eqv-groups}
heavily rely on a method of representing terms/polynomials of finite solvable group $\m G$
by bounded-depth circuits that use only modular gates.
Besides the obvious requirement that the circuit representing a group-polynomial $\po p$ has to compute the very same function as $\po p$ does, we also want to control the size of the circuit to be polynomial in terms of the size (length) of $\po p$.


A very similar approach can be found in \cite{Kompatscher19CC}, where M.\! Kompatscher considers generalizations of finite nilpotent groups, i.e.\! nilpotent algebras from the congruence modular varieties. He provides a method to rewrite circuits over such algebras to constant-depth circuits which, again, use only modulo-counting gates. These modular circuits, which appear in both of the mentioned cases, are known as CC-crcuits. However, since \cite{Kompatscher19CC} does not use the notion of a program/NUDFA, the author formulates his results for circuits over $\m A$ representing only some specific functions. For those functions, it is natural how to interpret Boolean values $0/1$ in the non-Boolean algebra $\m A$. In our paper, the notion of a program/NUDFa provides us with a formal framework which helps to relate the expressive power of algebraic structures to some standard circuit complexity classes. Here, we present a very precise characterization of functions computable by programs over algebras corresponding to polynomial-time cases of \progcsat{}.

\begin{thm}
\label{thm:2supernil-circuit-early}
Let $\m A$ be a finite nilpotent algebra from a congruence modular variety with  supernilpotent congruence $\sigma$ of $\m A$
such that cosets of $\alpha$ are of prime power size $p^{k}$ and  $\m A/\sigma$ is supernilpotent.
Then the function computable by a boolean program of size $\ell$ over the algebra $\m A$
can be also computed by an
$\ccand_d\circ\ccmod_{m}\circ\ccmod_p$-circuits of size $O(\ell^c)$
with $d,m,c$ being natural numbers depending only on $\m A$, and $m$ being relatively prime to $p$.
\end{thm}

This theorem not only is an interesting result on its own, but also is crucial in proving \cref{thm:cm-progcsat} and \ref{thm:cm-ceqv}. 


\section{Algebraic preliminaries}\label{sec-notions-1}
An algebra is a set called universe together with a finite set of operations acting on it called basic operations of the algebra. We usually use boldface letter to denote the algebra and the very same latter, but with a regular font, to denote its universe. $\pol{A}$ is polynomial clone of $\m A$, that is the set of all polynomial operations of $\m A$. An algebra $\m A$ is polynomially equivalent to an algebra $\m B$ if it is isomorphic
to an algebra which has the same set of polynomial operations as $\m B$.  An induced algebra $\m A|_S$ is a set $S$  with all polynomial operations of $\m A$ closed on $S$ (or in other word polynomial operations which for arguments from $S$ have value in $S$).  Idempotent  function is a function $f$ such that $f(f(x))=f(x)$.

In proves of intractability of \progcsat{} and \ceqv{} for algebras from congruence modular varieties the crucial role is played by Tame Congruence Theory (see \cite{hm} for details). This is a deep algebraic tool describing local behavior of finite algebras. TCT shows that locally  finite algebras behave in one of following five ways:
\begin{enumerate}
\item[{\tn 1}.]  a finite set with a group action on it,
\item[{\tn 2}.]  a finite vector space over a finite field,
\item[{\tn 3}.]  a two-element Boolean algebra,
\item[{\tn 4}.]  a two-element lattice,
\item[{\tn 5}.]  a two-element semilattice.
\end{enumerate}
By $\typset{\m A}$, let us denote a subset of $\{\tn 1,.., \tn 5\}$ which describes the local behaviours we can find in an algebra $\m A$. Note that in a case of algebra $\m A$ from a congruence modular variety only three types can appear in $\typset{\m A}$, that is types {\tn 2}, {\tn 3} and {\tn 4}. In case of types {\tn 3} and {\tn 4} there has to be two-element set $U$ (let's call element of $U$ as $0$ nad $1$) such that
\begin{itemize}
    \item there is a polynomial $\po e$ of $\m A$ fulfilling $\po e(A)=U$,
    \item there are polynomials of $\m A$ which behaves on $U$ like $\meet$ and $\join$,
    \item in case of type {\tn 3} there is also unary polynomial of $\m A$ which is a negation on $U$.
\end{itemize}

If we can find type {\tn 3} or {\tn 4} in $\typset{\m A}$, the complexity of both \progcsat{} and \ceqv{} is relatively easy to determine, as we shall see in forthcoming chapters. For this reason the most of the volume of the paper is devoted to algebras with $\typset{\m A} = \{\tn 2\}$. In the congruence modular variety those are precisely the solvable algebras \cite{fm, hm}. In fact, some of the earlier papers already dealt with algebras that are solvable but not nilpotent, so we will be mostly concerned with the notion of nilpotency and its properties.

Every solvable (so in particular - nilpotent) algebra in the congruence modular variety is Malcsev, i.e.\! it possesses a polynomial operation $\po d$ satisfying the following identity:
$\po d(y,x,x)=\po d(x,x,y)=y$. A standard example of Malcev algebras are groups with a Malcev term of the form $x \cdot y^{-1} \cdot z$. Unlike for groups, nilpotent algebras do not necessarly decompose into a direct product  of algebras of prime power order. However, the technique contained in this paper splits an algebra into slices on which such nice decomposition can be observed. 

To define this slicing properly, we need to consider congruences. Recall that congruence $\sigma$ of an algebra $\m A$ is an equivalence relation which is preserved by the operations of $\m A$. Such relations are naturally associatted with surjective homomorphisms from $\m A$ to $\m A/\sigma$ mapping $x$ to $[x]_{\sigma}$ (equivalence class of $x$ in $\sigma$), so congruencess are essentially generalization of normal subgroups of a group. Similarly as normal subgroups, congruences of an algebra form a lattice. From now on, we write $\con{\m A}$ for the set of all congruences of $\m A$. Every element of a finite lattice can be written as a meet (join) of meet-irreducible (join-irreducible) elements, i.e.\! elements which cannot be writen as a meet (join) of any other two elements of the lattice. These special elements, generating $\con A$, will play a significant role in our analysis.

For $\alpha, \beta \in \con{\m A}$, by $\intv{\alpha}{\beta}$ we mean a set of congruencess $\gamma$ such that $\alpha \leq \gamma \leq \beta$. In case when $\intv{\alpha}{\beta} = \{\alpha, \beta\}$, i.e.\! there are no congruencess between $\alpha$ and $\beta$, we call $\beta$ a cover of $\alpha$,  we call $\alpha$ a subcover of $\beta$, and we call $\alpha, \beta$ a covering pair. To highlight such a situation we write $\alpha \prec \beta$ for short. Whenever $\alpha$ is meet-irreducible (join-irreducible) congruence, then there is a unique congruence $\alpha^{+}$ ($\alpha^{-}$) such that $\alpha \prec \alpha^{+}$ ($\alpha^{-} \prec \alpha$). 
For a nilpotent algebra $\m A$  from CM wheneverver its congruencess $\alpha, \beta$ satisfy $\alpha \prec \beta$, the cosets (congruence classes) of $\beta/\alpha$ in $\m A/\alpha$ have equal sizes, being a power of some prime (see \cref{lm:simple-atom} and \cref{lm:simple-module-atom}). We later denote this prime by $\charr(\alpha, \beta)$ and call it a characteristic of a congruence cover $\alpha \prec \beta$. Moreover for arbitrary $\alpha < \beta$, we write $\charrset{\alpha, \beta}$ for the set of all possible prime characteristics of covering pairs, which are fully contained in $\intv{\alpha}{\beta}$.
We say that a pair of congruences $\alpha < \beta$ of a nilpotent algebra $\m A$ forms a Prime Uniform Product Interval (PUPI), if there are congruencess $\alpha < \alpha_1, \ldots, \alpha_k \leq \beta$ such that
\begin{itemize}
    \item $\bigvee \alpha_i = \beta$
    \item $\alpha_i \wedge (\bigvee_{j\neq i}\alpha_j) = \alpha$, for $i,j \in \{1..k\}$, \item For every $i$  we have $|\charrset{\alpha, \alpha_i}| = 1$.
\end{itemize}

We call a congruence $\beta$ supernilpotent whenever it is nilpotent and the interval $\intv{0_{\m A}}{\beta}$ is a PUPI, and we call an algebra $\m A$ supernilpotent, whenever $1_{\m A}$ is supernilpotent. 
Each supernilpotent algebra is isomorphic to a direct product of nilpotent algebras of prime power size. In this sense supernilpotence generalizes nilpotence for groups.
Note that this definition is equivalent to a standard definition of supernilpotent algebras in congruence modular varieties \cite{MayrS21}.

 We say that a nilpotent algebra $\m A$ has supernilpotent rank $k$ whenever $k$ is the smallest number for which we can find sequence of congruences $0_{\m A} = \alpha_0 < \alpha_1 < \ldots < \alpha_k = 1_{\m A}$ such that interval $\intv{\alpha_i}{\alpha_{i+1}}$ is a PUPI for each $0 \leq i < k$.  Note that for a finite nilpotent algebra $\m A$ we can always find such a finite sequence, since each covering pair forms a PUPI. This notion of supernilpotent rank of an algebra, later denoted by $\sr{\m A}$, proved to be extremely usuful in some very recent results on the computation complexity of circuit satisfiability problem \cite{IdziakKK20, Kompatscher21}. In fact, results for $\progcsat{}/\ceqv{}$ we present in this paper, are essentially about nilpotent algebras with $\sr{A} = 2$.   
\section{Polynomial equivalence}
Our characterization of polynomial time cases of \poleqv{} is achieved through a reduction to some special instances of  \polsat{}. It was noticed already in \cite{GoldmannR02} that \polsat{\m G} for finite group $\m G$ reduces in polynomial time to \progpolsat{\m G}. Very recently, the full characterization,  of groups for which \progpolsat{} can be solved in randomized polynomial time was shown under the assumptions of \rethh and \cdhh in
\cite{IdziakKKW22-icalp}. 

 \begin{thm}
\label{thm:progsat}
Let $\m G$ be a finite group.
Assuming \rethh and \cdhh
the problem
\progpolsat{\m G} is in  \rptime if and only if
$\m G/\m G_p$ is nilpotent for some
normal $p$-subgroup $\m G_p$ of $\m G$
(with $p$ being prime).
\end{thm}

\cref{thm:progsat} together with a result from \cite{{IdziakKKW22TOCS}} provides us with enough information to prove the \cref{eqv-groups}. 





\begin{proof}\textcolor{red}{TOPROVE 0}\end{proof}

\section{Program satisfiability}
\label{section:progcsat}
The goal of this section is to prove \cref{thm:cm-progcsat}. First we observe in \cref{fact:red-to-progcsat} that if a nilpotent algebra has a Malcev term then \csat{} and \ceqv{} for this algebra reduces to \progcsat{}. Thus, intractability of \csat{} or \ceqv{} implies intractability of \progcsat{} and conversely if \progcsat{} is in \rptime, so are \csat{} and \ceqv{}.



\begin{fact}
\label{fact:red-to-progcsat}
For a finite nilpotent Malcev algebra $\m A$ the problems $\csat{\m A}$ and $\ceqv{\m A}$
are Turing reducible to $\progcsat{\m A}$.
\end{fact}

\begin{proof}\textcolor{red}{TOPROVE 1}\end{proof}
In the proof of Theorem \ref{thm:cm-progcsat} we will use Fact \ref{fact:red-to-progcsat} to show that, under our assumptions, nilpotent Malcev algebra with tractable \progcsat{} has supernilpotent rank equal at most $2$. In section \ref{section:hard} we use advanced tools of Tame Congruence Theory and Commutator Theory to prove the following lemma which shows that not for every algebra with supernilpotent rank equal $2$ \progcsat{} is tractable.

\begin{lm}
\label{lm:hard}
Let $\m A$ be a finite nilpotent algebra from a \cm variety with $\sr{\m A}=2$ and tractable $\progcsat{\m A}$.

Then $\m A$ has a supernilpotent congruence $\alpha$ with cosets of prime power order such that quotient algebra $\m A/\alpha$ is also supernilpotent, or \rethh fails.
\end{lm}

The important ingredient of the proof of above lemma is an idea of Barrington et al \cite{BarringtonBR94} heavily explored in \cite{idziakKK22LICS} and \cite{IdziakKKW22-icalp} resulting in the following lemma.

\begin{lm}
\label{lm:pseudo-and}
Let $p$ be a prime number and $\nu \geq 1$ be an integer.
Then for each 3-CNF formula $\Phi(\o x)$ with $n$ variables
there is a polynomial $w^\Phi_{p}(\o x)$ over  $GF(p)$
of degree at most $O(p^\nu)$
such that for all $\o b \in \set{0,1}^n$ we have
\[
w^\Phi_{p}(\o b) =
\left\{
\begin{array}{ll}
0, &\mbox{if the number of unsatisfied (by $\o b$) clauses in $\Phi$}\\
    &\mbox{is divisible by $p^\nu$}\\
1, &\mbox{otherwise.}
\end{array}
\right.
\]
Moreover, computing $w^\Phi_{p}$ from $\Phi$ can be done in $2^{O(p^\nu(\log n+\log p))}$ steps.
\end{lm}

The power of \cref{lm:pseudo-and} can be observed when we use it simultaneously for two different primes, say $p_1$ and $p_2$. Then if for a given 3-CNF formula $\Phi$ with $m$ clauses we will choose  positive integers $\nu_1$, $\nu_2$ such that $p_i^{\nu_i-1}\leq \sqrt{m}< p_i^{\nu_i}$, we will get, by Chinese Remainder Theorem, that $\Phi$ is satisfied by $\o b\in\set{0,1}$ iff $w^\Phi_{p_1}(\o b) =w^\Phi_{p_2}(\o b) = 0$. Moreover, the lengths of $w^\Phi_{p_1}(\o b)$ and $w^\Phi_{p_2}(\o b)$ are subexponential in the size of $\Phi$. The core of the proof of \cref{lm:hard} is showing (with haevilly use of Tame Congruence Theory and Commutator Theory) that if nilpotent Malcev algebra $\m A$ with supernilpotentn rank $2$ has supernilpotent congruence $\alpha$ which cosets are not of prime power size and such that $\m A/\alpha$ is supernilpotent then we can simulate by programs over $\m A$ systems of equations in the form:
\[
w^\Phi_{p_1}(\o b) = 0,
\]
\[
w^\Phi_{p_2}(\o b) = 0.
\]

Now we are ready to prove \cref{thm:progsat}
\begin{proof}\textcolor{red}{TOPROVE 2}\end{proof}


\section{Circuit equivalence}
In this section we will prove \cref{thm:cm-ceqv}. To do it we will show that solving \ceqv{} for an algebra $\m A$ can be reduced to solving the very same problem for quotients of $\m A$ by a meet-irreducible congruences. Obviously, every quotient algebra by a meet-irreducible congruence has the smallest congruence bigger than identity relation. This observation plays a crucial role in the proof o \cref{thm:cm-ceqv}.

\begin{proof}\textcolor{red}{TOPROVE 3}\end{proof}


\section{Notation}

In this section we introduce detailed notation, needed in further sections of the paper.
\subsection*{Algebra} 
We use the standard universal algebraic notation the reader can find e.g. in \cite{BurrisSankappanavar}. Our results heavily rely on Tame Congruence Theory and Modular Commutator Theory which have a detailed description in \cite{hm} and \cite{fm} respectively. One can find the not too long summary of needed notions and facts  in \cite[Section 2]{IdziakK22}. Here we just recall, for readers convenience, the most important notation.


 For an algebra $\m A$ and $a,b\in A$ congruence $\Theta(a,b)$ is the smallest congruence containing the pair $(a,b)$. Note that every join-irreducible congruence is generated by a single pair of elements of the algebra. In particular, covers of the identity relation, called atoms, are generated by a single pair. 

 Minimal sets are the central notion of the Tame Congruence Theory. Formally, for an algebra $\m A$ and $\alpha,\beta\in \con {\m A}$ such that $\alpha < \beta$ we define the $(\alpha,\beta)$-minimal set $U\subseteq A$ as minimal, with respect to the inclusion, among sets of the form $\po f(A)$ for all unary polynomials  $\po f$ of  $\m A$ such that $\po f(\beta)\not\subseteq\alpha$. In this paper we consider $(\alpha,\beta)$-minimal sets for $\alpha\prec\beta$ only (i.e. for so-called prime quotients).  For every minimal set $U$ of $\m A$ there exists unary idempotent polynomial $\po e_U$ of $\m A$ such that $\po e_U(A)=U$. A trace of the $(\alpha,\beta)$-mimal set $U$ is a set $u/\alpha\cap U$ for some $u\in U$ such that $u/\alpha\cap U\not=u/\beta\cap U$. For $(\alpha, \beta)$-minimal set $U$ (or its trace $N$) of algebra $\m A$ we usually consider an induced algebra $\m A/\alpha|_{U/\alpha}$ ($\m A/\alpha|_{N/\alpha}$). The five ways of local behaviour of finite algebras mentioned in \cref{sec-notions-1} are exactly the types of algebras (or rather their polynomial clones) induced on minimal sets traces. Note that all traces of a minimal set are polynomially equivalent. In this paper we usually work with minimal sets of type $\tn 2$, that is in which algebras induced on traces are polynomially equivalent to one-dimensional vector spaces. Traces of a minimal set of type $\tn 2$ are of prime power order, this prime is called a characteristic of the minimal set. Note that all minimal sets taken with respect to the same prime quotients induce polynomially equivalent algebras. It allows us to define the type of prime quotient as a type of minimal sets taken with respect to that quotient. In case of prime quotients of type $\tn 2$ a prime characteristic of the quotient defined in \cref{sec-notions-1} is equal to the characteristic of minimal sets taken with respect to it.

Algebras belonging to congruence modular variety are a wide class of algebras for which Tame Congruence Theory and Commutator Theory work particularly well. In this context term ,,modular'' means that all algebras from the variety have modular congruence lattice i.e. for $\alpha,\beta, \gamma\in \con{A}$ $\alpha\leq\beta$ implies $\alpha\join(\gamma\meet\beta)=(\alpha\join\gamma)\meet\beta$.  Equivalently, lattice is modular if it has no elements $\alpha,\beta,\gamma$ for which $\alpha<\beta$, $\alpha\join\gamma=\beta\join\gamma$ and $\alpha\meet \gamma=\beta\meet \gamma$. Such a sublattice $\{\alpha, \beta, \gamma, \alpha\join\gamma, \alpha\meet\gamma\}$, if found in $\con{A}$, is called a \textit{pentagon}.   If $\intv{\alpha}{\beta}$ and $\intv{\gamma}{\delta}$ are intervals such that $\beta\meet\gamma = \alpha$ and
$\beta\join \gamma = \delta$, then $\intv{\alpha}{\beta}$ is said to transpose up to $\intv{\gamma}{\delta}$, written $\intv{\alpha}{\beta}\nearrow\intv{\gamma}{\delta}$ and $\intv{\gamma}{\delta}$ is said to transpose down to $\intv{\alpha}{\beta}$, written $\intv{\gamma}{\delta}\searrow\intv{\alpha}{\beta}$ and the two intervals are called transposes of one another. Two intervals are said
to be projective if one can be obtained from the other by a finite sequence of transposes. A fundamental fact in lattice theory is that a lattice is modular if and only if its projective intervals are isomorphic. In the case of congruence lattice of an algebra this isomorphism is even stronger since projective prime  quotients have exactly the same minimal sets and in a consequence the same types and, in case of type $\tn 2$, the same characteristics.

Malcev algebras are examples of algebras from congruence modular varieties. Such algebras form so called congruence permutable varieties. Congruences of algebras from this class commute i.e.\! for arbitrary congruences $\alpha$ and $\beta$ we have that  $\alpha\circ\beta=\beta\circ\alpha$.   

Supernilpotency plays a crucial role in our investigations. For a fixed algebra $\m A$ we write $\sigma$ for a biggest supernilpotent congruence and $\kappa$ for the smallest congruence such that $\m A/\kappa$ is supernilpotent. Moreover, we write $\sigma_p$ for the biggest supernilpotent congruence with cosets of size being a power of $p$. The existance of such biggest/smallest congruences follows from term definition of supernilpotence and its properties \cite[Corollary 6.6]{aichmud-2010}. 

\subsection*{Arithmetic operations and functions}
Sometimes we use addition in two different abelian groups  inside one formula. To avoid ambiguity, different symbols for different types of addition are used: $+, \sum$  for one type of addition (for instance in $\z_p$) and $\oplus, \bigoplus$ for the other (in $\z_m$). Moreover, for a natural number $m$ with a prime decomposition $p_1^{\alpha} \cdot p_2^{\alpha_2} \cdot \ldots \cdot p_s^{\alpha_s}$ we write $\sdiv m=\set{p_1,\ldots,p_\s}$ for the set of prime divisors of $m$ and $\pdiv m = p_1 \cdot \ldots \cdot p_s$ for the largest square-free divisor of $m$. We also use $\sdiv A$ for $\sdiv {|A|}$ and $\pdiv A$ for $\pdiv {|A|}$. Likewise, $\ar A$ denotes the maximal arity among basic operations of $\m A$ and $\ar f$ is the arity of the function $f$. For a set $X=\set{d_1,\ldots,d_s}$
and a function $f: X^k \map Y$ we associate to $f$ its
binary expansion $\fcirc{f} : \bool^{\card{X}\cdot k} \map Y$,
i.e., arbitrary function satisfying
\[
f(x_1,\ldots,x_k) =
\fcirc{f}(x_1 \beq d_1,\ldots,x_1 \beq d_s,\ldots,x_k \beq d_1,\ldots,x_k \beq d_s).
\]

\subsection*{Programs}
Recall that a program $(\po p, n, \iota, S)$ computes a Boolean function $\{0,1\}^n \rightarrow \{0,1\}$. From now on we write $\progb{\po p}{\iota}{S}(\o b)$ for evaluation of this function on a tuple $\o b \in \{0,1\}^n$. Moreover, if $S = \{c\}$ is one-element set, we simply write $\progb{\po p}{\iota}{\ccc}(\o b)$ instead of $\progb{\po p}{\iota}{\{\ccc\}}(\o b)$. With each program $(\po p, n, \iota, S)$ over $\m A$ we can naturally associate an inner function $\prog{\po p}{\iota}: \{0,1\}^n \rightarrow A$ satisfying $\prog{\po p}{\iota}(\o b) \in S$ iff $\progb{\po p}{\iota}{S}(\o b) = 1$. This function is computed by the expression 
$\po p(a^{x_1}(b^{x_1}),\ldots,a^{x_k}(b^{x_k}))$, where all $a^{x_i}, b^{x_i}$ are provided by the instructions $\iota(x) = (b^x, a^x)$. For a congruence $\sigma \in \con{\m A}$ and a program $(\po p, n, \iota, S)$ we can define the quotient program $(\po p/\sigma, n, \iota/\sigma, S/\sigma)$ of an algebra $\m A/\sigma$, by simply reinterpreting $\po p$ to be a circuit over $\m A/\sigma$ (algebras $\m A$ and $\m A/\sigma$ have the same signature), and taking $\iota(x) = (b^x, a^{x}/\sigma)$, and $S/\sigma = \{[s]_{\sigma}: s\in S\}$.
\subsection*{Circuits} In next chapters we use different types of gates to build bounded-depth circuits computing Boolean functions. For instance we write $\ccand_{d}$ to denote a gate which takes at most $d$ inputs and computes their conjunction and $\ccor_{d}$ for a gate computing at most $d$-ary disjunction. A $\ccmod_m$-type boolean gate is any unbouded fan-in gate, which sums the inputs modulo $m$ and returns $1$ iff the sum belongs to some accepting set $S\subseteq \z_m$. We allow different accepting sets for different gates. Other kinds of gates appearing in the paper are $\sumpk{p}{\nu}$ gates which also rely on modulo counting, but in a more complex way. Each such a gate takes $n$ inputs $b_1, \ldots, b_n$, and computes their affine combination $\alpha_1 b_1 + \ldots + \alpha_n b_n + d$  in $\z_p^v$. Here we interpret each $b_i$ as $v$-dimensional vector $(b_i, \ldots, b_i)$. We allow each $\alpha_i$ to be arbitrary endomorphism of the abelian group $\z_p^v$, which can be also viewed as arbitrary $v \times v$ matrix with coefficients from $\z_p$. Hence $\alpha_i b_i$ can be viewed as applying linear map $\alpha_i$ to $(b_i, \ldots, b_i)$.  In this way $\sumpk{p}{\nu}$ gate computes a function of type $\{0,1\}^n \rightarrow \z_{p}^{\nu}$. Additionally, for $c\in \z_p$ let $\sumpk{p}{\nu}^c$ denote a boolean variant of $\sumpk{p}{\nu}$ gate which returns value $1$ when the affine combination $\alpha_1 b_1 + \ldots + \alpha_n b_n + d$ evaluates to $c$, and returns $0$ otherwise. Note that for a field $\mathbb{F}$ with underlying group $\z_p^{v}$, every scalar $g \in F$ defines a linear map $x \mapsto g \cdot x$. Thus such $g$ (or rather its associated endomorphism) can be used as a coefficient $\alpha_i$ inside $\sumpk{p}{\nu}$ gate.  To simplify some of the later calculations, we will always assume that vector $(1,\ldots, 1)$ is the unit of such a field $\mathbb{F}$.  Note that for a field $\mathbb{F} = (F, \cdot, +, 0, 1)$, for each $e \in F$ we can define a new field $\mathbb{F}_e = (F, \circ, +, 0, e)$, by defining $x \circ y$ to be $x \cdot e^{-1} \cdot y $. 


It is indeed a valid assumption, since for every  non-zero element $g\in \z_p^{v}$ which is not a unit of the field $\mathbb{F}$ over $\z_p^{v}$, we can redefine multiplication to be $x \cdot g^{-1} \cdot y$ in terms of old multiplication, and fix the new inverse of an element $x$ to be $x^{-1}\cdot g^{2}$. One can easly check that such a rewriting of $\mathbb{F}$ defines a new field in which $g$ plays role of the unit (and we did not alter the underlying group).

Having all these gates, we build a bounded-depth circuits by listing types of gates which are allowed on each layer, starting with the input layer on the left, finishing with the output layer on the right. For instance $\ccand_d\circ\ccmod_m\circ\ccmod_p$ denotes a $3$-level circuit with inputs wired to some set of $\ccand_d$ gates, then this  $\ccand_d$ gates have wires to $\ccmod_m$ gates and on the output level there is one $\ccmod_p$ gate. We allow multiple wires between two gates.

\section{Useful algebraic facts}
\label{section:collection}

Before dealing with the remaining parts of the proofs, we present a number of lemmas describing certain useful aspects of the local behaviour of algebras. 


\begin{lm}
\label{lm:simple-atom}
Let $\m A$ be a finite algebra and $\beta$ be an atom in $\con A$.
Then for each element $\zero\in A$ the induced algebra $\m A|_{\zero/\beta}$
on the coset $\zero/\beta$
is either trivial or simple.
\end{lm}
\begin{proof}\textcolor{red}{TOPROVE 4}\end{proof}


The next lemma is a specialized version of some much more general facts known in Universal Algebra (see for instance \cite[Corollary 5.8]{fm}).


\begin{lm}
\label{lm:simple-module-atom}
Let $\m A$ be a finite nilpotent algebra from a congruence modular variety
with a Malcev term $\po d(x,y,z)$
and $\beta$ be an abelian atom in $\con A$.
Then for each element $\zero\in A$ the induced algebra $\m A|_{\zero/\beta}$ on the coset $\zero/\beta$ is polynomially equivalent to a simple module in which the underlying group structure is determined by the binary operation $x+y=\po d(x,\zero,y)$
and the corresponding group $(\zero/\beta; +,\zero)$
is isomorphic to some power of the group $(\z_p;+,0)$.
\end{lm}



\begin{proof}\textcolor{red}{TOPROVE 5}\end{proof}

Now we present yet another simple observation.


\begin{lm}
\label{lm:minset}
Let $\m A$ be a finite nilpotent Malcev algebra
and $\delta\prec\theta$ two of its congruences.
Then every element $e\in A$ belongs to some $(\delta,\theta)$-minimal set.
\end{lm}

\begin{proof}\textcolor{red}{TOPROVE 6}\end{proof}



Now we demonstrate that easiness of \progcsat{} can be transferred to quotients. 
\begin{fact}

\label{fact:quotient}
For an algebra $\m A$ and its congruence $\theta$
there is a polynomial time reduction from \progcsat{\m A/\theta} to \progcsat{\m A}.
\end{fact}

\begin{proof}\textcolor{red}{TOPROVE 7}\end{proof}

The following simple fact provides us with a way to reason about characteristics of intervals below a join irreducible congruence in the congruence lattice of an algebra.

\begin{fact}\label{fact:ji-PUPI}
Let $\m A$ be a solvable Malcev algebra and $\alpha,\beta\in \con{A}$ such that $\intv{\alpha}{\beta}$ is a \pupi.

Then,  for every $\gamma\in\con{A}$ join irreducible in $\intv{\alpha}{\beta}$ it follows that $|\charrset{\alpha,\gamma}|=1$.
\end{fact}
\begin{proof}\textcolor{red}{TOPROVE 8}\end{proof}


\section{ProgramCSat -- hardness}
\label{section:hard}


The next Lemma is modeled after  Lemma 3.1 from \cite{ikk:mfcs}. It provides us with a normal form for all the $s$-ary functions of type $\z_m^s \map \z_p$.

\begin{lm}
\label{lm-zpqe}
Let $m$ be a square-free positive integer and $p\nmid m$ be a prime.
Then every function $f: \z_m^s \map \z_p$ can be expressed by
\[
f(x_1,\ldots,x_s)=\sum_{(\o\beta,u)\in\m \z_m^s\times\z_m}
\mu_{\o\beta,u}\cdot \b\left(\bigoplus_{i=1}^s \beta_i x_i\oplus u\right),
\]
where
\begin{itemize}
    \item $\b:\z_m\map\z_p$ is given by $\b(0)=1$ and $\b(x)=0$ for $x \neq 0$,
    \item $\mu_{\o\beta,u}\in \z_p$,
    \item $\sum$ is the addition from $\mathbb{Z}_p$,
    \item $\bigoplus$ is the addition from $\mathbb{Z}_m$.
\end{itemize}
Moreover, coefficients $\mu_{\o\beta,u}\in \z_p$ are computable in $2^{O(s)}$ steps.
\end{lm}
\begin{proof}\textcolor{red}{TOPROVE 9}\end{proof}


\begin{proof}\textcolor{red}{TOPROVE 10}\end{proof}



\begin{lm}
\label{lm:beta-int}
Let $\m A$ be a finite nilpotent Malcev algebra,
$\alpha$ its join irreducible congruence with $\alpha^-$ being its unique subcover.
Moreover let $\beta\in \con A$ be a subcover of $\alpha^-$ with $\charr(\beta,\alpha^-)\neq\charr(\alpha^-,\alpha)$.
Then for any choice of $(c,d)\in\alpha\setm\alpha^-$ and $(e,a)\in\alpha^-\setm\beta$
every function $f: \set{c,d}^s \map \set{e,a}$ can be $\beta$-interpolated
by an $s$-ary polynomial $\po p$ of $\m A$,
i.e. $f(\o x)\congruent{\beta}\po p(\o x)$, whenever $\o x \in\set{c,d}^s$.
Moreover such a polynomial $\po p$ can be obtained from the function $f$ in $2^{O(s)}$ steps.
\end{lm}

\begin{proof}\textcolor{red}{TOPROVE 11}\end{proof}







\begin{proof}\textcolor{red}{TOPROVE 12}\end{proof}








\section{ProgramCSat -- easiness}
\label{section:easy}



We start this chapter by presenting number of usefull facts about low-depth circuits  which use only arithmetic operations and conjunction. Similar results can be found for instance in \cite{Grolmusz01, GrolmuszT00}.



\begin{lm}\label{lm:and-sum}
Every function of the form $\set{0,1}^n \map \z_p^k$
can be computed by an $\ccand_n\circ\sumpk{p}{k}$-circuit of size at most $2^n$.
\end{lm}

\begin{proof}\textcolor{red}{TOPROVE 13}\end{proof}

\begin{lm}\label{lm:normal-form}
Let $m$ be a square-free positive integer, $p\nmid m$ be a prime.
Then every function computable by a $\ccmod_m\circ\ccand_d$-circuit
can be also computed by a $\ccmod_m\circ\sumpk{p}{1}$-circuit of size $O(m^{d+1})$.
\end{lm}
\begin{proof}\textcolor{red}{TOPROVE 14}\end{proof}


\begin{lm}
\label{lm:unmod}
Every function computable by a $\ccmod_m\circ\ccmod_p$-circuit of size $\csize$
can be also computed by a $\ccmod_m\circ\sumpk{p}{1}$-circuit of size $O(m^p\csize^p)$.
\end{lm}
\begin{proof}\textcolor{red}{TOPROVE 15}\end{proof}


\begin{lm}\label{lm:apply_func}
Let $m$ be a square-free positive integer, $p\nmid m$ be a prime and $g$ be a $k$-ary boolean function.
If the functions $f_1, \ldots, f_k$ are computable by
$\ccmod_m\circ\ccmod_p$-circuits
of size $O(\lambda)$
then also $g(f_1,\ldots,f_k)$ is computable by such a circuit,
but of size $O(\lambda^{kp})$.
\end{lm}
\begin{proof}\textcolor{red}{TOPROVE 16}\end{proof}


\begin{lm}\label{lm:5to3}
Let $m$ be a square-free positive integer, $p\nmid m$ be a prime and $\ccc\in\z_p^\nu$.
Then every $\ccand_d\circ\ccmod_m\circ\ccmod_p
\circ\ccand_{d'}\circ\sumpk{p}{\nu}^{\ccc}$-circuit of size $\lambda$
can be replaced a
$\ccand_d\circ\ccmod_m\circ\ccmod_p$-circuit
of size $O(\lambda^{\nu d'p^3})$.
\end{lm}
\begin{proof}\textcolor{red}{TOPROVE 17}\end{proof}







Now we present a characterization of Boolean functions that can be represented in supernilpotent algebras. 


\begin{lm}
\label{lm:supernil-circuit}
Let $\m A$ be a finite supernilpotent Malcev algebra.
Then the functions computable by an $n$-ary boolean program
$\progg{\po p}{n}{\iota}{S}$ over the algebra $\m A$
can be also computed by
$\ccand_d\circ\ccmod_{\pdiv A}\circ\ccor_{\card S}$-circuits of size $O(n^d)$ with
$d\leq\card{A}^{1+\log\maxar A} = (2 \cdot \maxar A)^{\log\card{A}}$.
\end{lm}



\begin{proof}\textcolor{red}{TOPROVE 18}\end{proof}

\begin{thm}[\cref{thm:2supernil-circuit-early} restated]
\label{thm:2supernil-circuit}
Let $\m A$ be a finite nilpotent Malcev algebra with $\charrset{0,\kappa}=\set{p}$.
Then the function computable by a boolean program
$\progg{\po p}{n}{\iota}{S}$ of size $\ell$ over the algebra $\m A$
can be also computed by an
$\ccand_d\circ\ccmod_{\pdiv A/p}\circ\ccmod_p$-circuits of size $O(\ell^c)$
with $d\leq\card{A}^{1+\log\maxar A}$
and the degree $c$  bounded by $O(\card{A}\cdot\maxar{A})^{O(\log\card{A})}$.
\end{thm}



\begin{proof}\textcolor{red}{TOPROVE 19}\end{proof}


Note here that the degree of the polynomial bounding the size of the circuit,
i.e.  $O(\card{A}\cdot\maxar{A})^{O(\log\card{A})}$ have two sources.
The $\log\card{A}$ comes from the number of iterative use of claim (\ref{section:easy}.\ref{mod-beta}), i.e. from the hight of the congruence lattice of $\m A$.
The $(\card{A}\cdot\maxar{A})$-part is a consequence of arity of a binary expansion
$\fcirc{f} : \bool^{\card{A}\cdot\ar{f}} \map A$ coding basic operations $f$.
The careful reader can easily note some room for improvement here to
$O(\log\card{A}\cdot\maxar{A})^{O(\log\card{A})}$ by coding the elements of $A$ with $\log\card{A}$ bits and therefore shrinking the arity of $\fcirc{f}$ to $\log\card{A}\cdot\ar{f}$.
\section{Final Remarks}

Constant Degree Hypothesis plays crucial role in our proofs of \rptime algorithms existence. One can ask if this assumption is really needed. In fact, there are some  unconditional results. For example very recent paper \cite{KawalekKK19} shows that \ceqv{\m A} is in $\ptime$ whenever $\m A$  from CM is $2$-nilpotent, i.e. it has abelian congruence with abelian quotient. Also supernilpotent algebras admit (unconditional) polynomial-time algorithm for \csat{}/\ceqv{} \cite{aichmud-2010, komp2017, IdziakK22}, and if we allow random bits the time complexity drops down to linear \cite{KawalekK}. Unfortunately, it is not hard to construct for a given $d$, $m$, $p$ an algebra $\m A$ with supernilpotent rank equal $2$ such that functions computable by $\ccand_d \circ \ccmod_m \circ \ccmod_p$-circuit are exactly functions computable by, not too long, programs over $\m A$.  This, together with our results, shows that (under ETH) showing unconditional algorithms solving \progcsat{} for algebras from congruence modular variety with supernilpotent rank equal $2$ is equivalent to proving CDH. Hence, the natural question is if CDH holds.
\begin{prob}
Prove or disprove the Constant Degree Hyphothesis.
 \end{prob}

The natural next step in our investigations is to go outside congruence modular realm. The first problem in such a case is that Tame Congruence Theory and Commutator Theory (our heavily used tools) for arbitrary algebras do not work as well as for algebras from congruence modular varieties.  Moreover \cite[Example 2.8]{IdziakK22} shows that there is an algebra $\m A$ (not contained in congruence modular variety) and its congruence $\sigma$ such that $\csat{\m A}$ is in $\ptime$, while $\csat{\m A/\sigma}$ is \npc. This suggests that we cannot expect a nice characterization of polynomial-time cases for \csat{} outside CM. On the other hand, we are not aware of any such examples for \progcsat{} and \ceqv{}. In fact we saw that the hardness of $\progcsat{\m A/\sigma}$ implies the hardness for $\progcsat{\m A}$. This gives hope for characterization of tractable cases of \progcsat{} for general finite algebras.

\begin{prob}
 Characterize finite algebras with tractable \progcsat{}/\ceqv{}.
 \end{prob}
 

\bibliographystyle{alpha}
  \bibliography{equations}
\end{document}
