\section{Proof of \Cref{quasi_1}}
\label{sec: quasi_exact}

In this section, we provide the proof of \Cref{quasi_1}. 

We first prove the upper bound that every quasi-bipartite graph with $k$ terminals admits an exact contraction-based cut sparsifier on $k^{O(k^2)}$ vertices. 
Let $G$ be a quasi-bipartite graph with a set $T$ of $k$ terminals.
Throughout, we will assume that for every $\emptyset \subsetneq S\subsetneq T$, there is a unique min-cut in $G$ separating $S$ from $T\setminus S$. This can be achieved by slightly perturbing the edge weights $\set{w(e)}_{e\in E(G)}$.

We use the definition of \emph{profiles} studied in previous works \cite{hagerup1998characterizing,khan2014mimicking}.
For a vertex $v\in V(G)$, its profile $\pi^v$ is defined to be a $(2^{|T|}-2)$-dimensional vector whose coordinates are indexed by proper subsets $S$ of $T$. Specifically, for every $\emptyset \subsetneq S\subsetneq T$, $\pi^v_S=1$ iff $v$ lies on the side of $S$ in the $(S,T \setminus S)$ min-cut in $G$, otherwise $\pi^v_S=0$. 
Let $\Pi(G)$ be the collection of all distinct profiles of the vertices in $G$.
The following result was proved in \cite{hagerup1998characterizing}.

\begin{lemma}
Every graph $G$ admits a quality-$1$ contraction-based cut sparsifier of size $|\Pi(G)|$.
\end{lemma}

Therefore, to prove \Cref{quasi_1}, it is sufficient to prove the following lemma.

\begin{lemma} \label{quasi_profile}
For any quasi-bipartite graph $G$ with $k$ terminals, $|\Pi(G)|=k^{O(k^2)}$.
\end{lemma}

To prove \Cref{quasi_profile}, we will show that $\Pi(G)$, when viewed as a family of sets (instead of vectors), has bounded VC-dimension.
Specifically, consider the ground set $\uset=\set{S\mid \emptyset \subsetneq S\subsetneq T}$.
Now each profile $\pi^v$ naturally defines a subset of the ground set $\uset$: $\Pi^v=\set{S\mid \pi^v_S=1}$.
Abusing the notation, we also write $\Pi(G)=\set{\Pi^v\mid v\in V(G)}$.
We will show that the set family $\Pi(G)$ on the ground set $\uset$ has VC-dimension $O(k \log k)$. Then from Sauer-Shelah lemma \cite{shelah1972combinatorial,sauer1972density},  
$$|\Pi(G)|\le |\uset|^{O(k \log k)}\le (2^k)^{O(k \log k)} = k^{O(k^2)}.$$  

\paragraph{Shattering sets and VC dimension.} Let $\fset$ be a set family on the ground set $\uset$.
Let $U$ be a subset of $\uset$. We say that the family $\fset$ \emph{shatters} $U$, iff for every subset $U'\subseteq U$, there exists some set $F\in \fset$, such that $F\cap U=U'$. The VC-dimension of the family $\fset$ is the maximum size of a set $U$ shattered by $\fset$.

The remainder of this section is dedicated to the proof of the following claim.

\begin{claim}
\label{clm: VC-dim}
The VC-dimension of $\Pi(G)$ on the ground set $\uset$ is $O(k \log k)$.
\end{claim}
\begin{proof}
For every subset $S \subseteq T$, we denote by $w_v(S)$ the total weight of all the edges between $v$ and the terminals in $S$. Since $G$ is a quasi-bipartite graph, for every $S$, the value of $\pi^v_S$ only depends on $w_v(S)$ and $w_v(T\setminus S)$. Specifically, for every $v$, $\pi^v_S=1$ iff $w_v(S)>w_v(T\setminus S)$.

Recall that the ground set $\uset$ contains all proper subsets of $T$ as its elements. Here we consider the subsets of $\uset$. We say that two subsets $\uset_1,\uset_2\subseteq \uset$ are \emph{similar}, iff
\begin{itemize}
\item $\uset_1,\uset_2$ are disjoint subsets of $\uset$;
\item $|\uset_1|=|\uset_2|$; and
\item for every terminal $t\in T$, $\big|\set{S\in \uset_1\mid t\in S}\big|=\big|\set{S\in \uset_2\mid t\in S}\big|$.
\end{itemize}
The proof of \Cref{clm: VC-dim} is completed by the following two observations.


\begin{observation}
Let $\uset_1,\uset_2$ be a similar pair. Then any subset $\uset'\subseteq \uset$ that contains both $\uset_1$ and $\uset_2$ is not shattered by $\Pi(G)$.
\end{observation}
\begin{proof}
As $\uset_1$ and $\uset_2$ are similar, for every non-terminal $v\in V(G)$, $\sum_{S\in \uset_1} w_{v}(S) = \sum_{S \in \uset_2} w_{v}(S)$ and $\sum_{S \in \uset_1} w_v(T \setminus S) = \sum_{S \in \uset_2} w_v(T \setminus S)$ must hold. 
If there exists a set $\Pi^v\in \Pi(G)$ with $\Pi^v\cap \uset'=\uset_1$, then by definition,
$\pi^v_S=1$ for all $S \in \mathcal{U}_1$ and $\pi^v_S=0$ for all $S \in \mathcal{U}_2$. 
However, this means that
\[
\sum_{S\in \uset_1} w_{v}(S)> |\uset_1|\cdot \frac{w_{v}(T)}{2}= |\uset_2|\cdot \frac{w_{v}(T)}{2}
> \sum_{S\in \uset_2} w_{v}(S) = \sum_{S\in \uset_1} w_{v}(S),
\]
a contradiction.
\end{proof}

\begin{observation}
Every subset $\uset'\subseteq \uset$ with $|\uset'|>2k\log k$ admits a similar pair $\uset_1,\uset_2\subseteq \uset'$.
\end{observation}
\begin{proof}
Let $\uset'$ be a subset of $\uset$ with $|\uset'|>k\log k$.
First, there are $\binom{|\uset'|}{\floor{|\uset'|/2}}$ size-$\floor{|\uset'|/2}$ subsets of $\uset'$. Second, for such each such subset $\uset''$, if we define the $T$-dimensional vector $\alpha[\uset'']=(\alpha[\uset'']_t)_{t\in T}$ as
$\alpha[\uset'']_t=\big|\set{S\in \uset''\mid t\in S}\big|$, then there are a total of $(\floor{|\uset'|/2}+1)^k$ possible vectors $\alpha[\uset'']$.

This impies that, when $|\uset'|>2k\log k$, $\binom{|\uset'|}{\floor{|\uset'|/2}}>(\floor{|\uset'|/2}+1)^k$, and so there must exist a pair $\mathcal{U}'_1,\mathcal{U}'_2$  of distinct subsets of $\mathcal{U}'$, such that $\card{\mathcal{U}_1} = \card{\mathcal{U}_2} = \floor{|\uset'|/2}$ and the vectors $\alpha[\uset_1],\alpha[\uset_2]$ are identical.
It is then easy to verify that $\uset_1\setminus \uset_2,\uset_2\setminus \uset_1$ are a similar pair, both contained in $\uset'$. 
\end{proof}
\end{proof}


We now prove the lower bound that for every integer $k$, there exists a quasi-bipartite graph with $k$ terminals whose exact contraction-based cut sparsifier must contain $\Omega(2^k)$ vertices.


We construct a graph $G$ as follows. The terminal set contains $k$ vertices and is denoted by $T$. For every subset $\emptyset\subsetneq S \subsetneq T$ with even size, we add a new vertex $v_S$ and connect it to every vertex of $S$.
Every edge is given a capacity that is chosen uniformly at random from the interval $(1-2^{k},1+2^{-k})$. In this way we can guarantee that with high probability, sets $E',E''$ of edges of $G$ have the same total weight iff $E'=E''$.
And clearly, $G$ is a quasi-bipartite graph with $|V(G)|=\Omega(2^k)$.

The proof of the $\Omega(2^k)$ size lower bound is completed by the following claims.

\begin{claim}
\label{clm: profiles}
Every pair of distinct vertices have different profiles.
\end{claim}
\begin{proof}
Consider a pair $v=v_S,v'=v_{S'}$ of vertices in $G$.
As $G$ is a quasi-bipartite graph, the profile of $v$ only depends on incident edges of $v$.
We will show that there always exists a subset $U\subseteq T$ of terminals, such that $v,v'$ lie on different sides of the $(U,T\setminus U)$ min-cut in $G$.

Recall that $S,S'$ are distinct subsets of $T$ with even size. Assume w.l.o.g. that $S\setminus S'\ne \emptyset$. If $|S\cap S'|$ is even, then we let $U$ contains exactly half of the elements in $S\cap S'$. Now if $v$ lies on the $U$ side of the $(U,T\setminus U)$ min-cut in $G$, then we are done, as $v'$ lies on the $T\setminus U$ size of the min-cut, then we are done we add to $U$. Otherwise, we add all elements in $S\setminus S'$, ensuring that $v$ lies on the $T\setminus U$ side of the $(U,T\setminus U)$ min-cut while $v'$ lies on the $U$ side.
Similarly, if $|S\cap S'|$ is even, then we let $U$ contains $\frac{|S\cap S'|+1}{2}$ elements in $S\cap S'$, and we can adjust $U$ similarly (by optionally adding to it all elements of $S\setminus S'$) to ensure that $v,v'$ lie on different sides of the $(U,T\setminus U)$ min-cut in $G$.
\end{proof}



\begin{claim}
No pair of vertices can be contracted in an exact contraction-based cut sparsifier of $G$.
\end{claim}
\begin{proof}
Assume for contradiction that vertices $v_S,v_{S'}$ are contracted. We have shown in \Cref{clm: profiles} that there exists a subset $U$ of $T$, such that vertices $v_S,v_{S'}$ lie on different sides of the $(U,T\setminus U)$ min-cut in $G$.
Therefore, if $v_S,v_{S'}$ are merged in the sparsifier, then the set $E''$ of edges constituting the $(U,T\setminus U)$ min-cut in the cut sparsifier $H$ must be different from the set $E'$ of edges constituting the $(U,T\setminus U)$ min-cut in $G$, and therefore they must have different total weight (meaning that $H$ cannot be an exact cut sparsifier for $G$), since we have assumed that sets $E',E''$ of edges of $G$ have the same total weight iff $E'=E''$.
\end{proof}









\iffalse
In this section, we prove that any quasi-bipartite graph with $k$ terminals admits an exact contraction-based cut sparsifier of size $k^{O(k^2)}$. We used the framework of \cite{khan2014mimicking}: for any vertex $v$, and for any set of terminals $S \subseteq T$, denote $P_S(v)=1$ if in the min-cut that cuts $(S,T \setminus S)$, $v$ is on the side of $S$ (if there are different min-cuts, we choose the min-cut such that the side of $S$ has the least number of vertices, we can see later there are only one such min-cut), otherwise $P_S(v)=0$. We contract together the vertices that has the same $P_S$ value for every $S$. By doing so, for any $S \subseteq T$, any pair of vertices that on different side of the min-cut of $(S, T \setminus S)$ are not contracted together, so the min-cut size is not changed. Therefore the resulting graph is an exact cut sparsifier of the original graph. Therefore, we only need to bound the number of different possible $P_S$ values. Thus, to prove \Cref{quasi_1}, it is sufficient to prove the following lemma:

\begin{lemma} \label{quasi_profile}
    For any quasi-bipartite graph with $k$ terminals, there are at most $k^{O(k^2)}$ different $P_S$ values.
\end{lemma}

We can view different $P_S$ values as a set family, where the elements are the subsets of $T$, and for each $v$, we define the set $C_v$ as the set of $S$ such that $P_S(v)=1$. We prove that the VC-dimension of the set family is $O(k \log k)$, and thus by Sauer-Shelah lemma, the number of different $P_S$ values are at most $(2^k)^{O(k \log k)} = k^{O(k^2)}$.  

For any $S \subseteq T$, denote $w_S(v)$ as the total weight of the edges between $v$ and the terminals in $S$. Since the graph is a quasi-bipartite graph, there is no edge between no-terminals, which means for any $S$, the side of $v$ in the min-cut of $(S,T \setminus S)$ only depends on the weights of edges incident on $v$. Therefore, for any $v$, $P_S(v)=1$ iff $w_S(v) > w_{T \setminus S}(v)$ (note that when $w_S(v) = w_{T \setminus S}(v)$, $v$ is not on the side of $S$ because we choose the min-cut that minimizes the side of $S$). 

For any group of sets $\mathcal{S}$ and terminal $t$, denote $C_{\mathcal{S}}$ as the number of sets in $\mathcal{S}$ that contains $t$. Now consider two disjoint groups of sets $\mathcal{S}_1$ and $\mathcal{S}_2$ such that $\card{\mathcal{S}_1} = \card{\mathcal{S}_2}$ and for any terminal $t$, $C_{\mathcal{S}_1}(t)=C_{\mathcal{S}_2}(t)$. In other words, for any $v$, $\sum_{S\in \mathcal{S}_1} w_{S}(v) = \sum_{S \in \mathcal{S}_2} w_{S}(v)$ and $\sum_{S \in \mathcal{S}_1} w_{T \setminus S}(v) = \sum_{S \in \mathcal{S}_2} w_{T \setminus S}(v)$. Therefore, it is impossible that $P_{S}(v)=1$ for all $S \in \mathcal{S}_1$ and $P_{S}(v)=0$ for all $S \in \mathcal{S}_2$. This means that for any group of sets $\mathcal{S}$ that $\mathcal{S}_1,\mathcal{S}_2 \subseteq \mathcal{S}$, the set family cannot shatter $\mathcal{S}$.

Now consider any group of sets $\mathcal{S}$ that is shattered by the set family and contains even number of sets. Let $s=\card{\mathcal{S}}$. Consider a pair of groups $\mathcal{S}_1,\mathcal{S}_2 \subseteq \mathcal{S}$ such that $\card{\mathcal{S}_1} = \card{\mathcal{S}_2} = s/2$. If for any $t$, $C_{\mathcal{S}_1}(t) = C_{\mathcal{S}_2}(t)$, then $C_{\mathcal{S}_1 \setminus \mathcal{S}_2}(t) = C_{\mathcal{S}_2 \setminus \mathcal{S}_1}(t)$. Moreover, $\mathcal{S}_1 \setminus \mathcal{S}_2$ and $\mathcal{S}_2 \setminus \mathcal{S}_1$ are disjoint, which means the set family cannot shatter $\mathcal{S}$, a contradiction. Thus, for any pair of $\mathcal{S}_1$ and $\mathcal{S}_2$ such that $\card{\mathcal{S}_1} = \card{\mathcal{S}_2} = s/2$, there exists a $t$ such that $C_{\mathcal{S}_1}(t) \neq C_{\mathcal{S}_2}(t)$. For any $s/2$ size subset of $\mathcal{S}$, the number of sets that contains a terminal $t$ is at most $s/2$, thus there are at most $(s/2)^k$ different $C$ values for such subsets. On the other hand, the total number of $s/2$ subset of $\mathcal{S}$ is $\binom{s}{s/2} = 2^s/\sqrt{s}$, therefore, $2^s/\sqrt{S} \le (s/2)^k$. By calculation, we have $s/\log s < k + 1/2$ and thus $s < 2 k \log k$. This means that the VC-dimension of the set family is $O(k \log k)$ and therefore the number of different $P_S$ values is $k^{O(k^2)}$, which finishes the proof of \Cref{quasi_profile}.
\fi