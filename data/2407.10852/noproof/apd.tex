\section{Missing Proofs}

\subsection{Proof of \Cref{lem: divide}}
\label{apd: Proof of lem: divide}

Graph $H'$ is constructed as follows. The vertex set is $V(H')=\bigcup_{G'\in \gset}V(H_G')$. The terminal set is the union of $T$ and the set of all vertices of $G$ that appear in more than one graphs in $\gset$. The edge set of $H'$ is simply the union of all edges $E(H_{G'})$. Note that $V(H')$, according to our definition, indeed contains all vertices in any graph $G'\in \gset$, so the $H'$ is well-defined, and it is easy to verify that $|V(H')|\le \sum_{G'\in \gset}|V(H_{G'})|$.

It remains to show that the graph $H'$ constructed above is a $(1+\eps)$ cut sparsifier of $G$ with respect to $T$. Let $(T^1,T^2)$ be a partition of $T$ into two subsets.

On the one hand, consider a min-cut $\hat E$ in $G$ separating $T^1$ from $T^2$, and let $(V^1,V^2)$ be the vertex partition of $V(G)$ formed by the cut $\hat E$, where $T^1\subseteq V^1$ and $T^2\subseteq V^2$. We construct a cut $E'$ in $H'$ with $w(E')\le (1+\eps)\cdot w(\hat E)$ as follows. 
For each $G'\in \gset$, define $T^1_{G'}=T_{G'}\cap V^1$ and $T^2_{G'}=T_{G'}\cap V^2$, and let $E'_{G'}$ be a min-cut in $H'_{G'}$ separating $T^1_{G'}$ from $T^2_{G'}$. 
Then we let $E'=\bigcup_{G'\in \gset}E'_{G'}$.

First we show that $w(E')\le (1+\eps)\cdot w(\hat E)$.
Recall that the edge sets $\set{E(G')\mid G'\in \gset}$ partition $E(G)$. For each $G'\in \gset$, denote $\hat E_{G'}=\hat E\cap E(G')$, so the sets $\set{\hat E_{G'} \mid G'\in \gset}$ partitions $\hat E$ and so $w(\hat E)=\sum_{G'\in \gset}w(\hat E_{G'})$.
Note that for each $G'\in \gset$, the set $\hat E_{G'}$ separates $T^1_{G'}$ from $T^2_{G'}$, and by definition we have $w(E'_{G'})\le (1+\eps) w(\hat E_{G'})$. Altogether, $w(E')\le (1+\eps)\cdot w(\hat E)$.

Second we show that $E'$ indeed separates $T^1$ from $T^2$ in $H'$. For each $G'\in \gset$, denote by $(V^1_{G'},V^2_{G'})$ the partition formed by the cut $E'_{G'}$, where $T^1_{G'}\subseteq V^1_{G'}$ and $T^2_{G'}\subseteq V^2_{G'}$. We let $V^1_{H'}=\bigcup_{G'\in \gset}V^1_{G'}$ and $V^2_{H'}=\bigcup_{G'\in \gset}V^2_{G'}$.
It is easy to verify that (i) $V^1_{H'}$ and $V^2_{H'}$ partition $V(H')$; (ii) $T^1\subseteq V^1_{H'}$; (iii) $T^2\subseteq V^2_{H'}$; and (iv) all edges in $E_{H'}(V^1_{H'},V^2_{H'})$ lies in $E'$, so $E'$ indeed separates $T^1$ from $T^2$ in $H'$.

On the other hand, consider a min-cut $E'$ in $H'$ separating $T^1$ from $T^2$, and let $(V_{H'}^1,V_{H'}^2)$ be the corresponding vertex partition of $V(H')$, where $T^1\subseteq V^1$ and $T^2\subseteq V^2$. 
We construct a cut $\hat E$ in $G$ with $w(\hat E)\le (1+\eps)\cdot w(E')$ as follows. 
For each $G'\in \gset$, define $T^1_{G'}=T_{G'}\cap V^1_{H'}$ and $T^2_{G'}=T_{G'}\cap V^2_{H'}$, and let $\hat E_{G'}$ be a min-cut in $G'$ separating $T^1_{G'}$ from $T^2_{G'}$. 
Then we let $\hat E=\bigcup_{G'\in \gset}\hat E_{G'}$.
The arguments for showing that $w(\hat E)\le (1+\eps)\cdot w(E')$ and that $\hat E$ indeed separates $T^1$ from $T^2$ in $G$ are symmetric.

\subsection{Proof of \Cref{lem: pattern cover}}
\label{apd: Proof of lem: pattern cover}

Let $P$ be the $\Phi$-pattern shortest path connecting $v$ to any vertex on $R$. Let $z$ be the $R$-endpoint of $P$, and let $\ell$ be the total length of $P$.
Vertex $z$ separates $R$ into two subpaths, that we call the \emph{forward} subpath from $z$ to the one endpoint of $R$ and the \emph{backward} subpath from $z$ to the other endpoint of $R$.
We now iteratively computes a set $C_F$ of vertices in the forward subpath as follows.
Initially set $C_F=\emptyset$ and a variable vertex $u=z$. In each iteration, we find the first vertex $u'$ lying in the forward direction from $u$, such that 
$\dist^{\Phi}(v,u)+\dist(u,u')> (1+\eps)\cdot \dist^{\Phi}(v,u')$. We add $u$ to $C_F$ and update $u\leftarrow u'$. We continue until we cannot find any such $u'$ in the forward path and obtain the resulting $C_F$.
We then computes a set $C_B$ of vertices in the back subpath in the same way, and eventually returns $C(v,R,\Phi)=C_F\cup C_B$ as the $\Phi$-respecting $\eps$-cover.
Clearly, by definition we indeed get a $\Phi$-respecting $\eps$-cover of $v$ in $R$. The next claim shows that $|C(v,R,\Phi)|=O(1/\eps)$.

\begin{claim}
The number of iterations  is $O(1/\eps)$.
\end{claim}
\begin{proof}\textcolor{red}{TOPROVE 0}\end{proof}





