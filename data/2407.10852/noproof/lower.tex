\section{Proof of \Cref{main: lower}}
\label{sec: lower}

In this section we prove \Cref{main: lower}, showing that for every $k\ge 1$ and $\eps>0$, there exists a quasi-bipartite graph with $k$ terminals, whose quality-$(1+\eps)$ contraction-based cut sparsifier must contain $k^{\Omega\big(\frac{1}{\eps \log(1/\eps)}\big)}$ vertices. We will make use of the following intermediate problem, which is a variant of the $0$-Extension problem.

\newcommand{\bhc}{\textnormal{\textsf{BHC}}}

\paragraph{Boolean Hypercube Contraction ($\bhc$).}
In an instance of $\bhc$, the input consists of
\begin{itemize}
	\item a graph $G=(V,E)$ with $V=\set{0,1}^d$, namely each vertex is a $d$-dimensional $0/1$-string;
	\item a set $T\subseteq V$ of $k$ terminals, such that every edge is incident to some terminal.
\end{itemize}
A solution is a mapping $f:V\to V$, such that for each terminal $t\in T$, $f(t)=t$. For solution $f$, its
\begin{itemize}
\item \emph{cost} is defined as $\vol(f)=\sum_{(u,v)\in E}||f(u)-f(v)||_1$; and
\item \emph{size} is defined as $|\set{f(v)\mid v\in V}|$, namely the size of its image set.
\end{itemize}

The goal is to compute a solution $f$ with small size and cost. Specifically, the following ratio called \emph{stretch}, is to be minimized:
\[\rho=\frac{\sum_{(u,v)\in E}||f(u)-f(v)||_1}{\sum_{(u,v)\in E}||u-v||_1}.\]

In fact, the $\bhc$ problem is the special case of the $0$-Extension problem first introduced by Karzanov \cite{karzanov1998minimum}, as a generalization of the multi-way cut problem and a special case of the metric labeling problem.
The $0$-Extension problem and some of its variants were shown to be concretely connected to cut/flow sparsifiers \cite{moitra2009approximation,leighton2010extensions,andoni2014towards,chen2024lower,chen20241+}. Specifically, $\bhc$ focuses on searching for ``canonical solutions'' of the \emph{$0$-Extension with Steiner Nodes} problem studied \cite{chen2024lower}, and moreover, instead of allowing the input graph to be any general graph, $\bhc$ only consider instances on the boolean hypercube.


The proof of \Cref{main: lower} consists of the following two lemmas.

\begin{lemma}
If for every $k\ge 1$ and $q\ge 1$, every $k$-terminal quasi-bipartite graph admits a quality-$q$ contraction-based cut sparsifier of size $s(k,q)$, then every $k$-terminal quasi-bipartite instance of $\bhc$ admits a solution of stretch $q$ and size $s(k,q)$.
\end{lemma}

\begin{proof}\textcolor{red}{TOPROVE 0}\end{proof}

\begin{lemma}
For every $k\ge 1$ and $\eps>0$, there exists a $k$-terminal quasi-bipartite instance of $\bhc$, such that any solution of stretch $(1+\eps)$ and must have size at least $k^{\Omega(1/(\eps\log (1/\eps)))}$.
\end{lemma}

\begin{proof}\textcolor{red}{TOPROVE 1}\end{proof}







\iffalse

In this section, we prove the lower bound for contraction-based $(1+\eps)$-cut sparsifier for quasi-bipartite graphs. We first convert the cut sparsifier problem to a special case of $0$-extension with Steiner nodes problem.

In the 0-extension problem, we are given a metric $D$ on $n$ points. The set of points is deonted by $V$, and there is a terminal set $T \subseteq V$. We are also given a weighted graph $G=(V,E,w)$. We say the weight of the graph is the sum of $D(u,v)\cdot w(u,v)$ for every edge $(u,v)\in E$. The goal is to move each non-terminal point in $V$ to a terminal in $T$, so that the weight of the graph is not increased too much. In 0-extenison with Steiner nodes problem, the input and the goal is the same, but now we can set up several Steiner nodes in the metric, and we can move the points not only to terminals, but also the steiner points we set up in the metric. 

We will focus on the special case of the 0-extension with Steiner nodes problem when the metric is induced by the distances between the vertices of a hyper-cube. Which means that each point, and the potential Steiner nodes we can set up, can all defined by a $d$-dimensional binary vector, and the distance between two point is the $\ell_1$-distance between them.

Given a graph $G$, a terminal set $T$, and $d$ terminal cuts, we can set up a 0-extension with Steiner nodes problem as follows: arbitrarily select a teminal $t_0$, and set the $d$-dimensional vector represent it as the $0$-vector. For each vertex in the graph, the $i^{th}$ entry of the vector is $0$ if it is on the same side with $t_0$ in the terminal cut, otherwise it is $1$. By definition, the weight of the graph in the 0-extension instance is exactly the total weight of the terminal cuts.

The reduction from 0-extension with Steiner nodes problem to the cut sparsifier problem is the reverse of the above process. Given a metric that is induced by the $\ell_1$ distance among binary vectors, a graph whose vertex set is the points in the metric and a terminal sets, we consider the cut sparsifier problem with the same graph and terminal sets. For the $i^{th}$ dimension, let $C_i$ be the terminal cuts such that one side has all the terminal whose $i^{th}$ dimension is $0$ and the other side is all the terminal whose $i^{th}$ dimension is $1$. The weight of graph in the 0-extension instance is at least the total weight of the sum of these terminal cuts. Now suppose we have a $\alpha$-approximation cut sparsifier, the sum of these terminal cuts are at most $\alpha$ times the original weight. So we can construct a $\alpha$-factor 0-extension solution. Therefore, to prove \Cref{main: lower}, it is sufficient to give a 0-extension with Steiner nodes instance whose metric is given by the $\ell_1$ distance among binary vectors.

In the rest of this section, we give a hard instance for 0-extension with Steiner nodes problem. We consider the number of terminals $k$ as the number that is $k=2\binom{d}{\eps d}+2$ for some integer $d$. Let the metric be the $d$ dimensional hypercube and we have two terminal sets $T_0$ and $T_1$, they are the vectors with exactly $\eps d$ entries of 0 or exactly $\eps d$ entries of $1$, and two spectial terminal $t_0$ and $t_1$ whose vector consistants of all 0 and all 1. Let all other points in the graph as the vectors with exactly $d/2$ entries of 1. For any point $x$, we connect $x$ with all terminals whose distance to $x$ is $(1/2-\eps)d$. There are $2\binom{d/2}{\eps d}$ such terminals. We also add two edges of weight $\binom{d/2}{\eps d}$ from $x$ to the spectial terminals $t_0$ and $t_1$. We can see these edges as the shortest paths between $t_0$ and terminals in $T_0$, and the paths between $t_1$ and terminals in $T_1$, that go through $x$. These pahts has length $(1-\eps)d$.

Now consider the case when we only allow $2^{d/6}$ steiner points. Since for any point in the hypercube, there are at most $\sum_{i=0}^{d/100} \binom{d}{i}<2^{d/5}$ points that are at most $d/100$ distance away from it, this means that most of the point in the graph will be moved to a Steiner node that is at least $d/100$ away. For any such point $x$, and it has been moved to a Steiner point $y$. Without lose of generality assume there are at least $d/200$ entries on $x$ is $0$ and on $y$ is $1$. If we randomly sample a terminal $t$ in $T_0$ that has an edge to $x$, in expectation, there are at least $\eps d/100$ entries on $t$ is $0$ but on $y$ is $1$. Thus, the average length of the paths between $T_0$ and $t_0$ that go through $x$ is now at least $(1-49eps/50)d$, therefore, the average length of paths that go through $x$ has increased by a factor of $1+\Omega(\eps)$. Therefore, any solution with $2^{d/6}$ Steiner nodes has a factor of $(1+\Omega(\eps))$. Since we have $k = 2\binom{d}{\eps d}+2$ terminals, so the number of Steiner nodes we need for a $(1+\eps)$-factor solution is $k^{O(1/\eps)}$.

\fi
