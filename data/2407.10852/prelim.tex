\section{Preliminaries}
\label{sec: prelim}

By default, all logarithms are to the base of $2$. 
All graphs are undirected and simple.

Let $G=(V,E,w)$ be an edge-weighted graph, where each edge $e\in E$ has weight $w_e$. 
In this paper, weights can be either representing capacities (in the context of cut values) or lengths (in the context of shortest-path distances).
For a vertex $v\in V$, we denote by $\deg_G(v)$ the degree of $v$ in $G$.
%For each pair $S,T\subseteq V$ of disjoint subsets, we denote by $E_{G}(S,T)$ the set of edges in $G$ with one endpoint in $S$ the other endpoint in $T$.
For a pair $v,v'$ of vertices in $G$, we use $\dist_{G}(v,v')$ (or $\dist_{w}(v,v')$) to denote the shortest-path distance between $v$ and $v'$ in $G$, viewing edge weights as lengths.
%We define the \emph{diameter} of $G$ as $\diam(G)=\max_{v,v'\in V}\set{\dist_G(v,v')}$, and we define the \emph{girth} of $G$, denoted by $\gir(G)$, as the minimum weight of any cycle in $G$.
We may omit the subscript $G$ in the above notations when the graph is clear from the context.

%Weight $w$ on edges, both for distance and capacity

\iffalse
\paragraph{Chernoff Bound.}
We will use the following form of Chernoff bound (see e.g.\cite{dubhashi2009concentration}).
\begin{lemma}[Chernoff Bound] \label{prop:chernoff}
	Let $X_1,\dots X_n$ be independent random variables taking values in $\{0,1\}$. Let $X$ denote their sum and let $\mu=\ex{X}$ denote the sum's expected value. Then for any $\delta>0$, 
	$$
	\pr{X > (1+\delta) \mu} < \left( \frac{e^{\delta}}{(1+\delta)^{1+\delta}}\right)^{\mu}.
	$$
\end{lemma}
\fi



\paragraph{Cut sparsifiers, contraction-based sparsifiers.}
Let $G$ be a graph and let $T$ be a subset of its vertices called \emph{terminals}. 
%
Let $H$ be a graph with $T\subseteq V(H)$. We say that $H$ is a \emph{cut sparsifier} of $G$ with respect to $T$ with \emph{quality} $q\ge 1$, iff for any partition of $T$ into two non-empty subsets $T_1,T_2$, 
\[\mc_H(T_1,T_2)\le \mc_G(T_1,T_2)\le q\cdot\mc_H(T_1,T_2).\]


We will use the following observation, which is immediate from the definition of cut sparsifiers.
\begin{observation}
\label{obs: chain}
If $(G_1,T)$ is a quality-$q_1$ cut sparsifier of $(G,T)$, and $(G_2,T)$ is a quality-$q_2$ cut sparsifier of $(G_1,T)$, then $(G_2,T)$ is a quality-$q_1q_2$ cut sparsifier of $(G,T)$.
\end{observation}

We say that a graph $H$ is a \emph{contraction-based sparsifier of $G$ with respect to $T$}, iff there exists a partition $\lset$ of vertices in $G$ into subsets where different terminals in $T$ lie in different subsets in $\lset$, and $H$ is obtained from $G$ by contracting, for each $L\in\lset$, vertices in $L$ into a supernode $u_L$, keeping parallel edges and discarding self-loops. %For each $t\in T$, if $F(t)$ is the (unique) cluster in $\fset$ that contains  $t$, then the node $u_{F(t)}$ in $H$ is identified with $t$.

