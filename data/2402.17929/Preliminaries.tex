%!TEX root = mainEV.tex

\section{Preliminaries}
\label{sec:prelims}
Let $\AA_0$ denote the initial matrix. Let $\lambda_0 = \lambda_{\max}(\AA_{0}) = \|\AA_{0}\|$ denote the maximum eigenvalue of the initial matrix $\AA_0$.
The following are the key definitions we use in our analysis.

\begin{definition}[$\epsilon$-max span and dimension] We define $\Span(\epsilon,\AA)$ to denote the space spanned by all eigenvectors of $\AA$ corresponding to eigenvalues $\lambda$ satisfying $\lambda \geq (1-\epsilon)\lambda_0$. Let $\dim{\epsilon,\AA}$ to denote the dimension of the space $\Span(\epsilon,\AA)$.
\end{definition}

We emphasize that $\lambda_0$ depends only on $\AA_0$. So, it is a static value that does not change through time. 
We will use the following linear algebraic notations.

\begin{definition}\label{def:subspace}Let $S_1$ and $S_2$ be two subspaces of a vector space $S$. The sum $S_1+S_2$ is the space,
\[
S_1+S_2 = \{s=s_1+s_2: s_1\in S_1, s_2 \in S_2\}.
\]
The complement $\overline{S}$ of $S$ is the vector space such that $S+\overline{S} = \mathbb{R}^n$, and $S\cap \overline{S} = \{0\}$. The difference, $S_1-S_2$ is defined as,
\[
S_1-S_2 = S_1\cap \overline{S_2}.
\]
\end{definition}

Next, we list standard facts about high-dimensional probability needed in our analysis.

\begin{lemma}[Chernoff Bound]\label{lem:Bernstein} Let $x_1,\cdots x_m$ be independent random variables such that $a\leq x_i\leq b$ for all $i$. Let $x = \sum_i x_i$ and let $\mu = \av[x]$. Then for all $\delta>0$,
\[
\Pr[x\geq(1+ \delta) \mu] \leq \exp \left(- \frac{2\delta^2\mu^2 }{m(b-a)^2} \right)
\]
\[
\Pr[x \leq (1-\delta) \mu] \leq \exp \left(- \frac{\delta^2\mu^2 }{m(b-a)^2} \right)
\]
\end{lemma}

\begin{lemma}[Norm of Gaussian Vector]\label{lem:NormG}
A random vector $\vv \in \mathbb{R}^n$ with every coordinate chosen from a normal distribution, $N(0,1)$ satisfies,
\[
\Pr[|\|\vv\|^2- n| \leq 2(1+\delta)\delta\cdot n ] \geq 1- e^{-\delta^2  n}.
\]
\end{lemma}
\begin{proof}
The vector $\vv$ has entries that are from $N(0,1)$. Now, every $\vv_i^2$ follows a $\chi^2$ distribution. From Lemma 1 of \cite{laurent2000adaptive} we have the following tail bound for a sum of $\chi^2$ random variables,
\[
\Pr[|\sum_i \vv^2_i - n| > 2\sqrt{nx} + 2x]\leq e^{-x}.
\]
Choosing $x = n\delta^2$ gives,
\[
\Pr[|\|\vv\|^2 - n| \leq 2\delta(1+\delta) n]\geq 1-e^{-\delta^2 n},
\] 
as required.
\end{proof}

\begin{lemma}[Distribution of $\chi^2$ Variable]\label{lem:chi}
Let $x\sim N(0,1)$ be a gaussian random variable. Then,
\[
\Pr\left[x^2\geq \frac{1}{n^4}\right] \geq 1- \frac{1}{n^{2}}.
\]
\end{lemma}
\begin{proof}
The probability distribution function for $y = x^2$ is given by,
\[
\ff(y) = \frac{1}{\sqrt{2}\Gamma(\frac{1}{2})}y^{-\frac{1}{2}}e^{-\frac{y}{2}}.
\]
It is known that $\Gamma(\frac{1}{2}) = \sqrt{\pi}$. Now, 
\begin{align*}
\Pr\left[x^2\leq \frac{1}{n^4}\right] =  \int_{0}^{1/n^4}\frac{1}{\sqrt{2}\Gamma(\frac{1}{2})}y^{-\frac{1}{2}}e^{-\frac{y}{2}}dy
\leq  \frac{e^0}{\sqrt{2\pi}} \int_{0}^{1/n^4} y^{-\frac{1}{2}}dy
=  \sqrt{\frac{2}{\pi}} \cdot \frac{1}{n^{2}}
\leq  \frac{1}{n^{2}}.
\end{align*}
\end{proof}


