%!TEX root = mainEV.tex


\section{Algorithms against an Oblivious Adversary}\label{sec:Obl}

To prove \Cref{thm:upper}, we first reduce Problem~\ref{prob:dyn} to solving a normalized threshold version of the problem where we assume that initially, the maximum eigenvalue is not much bigger than one. Then we want to maintain a certificate that 
the maximum eigenvalue is not much less than one until no such certificate exists. This is formalized below.
%
%
\begin{problem}[DecMaxEV($\epsilon,\AA_0,\vv_1,\cdots,\vv_T$)]\label{def:DecMaxEval} Let $\AA_0$ be an $n\times n$ symmetric PSD matrix such that $ \lambda_{\max}(\AA_0) \leq 1 + \frac{\epsilon}{\log n}$. The {\sc DecMaxEV}($\epsilon,\AA_0,\vv_1,\cdots,\vv_T$) problem asks to find for every $t$, a vector $\ww_t$ such that 
\[
\|\ww_t\| = 1 \quad \text{and} \quad \ww_t^{\top}\AA_t \ww_t \geq 1-40\epsilon,
\]
or return {\sc False} indicating that $\lambda_{\max}(\AA_t)\le 1- \frac{\eps}{\log n}$.
\end{problem} 
We defer the proof of the standard reduction stated below to the appendix.
%
\begin{restatable}{lemma}{Decision}\label{lem:Decision}
Given an algorithm $\mathcal{A}$ that solves the decision problem {\sc DecMaxEV}($\epsilon,\AA_0,\vv_1,\cdots,\vv_T$) (Definition~\ref{def:DecMaxEval}) for any $\epsilon>0$, $\AA_0 \succeq 0$ and vectors $\vv_1,\cdots,\vv_T$ in time $\mathcal{T}$, we can solve Problem~\ref{prob:dyn} in total time $O\left(\frac{\log^2 n\log\frac{n}{\epsilon}}{\epsilon}\cdot nnz(\AA_0) + \frac{\log n}{\epsilon}\log \frac{\lambda_{\max}(\AA_0)}{\lambda_{\max}(\AA_T)}\mathcal{T}\right)$.
\end{restatable}
%
%
Next, we describe \Cref{alg:PowerMethod} which can be viewed as an algorithm for \Cref{def:DecMaxEval} when there are no updates. 
Our algorithm essentially applies the {\it power iteration}, which is a standard algorithm used to find an approximate maximum eigenvalue and eigenvector of a matrix. In the algorithm, we make $R = O(\log n)$ copies to boost the probability.
%
%
\begin{algorithm}
\caption{{\sc DecMaxEV} with no update}\label{alg:PowerMethod}
 \begin{algorithmic}[1]
\Procedure{PowerMethod}{$\epsilon, \AA$}
\State $R \leftarrow 10\log n, r_0 \leftarrow 1$
\State $K \leftarrow \frac{4\log \frac{n}{\epsilon}}{\epsilon}$
\For{$r = 1,\cdots , R$}
\State $\vv^{(0,r)} \leftarrow $ random vector with coordinates chosen from $N(0,1)$\label{algline:randomInit}
\For{$k = 1:K$}
\State $\vv^{(k,r)}\leftarrow \AA\vv^{(k-1,r)}$
\EndFor
\State $\ww^{(r)} \leftarrow \frac{\vv^{(K,r)}}{\|\vv^{(K,r)}\|}$ 
\EndFor
\State $\WW = [\ww^{(1)},\dots,\ww^{(R)}]$\label{line:before case}
\If{ $(\ww^{(r)})^{\top}\AA\ww^{(r)} < 1-\epsilon$ for all $r\le R$}\label{line: if all w 1-eps}
\State \Return {\sc False}
\Else
\State $r_0\leftarrow$ smallest $r$ such that $(\ww^{(r)})^{\top}\AA\ww^{(r)} \geq 1-5\epsilon$
\State \Return $[r_0,\WW]$
\EndIf
\EndProcedure 
 \end{algorithmic}
\end{algorithm}
%
%
Below, we state the guarantees of the power method.

\begin{restatable}{lemma}{PowerMethod}\label{thm:StaticPowerMain}
Let $\epsilon>0$ and $\AA \succeq 0$. Let $\WW$ be as defined in Line~\ref{line:before case} in the execution of {\sc PowerMethod}($\epsilon,\AA$). With probability at least $1-1/n^{10}$, for some $\ww \in \WW$, it holds that $\ww^{\top}\AA\ww \geq (1-\frac{\epsilon}{2})\lambda_{\max}(\AA)$. The total time taken by the algorithm is at most $O\left(\frac{nnz(\AA)\log n\log \frac{n}{\epsilon}}{\epsilon}\right)$.

Furthermore, let $\lambda_i$ and $\uu_i$ denote the eigenvalues and eigenvectors of $\AA$. For all $i$ such that $\lambda_i (\AA) \leq \frac{\lambda_{\max}(\AA)}{2}$, with probability at least $1-2/n^{10}$, $\left[\ww^{\top}\uu_i\right]^2\leq \frac{1}{n^8}\cdot \frac{\lambda_i}{\lambda_1}$.
\end{restatable}
 We note that the last line of the above lemma is saying that the vectors returned by the power method satisfy \Cref{def:super}, which we state for completeness but is not required by our algorithm. The following result is a direct consequence of Lemma~\ref{thm:StaticPowerMain}.

 \begin{corollary}\label{thm:StaticPower}
 Let $\epsilon>0, \AA \succeq 0$. Let $\WW$ be as defined in Line~\ref{line:before case} in the execution of {\sc PowerMethod}($\epsilon,\AA$). If $\lambda_{\max}(\AA) \ge 1-\epsilon$, then with probability at least $1-1/n^{10}$, $\ww^{\top}\AA\ww\geq 1-5\epsilon$ for some $\ww\in \WW$. Furthermore, if $\lambda_{\max}(\AA) \ge 1-\epsilon/\log n$, then with probability at least $1-1/n^{10}$, $\ww^{\top}\AA\ww\geq 1-\epsilon$ for some $\ww\in \WW$. The total time taken by the algorithm is at most $O\left(\frac{nnz(\AA)\log n\log \frac{n}{\epsilon}}{\epsilon}\right)$.
 \end{corollary}

Observe that, if the algorithm returns $[r_0,\WW]$, then $(\ww^{(r)})^{\top}\AA\ww^{(r)}\geq 1-5\epsilon$ for $r=r_0$, and $\ww^{(r_0)}$ is therefore a solution to Problem~\ref{def:DecMaxEval} when there is no update. The power method and its analysis are standard, and we thus defer the proof of \Cref{thm:StaticPowerMain} to the appendix. 




Next, in \Cref{alg:Init,alg:DynamicMaxPM} we describe an algorithm for Problem~\ref{def:DecMaxEval} when we have an online sequence of updates $\vv_1,\cdots, \vv_T$. 
%
The algorithm starts by initializing $R = O(\log n)$ copies of the approximate maximum eigenvectors from the power method. Given a sequence of updates, as long as one of the copies is the witness that the current matrix $\AA_t$ still has a large eigenvalue, i.e., there exists $r$ where $(\ww^{(r)})_t^{\top}\AA_t\ww^{(r)}_t\geq 1-40\epsilon$, we can just return $\ww^{(r)}$ as the solution to Problem~\ref{def:DecMaxEval}. 
Otherwise, $(\ww^{(r)})_t^{\top}\AA_t\ww^{(r)}_t< 1-40\epsilon$ for all $r \le R$ and none of the vectors from the previous call to the power method are a witness of large eigenvalues anymore. In this case, we simply recompute these vectors by calling the power method again. If the power method returns that there is no large eigenvector, then we return {\sc False} from now. Otherwise, we continue in the same manner. 
%
Note that our algorithm is very simple, but as we will see, the analysis is not straightforward.
%
\begin{algorithm}
\caption{Initialization}\label{alg:Init}
 \begin{algorithmic}[1]
 \Procedure{Init}{$\epsilon, \AA_{0}$}
\State $\WW \leftarrow$ {\sc PowerMethod}($\epsilon,\AA_0$)\label{algline:PMInit}
\State \Return $\WW$
\EndProcedure
\end{algorithmic}
\end{algorithm}


\begin{algorithm}
\caption{Update algorithm at time $t$ ($A_{t-1},r_t,\WW_{t-1}= [w^{(r)}_{t-1}: r= 1,\cdots R], \eps$ are maintained)}\label{alg:DynamicMaxPM}
 \begin{algorithmic}[1]
\Procedure{Update}{$\vv_t$}
\State $\AA_t \leftarrow \AA_{t-1}-\vv_t\vv_t^{\top}$
\If{$ (\ww^{(r)}_{t-1})^{\top}\AA_t\ww^{(r)}_{t-1} <1-40\epsilon$ for all $r \le R$}\label{algline:Check}
\State $[r_t,\WW_t]\leftarrow$ {\sc PowerMethod}($\epsilon,\AA_t$)\label{algline:PM}
\If{{\sc PowerMethod}($\epsilon,\AA_t$) returns {\sc False}}
\State \Return {\sc False} for all further updates
\EndIf
\Else
\State $r_t\leftarrow $ smallest $r$ such that $(\ww^{(r)}_{t-1})^{\top}\AA_t\ww^{(r)}_{t-1} \geq 1-40\epsilon$
\State $\WW_t \leftarrow \WW_{t-1}$
\EndIf
\State \Return $[r_t,\WW_t]$
\EndProcedure 
 \end{algorithmic}
\end{algorithm}

\subsection{Proof Overview}
The overall proof of \Cref{thm:upper}, including the proof of correctness and the runtime depends on the number of executions in Line~\ref{algline:PM} in Algorithm~\ref{alg:DynamicMaxPM}. If the number of executions of Line~\ref{algline:PM} is bounded by $\poly(\log n/\epsilon)$, then the remaining analysis is straightforward. Therefore, the majority of our analysis is dedicated to proving this key lemma, i.e., $\poly(\log n/\epsilon)$ bound on the number of calls to the power method:
\begin{lemma}[Key Lemma]\label{lem:BoundW}
The number of executions of Line~\ref{algline:PM} over all updates is bounded by $O(\log n\log^5\frac{n}{\epsilon}/\epsilon^2)$ with probability at least $1-\frac{1}{n}$.
\end{lemma}
Given the key lemma, the correctness and runtime analyses are quite straightforward and are presented in \Cref{sec:correct}. We now give an overview of the proof of \Cref{lem:BoundW}.


Let us consider what happens between two consecutive calls to Line~\ref{algline:PM}, say at $\AA$ and $\AAtil = \AA - \sum_{i=1}^k\vv_i\vv_i^{\top}$. We first define the following subspaces of $\AA$ and $\AAtil$.
Recall \Cref{def:subspace}, which we use to define the following subspaces.

\begin{definition}[Subspaces of $\AA$ and $\AAtil$]\label{def:SpaceA} Given $\epsilon>0$, $\AA$, and $\AAtil$ define for $\nu = 0,1,\cdots, 15\log \frac{n}{\epsilon}-1$:
\[
T_{\nu} = \Span\left(\frac{(\nu+1)\epsilon}{5\log \frac{n}{\epsilon}},\AA\right) -\Span\left(\frac{\nu\epsilon}{5\log \frac{n}{\epsilon}},\AA\right),
\]
and,
\[
\tilde{T}_{\nu} = \Span\left(\frac{(\nu+1)\epsilon}{5\log \frac{n}{\epsilon}},\AAtil\right)-\Span\left(\frac{\nu\epsilon}{5\log \frac{n}{\epsilon}},\AAtil\right).
\]
That is, the space $T_{\nu}$ and $\Ttil_\nu$ are spanned by eigenvectors of $\AA$ and $\AAtil$, respectively, corresponding to eigenvalues between $\left(1-(\nu+1)\frac{\epsilon}{5\log \frac{n}{\epsilon}}\right)\lambda_0$ and $\left(1-\nu\frac{\epsilon}{5\log \frac{n}{\epsilon}}\right)\lambda_0$.

Let $d_{\nu} = \dim{T_{\nu}}$ and $\tilde{d}_{\nu} = \dim{\tilde{T}_{\nu}}$. Also define,
\[
\tilde{T} = \Span(3\epsilon,\AAtil),\quad  T = \Span(3\epsilon,\AA),
\]
and let $d = \dim{T}$, $\tilde{d} = \dim{\tilde{T}}$.
\end{definition}

Observe that $T = \sum_{\nu = 0}^{15\log\frac{n}{\eps}-1}T_\nu$ and similarly $\Ttil = \sum_{\nu = 0}^{15\log\frac{n}{\eps}-1}\Ttil_\nu$. We next define some indices/levels corresponding to large subspaces, which we call ``important levels''.
\begin{definition}[Important $\nu$]\label{def:impNu} We say a level $\nu$ is important if,
\[d_{\nu} \geq \frac{\epsilon}{600\log^3 \frac{n}{\epsilon}} \sum_{\nu'<\nu}d_{\nu'}.
\]
We will use $\mathcal{I}$ to denote the set of $\nu$ that are important.
\end{definition}
%

The main technical lemma that implies \Cref{lem:BoundW} is the following:


\begin{restatable}[Measure of Progress]{lemma}{progress}\label{lem:EigenspaceChange}
	Let $\epsilon>0$ and let $\WW=[\ww^{(1)},\dots,\ww^{(R)}]$ be as defined in Line~\ref{line:before case} in the execution of {\sc PowerMethod}($\epsilon,\AA$). Let $\vv_{1},\cdots,\vv_{k}$ be a sequence of updates generated by an oblivious adversary and define $\AAtil=\AA-\sum_{i=1}^{k}\vv_{i}\vv_{i}^{\top}$.
	
	Suppose that $\lambda_{\max}(\AA)\ge 1-\epsilon$ and $\ww^{\top}\AAtil\ww<1-40\epsilon$ for all $\ww\in\WW$. Then, with probability at least $1-\frac{50\log\frac{n}{\epsilon}}{n^{2}}$, for some $\nu\in\mathcal{I}$,
	\begin{itemize}
		\item $\dim{T_{\nu}-\tilde{T}}\geq\frac{\epsilon}{300\log\frac{n}{\epsilon}}d_{\nu}$ if $d_{\nu}\geq\frac{3000\log n\log\frac{n}{\epsilon}}{\epsilon}$, or
		\item $\dim{T_{\nu}-\tilde{T}}\geq1$ if $d_{\nu}<\frac{3000\log n\log\frac{n}{\epsilon}}{\epsilon}$.
	\end{itemize}
\end{restatable}


We prove this lemma in \Cref{sec:progress}. Intuitively speaking, it means that, whenever  Line~\ref{algline:PM} of \Cref{alg:DynamicMaxPM} is executed, there is some important level $\nu$ such that an $\Omega(\eps/\poly \log(n/\eps))$-fraction of eigenvalues of $\AA$ at level $\nu$ have decreased in value. This is the crucial place where we exploit an oblivious adversary.

Given \Cref{lem:EigenspaceChange}, the remaining proof of \Cref{lem:BoundW} follows a potential function analysis which is presented in detail in Section~\ref{sec:proof key}. We consider potentials $\Phi_j = \sum_{\nu=0}^jd_{\nu}$ for $j = 0,\cdots,15\log\frac{n}{\epsilon}-1$. The main observation is that $\Phi_j$ is non-increasing over time for all $j$, and whenever there exists $\nu_0\in \mathcal{I}$ that satisfies the condition of Lemma~\ref{lem:EigenspaceChange}, $\Phi_{\nu_0}$ decreases by $\dim{T_{\nu_0}-\tilde{T}}$. Since $\dim{T_{\nu_0}-\tilde{T}} \geq \Omega(\epsilon/\poly\log(n/\epsilon))d_{\nu_0}$ and $\nu_0\in \mathcal{I}$, i.e., $\Phi_{\nu_0} = \sum_{\nu<\nu_0}d_{\nu} + d_{\nu_0} \leq d_{\nu_0} \left(\frac{O(\log^3\frac{n}{\epsilon})}{\epsilon}+1\right) $,  we can prove that $\Phi_{\nu_0}$ decreases by a multiplicative factor of $\Omega(1-\epsilon^2/\poly\log(n/\epsilon))$. As a result, every time our algorithm executes Line~\ref{algline:PM}, $\Phi_{j}$ decreases by a multiplicative factor for some $j$, and since we have at most $15\log\frac{n}{\epsilon}$ values of $j$, we can only have $\poly(\log n/\epsilon)$ executions of Line~\ref{algline:PM}.

It remains to describe how we prove Lemma~\ref{lem:EigenspaceChange} at a high level. We can write $\ww^{\top}\AAtil\ww$ for any $\ww \in \WW$ as
\[
\ww^{\top}\AAtil\ww = \ww^{\top}\AA\ww - \ww^{\top}\VV\ww,
\]
for $\VV = \sum_{i=1}^k\vv_i\vv_i^{\top}$. 
Our strategy is to show that:
\begin{align*}\label{star}
        &\text{If } \dim{T_{\nu}-\tilde{T}}\text{ does not satisfies the inequalities in \Cref{lem:EigenspaceChange} for all }\nu\in \mathcal{I},\\ 
        &\text{then }\ww^{\top}\VV\ww \leq 35\epsilon \text{ for all } \ww \in \WW. \tag{$\star$}
\end{align*}    
Given \eqref{star} as formalized later in \Cref{cl:progress}, we can conclude \Cref{lem:EigenspaceChange} because, from the definition of $\AA$ and $\AAtil$, we have that for some $\ww\in \WW$, $\ww^{\top}\AA\ww \geq 1-5\epsilon$ by \Cref{thm:StaticPower} and $\ww^{\top}\AAtil\ww <1-40\epsilon$. As a result for this $\ww$, $\ww^{\top}\VV\ww >35\epsilon$. Now, by contra-position of \eqref{star}, we have that $\dim{T_{\nu}-\tilde{T}}$ is large for some $\nu \in \mathcal{I}$.

To prove \eqref{star}, we further decompose $\ww^{\top}\VV\ww$ as
\[
\ww^{\top}\VV\ww = \ww^{\top}\VV_{\tilde{T}}\ww+\sum_{\nu = 0}^{15\log\frac{n}{\epsilon}-1}\ww^{\top}\VV_{T_{\nu} -\tilde{T}}\ww+ \ww^{\top}\VV_{\overline{T}}\ww.
\]

In the above equation, $\VV_{\tilde{T}}= \Pi_{\tilde{T}}\VV\Pi_{\tilde{T}},\VV_{T_{\nu} -\tilde{T}}=\Pi_{\nu}\VV\Pi_{\nu}$, and $\VV_{\overline{T}} = \Pi_{\overline{T}}\VV\Pi_{\overline{T}}$ where $\Pi_{\tilde{T}},\Pi_{\nu},\Pi_{\overline{T}}$ denote the projections matrices that project any vector onto the spaces $\tilde{T}$, $T_{\nu}-\tilde{T}$, and $\overline{T}$ respectively\footnote{Suppose a subspace $S$ is spanned by vectors $\uu_1,\dots,\uu_k$. Let $\UU = [\uu_1,\dots,\uu_k]$. Recall that the projection matrix onto $S$ is $\UU(\UU^\top \UU)^{-1} \UU^\top$.}. Refer to Section~\ref{sec:progress} for proof of why such a split is possible. Our proof of \eqref{star} then bounds the terms on the right-hand side. Let us consider each term separately.
\begin{enumerate}
	\item $\ww^{\top}\VV_{\tilde{T}}\ww$: We prove that this is always at most $10\epsilon(1+\epsilon)$ (Equation~\eqref{eq:V2}). From the definition of $\VV,$ 
	\[
	\ww^{\top}\VV_{\tilde{T}}\ww = \ww^{\top}\Pi_{\tilde{T}}\AA\Pi_{\tilde{T}}\ww - \ww^{\top}\Pi_{\tilde{T}}\AAtil\Pi_{\tilde{T}}\ww.
	\] 
	Since $\Pi_{\tilde{T}}\ww$ is the projection of $\ww$ along the large eigenspace of $\AAtil$, the second term on the right-hand side above is large, i.e. $\geq (1- 10\epsilon)\lambda_0\|\Pi_{\tilde{T}}\ww\|^2$. The first term on the right-hand side can be bounded as, $\ww^{\top}\Pi_{\tilde{T}}\AA\Pi_{\tilde{T}}\ww \leq \|\AA\|\|\Pi_{\tilde{T}}\ww\|^2 \leq \lambda_0 \|\Pi_{\tilde{T}}\ww\|^2$.
 Therefore the difference on the right-hand side is at most $10\epsilon\lambda_0\|\Pi_{\tilde{T}}\ww\|^2 \leq 10 \epsilon\lambda_0\|\ww\|^2 = 10\epsilon \lambda_0 \leq 10\epsilon(1+\epsilon)$.
	\item $\ww^{\top}\VV_{\overline{T}}\ww$: Observe that this term corresponds to the projection of $\ww$ along the space spanned by the eigenvalues of $\AA$ of size at most $1-3\epsilon$. Let $\uu_i$ and $\lambda_i$ denote an eigenvector and eigenvalue pair with $\lambda_i<1-3\epsilon$. Since the power method can guarantee that $\ww^{\top}{\uu_i}\approx\lambda_i^{2K}$, we have  $\lambda_i^{2K} \leq (1-3\epsilon)^{2K}\leq \poly\left(\frac{\epsilon}{n}\right)$ is tiny. So we have that $\ww^{\top}\VV_{\overline{T}}\ww\leq \epsilon$ (Lemma~\ref{lem:boundLowEV}).

	\paragraph{}Before we look at the final case, we define a basis for the space $T_{\nu}$.
	\begin{definition}[Basis for $T_{\nu}$]\label{def:Basis} Let $T_{\nu}$ be as defined in Definition~\ref{def:SpaceA}.  Define indices $a_{\nu}$ and $b_{\nu}$ with $b_{\nu}-a_{\nu}+1 = d_{\nu}$ such that the basis of $T_{\nu}$ is given by $\uu_{a_{\nu}},\cdots, \uu_{b_{\nu}}$, where $\uu_1,\uu_2,\cdots,\uu_n$ are the eigenvectors of $\AA$ in decreasing order of eigenvalues.
	\end{definition}

    \item $\ww^{\top}\VV_{T_{\nu} -\tilde{T}}\ww$: For this discussion, we will ignore the constant factors and assume that the high probability events hold. Let $\Pi_{\nu}$ denote the projection matrix for the space $T_{\nu}-\Ttil$. Observe that $\ww^{\top}\VV_{T_{\nu}-\Ttil}\ww=\ww^{\top}\Pi_{\nu}\VV\Pi_{\nu}\ww\le\|\VV\|\|\Pi_{\nu}\ww\|^2 \le (1+\epsilon)\|\Pi_{\nu}\ww\|^2$, where the last inequality is because $\AAtil=\AA-\VV\succeq0$, and therefore, $\|\VV\|\le\|\AA\|\le(1+\epsilon)$. Hence, it suffices to bound $\|\Pi_{\nu}\ww\|^2 = O(\epsilon)$. 

We can write $\ensuremath{\ww=\frac{\sum_{i=1}^{n}\lambda_{i}^{K}\alpha_{i}\uu_{i}}{\sqrt{\sum_{i=1}^{n}\lambda_{i}^{2K}\alpha_{i}^{2}}}}$ where $\lambda_{i},\uu_{i}$'s are the eigenvalues and eigenvectors of $\AA$ and $\alpha_{i}\sim N(0,1)$. 
Define $\zz = \sum_{i=1}^n z_i \uu_i$ where $z_i = \lambda_i^K\alpha_i$. That is, 
$\ww=\frac{\zz}{\|\zz\|}$. Since $\|\Pi_{\nu}\ww\| = \|\Pi_{\nu}\zz\|/\|\zz\|$, it suffices to show that $\|\Pi_{\nu}\zz\|^2 \le O(\epsilon) \| \zz\|^2$. We show this in two separate cases. In both cases, we start with the following bound
\[
\|\Pi_{\nu}\zz\|^{2}\le\lambda_{a_{\nu}}^{2K}\cdot\dim{T_{\nu}-\tilde{T}},
\]
which holds with high probability. To see this, let $\boldsymbol{g}_{\nu}\sim N(0,1)$ be a gaussian vector in the space $T_{\nu}-\tilde{T}$. 
We can couple $\textbf{g}_\nu$ with $\Pi_{\nu}\zz$ so that $\Pi_{\nu}\zz$ is dominated by $\lambda_{a_{\nu}}^{K}\cdot \boldsymbol{g}_{\nu}$. So $\|\Pi_{\nu}\zz\|^{2}\le\lambda_{a_{\nu}}^{2K}\|\boldsymbol{g}_{\nu}\|^{2}$. By \Cref{lem:NormG}, the norm square of gaussian vector is concentrated to its dimension so $\|\boldsymbol{g}_{\nu}\|^{2}\le\dim{T_{\nu}-\tilde{T}}$ with high probability, thus proving the inequality. Next, we will bound $\dim{T_{\nu}-\tilde{T}}$ in terms of $\|\zz\|$ in two cases. 

\paragraph{When $\nu\protect\notin{\cal I}$ (\Cref{lem:NotImp}):}

From the definition of the important levels, we have 
\[
\dim{T_{\nu}-\tilde{T}}\leq d_{\nu}\leq\frac{O(\epsilon)}{\log^{3}\frac{n}{\epsilon}}\sum_{\nu'<\nu}d_{\nu'}.
\]
Now, we have $\sum_{\nu'<\nu}d_{\nu'}\approx\sum_{i=1}^{b_{\nu-1}}\alpha_{i}^{2}$ because $\alpha_{i}\sim N(0,1)$ is gaussian and the norm square of gaussian vector is concentrated to its dimension (\Cref{lem:NormG}). Since $\alpha_{i}=z_{i}/\lambda_{i}^{K}$, we have that
\[
\sum_{\nu'<\nu}d_{\nu'}\approx\sum_{i=1}^{b_{\nu-1}}\alpha_{i}^{2}=\sum_{i=1}^{b_{\nu-1}}\frac{z_{i}^{2}}{\lambda_{i}^{2K}}\le\|\zz\|^{2}/\lambda_{b_{\nu-1}}^{2K}.
\]
Therefore, we have 
\[
\|\Pi_{\nu}\zz\|^2 \le\lambda_{a_{\nu}}^{2K}\dim{T_{\nu}-\tilde{T}}\le\left(\frac{\lambda_{a_{\nu}}}{\lambda_{_{b_{\nu-1}}}}\right)^{2K}\frac{O(\epsilon)}{\log^{3}\frac{n}{\epsilon}}\|\zz\|^{2}\le O(\epsilon)\|\zz\|^{2}
\]
where the last inequality is trivial because $\lambda_{b_{\nu-1}}\ge\lambda_{a_{\nu}}$ by definition.

\paragraph{When $\nu\in{\cal I}$ (\Cref{lem:GaussianProjD}):}

In this case, according to \eqref{star}, we can assume $$\dim{T_{\nu}-\tilde{T}}\lesssim\epsilon d_{\nu}.$$
Again, by \Cref{lem:NormG}, we have that $d_{\nu}\approx\sum_{i=a_{\nu}}^{b_{\nu}}\alpha_{i}^{2}$ because $\alpha_{i}\sim N(0,1)$ is gaussian. Since $\alpha_{i}=z_{i}/\lambda_{i}^{K}$, we have
\[
d_{\nu}\approx\sum_{i=a_{\nu}}^{b_{\nu}}\alpha_{i}^{2}=\sum_{i=a_{\nu}}^{b_{\nu}}\frac{z_{i}^{2}}{\lambda_{i}^{2K}}\le\|\zz\|^{2}/\lambda_{b_{\nu}}^{2K}.
\]
Therefore,
\[
\|\Pi_{\nu}\zz\|^2\le\lambda_{a_{\nu}}^{2K}\dim{T_{\nu}-\tilde{T}}\le\left(\frac{\lambda_{a_{\nu}}}{\lambda_{_{b_{\nu}}}}\right)^{2K}\epsilon\|\zz\|^{2}\le O(\epsilon)\|z\|^{2}
\]
where the last inequality is because $\ensuremath{\left(\frac{\lambda_{a_{\nu}}}{\lambda_{b_{\nu}}}\right)^{2K}\leq\left(\frac{1-\frac{\nu\epsilon}{5\log\frac{n}{\epsilon}}}{1-\frac{(\nu+1)\epsilon}{5\log\frac{n}{\epsilon}}}\right)^{2K}\approx\left(1+\frac{\epsilon}{2\log\frac{n}{\epsilon}}\right)^{2K}\approx O(1)}.$

\end{enumerate}
From these three cases, we can conclude that if $\dim{T_{\nu}-\tilde{T}}$ is small for all $\nu\in \mathcal{I}$, then $\ww^{\top}\VV\ww \leq 35\epsilon$, proving our claim.

In the remaining sections, 
we give formal proofs of the claims made in this section. In \Cref{sec:correct}, we prove the main result, \Cref{thm:upper}, assuming the key lemma. In \Cref{sec:proof key}, we prove the key lemma, \Cref{lem:BoundW}, assuming the \Cref{lem:EigenspaceChange}. Finally, we prove \Cref{lem:EigenspaceChange} in \Cref{sec:progress}


\subsection{Proof of the Main \Cref{thm:upper} assuming the Key \Cref{lem:BoundW}}\label{sec:correct}

Here, we formally prove \Cref{thm:upper} assuming the key \Cref{lem:BoundW}. We will first prove the correctness and then bound the total runtime.


\paragraph{Correctness.}
The following formalizes the correctness guarantee of  Algorithm~\ref{alg:DynamicMaxPM}. 

\begin{lemma}\label{lem:DynCorrectAns}
Let $\epsilon>1/n$. With probability at least $1-1/n$, the following holds for all time step $t \ge 1$: 
if the maximum eigenvalue of $\AA_t$ is at least $1-\frac{\epsilon}{\log n}$, {\sc Update}($\vv_t$) returns $[r_t,\WW_t]$.
\end{lemma}
\begin{proof}\textcolor{red}{TOPROVE 0}\end{proof}



\paragraph{Runtime.}\label{sec:runtime}

Next, we bound the runtime of the various lines of Algorithm~\ref{alg:DynamicMaxPM}. 
%
\begin{lemma}\label{lem:SameW}
For a fixed $\ww$ and any $t$, we can update $\ww^{\top}\AA_{t-1}\ww$ to $\ww^{\top}\AA_t\ww$ in time $O(nnz(\vv_t))$.
\end{lemma}
\begin{proof}\textcolor{red}{TOPROVE 1}\end{proof}
%
\begin{lemma}\label{lem:DiffW}
Fix time $t$. Given $\ww$ as input, we have that $\ww^{\top}\AA_t\ww$ and $\AA_t\ww$ can be computed $O\left(nnz(\AA_0) +\sum_{i=1}^t nnz(\vv_i)\right)$ time.
\end{lemma}
\begin{proof}\textcolor{red}{TOPROVE 2}\end{proof}
%
\begin{lemma}\label{lem:PMt}For any time $t$, we can implement {\sc PowerMethod}($\epsilon,\AA_t$) in time at most 
\[
O\left(\frac{\log n\log \frac{n}{\epsilon}}{\epsilon}\left(nnz(\AA_0) +\sum_{i=1}^t nnz(\vv_i)\right)\right).
\]
\end{lemma}
\begin{proof}\textcolor{red}{TOPROVE 3}\end{proof}

 Given the above results, we can now prove Theorem~\ref{thm:upper}.
\subsubsection*{Proof of Theorem~\ref{thm:upper}}
\begin{proof}\textcolor{red}{TOPROVE 4}\end{proof}


%!TEX root = mainEV.tex


\subsection{Proof of the Key Lemma~\ref{lem:BoundW}: Few Executions of Line~\ref{algline:PM}}

\label{sec:proof key}

In this section, we prove Lemma~\ref{lem:BoundW} assuming the Progress Lemma~\ref{lem:EigenspaceChange}. 
%
Let us first recall the precise definition of $\AA$ and $\AAtil$. Suppose we execute Line~\ref{algline:PM} at update $t_{0}$. Now consider a sequence of updates, $\vv_{t_{0}+1},\cdots,\vv_{t_{0}+k}$, and let $\AA_{t_{0}+k}=\AA_{t_{0}}-\sum_{i=1}^{k}\vv_{t+i}\vv_{t+i}^{\top}.$ Suppose the next execution of Line~\ref{algline:PM} happens at $t_{0}+k$. For this to happen we must have that $\ww_{t_{0}}^{\top}\AA_{t_{0}+k}\ww_{t_{0}}<1-40\epsilon$ for all $\ww_{t_{0}}\in\WW_{t_{0}}$. We let $\AA=\AA_{t_{0}}$ and $\AAtil=\AA_{t_{0}+k}$. 

Next, recall Definition~\ref{def:SpaceA} and define $T_{\leq i}=\sum_{\nu=0}^{i}T_{\nu}$, $\tilde{T}_{\leq i}=\sum_{\nu=0}^{i}\tilde{T}_{\nu}$. The following observation will motivate our potential function analysis.
\begin{lemma}
	\label{lem:Monotone} For all $i\leq15\log\frac{n}{\epsilon}$, 
	\[
	\dim{T_{\le i}}\ge\dim{\Ttil_{\le i}}.
	\]
	Let $\nu_{0}$ be such that $\dim{T_{\nu_{0}}-\tilde{T}}>0$. For any $i\ge\nu_{0}$, we have
	\[
	\dim{T_{\le i}}\geq\dim{\Ttil_{\le i}}+\dim{T_{\nu_{0}}-\Ttil}.
	\]
	
\end{lemma}

\begin{proof}
	We claim that $\tilde{T}_{\leq i}\subseteq T_{\leq i}$, which implies the first claim. This is because the updates are decreasing. So, if $\vv^{\top}\AAtil\vv\geq1-\frac{(i+1)\epsilon}{5\log\frac{n}{\epsilon}}$ then $\vv^{\top}\AA\vv\geq1-\frac{(i+1)\epsilon}{5\log\frac{n}{\epsilon}}$. That is, if $\vv\in\Ttil_{\le i}$, then $\vv\in T_{\leq i}$. For the second part, we have $T_{\leq i}=T_{\leq i}\cap\tilde{T}_{\leq i}+(T_{\leq i}-\tilde{T}_{\leq i})\supseteq\tilde{T}_{\leq i}+(T_{\nu_{0}}-\tilde{T})$ because $T_{\nu_{0}}\subseteq T_{\le i}$ and $\Ttil_{\le i}\subseteq\Ttil$. Since $\tilde{T}_{\leq i}\cap(T_{\nu_{0}}-\tilde{T})=\emptyset$, we can conclude the second part.
\end{proof}

\paragraph{The Potentials.}
The above lemma and \Cref{lem:EigenspaceChange} motivate the following potentials.
For every $j\le15\log\frac{n}{\epsilon}$, we define the potentials $\Phi_{j}=\dim{T_{\le j}}$ and $\tilde{\Phi}_{j}=\dim{\Ttil_{\le j}}$. For all $j$, the potential may only decrease, i.e., $\Phitil_{j}\le\Phi_{j}$ by \Cref{lem:Monotone}. Also, clearly, $\Phi_{j}\le n$. 

We will show that for each execution of Line~\ref{algline:PM}, $\Phitil_{j}$ decreases from $\Phi_j$ by a significant factor with high probability for some $j$. This will bound the number of executions.

Consider any important level $\nu_{0} \in \cal{I}$. We have two observations:
\begin{enumerate}
	\item \label{enu:phi 1} $\Phitil_{\nu_{0}}\leq\Phi_{\nu_{0}}-\dim{T_{\nu_{0}}-\tilde{T}}$, and 
	\item \label{enu:phi 2} $d_{\nu_{0}}\geq\Omega(1)\frac{\epsilon}{\log^{3}\frac{n}{\epsilon}}\Phi_{v_{0}}$. 
\end{enumerate}
The first point follows directly from the second part of Lemma~\ref{lem:Monotone}. Moreover, since $\nu_{0}\in\mathcal{I}$ is important, we have $d_{\nu_{0}}\geq\frac{\epsilon}{600\log^{3}\frac{n}{\epsilon}}\sum_{\nu'<\nu}d_{\nu'}.$ Therefore, 
\[
\Phi_{\nu_{0}}=\sum_{\nu=1}^{\nu_{0}}d_{\nu}=\sum_{\nu'<\nu_{0}}d_{\nu'}+d_{\nu_{0}}\leq\left(\frac{600\log^{3}\frac{n}{\epsilon}}{\epsilon}+1\right)d_{\nu_{0}},
\]
implying the second point. 
%
Combining these observations with the Progress \Cref{lem:EigenspaceChange}, we have:
\begin{lemma}
	\label{claim:potential decrease}Suppose that $\lambda_{\max}(\AA)\ge1-\epsilon$ and $\ww^{\top}\AAtil\ww<1-40\epsilon$ for all $\ww\in\WW$. With probability at least $1-50\log\frac{n}{\epsilon}/n^{2}$, there is a level $\nu_{0}$ such that either 
	\begin{itemize}
		\item $\Phitil_{\nu_{0}}\le\left(1-\frac{\Omega(\epsilon^{2})}{\log^{4}\frac{n}{\epsilon}}\right)\Phi_{\nu_{0}}$, or 
		\item $\Phitil_{\nu_{0}}\le\Phi_{\nu_{0}}-1$ and $\Phi_{\nu_{0}}\leq O(1)\frac{\log^{4}\frac{n}{\epsilon}\log n}{\epsilon^{2}}.$
	\end{itemize}
\end{lemma}

\begin{proof}
	Given the assumption, it follows from \Cref{lem:EigenspaceChange} that with probability at least $1-50\log\frac{n}{\epsilon}/n^{2}$, there is an important level $\nu_{0}\in\mathcal{I}$ be such that either $d_{\nu_{0}}\geq\frac{3000\log n\log\frac{n}{\epsilon}}{\epsilon}$ and $\dimm(T_{\nu_{0}}-\tilde{T})\geq\frac{\epsilon}{300\log\frac{n}{\epsilon}}d_{\nu_{0}}$ or $d_{\nu_{0}}<\frac{3000\log n\log\frac{n}{\epsilon}}{\epsilon}$ and $\dimm(T_{\nu_{0}}-\tilde{T})\geq1$. So, by calculation, we have the following.
	\begin{itemize}
		\item If $d_{\nu_{0}}\geq\frac{3000\log n\log\frac{n}{\epsilon}}{\epsilon}$, we have $\Phitil_{\nu_{0}}\leq\Phi_{\nu_{0}}-\frac{\epsilon}{300\log\frac{n}{\epsilon}}d_{\nu_{0}}\le\Phi_{\nu_{0}}(1-\frac{\Omega(\epsilon^{2})}{\log^{4}\frac{n}{\epsilon}})$ where the first inequality is by (\ref{enu:phi 1}) and \Cref{lem:EigenspaceChange}. The second is by (\ref{enu:phi 2}). 
		\item If $d_{\nu_{0}}<\frac{3000\log n\log\frac{n}{\epsilon}}{\epsilon}$, we have $\Phitil_{\nu_{0}}\le\Phi_{\nu_{0}}-1$ by (\ref{enu:phi 1}) and \Cref{lem:EigenspaceChange}. In this case, we also have $\Phi_{\nu_{0}}\leq O(1)\frac{\log^{4}\frac{n}{\epsilon}\log n}{\epsilon^{2}}$ by (\ref{enu:phi 2}). 
	\end{itemize}
\end{proof}
We are now ready to prove Lemma~\ref{lem:BoundW}. 

\subsubsection*{Proof of Lemma~\ref{lem:BoundW}.}

First, observe that whenever $\lambda_{\max}(\AA)<1-\epsilon$, by Line~\ref{line: if all w 1-eps} of \Cref{alg:PowerMethod}, we will always return $\textsc{False}$ at the next execution of Line \ref{algline:PM} of \Cref{alg:DynamicMaxPM}.

Therefore, it suffices to bound the number of executions while $\lambda_{\max}(\AA)\ge1-\epsilon$. We execute Line~\ref{algline:PM} only if $\ww^{\top}\AAtil\ww<1-40\epsilon$ for all $\ww\in\WW$. When this happens, there exists a level $j$ where the potential $\Phi_{j}$ significantly decreases according to \Cref{claim:potential decrease} with probability at least $1-50\log\frac{n}{\epsilon}/n^{2}$. For each level $j$, this can happen at most $L=O(\frac{\log n\log^{4}\frac{n}{\epsilon}}{\epsilon^{2}})$ times because for every $j$, $\Phi_{j}$ is an integer that may only decrease and is bounded by $n$. Suppose for contradiction that there are more than $L\times15\log\frac{n}{\epsilon}$ executions of Line~\ref{algline:PM}. So, with probability at least $1-L\cdot50\log\frac{n}{\epsilon}/n^{2}\ge1/n$, there exists a level $j$ where $\Phi_{j}$ decreases according to \Cref{claim:potential decrease} strictly more than $L$ times. This is a contradiction. 











\subsection{Proof of the Progress Lemma~\ref{lem:EigenspaceChange}}\label{sec:progress}
It remains to prove the Progress Lemma. We first restate the lemma here.

\progress*
Recall the definitions of subspaces $T,T_\nu,\Ttil$ and $\overline{T}$ in \Cref{def:SpaceA}.
We will first state the following claim and show that this is sufficient to prove \Cref{lem:EigenspaceChange}. After concluding the proof of \Cref{lem:EigenspaceChange}, we would prove the claim.

\begin{claim}\label{cl:progress} Suppose that $\lambda_{\max}(\AA) \geq 1-\epsilon$. If for every $\nu \in \mathcal{I}$, 
\begin{itemize}
\item $\dim{T_{\nu} -\tilde{T}} < \frac{\epsilon}{300\log\frac{n}{\epsilon}} d_{\nu}$ if $d_{\nu} \geq \frac{3000\log n\log\frac{n}{\epsilon}}{\epsilon}$, and 
\item $\dim{T_{\nu} -\tilde{T}} < 1$ if $d_{\nu} < \frac{3000\log n\log\frac{n}{\epsilon}}{\epsilon}$.
\end{itemize}
Then, with probability at least $1-\frac{40\log\frac{n}{\epsilon}}{n^2}$, $\ww^{\top}\VV\ww\leq 35 \epsilon$ for all $\ww\in \WW$.
\end{claim}
\subsubsection*{Proof of \Cref{lem:EigenspaceChange} using \Cref{cl:progress}}
Suppose for contradiction that \Cref{lem:EigenspaceChange} does not hold, i.e., the conditions on $\dim{T_\nu-\Ttil}$ of \Cref{cl:progress} hold for all $\nu \in \cal{I}$.
 On one hand, since $\lambda_{\max}(\AA)\geq 1-\epsilon$, \Cref{cl:progress} says that $\ww^{\top}\VV\ww\leq 35 \epsilon$ for all $\ww \in \WW$ with probability at least $1-\frac{40\log\frac{n}{\epsilon}}{n^2}$.
 From the assumption of \Cref{lem:EigenspaceChange}, we have $\ww^{\top}\AAtil\ww < 1- 40\epsilon$ for all $\ww\in \WW$ as well, this implies that, for all $\ww \in \WW$,
\[
\ww^{\top}\AA\ww < \ww^{\top}\AAtil\ww +\ww^{\top}\VV\ww < 1-5\epsilon.
\]
On the other hand, since $\lambda_{\max}(\AA)\geq 1- \epsilon$, we must have  $\ww^{\top}\AA\ww \geq 1-5\epsilon$ for some $\ww\in \WW$ with probability at least $1-1/n^2$ by  \Cref{thm:StaticPower}.
This gives a contradiction. 

Therefore, we conclude that, with probability at least $1-\frac{40\log\frac{n}{\epsilon}+1}{n^2}$, the conditions of \Cref{cl:progress} must be false for some $\nu \in \mathcal{I}$. That is, we have
\begin{itemize}
\item $\dim{T_{\nu} -\tilde{T}} \geq \frac{\epsilon}{300\log\frac{n}{\epsilon}} d_{\nu}$ if $d_{\nu} \geq \frac{3000\log n\log\frac{n}{\epsilon}}{\epsilon}$, and 
\item $\dim{T_{\nu} -\tilde{T}} \geq 1$ if $d_{\nu} < \frac{3000\log n\log\frac{n}{\epsilon}}{\epsilon}$.
\end{itemize}
This concludes the proof of \Cref{lem:EigenspaceChange}.


\subsubsection*{Setting Up for the Proof of \Cref{cl:progress}}

Recall the definitions of subspaces $T,T_\nu,\Ttil$ and $\overline{T}$ in \Cref{def:SpaceA}.

\begin{proposition}\label{prop:decomp space}
We can cover the entire space with the following subspaces
\begin{equation}\label{eq:SplitSpace}
 \mathbb{R}^n = T + \overline{T} = \tilde{T} + \sum_{\nu = 0}^{15\log \frac{n}{\epsilon}-1}(T_{\nu} -\tilde{T}) + \overline{T}
\end{equation}
where all subspaces in the sum are mutually orthogonal.
\end{proposition}
\begin{proof}
    It suffices to show that $T = \tilde{T} + \sum_{\nu = 0}^{15\log \frac{n}{\epsilon}-1}(T_{\nu} -\tilde{T})$.
    To see this, note that $\tilde{T}\subseteq T$. Therefore, we have $T = (T-\tilde{T}) + T\cap \tilde{T} = (T - \tilde{T}) + \tilde{T}$. We also know that $ T = \sum_{\nu=0}^{15\log \frac{n}{\epsilon}-1}T_{\nu}$ and this gives
$T - \tilde{T} = \sum_{\nu=0}^{15\log \frac{n}{\epsilon}-1} (T_{\nu}-\tilde{T})$, which concludes the proof.
\end{proof}


\paragraph{Notation.}The goal of \Cref{cl:progress} is to bound $\ww^{\top}\VV\ww\leq 35 \epsilon$ for all $\ww\in \WW$. We will use the following notations.
\begin{itemize}
    \item Let $\Pi_{\tilde{T}},\Pi_{\nu},\Pi_{\overline{T}}$ denote projection matrices to the subspaces $\tilde{T}$, $T_{\nu}-\tilde{T}$, and $\overline{T}$ respectively.
    \item  Define $\VV_{\tilde{T}} = \Pi_{\tilde{T}}\VV\Pi_{\tilde{T}}$, $\VV_{T_{\nu}-\tilde{T}} = \Pi_{\nu}\VV\Pi_{\nu}$, and $\VV_{\overline{T}} = \Pi_{\overline{T}}\VV\Pi_{\overline{T}}$.
\end{itemize}
By \Cref{prop:decomp space}, for any $\ww\in \WW$, we can decompose $\ww^{\top}\VV\ww$  as 
\[
\ww^{\top}\VV\ww = \ww^{\top}\VV_{\tilde{T}}\ww+\sum_{\nu = 0}^{15\log\frac{n}{\epsilon}-1}\ww^{\top}\VV_{T_{\nu} -\tilde{T}}\ww+ \ww^{\top}\VV_{\overline{T}}\ww.
\]
Our strategy is to upper bound each term one by one. Bounding $\ww^{\top}\VV_{\tilde{T}}\ww$ is straightforward, but bounding other terms requires technical helper lemmas.
\Cref{lem:boundLowEV} is needed for bounding $\ww^{\top}\VV_{\overline{T}}\ww$.
\Cref{lem:GaussianProjD,lem:NotImp} are helpful for bounding $\ww^{\top}\VV_{T_{\nu} -\tilde{T}}\ww$ when $\nu \in \cal{I}$ and when $\nu \notin \cal{I}$, respectively.


\subsubsection*{Helper Lemmas for \Cref{cl:progress}}
In all the statements of the helper lemmas below. Let $\WW$ be as defined in Line~\ref{line:before case} in the execution of \textsc{PowerMethod}($\epsilon,\AA$). Consider any fixed $\ww\in\WW$. Observe that we can write 
\begin{equation}\label{eq:rewrite w}  
\ww=\sum_{i=1}^{n}\frac{\lambda_{i}^{K}\alpha_{i}\uu_{i}}{\sqrt{\sum_{j}\lambda_{j}^{2K}\alpha_{j}^{2}}}
\end{equation}
where $K=\frac{4\log\frac{n}{\epsilon}}{\epsilon}$, $\alpha_{i}\sim N(0,1)$ are gaussian random variables, and $\lambda_i$ and $\uu_{i}$ are the $i$-th eigenvalue and eigenvector of $\AA$, respectively.

The following lemma shows that the projection of $\ww$ on $\overline{T}$ is always small. At a high level, 
since $\overline{T}$ is spanned by the eigenvectors with the small eigenvalues, the power method guarantees with high probability that the direction of $\ww$ along these eigenvectors will be exponentially small in the number of iterations $K$.
Recall that $\Pi_{\overline{T}}$ is the projection matrix to the space $\overline{T}$.
\begin{lemma}\label{lem:boundLowEV}
If $\lambda_{\max}(\AA)\geq 1-\epsilon$, then
\[
\Pr\left[\|\Pi_{\overline{T}}\ww\|^2 \leq \frac{\epsilon^2}{4}\right]\geq 1 - \frac{3}{n^2}.
\]
\end{lemma}
\begin{proof}
From the definition of $\overline{T}$ and $d$ defined in \Cref{def:SpaceA}, we have $\Pi_{\overline{T}}\ww = \frac{\sum_{i>d} \lambda_i^{K}\alpha_i\uu_i}{\sqrt{\sum_{j=1}^n \lambda_j^{2K}\alpha_i^2}}$ by \Cref{eq:rewrite w}. Hence,  
\[
\|\Pi_{\overline{T}}\ww\|^2 = \frac{\sum_{i>d}\lambda_i^{2K}\alpha_i^2}{\sum_{i=1}^n\lambda_i^{2K}\alpha_i^2}.
\]
First, we give a crude lower bound for the denominator. We have $$\sum_{i=1}^n\lambda_i^{2K}\alpha_i^2 \geq \frac{\lambda_1^{2K}}{n^4}$$ with probability at least $1-1/n^{2}$. This is because $\sum_{i=1}^n\lambda_i^{2K}\alpha_i^2 \geq \lambda_1^{2K}\alpha_1^2$ and,  since $\alpha_1^2 \sim \chi^2_1$,  we have $\alpha_1^2 \geq 1/n^4$ with probability at least $1-1/n^{2}$ by \Cref{lem:chi}.

Next, we upper bound the numerator as 
\[
\sum_{i>d}\lambda_i^{2K}\alpha_i^2 \leq \lambda_{d+1}^{2K}\sum_{i=1}^n\alpha_i^2.
\]
From \Cref{lem:NormG}, $\sum_{i=1}^n\alpha_i^2 \leq 2 n$ with probability at least $1-1/n^2$. Also note that, since $\lambda_{\max}(\AA)\geq 1-\epsilon \geq \lambda_0 (1-2\epsilon)$. We now have with probability $1-3/n^2$,
\[
\|\Pi\ww\|^2 \leq \left(\frac{\lambda_{d+1}}{\lambda_1}\right)^{2K}\cdot 2n^5 \leq  2n^5\left(\frac{\lambda_{d+1}}{\lambda_0 (1-2\epsilon)}\right)^{2K} \leq 2 n^5 \left(\frac{1-3\epsilon}{1-2\epsilon}\right)^{2K} \leq 2 n^5 \frac{\epsilon^6}{n^6}\leq \frac{\epsilon^2}{4}.\qedhere
\]
\end{proof}


The next two helper lemmas are to show that the projection of $\ww$ to $T_{\nu}- \tilde{T}$ is small. To do this, we introduce some more notations and one proposition. 
For any level $\nu$, we will use $q_{\nu}$ to denote,
\begin{equation}
    q_{\nu} \defeq \dim{T_{\nu}-\tilde{T}}.
\end{equation}
Let 
\begin{equation}\label{def:z}
\zz=\sum_{i}z_{i}\uu_{i}\text{ where }z_{i}=\lambda_{i}^{K}\alpha_{i}.
\end{equation} 
Therefore, $\ww=\zz/\|\zz\|$.

We now bound the norm of the projection of $\zz$ to $T_\nu - \Ttil$. The proof is based on the fact that $\zz$ is a {\it scaled} gaussian random vector, and the projection of a gaussian on a $q_{\nu}$-dimensional subspace should have norm proportional to $q_{\nu}$. 
\begin{proposition}\label{lem:projZ}
For $\zz$ as defined in~\eqref{def:z}, we have
\[
\|\Pi_{\nu}\zz\|^2 \leq \lambda_{a_{\nu}}^{2K} \cdot \sum_{j=a_{\nu}}^{b_{\nu}}\alpha_j^2.
\] 
Furthermore, if $q_{\nu}\geq 10 \log n$, then with probability at least $1-1/n^2$,
\[
\|\Pi_{\nu}\zz\|^2 \leq 2q_{\nu}\cdot \lambda_{a_{\nu}}^{2K}.
\]
\end{proposition}
\begin{proof}
Let $\Pi_{\nu}^{\text{full}}$ be a projection matrix to the subspace $T_{\nu}$. Recall from \Cref{def:Basis} that $\uu_{a_{\nu}},\dots,\uu_{b_{\nu}}$ form an orthonormal basis of $T_\nu$ and so we have $\Pi_{\nu}^{\text{full}}=\sum_{j=a_{\nu}}^{b_{\nu}}\uu_{j}\uu_{j}^{\top}$ and so $$\Pi_{\nu}^{\text{full}}\zz=\sum_{j=a_{\nu}}^{b_{\nu}}(\lambda_{j}^{K}\alpha_{j})\uu_{i}.$$ Since $T_{\nu}-\Ttil\subseteq T_{\nu}$, we have 
\[
\|\Pi_{\nu}\zz\|^{2}\le\|\Pi_{\nu}^{\text{full}}\zz\|^{2} = \lambda_{a_{\nu}}^{2K}\sum_{j=a_{\nu}}^{b_{\nu}}\alpha_{j}^{2},
\]
which proves the first part of the lemma. 

Before proving the second part, we consider the vector $\yy=\sum_{i}\alpha_{i}\uu_{i}$. Since $\alpha_{i}\sim N(0,1)$ for all $i$, we also have $\Pi_{\nu}\yy\sim N(0,1)$ is a gaussian in a $q_{\nu}$-dimensional space. So by \Cref{lem:NormG}, we have
\[
\|\Pi_{\nu}\yy\|^{2}\le2q_{\nu}
\]
with probability at least $1-e^{-q_{\nu}/4}\ge1-e^{-2\log n}=1-1/n^{2}$. 

To prove the second part, observe that $(\lambda_{a_{\nu}}^{K}\cdot\yy)$ ``dominates'' $\Pi_{\nu}^{\text{full}}\zz$ in every coordinate, i.e,. the coefficient of each $\uu_{i}$ in $(\lambda_{a_{\nu}}^{K}\cdot\yy)$ is at least that of $\Pi_{\nu}^{\text{full}}\zz$ for every $i$. Therefore, $\|\PP(\Pi_{\nu}^{\text{full}}\zz)\|\le\|\PP(\lambda_{a_{\nu}}^{K}\cdot\yy)\|$ for any projection matrix $\PP$. Since $T_{\nu}-\Ttil\subseteq T_{\nu}$, we have $\Pi_{\nu}\zz=\Pi_{\nu}\Pi_{\nu}^{\text{full}}\zz$. Therefore, we can conclude that 
\[
\|\Pi_{\nu}\zz\|^{2}=\|\Pi_{\nu}\Pi_{\nu}^{\text{full}}\zz\|^{2}\le\|\Pi_{\nu}(\lambda_{a_{\nu}}^{K}\cdot\yy)\|^{2}
\]
which is at most $\lambda_{a_{\nu}}^{2K}\cdot2q_{\nu}$ with probability at least $1-1/n^{2}$. \qedhere
\end{proof}
The following lemma shows that the projection of $\ww$ on $T_{\nu}-\Ttil$ is small when $\dim{T_\nu - \Ttil}:= q_\nu$ is roughly at most an $\epsilon$-factor of $\dim{T_\nu}:= d_\nu$, and $q_\nu$ is still at least logarithmic.
We will use this lemma to characterize the projection of $\ww$ on $T_{\nu}$ for $\nu \in \mathcal{I}$. 
Recall that $\Pi_\nu$ is a projection matrix that projects any vector to the space $T_{\nu}-\Ttil$.
\begin{lemma}\label{lem:GaussianProjD}

If $ 10 \log n \leq q_{\nu} \leq \frac{\epsilon}{300\log\frac{n}{\epsilon}}d_{\nu}$, then
\[
\Pr\left[\norm{\Pi_{\nu}\ww}^2 \leq \frac{\epsilon}{\log\frac{n}{\epsilon}} \right] \geq 1-\frac{2}{n^{2}}.
\]
\end{lemma}
\begin{proof}
Since $\ww=\zz/\|\zz\|$, it is equivalent to show that, with probability $1-\frac{2}{n^{2}}$, $$\|\Pi_{\nu}\zz\|^{2}\leq\frac{\epsilon}{\log\frac{n}{\epsilon}}\|\zz\|^{2}.$$ 

We first bound $\|\Pi_\nu \zz\|$ in terms of $d_\nu$. As $q_{\nu} \ge 10\log n$, by \Cref{lem:projZ}, we have with probability at least $1-1/n^{2}$, 
\[
\|\Pi_{\nu}\zz\|^{2}\leq\lambda_{a_{\nu}}^{2K}\cdot2q_{\nu}\le\lambda_{a_{\nu}}^{2K}\cdot\frac{\epsilon}{150\log\frac{n}{\epsilon}}d_{\nu}.
\]
where the second inequality follows from the assumption $q_{\nu}\leq\frac{\epsilon}{300\log\frac{n}{\epsilon}}d_{\nu}$. 

Next, we bound $d_{\nu}$ in terms of $\|z\|$. Consider the $d_{\nu}$-dimensional gaussian vector with coordinate $\alpha_{i}$ for $i=a_{\nu},\dots,b_{\nu}$. Applying \Cref{lem:NormG} to this vector with $\delta=1/10$, we have $\Pr[\sum_{i=a_{\nu}}^{b_{\nu}}\alpha_{i}^{2}\geq (1-\frac{1}{2})d_{\nu}]\geq1-e^{-d_{\nu}/100}\geq1-\frac{1}{n^{2}}$ where the last inequality used that $d_{\nu}\ge 3000\log n$. With probability $1-1/n^{2}$, we now have
\[
d_{\nu}\le2\sum_{i=a_{\nu}}^{b_{\nu}}\alpha_{i}^{2}=2\sum_{i=a_{\nu}}^{b_{\nu}}\frac{z_{i}}{\lambda_{i}^{2K}}\le\frac{2}{\lambda_{b_{\nu}}^{2K}}\|\zz\|^{2}
\]
Combining the two inequalities, we can conclude that, with probability at least $1-2/n^{2}$, 
\[
\|\Pi_{\nu}\zz\|^{2}\le\left(\frac{\lambda_{a_{\nu}}}{\lambda_{b_{\nu}}}\right)^{2K}\frac{\epsilon}{75\log\frac{n}{\epsilon}}\|\zz\|^{2}\le\frac{\epsilon}{\log\frac{n}{\epsilon}}\|\zz\|^{2}
\]
as desired. To see the last inequality, recall from \Cref{def:Basis} that $\lambda_{a_{\nu}}\leq\left(1-\frac{\nu\epsilon}{5\log\frac{n}{\epsilon}}\right)\lambda_{0}$ and $\lambda_{b_{\nu}}\geq\left(1-\frac{(\nu+1)\epsilon}{5\log\frac{n}{\epsilon}}\right)\lambda_{0}$. So $\frac{\lambda_{a_{\nu}}}{\lambda_{b_{\nu}}}\le1+\frac{\epsilon}{2\log\frac{n}{\epsilon}}$ and, hence, 
\[
\left(\frac{\lambda_{a_{\nu}}}{\lambda_{b_{\nu}}}\right)^{2K}\leq\left(1+\frac{\epsilon}{2\log\frac{n}{\epsilon}}\right)^{2K}\le e^{4}\approx54.6. \qedhere
\]
\end{proof}

We next prove that for all $\nu\notin \mathcal{I}$, arbitrary projections of $\ww$ on $T_{\nu}-\Ttil$ are always small. This proof uses a similar idea as that of the previous lemma and the main difference is that we can use the fact that $d_{\nu}$ is small for $\nu\notin \mathcal{I}$ to additionally show that the projection of $\ww$ is small even for small dimensional arbitrary subspaces of $T_{\nu}$. Recall that $\Pi_\nu$ is a projection matrix to the space $T_{\nu}-\Ttil$.
%
\begin{restatable}{lemma}{NotImp}\label{lem:NotImp}
For any non-important level, $\nu \notin \cal{I}$,
\[
\Pr\left[\norm{\Pi_{\nu}\ww}^2 \leq \frac{\epsilon}{\log\frac{n}{\epsilon}} \right] \geq 1-\frac{1}{n^{2}}.
\]
\end{restatable}

\begin{proof}
Again, it is sufficient to prove for $\zz$ as defined in \Cref{def:z},
\[
\Pr\left[\norm{\Pi_{\nu}\zz}^2 \leq \frac{\epsilon}{\log\frac{n}{\epsilon}}\|\zz\|^2 \right] \geq 1-\frac{1}{n^{2}}.
\]
In this proof, we consider the case of $q_{\nu}\geq 20\log n$ and $q_{\nu}<20\log n$ separately. Let us first look at $q_{\nu}\geq 20\log n$.
\paragraph{Case $q_{\nu}\geq 20\log n$:}
Our strategy will be to first bound $\|\Pi_{\nu}\zz\|^2$ by $d_{\nu}$, which can be further bounded by $\sum_{\nu'<\nu}d_{\nu'}$.
From \Cref{lem:projZ}, with probability at least $1-1/n^2$,
\[
\|\Pi_{\nu}\zz\|^2 \leq 2q_{\nu}\lambda_{a_{\nu}}^{2K}.
\]
Since $T_{\nu}-\tilde{T}\subseteq T_{\nu}$, $q_{\nu}\leq d_{\nu}$. Furthermore, since $\nu \notin \mathcal{I}$, $d_{\nu}\leq \frac{\epsilon}{600\log^3 \frac{n}{\epsilon}}\sum_{\nu'<\nu}d_{\nu'}$. Using these bounds, we then have with probability $1-1/n^2$,  
\[
\|\Pi_{\nu}\zz\|^2 \leq \frac{\epsilon}{300\log^3 \frac{n}{\epsilon}}\lambda_{a_{\nu}}^{2K}\sum_{\nu'<\nu}d_{\nu'}.
\]
We next bound $\sum_{\nu'<\nu}d_{\nu'}$ in terms of $\|\zz\|$. Consider the $\sum_{\nu'<\nu}d_{\nu'}$ dimensional gaussian vector with coordinates $\alpha_i$, for $i = 1, \cdots, b_{\nu-1}$. Applying \Cref{lem:NormG} to this vector with $\delta=1/3$ gives,
\[
\Pr\left[\sum_{i=1}^{b_{\nu-1}} \alpha_i^2 \geq \left(1-\frac{8}{9}\right)\sum_{\nu'<\nu}d_{\nu'}\right] \geq 1- e^{-\frac{\sum_{\nu'<\nu}d_{\nu'}}{9}} \geq 1-\frac{1}{n^2}.
\]
In the last inequality we used that $\sum_{\nu'<\nu}d_{\nu'}\geq d_{\nu}$ and $d_{\nu}\geq q_{\nu}\geq 20\log n$. We now know that with probability at least $1-1/n^2$,
\[
\sum_{\nu'<\nu}d_{\nu'} \leq 9\sum_{i=1}^{b_{\nu-1}}\alpha_i^2 = 9\sum_{i=1}^{b_{\nu-1}}\frac{\zz_i^2}{\lambda_i^{2K} }\leq 9\frac{\|\zz\|^2}{\lambda_{b_{\nu-1}}^{2K}}.
\]
Therefore, with a probability of at least $1-2/n^2$,
\[
\|\Pi_{\nu}\zz\|^2 \leq \frac{\epsilon}{15\log^3 \frac{n}{\epsilon}}\left(\frac{\lambda_{a_{\nu}}}{\lambda_{b_{\nu-1}}}\right)^{2K}\|\zz\|^2 \leq  \frac{\epsilon}{15\log^3 \frac{n}{\epsilon}}\|\zz\|^2.
\]
The last inequality follows from the fact $\lambda_{a_{\nu}} \leq \lambda_{b_{\nu-1}}$.
\paragraph{Case $q_{\nu} <20 \log n$:} In this case, since $q_{\nu}<20\log n$, we apply the first part of \Cref{lem:projZ} to get
\[
\|\Pi_{\nu}\zz\|^2\leq  \lambda_{a_{\nu}}^{2K} \sum_{j=a_{\nu}}^{b_{\nu}}\alpha_j^2.
\]
Observe that in this case $d_{\nu}$ can be less than $20\log n$. We also know that since $\nu \notin \mathcal{I}$, we can bound $d_{\nu}$ by $\sum_{\nu'<\nu}d_{\nu'}$. In our analysis, we consider the value of $\sum_{\nu'<\nu}d_{\nu}$ and further split it into two parts based on whether $\sum_{\nu'<\nu}d_{\nu}$ is large or small.
\begin{itemize}
    \item $\sum_{\nu'<\nu} d_{\nu'} \geq 20 \log n$: Our strategy would be to first bound $\|\Pi_{\nu}\zz\|^2$ by $\sum_{\nu'<\nu}d_{\nu}$, and then bound $\sum_{\nu'<\nu}d_{\nu}$ by $\|\zz\|^2$. Note that since $\alpha_i$'s are gaussian random variables, $\sum_{j=a_{\nu}}^{b_{\nu}}\alpha_j^2$ follows a $\chi^2_k$ distribution with $k = d_{\nu}$. Therefore, $\sum_{j=a_{\nu}}^{b_{\nu}}\alpha_j^2 \leq d_{\nu} \log n$ with probability at least $1-1/n^2$. Further since $d_{\nu}\leq \frac{\epsilon}{600 \log^3 \frac{n}{\epsilon}}\sum_{\nu'<\nu} d_{\nu'}$, we get with probability at least $1-2/n^2$,
    \[
    \|\Pi_{\nu}\zz\|^2\leq  \lambda_{a_{\nu}}^{2K} \cdot \log n\cdot d_{\nu} \leq \lambda_{a_{\nu}}^{2K}\frac{\epsilon}{600\log \frac{n}{\epsilon}}\sum_{\nu'<\nu} d_{\nu'}.
    \]
    Now, since $\sum_{\nu'<\nu} d_{\nu'} \geq 20 \log n$ we use Lemma~\ref{lem:NormG} again with $\delta = 1/3$, on a vector with coordinates $\alpha_i$'s for $i = 1,\cdots,b_{\nu-1}$ to get with probability at least $1-1/n^2$, 
    \[
    \sum_{\nu'<\nu} d_{\nu'} \leq 9\sum_{i=1}^{b_{\nu-1}}\alpha_i^2=9\sum_{i=1}^{b_{\nu-1}}\frac{z_i^2}{\lambda_i^{2K}} \leq 9\frac{\|\zz\|^2}{\lambda_{b_{\nu-1}}^{2K}}.
    \]
    Plugging this back, we get with probability at least $1-3/n^2$,
    \[
    \|\Pi_{\nu}\zz\|^2\leq \frac{\epsilon}{60\log \frac{n}{\epsilon}} \left(\frac{\lambda_{a_{\nu}}}{\lambda_{b_{\nu-1}}}\right)^{2K}\|\zz\|^2 \leq\frac{\epsilon}{60\log \frac{n}{\epsilon}}\|\zz\|^2.
\]
Last inequality follows from $\lambda_{a_{\nu}} \leq \lambda_{b_{\nu-1}}$.
\item $\sum_{\nu'<\nu} d_{\nu'} < 20 \log n$: Since $\nu\notin \mathcal{I}$, we know that 
\[
d_{\nu}\leq \frac{\epsilon}{600 \log^3\frac{n}{\epsilon}}\sum_{\nu'<\nu}d_{\nu'}.
\]
Since $\sum_{\nu'<\nu} d_{\nu'} < 20 \log n$, we must then have that,
\[
d_{\nu}\leq \frac{\epsilon}{30 \log^2\frac{n}{\epsilon}} <1.
\]
Since the dimension $d_{\nu}$ must be an integer, it must be the case that $d_{\nu} = 0$, and therefore, $\|\Pi_{\nu}\zz\| = 0.$
\end{itemize}
\end{proof}


%
\paragraph{Proof of \Cref{cl:progress}.}
 We are now ready to finally prove \Cref{cl:progress}.

\begin{proof}

We want to show that if $\lambda_{\max}(\AA)\geq 1- \epsilon$ and $\dim{T_{\nu}-\tilde{T}}$ is small for all $\nu\in \mathcal{I}$ as stated in \Cref{cl:progress}, then $\ww^{\top}\VV\ww\leq 35 \epsilon$ for all $\ww\in \WW$ with high probability. Recall that
\[
\ww^{\top}\VV\ww = \ww^{\top}\VV_{\tilde{T}}\ww+\sum_{\nu = 0}^{15\log\frac{n}{\epsilon}-1}\ww^{\top}\VV_{T_{\nu} -\tilde{T}}\ww+ \ww^{\top}\VV_{\overline{T}}\ww.
\]
Let us upper bound each term in the sum below.
\begin{enumerate}
    
    \item \textbf{$\ww^\top\VV_{\tilde{T}}\ww$:} 
    We have 
    \[
        \ww^\top \VV_{\Ttil} \ww = (\Pi_{\tilde{T}}\ww)^{\top}\VV\Pi_{\tilde{T}}\ww =
        (\Pi_{\tilde{T}}\ww)^{\top}\AA\Pi_{\tilde{T}}\ww - (\Pi_{\tilde{T}}\ww)^{\top}\AAtil\Pi_{\tilde{T}}\ww. 
    \]
    From the definition of $\tilde{T}$ (see \Cref{def:SpaceA}), we know that $(\Pi_{\tilde{T}}\ww)^{\top}\AAtil\Pi_{\tilde{T}}\ww \geq \left(1-10\epsilon\right)\lambda_0\|\Pi_{\tilde{T}}\ww\|^2 $ because $\Pi_{\tilde{T}}\ww \in \Ttil = \Span(3\eps,\AAtil).$
    We also know that $(\Pi_{\tilde{T}}\ww)^{\top}\AA\Pi_{\tilde{T}}\ww \leq \lambda_0 \|\Pi_{\tilde{T}}\ww\|^2$. So 
    \begin{equation}\label{eq:V2}
    \ww^{\top}\VV_{\tilde{T}}\ww \leq 10\epsilon\lambda_0 \|\Pi_{\tilde{T}}\ww\|^2\leq 10\epsilon(1+\epsilon)\|\ww\|^2 =10\epsilon (1+\epsilon).
  \end{equation}
\item \textbf{$\ww^\top\VV_{\overline{T}}\ww$:} Since $\ww\in \WW$, from Lemma~\ref{lem:boundLowEV}, with probability at least $1-\frac{3}{n^2}$, $\|\Pi_{\overline{T}}\ww\|\leq \epsilon/2$. Now, 
    \begin{equation}\label{eq:V1}
        \ww^{\top}\VV_{\overline{T}}\ww = (\Pi_{\overline{T}}\ww)^{\top}\VV(\Pi_{\overline{T}}\ww)\leq \|\Pi_{\overline{T}}\ww\|^2\|\VV\| \leq \frac{\epsilon^2}{4}\lambda_0 \leq \frac{\epsilon^2}{2},
    \end{equation}
    where we used that $\|\VV\|\leq \lambda_0$ since $\AAtil = \AA-\VV \succeq 0$ and $\lambda_0\leq 1+\epsilon/\log n$.
    %
    \item \textbf{$\ww^\top\VV_{T_{\nu} -\tilde{T}}\ww$ when $\nu \in \mathcal{I}$:} Note that the dimension of the space $T_{\nu} -\tilde{T}$ is small. We now have,
    \[
    \ww^{\top}\VV_{T_{\nu} -\tilde{T}}\ww = \ww^{\top}\Pi_{\nu}\VV\Pi_{\nu}\ww.
    \]
    We will now consider the large $d_{\nu}$ and small $d_{\nu}$ cases separately.
    \paragraph{Large dimension: $d_{\nu}\geq \frac{3000\log n\log\frac{n}{\epsilon}}{\epsilon}$.}
    In this case, $\dim{T_{\nu} -\tilde{T}} \leq \frac{\epsilon}{300\log\frac{n}{\epsilon}} d_{\nu}$. We can now apply Lemma~\ref{lem:GaussianProjD}, which gives with probability at least $1-\frac{2}{n^2}$,
    \[
\|\Pi_{\nu}\ww\|^2 \leq \frac{\epsilon}{\log\frac{n}{\epsilon}}.
    \]

    Now, using this value,
    \begin{equation}\label{eq:V3}
    \ww^{\top}\VV_{T_{\nu} -\tilde{T}}\ww \leq \|\Pi_{\nu}\ww\|^2 \|\VV\| \leq \frac{\epsilon}{\log\frac{n}{\epsilon}} \|\VV\|\leq  \frac{\epsilon}{\log\frac{n}{\epsilon}} \lambda_0 \leq  \frac{\epsilon(1+\epsilon)}{\log\frac{n}{\epsilon}}.
    \end{equation}
    As in case 2, we again used the fact that $\|\VV\| \le \lambda_0 \le (1+\eps)$.
    \paragraph{Small dimension: $d_{\nu}<\frac{3000\log n\log\frac{n}{\epsilon}}{\epsilon}$.}
    In this case, $\dim{T_{\nu} -\tilde{T}}<1.$ Therefore, the space $T_{\nu} -\tilde{T}$ is empty and as a result, 
    \begin{equation}\label{eq:V4}
    \ww^{\top}\VV_{T_{\nu} -\tilde{T}}\ww  = 0.
    \end{equation}
\item \textbf{$\ww^\top\VV_{T_{\nu} -\tilde{T}}\ww$ when $\nu \notin \mathcal{I}$:}
From Lemma~\ref{lem:NotImp}, $\|\Pi_{\nu}\ww\|^2 \leq \frac{\epsilon}{\log\frac{n}{\epsilon}}$ with probability at least $1-1/n^2$. Since $\|\VV\|\leq \lambda_0\leq 1+\epsilon$, we get,
\begin{equation}\label{eq:V5}
\ww^{\top}\VV_{T_{\nu} -\tilde{T}}\ww \leq \|\Pi_{\nu}\ww\|^2 \|\VV\| \le \frac{\epsilon(1+\epsilon)}{\log\frac{n}{\epsilon}}.
\end{equation}
\end{enumerate}
We now combine all the cases. We have for both large and small $d_{\nu}$ from Equations~\eqref{eq:V2},\eqref{eq:V1},\eqref{eq:V3},\eqref{eq:V4} and \eqref{eq:V5}, with probability at least $1-\frac{40\log\frac{n}{\epsilon}}{n^2}$ for any $\ww\in \WW$,
\[
\ww^{\top}\VV\ww \leq \ww^{\top}\VV_{\overline{T}}\ww + \ww^{\top}\VV_{\tilde{T}}\ww + \sum_{\nu = 0}^{15\log\frac{n}{\epsilon}-1}\ww^{\top}\VV_{T_{\nu} -\tilde{T}}\ww \leq \frac{\epsilon^2}{2} + 10\epsilon(1+\epsilon) + 10\epsilon(1+\epsilon) \leq 35\epsilon.
\]

\end{proof}



\section{Conditional Lower Bounds for an Adaptive Adversary}\label{sec:Adap}

In this section, we will prove a conditional hardness result for algorithms against adaptive adversaries. In particular, we will prove \Cref{thm:lower}.
Consider \Cref{alg:red} for solving \Cref{prob:factor}. 
The only step in \Cref{alg:red} whose implementation is not specified is Line~\ref{line:approx eig}. We will implement this step using an algorithm for \Cref{prob:dyn}.



\begin{algorithm}
\caption{Algorithm for Checking PSDness}\label{alg:red}
 \begin{algorithmic}[1]
\Procedure{CheckPSD}{$\delta,\kappa,\AA$}
\State $\epsilon \leftarrow \min\{1- n^{-o(1)}, (1-\delta)/(1+\delta)\}$
\State $T \leftarrow \frac{2n}{\epsilon(1-\epsilon)^2}\log \frac{\kappa}{\delta}$
\State $\AA_0 \leftarrow \AA$
\State $\mu_0 = 0, \ww_0 = 0$
\For{$t  = 1,2,\cdots,T$}
\State $\AA_t = \AA_{t-1} - \frac{\mu_{t-1}}{10}\ww_{t-1}\ww_{t-1}^{\top}$\label{line:red update}
\State $(\mu_t,\ww_t)\leftarrow \epsilon$-approximate maximum eigenvalue and eigenvector of $\AA_t$ (Equations~\eqref{eq:epsEigvalue},\eqref{eq:epsEigvec})\label{line:approx eig}
\If{$\mu_t<0$}
\State \Return {\sc False}:$\AA$ is not PSD
\EndIf
\EndFor
\State {$\sigma^2 \gets $PowerMethod($\epsilon,\AA_T^{\top}\AA_T$)}\label{line:last check}
\If{$0\leq \sigma \leq \frac{(1+\epsilon)\mu_1\delta}{\kappa}$}\label{line:red lastcheck}
\State \Return $\XX=\frac{1}{\sqrt{10}}\begin{bmatrix}\sqrt{\mu_1}\ww_1 & \sqrt{\mu_2}\ww_2 & \cdots & \sqrt{\mu_T}\ww_T\end{bmatrix}$
\Else
\State \Return {\sc False}:$\AA$ is not PSD
\EndIf
\EndProcedure 
 \end{algorithmic}
\end{algorithm}

\paragraph{High-level idea.}
 Overall for our hardness result, we use the idea that an adaptive adversary can use the maximum eigenvectors returned to perform an update. This can happen $n$ times and in the process, we would recover the entire eigen-decomposition of the matrix, which is hard. Now consider Algorithm~\ref{alg:red}. We claim that \Cref{alg:red} solves \Cref{prob:factor}. 
At the first glance, this claim looks suspicious because the input matrix for \Cref{prob:factor} might not be PSD, but the dynamic algorithm for \Cref{prob:dyn} at Line~\ref{line:approx eig} has any guarantees only when the matrices remain PSD. 
However, the reduction does work by crucially exploiting \Cref{prop:assume}. The high-level idea is as follows. 
\begin{itemize}
    \item If the input matrix $\AA$ is initially PSD, then we can show that $\AA_t$ remains PSD for all $t$  by exploiting \Cref{prop:assume}, (see \Cref{lem:orthoUpdate}). So, the approximation guarantee of the algorithm at Line~\ref{line:approx eig} is valid at all steps.
    From this guarantee, $\|\AA_T\|$ must be tiny since we keep decreasing the approximately maximum eigenvalues (see \Cref{lem:RedAns}). At the end, the reduction will return $\XX$.
    \item If the input matrix $\AA$ is initially \emph{not} PSD, there must exist a direction $\vv$ such that $\vv^{\top}\AA\vv <0$. Since in the reduction, we update $\AA_T = \AA - \WW$ for some $\WW \succeq 0$, we must have that $\vv^{\top}\AA_T\vv <\vv^{\top}\AA\vv$. That is, this negative direction remains negative or gets even more negative. It does not matter at all what guarantees the algorithm at Line~\ref{line:approx eig} has. We still have that $\|\AA_T\|$ cannot be tiny. We can distinguish whether $\|\AA_T\|$ is tiny or not using the static power method at Line~\ref{line:last check}, and, hence, we will return {\sc False} in this case (see \Cref{lem:RedAns}).
\end{itemize}



% {\color{blue} We now give a high-level reasoning to why our \Cref{alg:red} returns the right answer before we formally prove the reduction. When the input matrix $\AA$ is PSD, then \Cref{def:super} guarantees that $\AA_t$ remains PSD for all $t$ (see Lemma~\ref{lem:orthoUpdate}) and $\|\AA_T\|$ is small in the end (see Lemma~\ref{lem:RedAns}) as required. When $\AA$ is not PSD, there must exist a direction $\vv$ such that $\vv^{\top}\AA\vv <0$. Since in the algorithm, we create $\AA_T = \AA - \WW$ for $\WW \succeq 0$, we must have that $\vv^{\top}\AA_T\vv <\vv^{\top}\AA\vv.$ Now, since there is a large negative eigenvalue of $\AA$ and $\AA_T$, the eigenvalue with the largest magnitude ($\sigma$ computed in the end) must be large. Therefore, the algorithm must return that $\AA$ is not PSD. We now give the formal reduction.}


\paragraph{Analysis.}
We prove the guarantees of the output of Algorithm~\ref{alg:red} when $\ww_t$'s satisfy \Cref{def:super} for all $t$.


\begin{lemma}\label{lem:orthoUpdate}
In Algorithm~\ref{alg:red}, let $\ww_t$'s, $t=1,\cdots, T$ be generated such that they additionally satisfy \Cref{def:super}. If $\AA_0\succeq 0$, then $\AA_t \succeq 0$ for all $t$.
\end{lemma}
 We would like to point out that our parameter $\epsilon$ is quite large. This just implies that our reduction can work even if we find crude approximations to the maximum eigenvector as long as this is along the directions with large eigenvalue, since $\ww$ also has to satisfy \Cref{def:super}.
\begin{proof}\textcolor{red}{TOPROVE 5}\end{proof}


\begin{lemma}\label{lem:RedAns}
In Algorithm~\ref{alg:red}, let $\ww_t$'s, $t=1,\cdots, T$ be generated such that they additionally satisfy \Cref{def:super}. 
\begin{itemize}
    \item If $\AA\succeq 0$, then \Cref{alg:red} returns $\XX$ such that $\|\AA-\XX\XX^{\top}\|\leq \delta \min_{\|\xx\|=1}\|\AA\xx\|$. 
    \item If $\AA$ is not psd, then \Cref{alg:red} returns {\sc False}.
\end{itemize}
\end{lemma}
\begin{proof}\textcolor{red}{TOPROVE 6}\end{proof}


\paragraph{Proof of Theorem~\ref{thm:lower}.} 
We are now ready to prove our conditional lower bound.
\begin{proof}\textcolor{red}{TOPROVE 7}\end{proof}








