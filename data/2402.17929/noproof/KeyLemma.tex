%!TEX root = mainEV.tex


\subsection{Proof of the Key Lemma~\ref{lem:BoundW}: Few Executions of Line~\ref{algline:PM}}

\label{sec:proof key}

In this section, we prove Lemma~\ref{lem:BoundW} assuming the Progress Lemma~\ref{lem:EigenspaceChange}. 
%
Let us first recall the precise definition of $\AA$ and $\AAtil$. Suppose we execute Line~\ref{algline:PM} at update $t_{0}$. Now consider a sequence of updates, $\vv_{t_{0}+1},\cdots,\vv_{t_{0}+k}$, and let $\AA_{t_{0}+k}=\AA_{t_{0}}-\sum_{i=1}^{k}\vv_{t+i}\vv_{t+i}^{\top}.$ Suppose the next execution of Line~\ref{algline:PM} happens at $t_{0}+k$. For this to happen we must have that $\ww_{t_{0}}^{\top}\AA_{t_{0}+k}\ww_{t_{0}}<1-40\epsilon$ for all $\ww_{t_{0}}\in\WW_{t_{0}}$. We let $\AA=\AA_{t_{0}}$ and $\AAtil=\AA_{t_{0}+k}$. 

Next, recall Definition~\ref{def:SpaceA} and define $T_{\leq i}=\sum_{\nu=0}^{i}T_{\nu}$, $\tilde{T}_{\leq i}=\sum_{\nu=0}^{i}\tilde{T}_{\nu}$. The following observation will motivate our potential function analysis.
\begin{lemma}
	\label{lem:Monotone} For all $i\leq15\log\frac{n}{\epsilon}$, 
	\[
	\dim{T_{\le i}}\ge\dim{\Ttil_{\le i}}.
	\]
	Let $\nu_{0}$ be such that $\dim{T_{\nu_{0}}-\tilde{T}}>0$. For any $i\ge\nu_{0}$, we have
	\[
	\dim{T_{\le i}}\geq\dim{\Ttil_{\le i}}+\dim{T_{\nu_{0}}-\Ttil}.
	\]
	
\end{lemma}

\begin{proof}\textcolor{red}{TOPROVE 0}\end{proof}

\paragraph{The Potentials.}
The above lemma and \Cref{lem:EigenspaceChange} motivate the following potentials.
For every $j\le15\log\frac{n}{\epsilon}$, we define the potentials $\Phi_{j}=\dim{T_{\le j}}$ and $\tilde{\Phi}_{j}=\dim{\Ttil_{\le j}}$. For all $j$, the potential may only decrease, i.e., $\Phitil_{j}\le\Phi_{j}$ by \Cref{lem:Monotone}. Also, clearly, $\Phi_{j}\le n$. 

We will show that for each execution of Line~\ref{algline:PM}, $\Phitil_{j}$ decreases from $\Phi_j$ by a significant factor with high probability for some $j$. This will bound the number of executions.

Consider any important level $\nu_{0} \in \cal{I}$. We have two observations:
\begin{enumerate}
	\item \label{enu:phi 1} $\Phitil_{\nu_{0}}\leq\Phi_{\nu_{0}}-\dim{T_{\nu_{0}}-\tilde{T}}$, and 
	\item \label{enu:phi 2} $d_{\nu_{0}}\geq\Omega(1)\frac{\epsilon}{\log^{3}\frac{n}{\epsilon}}\Phi_{v_{0}}$. 
\end{enumerate}
The first point follows directly from the second part of Lemma~\ref{lem:Monotone}. Moreover, since $\nu_{0}\in\mathcal{I}$ is important, we have $d_{\nu_{0}}\geq\frac{\epsilon}{600\log^{3}\frac{n}{\epsilon}}\sum_{\nu'<\nu}d_{\nu'}.$ Therefore, 
\[
\Phi_{\nu_{0}}=\sum_{\nu=1}^{\nu_{0}}d_{\nu}=\sum_{\nu'<\nu_{0}}d_{\nu'}+d_{\nu_{0}}\leq\left(\frac{600\log^{3}\frac{n}{\epsilon}}{\epsilon}+1\right)d_{\nu_{0}},
\]
implying the second point. 
%
Combining these observations with the Progress \Cref{lem:EigenspaceChange}, we have:
\begin{lemma}
	\label{claim:potential decrease}Suppose that $\lambda_{\max}(\AA)\ge1-\epsilon$ and $\ww^{\top}\AAtil\ww<1-40\epsilon$ for all $\ww\in\WW$. With probability at least $1-50\log\frac{n}{\epsilon}/n^{2}$, there is a level $\nu_{0}$ such that either 
	\begin{itemize}
		\item $\Phitil_{\nu_{0}}\le\left(1-\frac{\Omega(\epsilon^{2})}{\log^{4}\frac{n}{\epsilon}}\right)\Phi_{\nu_{0}}$, or 
		\item $\Phitil_{\nu_{0}}\le\Phi_{\nu_{0}}-1$ and $\Phi_{\nu_{0}}\leq O(1)\frac{\log^{4}\frac{n}{\epsilon}\log n}{\epsilon^{2}}.$
	\end{itemize}
\end{lemma}

\begin{proof}\textcolor{red}{TOPROVE 1}\end{proof}
We are now ready to prove Lemma~\ref{lem:BoundW}. 

\subsubsection*{Proof of Lemma~\ref{lem:BoundW}.}

First, observe that whenever $\lambda_{\max}(\AA)<1-\epsilon$, by Line~\ref{line: if all w 1-eps} of \Cref{alg:PowerMethod}, we will always return $\textsc{False}$ at the next execution of Line \ref{algline:PM} of \Cref{alg:DynamicMaxPM}.

Therefore, it suffices to bound the number of executions while $\lambda_{\max}(\AA)\ge1-\epsilon$. We execute Line~\ref{algline:PM} only if $\ww^{\top}\AAtil\ww<1-40\epsilon$ for all $\ww\in\WW$. When this happens, there exists a level $j$ where the potential $\Phi_{j}$ significantly decreases according to \Cref{claim:potential decrease} with probability at least $1-50\log\frac{n}{\epsilon}/n^{2}$. For each level $j$, this can happen at most $L=O(\frac{\log n\log^{4}\frac{n}{\epsilon}}{\epsilon^{2}})$ times because for every $j$, $\Phi_{j}$ is an integer that may only decrease and is bounded by $n$. Suppose for contradiction that there are more than $L\times15\log\frac{n}{\epsilon}$ executions of Line~\ref{algline:PM}. So, with probability at least $1-L\cdot50\log\frac{n}{\epsilon}/n^{2}\ge1/n$, there exists a level $j$ where $\Phi_{j}$ decreases according to \Cref{claim:potential decrease} strictly more than $L$ times. This is a contradiction. 











\subsection{Proof of the Progress Lemma~\ref{lem:EigenspaceChange}}\label{sec:progress}
It remains to prove the Progress Lemma. We first restate the lemma here.

\progress*
Recall the definitions of subspaces $T,T_\nu,\Ttil$ and $\overline{T}$ in \Cref{def:SpaceA}.
We will first state the following claim and show that this is sufficient to prove \Cref{lem:EigenspaceChange}. After concluding the proof of \Cref{lem:EigenspaceChange}, we would prove the claim.

\begin{claim}\label{cl:progress} Suppose that $\lambda_{\max}(\AA) \geq 1-\epsilon$. If for every $\nu \in \mathcal{I}$, 
\begin{itemize}
\item $\dim{T_{\nu} -\tilde{T}} < \frac{\epsilon}{300\log\frac{n}{\epsilon}} d_{\nu}$ if $d_{\nu} \geq \frac{3000\log n\log\frac{n}{\epsilon}}{\epsilon}$, and 
\item $\dim{T_{\nu} -\tilde{T}} < 1$ if $d_{\nu} < \frac{3000\log n\log\frac{n}{\epsilon}}{\epsilon}$.
\end{itemize}
Then, with probability at least $1-\frac{40\log\frac{n}{\epsilon}}{n^2}$, $\ww^{\top}\VV\ww\leq 35 \epsilon$ for all $\ww\in \WW$.
\end{claim}
\subsubsection*{Proof of \Cref{lem:EigenspaceChange} using \Cref{cl:progress}}
Suppose for contradiction that \Cref{lem:EigenspaceChange} does not hold, i.e., the conditions on $\dim{T_\nu-\Ttil}$ of \Cref{cl:progress} hold for all $\nu \in \cal{I}$.
 On one hand, since $\lambda_{\max}(\AA)\geq 1-\epsilon$, \Cref{cl:progress} says that $\ww^{\top}\VV\ww\leq 35 \epsilon$ for all $\ww \in \WW$ with probability at least $1-\frac{40\log\frac{n}{\epsilon}}{n^2}$.
 From the assumption of \Cref{lem:EigenspaceChange}, we have $\ww^{\top}\AAtil\ww < 1- 40\epsilon$ for all $\ww\in \WW$ as well, this implies that, for all $\ww \in \WW$,
\[
\ww^{\top}\AA\ww < \ww^{\top}\AAtil\ww +\ww^{\top}\VV\ww < 1-5\epsilon.
\]
On the other hand, since $\lambda_{\max}(\AA)\geq 1- \epsilon$, we must have  $\ww^{\top}\AA\ww \geq 1-5\epsilon$ for some $\ww\in \WW$ with probability at least $1-1/n^2$ by  \Cref{thm:StaticPower}.
This gives a contradiction. 

Therefore, we conclude that, with probability at least $1-\frac{40\log\frac{n}{\epsilon}+1}{n^2}$, the conditions of \Cref{cl:progress} must be false for some $\nu \in \mathcal{I}$. That is, we have
\begin{itemize}
\item $\dim{T_{\nu} -\tilde{T}} \geq \frac{\epsilon}{300\log\frac{n}{\epsilon}} d_{\nu}$ if $d_{\nu} \geq \frac{3000\log n\log\frac{n}{\epsilon}}{\epsilon}$, and 
\item $\dim{T_{\nu} -\tilde{T}} \geq 1$ if $d_{\nu} < \frac{3000\log n\log\frac{n}{\epsilon}}{\epsilon}$.
\end{itemize}
This concludes the proof of \Cref{lem:EigenspaceChange}.


\subsubsection*{Setting Up for the Proof of \Cref{cl:progress}}

Recall the definitions of subspaces $T,T_\nu,\Ttil$ and $\overline{T}$ in \Cref{def:SpaceA}.

\begin{proposition}\label{prop:decomp space}
We can cover the entire space with the following subspaces
\begin{equation}\label{eq:SplitSpace}
 \mathbb{R}^n = T + \overline{T} = \tilde{T} + \sum_{\nu = 0}^{15\log \frac{n}{\epsilon}-1}(T_{\nu} -\tilde{T}) + \overline{T}
\end{equation}
where all subspaces in the sum are mutually orthogonal.
\end{proposition}
\begin{proof}\textcolor{red}{TOPROVE 2}\end{proof}


\paragraph{Notation.}The goal of \Cref{cl:progress} is to bound $\ww^{\top}\VV\ww\leq 35 \epsilon$ for all $\ww\in \WW$. We will use the following notations.
\begin{itemize}
    \item Let $\Pi_{\tilde{T}},\Pi_{\nu},\Pi_{\overline{T}}$ denote projection matrices to the subspaces $\tilde{T}$, $T_{\nu}-\tilde{T}$, and $\overline{T}$ respectively.
    \item  Define $\VV_{\tilde{T}} = \Pi_{\tilde{T}}\VV\Pi_{\tilde{T}}$, $\VV_{T_{\nu}-\tilde{T}} = \Pi_{\nu}\VV\Pi_{\nu}$, and $\VV_{\overline{T}} = \Pi_{\overline{T}}\VV\Pi_{\overline{T}}$.
\end{itemize}
By \Cref{prop:decomp space}, for any $\ww\in \WW$, we can decompose $\ww^{\top}\VV\ww$  as 
\[
\ww^{\top}\VV\ww = \ww^{\top}\VV_{\tilde{T}}\ww+\sum_{\nu = 0}^{15\log\frac{n}{\epsilon}-1}\ww^{\top}\VV_{T_{\nu} -\tilde{T}}\ww+ \ww^{\top}\VV_{\overline{T}}\ww.
\]
Our strategy is to upper bound each term one by one. Bounding $\ww^{\top}\VV_{\tilde{T}}\ww$ is straightforward, but bounding other terms requires technical helper lemmas.
\Cref{lem:boundLowEV} is needed for bounding $\ww^{\top}\VV_{\overline{T}}\ww$.
\Cref{lem:GaussianProjD,lem:NotImp} are helpful for bounding $\ww^{\top}\VV_{T_{\nu} -\tilde{T}}\ww$ when $\nu \in \cal{I}$ and when $\nu \notin \cal{I}$, respectively.


\subsubsection*{Helper Lemmas for \Cref{cl:progress}}
In all the statements of the helper lemmas below. Let $\WW$ be as defined in Line~\ref{line:before case} in the execution of \textsc{PowerMethod}($\epsilon,\AA$). Consider any fixed $\ww\in\WW$. Observe that we can write 
\begin{equation}\label{eq:rewrite w}  
\ww=\sum_{i=1}^{n}\frac{\lambda_{i}^{K}\alpha_{i}\uu_{i}}{\sqrt{\sum_{j}\lambda_{j}^{2K}\alpha_{j}^{2}}}
\end{equation}
where $K=\frac{4\log\frac{n}{\epsilon}}{\epsilon}$, $\alpha_{i}\sim N(0,1)$ are gaussian random variables, and $\lambda_i$ and $\uu_{i}$ are the $i$-th eigenvalue and eigenvector of $\AA$, respectively.

The following lemma shows that the projection of $\ww$ on $\overline{T}$ is always small. At a high level, 
since $\overline{T}$ is spanned by the eigenvectors with the small eigenvalues, the power method guarantees with high probability that the direction of $\ww$ along these eigenvectors will be exponentially small in the number of iterations $K$.
Recall that $\Pi_{\overline{T}}$ is the projection matrix to the space $\overline{T}$.
\begin{lemma}\label{lem:boundLowEV}
If $\lambda_{\max}(\AA)\geq 1-\epsilon$, then
\[
\Pr\left[\|\Pi_{\overline{T}}\ww\|^2 \leq \frac{\epsilon^2}{4}\right]\geq 1 - \frac{3}{n^2}.
\]
\end{lemma}
\begin{proof}\textcolor{red}{TOPROVE 3}\end{proof}


The next two helper lemmas are to show that the projection of $\ww$ to $T_{\nu}- \tilde{T}$ is small. To do this, we introduce some more notations and one proposition. 
For any level $\nu$, we will use $q_{\nu}$ to denote,
\begin{equation}
    q_{\nu} \defeq \dim{T_{\nu}-\tilde{T}}.
\end{equation}
Let 
\begin{equation}\label{def:z}
\zz=\sum_{i}z_{i}\uu_{i}\text{ where }z_{i}=\lambda_{i}^{K}\alpha_{i}.
\end{equation} 
Therefore, $\ww=\zz/\|\zz\|$.

We now bound the norm of the projection of $\zz$ to $T_\nu - \Ttil$. The proof is based on the fact that $\zz$ is a {\it scaled} gaussian random vector, and the projection of a gaussian on a $q_{\nu}$-dimensional subspace should have norm proportional to $q_{\nu}$. 
\begin{proposition}\label{lem:projZ}
For $\zz$ as defined in~\eqref{def:z}, we have
\[
\|\Pi_{\nu}\zz\|^2 \leq \lambda_{a_{\nu}}^{2K} \cdot \sum_{j=a_{\nu}}^{b_{\nu}}\alpha_j^2.
\] 
Furthermore, if $q_{\nu}\geq 10 \log n$, then with probability at least $1-1/n^2$,
\[
\|\Pi_{\nu}\zz\|^2 \leq 2q_{\nu}\cdot \lambda_{a_{\nu}}^{2K}.
\]
\end{proposition}
\begin{proof}\textcolor{red}{TOPROVE 4}\end{proof}
The following lemma shows that the projection of $\ww$ on $T_{\nu}-\Ttil$ is small when $\dim{T_\nu - \Ttil}:= q_\nu$ is roughly at most an $\epsilon$-factor of $\dim{T_\nu}:= d_\nu$, and $q_\nu$ is still at least logarithmic.
We will use this lemma to characterize the projection of $\ww$ on $T_{\nu}$ for $\nu \in \mathcal{I}$. 
Recall that $\Pi_\nu$ is a projection matrix that projects any vector to the space $T_{\nu}-\Ttil$.
\begin{lemma}\label{lem:GaussianProjD}

If $ 10 \log n \leq q_{\nu} \leq \frac{\epsilon}{300\log\frac{n}{\epsilon}}d_{\nu}$, then
\[
\Pr\left[\norm{\Pi_{\nu}\ww}^2 \leq \frac{\epsilon}{\log\frac{n}{\epsilon}} \right] \geq 1-\frac{2}{n^{2}}.
\]
\end{lemma}
\begin{proof}\textcolor{red}{TOPROVE 5}\end{proof}

We next prove that for all $\nu\notin \mathcal{I}$, arbitrary projections of $\ww$ on $T_{\nu}-\Ttil$ are always small. This proof uses a similar idea as that of the previous lemma and the main difference is that we can use the fact that $d_{\nu}$ is small for $\nu\notin \mathcal{I}$ to additionally show that the projection of $\ww$ is small even for small dimensional arbitrary subspaces of $T_{\nu}$. Recall that $\Pi_\nu$ is a projection matrix to the space $T_{\nu}-\Ttil$.
%
\begin{restatable}{lemma}{NotImp}\label{lem:NotImp}
For any non-important level, $\nu \notin \cal{I}$,
\[
\Pr\left[\norm{\Pi_{\nu}\ww}^2 \leq \frac{\epsilon}{\log\frac{n}{\epsilon}} \right] \geq 1-\frac{1}{n^{2}}.
\]
\end{restatable}

\begin{proof}\textcolor{red}{TOPROVE 6}\end{proof}


%
\paragraph{Proof of \Cref{cl:progress}.}
 We are now ready to finally prove \Cref{cl:progress}.

\begin{proof}\textcolor{red}{TOPROVE 7}\end{proof}

