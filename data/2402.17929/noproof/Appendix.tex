%!TEX root = mainEV.tex

\section{Omitted Proofs}
\Decision*
\begin{proof}\textcolor{red}{TOPROVE 0}\end{proof}

\subsection*{Power Method}

The following lemma proves that at least one initial random vector has a component along the eigenspace of the maximum eigenvalue.
\begin{lemma}\label{lem:PMHighComp}
Let $\uu_1,\cdots \uu_l$ denote the maximum eigenvectors of $\AA$ and let $\vv^{(0)} \in \mathbb{R}^n$ denote the vector with entries sampled indenpendently from $N(0,1)$, i.e., $\vv^{(0)} = \sum_{i=1}^n\alpha_i\uu_i$, where $\alpha_i \sim N(0,1)$, $\uu_i$'s are eigenvectors of $\AA$. Then, with probability at least $3/4$, $\sum_{i=1}^l\alpha_i^2 \geq \frac{1}{25}$.
\end{lemma}
\begin{proof}\textcolor{red}{TOPROVE 1}\end{proof}

\PowerMethod*
\begin{proof}\textcolor{red}{TOPROVE 2}\end{proof}

